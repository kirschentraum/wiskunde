\documentclass[12pt, a4paper, oneside]{book}

\usepackage[UTF8]{ctex}
\usepackage{bm}
\usepackage{amsmath}
\usepackage{amssymb}
\usepackage[colorlinks,linkcolor=black]{hyperref}
\usepackage{amsthm}
\usepackage{verbatim}
\usepackage{titletoc}
\usepackage{titlesec}
\usepackage{imakeidx}
\usepackage{multicol}
\usepackage{geometry}

\geometry{a4paper,scale=0.8}

\newtheorem{metadef}{元数学定义}
\newtheorem{sign}{记号定义}
\newtheorem{CS}{替代规则}
\newtheorem{CScor}{补充替代规则}
\newtheorem{CF}{形成规则}
\newtheorem{CFcor}{补充形成规则}
\newtheorem{C}{证明规则}
\newtheorem{Ccor}{补充证明规则}
\newtheorem{gramdef}{语法定义}
\newtheorem{gramthe}{语法定理}
\newtheorem{gramcor}{补充语法定理}
\newtheorem{Sch}{公理模式}
\newtheorem{ex}{显式公理}
\newtheorem{theo}{定理}
\newtheorem{cor}{补充定理}
\newtheorem{de}{定义}
\newtheorem{STdef}{结构定义}
\newtheorem{CST}{结构规则}
\newtheorem{CSTcor}{补充结构规则}
\newtheorem{exer}{习题}

\title{布尔巴基数学基础(第1卷)集合论学习笔记}
\author{何嘉惠,黄芸芸}
\date{二零二四年十月}

\begin{document}

	\pagestyle{plain}
	\frontmatter
	\hypersetup{pageanchor=false}
	\maketitle
	\hypersetup{pageanchor=true}

	\chapter{说明}
		本文系《布尔巴基数学基础(第1卷)集合论》(施普林格出版社2006年版)学习笔记,包括下列内容:
		\par
		元数学(métamathématique):
		\par
		(1)元数学定义1--65:大部分系原书内容,部分做了调整;
		\par
		(2)记号定义1--23:大部分系原书内容,部分做了调整;
		\par
		(3)替代规则(critère de substitution)1--12:按原书编号;
		\par
		(4)补充替代规则1--8:即原书未编号或自行补充的替代规则,一项规则为公理模式的命题,也纳入补充替代规则;
		\par
		(5)形成规则(critère formatif)1--13:按原书编号;
		\par
		(6)补充形成规则1--2:即原书未编号或自行补充的形成规则;
		\par
		(7)证明规则(critère déductif)1--63:按原书编号;
		\par
		(8)补充证明规则1--91:即原书未编号或自行补充的证明规则;
		\par
		(9)语法定义1--10:系原书第一章附录的定义;
		\par
		(10)语法定理1--8:系原书第一章附录的定理;
		\par
		(11)补充语法定理1--2:即自行补充的语法定理;
		\par
		数学(mathématique):
		\par
		(1)定义1--206,原书提及的定义,部分做了调整;
		\par
		(2)公理模式1--8,原书使用的8个公理模式;
		\par
		(3)显式公理1--4,原书使用的4个显式公理;
		\par
		(4)定理1--193,原书以斜体标明的定理;
		\par
		(5)补充定理1--432,原书未以斜体标明的内容或自行补充的定理;
		\par
		(6)结构定义1--39:系原始第四章的定义;
		\par
		(7)结构规则1--23:按原书第四章的编号;
		\par
		(8)补充结构规则1--14:即原书未编号或自行补充的结构规则.
		\par
		习题(exercice)1--212:按照原书顺序编号,它可能属于数学也可能属于元数学,不做区分.其中部分习题的表述做了调整.
		\par
		注:元数学定义中可能出现“自然数”等概念,但与数学中的“自然数”等概念无关,不存在循环定义.

	\newpage
	\tableofcontents
	\mainmatter
	
	\chapter{形式数学的描述(Description de la mathématique formelle)}
		\section{项和公式(Termes et relations)}
			\begin{metadef}
				\textbf{理论(théorie)}
				\par
				理论是指通过预先确定的“特别符号”、“公理模式”、“显式公理”三栏内容生成的规则体系.
			\end{metadef}
			\par
			注:理论是元数学的基本概念,在元数学中,实际上只能描述而无法定义.其中提到的“特别符号”、“公理模式”、“显式公理”将在后面定义.
			
			\begin{metadef}
				\textbf{特别符号(signe spécifique)}
				\par
				特别符号是指理论的“特别符号”一栏列举的字符.
			\end{metadef}

			\begin{metadef}
				\textbf{特别符号的元(poid de signe spécifique)}
				\par
				特别符号的元,是指理论列举特殊符号时,同时列出的对应自然数.
			\end{metadef}		

			\begin{metadef}
				\textbf{逻辑符号(signe logique)}
				\par
				逻辑符号是指$\Box$、$\tau$、$\lor$、$\neg$四个符号.
			\end{metadef}
			\par		
			注:
			\par
			$\lor$表示“析取”,$\neg$表示“否定”.
			\par
			$\tau$与某种性质相关联:令论域为良序集,如果论域中存在使该性质成立的对象,则它表示论域中具备此种性质的第一个对象;如果论域中不存在使该性质成立的对象,则它论域中的第一个对象.并且,论域的第一个对象除了与自身相等外,不具备任何其他性质.
			\par
			$\Box$必须与左侧的某一个$\tau$用连线连接,表示$\tau$的参数;$\tau$可以与右侧的多个$\Box$连接,也可以独立存在,表示没有参数.

			\begin{metadef}
				\textbf{字母(lettre)}
				\par
				字母是指大写或小写拉丁字母,其可以附带一个自然数下标,也可以附带有限个单引号的上标.
			\end{metadef}				
			\par
			注:字母的数量为可数集即可,按照习惯,仅允许附带一个自然数下标和有限个单引号上标.

			\begin{metadef}
				\textbf{符号(signe)}
				\par
				逻辑符号、字母,特别符号统称符号.
			\end{metadef}		

			\begin{metadef}
				\textbf{语句(assemblage), 连线(lien)}
				\par
				有限符号序列,可以不附加任何线,也可以附加一条或多条线,但任何一条线只能把一个符号和另一个符号连接起来,称为语句.语句附加的线,称为连线.
			\end{metadef}

			\begin{sign}
				\textbf{符号序列的连接(connexion de suites de signes)}
				\par
				令$A$、$B$为符号有限序列,$AB$表示将符号序列$B$依次写在符号序列$A$的右边而得到的序列.
			\end{sign}
			
			\begin{sign}
				\textbf{蕴涵(implication)}
				\par
				在语句中,“$\lor\neg$”简记为“$\Rightarrow$”.
			\end{sign}		

			\begin{sign}
				\textbf{中序表达式(notation infixée)}
				\par
				语句也可以用中序表达式表示.
				\par
				中序表达式的运算符,分为五个优先级.
				\par
				中序表达式的运算符,在定义时没有特别说明优先级的,如果该运算符有两个参数,且没有使用上标、下标或者$\langle\rangle$、$()$、$\{\}$,则为第二优先级;其他情况下,为第一优先级.
				\par
				其中,第二优先级、第三优先级、第四优先级、第五优先级的运算符均为左结合,第一优先级运算符均为右结合.可以前后加括号改变符号优先级.
			\end{sign}
			注:
			\par
			语句采用前序表达式,但为符合通常的书写习惯,故设置改写为中序表达式的规则.
			\par
			原则上,只有一个参数,或者包含上下标或各种括号,为第一优先级;对两个项进行运算得到一个新项,为第二优先级;对两个项进行运算得到一个公式,为第三优先级;析取和合取为第四优先级;蕴含和等价为第五优先级.

			\begin{sign}
				\textbf{取得对象(obtention de l'objet)}
				\par
				令$A$为语句,$x$为字母,则用$\tau_x(A)$表示这样的语句:语句$\tau A$当中所有的符号$x$替代为符号$\Box$,并将替代得到的$\Box$和开头的$\tau$连线.
			\end{sign}
			
			\begin{sign}
				\textbf{替代(substitution)}
				\par
				令$A$、$B$为语句,$x$为字母,则用$(B|x)A$表示以$B$替代$A$当中所有的符号$x$而得到的语\\句.
			\end{sign}
			注:原书还使用了带参数的语句取代替代记号,本文不采用这种写法,仍使用替代记号.

			\begin{CScor}\label{CScor1}
				\hfill\par
				令$A$为语句,$x$为字母,则$(x|x)A$和$A$相同.
			\end{CScor}		
			证明:根据定义可证.

			\begin{CS}\label{CS1}
				\hfill\par
				令$A$、$B$为语句,$x$、$x'$为字母,如果$A$不包含$x'$,则$(B|x)A$和$(B|x')(x'|x)A$相同.
			\end{CS}		
			证明:对$A$的长度用数学归纳法可证.

			\begin{CScor}\label{CScor2}
				\hfill\par
				令$A$为语句,$x$、$y$为字母,且$A$不包含$x$,则$(y|x)A$和$A$相同.
			\end{CScor}		
			证明:根据定义可证.
			

			\begin{CScor}\label{CScor3}
				\hfill\par
				令$A$为语句,$x$、$x'$为字母,如果$A$不包含$x'$,则$(x|x')(x'|x)A$和$A$相同.
			\end{CScor}		
			证明:根据替代规则\ref{CS1}、补充替代规则\ref{CScor1}可证.

			\begin{CS}\label{CS2}
				\hfill\par
				令$A$、$B$、$C$为语句,$x$、$y$为不同字母,如果$B$不包含$y$,则$(B|x)(C|y)A$和$(C'|y)(B|x)\\A$相同,其中$C'$为$(B|x)C$.
			\end{CS}		
			证明:对$A$的长度用数学归纳法可证.

			\begin{CScor}\label{CScor4}
				\hfill\par
				令$A$为语句,$x$、$y$、$z$为不同字母,且$A$不包含$z$,则$(y|z)(x|y)(z|x)A$和$(x|z)(y|x)(z|y)\\A$相同.
			\end{CScor}		
			证明:
			\par
			令$u$为不同于$x$、$y$、$z$的字母,且$A$不包含$u$.
			\par
			根据替代规则\ref{CS1},$(x|z)(y|x)(z|y)A$和$(x|z)(y|u)(u|x)(z|y)$相同;
			\par
			根据替代规则\ref{CS2},$(x|z)(y|u)(u|x)(z|y)$和$(y|u)(x|z)(z|y)(u|x)A$相同;
			\par
			根据替代规则\ref{CS1},$(y|u)(x|z)(z|y)(u|x)A$和$(y|u)(x|y)(u|x)A$相同;
			\par
			根据补充替代规则\ref{CScor3},$(y|u)(x|y)(u|x)A$和$(y|u)(x|y)(u|z)(z|u)(u|x)A$相同;
			\par
			根据替代规则\ref{CS1},$(y|u)(x|y)(u|z)(z|u)(u|x)A$和$(y|u)(u|z)(x|y)(z|u)(u|x)A$相同;
			\par
			根据替代规则\ref{CS1},$(y|u)(u|z)(x|y)(z|u)(u|x)A$和$(y|z)(x|y)(z|x)A$相同.
			\par
			得证.
			\par
			注:$(y|z)(x|y)(z|x)A$和$(x|z)(y|x)(z|y)A$即为将语句$A$中的$x$和$y$交换后得到的语句.

			\begin{CS}\label{CS3}
				\hfill\par
				令$A$为语句,$x$、$x'$为字母,如果$A$不包含$x'$,则$\tau_x(A)$和$\tau_{x'}(A')$相同,其中$A'$为$(x'|x)A$.
			\end{CS}
			证明:对$A$的长度用数学归纳法可证.

			\begin{CS}\label{CS4}
				\hfill\par
				令$A$、$B$为语句,$x$、$y$为不同字母,如果$B$不包含$x$,则$(B|y)\tau_x(A)$和$\tau_x(A')$相同 ,其中$A'$\\为$(B|y)A$.
			\end{CS}
			证明:对A的长度用数学归纳法可证.

			\begin{CS}\label{CS5}
				\hfill\par
				令$A$、$B$、$C$为语句,$x$为字母,$s$为特别符号,则$(C|x)(\neg A)$、$(C|x)(\lor AB)$、$(C|x)(\Rightarrow AB)$、$(C|x)(sAB)$分别和$\neg A'$、$\lor A'B'$、$\Rightarrow A'B'$、$'B'$相同,其中$A'$为$(C|x)A$,$B'$为$(C|x)B$.
			\end{CS}
			证明:对A的长度用数学归纳法可证.

			\begin{metadef}
				\textbf{第一类语句(assemblage de première espèce),第二类语句\\(assemblage de deuxième espèce)}
				\par
				以$\tau$开头的语句或单个字母,称为第一类语句,其他语句称为第二类语句.
			\end{metadef}

			\begin{metadef}
				\textbf{构造(construction formative)}
				理论$M$的一个构造是指语句有限序列,并且序列中的语句$A$均符合下列规则之一:
				\par
				(1)$A$是字母;
				\par
				(2)存在位于$A$之前的第二类语句$B$,使$A$为$\neg B$;
				\par
				(3)存在位于$A$之前的第二类语句$B$和$C$,使$A$为$\lor BC$;
				\par
				(4)存在位于$A$之前的第二类语句$B$,以及字母$x$,使$A$为$\tau_x(B)$;
				\par
				(5)	存在一个$n$元特别符号$s$,以及位于$A$之前的$n$个第一类语句$A_1$、$A_2$、$\cdots$、$A_n$,使$A$为$sA_1A_2\cdots A_n$.
			\end{metadef}
			
			\begin{metadef}
				\textbf{项(terme)、公式(relation)}
				\par
				理论$M$的项是指,该理论的某个构造中的第一类语句;理论$M$的公式是指,该理论的某个构造中的第二类语句.
			\end{metadef}

			\begin{CF}\label{CF1}
				\hfill\par
				如果$A$和$B$都是理论$M$的公式,则$\lor AB$也是理论$M$的公式.
			\end{CF}
			证明:将包含$A$的构造和包含$B$的构造合在一起.由于$A$和$B$都是公式,因此可以加入语句$\lor AB$,故$\lor AB$也是公式.

			\begin{CF}\label{CF2}
				\hfill\par
				如果$A$是理论$M$的公式,则$\neg A$也是理论$M$的公式.
			\end{CF}
			证明:在包含$A$的构造中,由于$A$是公式,因此可以加入语句$\neg A$,故$\neg A$也是公式.

			\begin{CF}\label{CF3}
				\hfill\par
				如果$A$是理论$M$的公式,$x$为字母,则$\tau_x(A)$是理论$M$的项.
			\end{CF}
			证明:在包含$A$的构造中,由于$A$是公式,因此可以加入语句$\tau_x(A)$,故$\tau_x(A)$是项.

			\begin{CF}\label{CF4}
				\hfill\par
				如果$A_1$、$A_2$、$\cdots$、$A_n$都是理论$M$的项,$s$是理论$M$的$n$元特别符号,则$sA_1A_2\cdots A_n$是理论$M$的公式.
			\end{CF}
			证明:将包含$A_1$、$A_2$、$\cdots$、$A_n$的各构造合在一起.由于$A_1$、$A_2$、$\cdots$、$A_n$都是项,因此可以加入语句$sA_1A_2\cdots A_n$,故$sA_1A_2\cdots A_n$是公式.

			\begin{CF}\label{CF5}
				\hfill\par
				如果$A$和$B$都是理论$M$的公式,则$\Rightarrow AB$也是理论$M$的公式.
			\end{CF}
			证明:$\Rightarrow AB$即$\lor\neg AB$,根据形成规则\ref{CF2}、形成规则\ref{CF1}可证.

			\begin{CF}\label{CF6}
				\hfill\par
				如果$A_1$、$A_2$、$\cdots$、$A_n$是理论$M$的一个构造,$x$、$y$为字母,并且$A_1$、$A_2$、$\cdots$、$A_n$均不包含$y$,则$(y|x)A_1$、$(y|x)A_2$、$\cdots$、$(y|x)A_n$也组成理论$M$的一个构造.
			\end{CF}
			证明:
			\par
			$A_1$只能是字母,因此$(y|x)A_1$也是字母,故命题对$A_1$成立.
			\par
			设命题对$A_1$、$A_2$、$\cdots$、$A_i-1$均成立,考虑$A_i$:
			\par
			若$A_i$是字母,则$(y|x)A_i$也是字母.
			\par
			若$A_i$是$\neg B$、$\lor BC$或$sA_1A_2\cdots A_j$的形式,根据替代规则\ref{CS5},$(y|x)A_i$也是同类的形式.
			\par
			若$A_i$是$\tau_z(A_j)$的形式:
			\par
			(1)如果$z$与$x$、$y$均不相同,根据替代规则\ref{CS4},$(y|x)(\tau_z(A_j))$和$\tau_z(A_j)$相同,故符合构造中语句的条件;
			\par
			(2)如果$z$是$x$,即$A_i$是$\tau_x(A_j)$,根据替代规则\ref{CS3},$\tau_x(A_j)$和$\tau_y((y|x)A_j)$相同.又因为\\$\tau_x(A_j)$不包含$x$,根据补充替代规则\ref{CScor2},$\tau_x(A_j)$和$(y|x)\tau_x(A_j)$相同.因此,$(y|x)A_i$和\\$\tau_y((y|x)A_j)$相同,符合构造中语句的条件;
			\par
			(3)如果$z$是$y$,即$A_i$是$\tau_y(A_j)$,即$\tau A_j$,则$(y|x)A_i$为$(y|x)\tau A_j$,即$\tau(y|x)A_j$,符合构造中语句的条件.
			
			\begin{CF}\label{CF7}
				\hfill\par
				如果$A$是理论$M$的公式(或项),$x$、$y$为字母,则$(y|x)A$也是理论$M$的公式(或项).
			\end{CF}
			证明:
			\par
			令$A_1$、$A_2$、$\cdots$、$A_n$为包含$A$的构造.
			\par
			$A_1$只能是字母,因此$(y|x)A_1$也是字母,为项,故命题对$A_1$成立.
			\par
			设命题对$A_1$、$A_2$、$\cdots$、$A_{i-1}$均成立,考虑$A_i$:
			\par
			若$A_i$是字母,则$(y|x)A_i$也是字母,为项.			
			\par
			若$A_i$是$\neg B$、$\lor BC$或$sA_1A_2\cdots A_j$的形式,根据替代规则\ref{CS5},$(y|x)A_i$也是同类的形式,根据形成规则\ref{CF2}、形成规则\ref{CF3}、形成规则\ref{CF4},为公式.
			\par
			若$A_i$是$\tau_z(A_j)$的形式:
			\par
			(1)如果$z$与$x$、$y$均不相同,根据替代规则\ref{CS4},$(y|x)(\tau_z(A_j))$和$\tau_z(A_j)$相同,根据形成规则\ref{CF3},$(y|x)A_i$为项;
			\par
			(2)如果$z$是$x$,即$A_i$是$\tau_x(A_j)$,根据替代规则\ref{CS3},$\tau_x(A_j)$和$\tau_y((y|x)A_j)$相同.又因为\\$\tau_x(A_j)$不包含$x$,根据补充替代规则\ref{CScor2},$\tau_x(A_j)$和$(y|x)\tau_x(A_j)$相同.因此,$(y|x)A_i$和\\$\tau_y((y|x)A_j)$相同,故$(y|x)A_i$为项;
			\par
			(3)如果$z$是$y$,令$u$是不同于$x$、$y$且不出现在$A_1$、$A_2$、$\cdots$、$A_j$中的字母,根据形成规则\ref{CF6},$(y|x)(u|y)A_1$、$(y|x)(u|y)A_2$、$\cdots$、$(y|x)(u|y)A_j$也组成$M$的一个构造,根据替代规则\ref{CS4}、替代规则\ref{CS3},$\tau_u((y|x)(u|y)A_j)$和$(y|x)\tau_y(A_j)$相同,即和$(y|x)A_i$相同,因此$(y|x)A_i$为项.

			\begin{CF}\label{CF8}
				\hfill\par
				如果$A$是理论$M$的公式(或项),$x$为字母,$T$为理论$M$的项,则$(T|x)A$也是理论$M$的公式(或项).
			\end{CF}
			证明:
			\par
			令$A_1$、$A_2$、$\cdots$、$A_n$为包含$A$的构造,$x_1$、$x_2$、$\cdots$、$x_p$是出现在$T$当中的所有不同字母,$x_1'$、$x_2'$、$\cdots$、$x_p'$是和$x_1$、$x_2$、$\cdots$、$x_p$均不同且不出现在$A_1$、$A_2$、$\cdots$、$A_n$中的字母.
			\par
			根据形成规则\ref{CF7},$(x_1'|x_1)(x_2'|x_2)\cdots(x_p'|x_p)T$也是项,记作$T'$.
			\par
			根据形成规则1,$(T|x)A$与$(x_1|x_1') (x_2|x_2')\cdots(x_p|x_p')(T'|x)A$相同.
			\par
			设命题对$A_1$、$A_2$、$\cdots$、$A_{i-1}$均成立,考虑$A_i$:
			\par
			若$A_i$是字母,则$(T'|x)A_i$或者为字母,或者为$T$,故$(T'|x)A_i$为项;
			\par
			若$A_i$是$\neg B$、$\lor BC$或$sA_1A_2\cdots A_j$的形式,根据替代规则\ref{CS5},$(T'|x)A_i$也是同类的形式,根据形成规则\ref{CF2}、形成规则\ref{CF3}、形成规则\ref{CF4},$(T'|x)A_i$为公式.
			\par
			若$A_i$是$\tau_z(A_j)$的形式:
			\par
			(1)如果$z$与$x$不同,且不出现在$T'$中,根据替代规则\ref{CS4},$(T'|x)\tau_z(A_j)$和$\tau_z((T'|x)A_j)$相同,根据形成规则\ref{CF3},$(T'|x)A_i$为项;
			\par
			(2)如果$z$是$x$,即$A_i$是$\tau_x(A_j)$,不包含$x$,故$A_i$和$(T'|x)A_i$相同,因此,$(T'|x)A_i$为项;
			\par
			(3)如果$z$出现在$T'$中,因此$z$不出现在$A_1$、$A_2$、$\cdots$、$A_n$中.令$u$是不出现在$(T'|x)A_1$、$(T'|x)A_2$、$\cdots$、$(T'|x)A_n$的字母.此时,$\tau_z(A_j)$即$\tau A_j$,$(T'|x)A_i$即$\tau(T'|x)A_j$,也即\\$\tau _u((T'|x)A_j)$,根据形成规则\ref{CF3},$(T'|x)A_i$为项为项.
			\par
			综上,$(T'|x)A_i$为项(或公式),根据形成规则\ref{CF7},$(T|x)A_i$为项(或公式).
			
			\begin{CFcor}\label{CFcor1}
				\hfill\par
				$A$是理论$M$的项或公式,则$A$的每个符号$\Box$都与左侧某一个符号$\tau$连线;且每个符号$\tau$或者未连线,或者与右边一个或多个符号$\Box$连线.除此之外,没有其他连线.
			\end{CFcor}
			证明:
			\par
			令$A_1$、$A_2$、$\cdots$、$A_n$为包含$A$的构造.
			\par
			假设命题对于前$k$个语句成立,对于$A_{k+1}$:
			\par
			如果$A_{k+1}$是字母、$\neg B$、$\lor BC$、$sD_1D_2\cdots D_n$的形式,由于$B$、$C$、$D_1$、$D_2$、$\cdots$、$D_n$\\都是构造中在$A_{k+1}$之前的语句,满足上述性质,且于$A_{k+1}$未添加其他连线,因此命题对\\$A_{k+1}$也成立.
			\par
			如果于$A_{k+1}$是$\tau_x(B)$的形式,除了$B$已有的连线满足上述性质外,$A$开头的$\tau$仅与$B$中的$x$\\替代得到的$\Box$连线,替代得到的$\Box$都与开头的$\tau$连线,除此之外未添加其他连线,故命题对\\$A_{k+1}$也成立.

			\begin{CFcor}\label{CFcor2}
				\hfill\par
				$A$是理论$M$的项或公式,则$A$不能以$\Box$开头.
			\end{CFcor}
			证明:根据定义可证.

			\begin{exer}\label{exer1}
				\hfill\par
				理论$M$没有特别符号,求证:$M$中没有公式,并且所有的的项都是单个字母.
			\end{exer}
			证明:考虑任意构造,用数学归纳法可证明构造中的每个语句都只能是单个字母.

			\begin{exer}\label{exer2}
				\hfill\par
				$A$是理论$M$的项或公式,求证:$A$的每个符号$\Box$都与左侧某一个符号$\tau$连线;且每个符号$\tau$或者未连线,或者与右边一个或多个符号$\Box$连线.除此之外,没有其他连线.
			\end{exer}
			证明:即补充形成规则\ref{CFcor1}.

			\begin{exer}\label{exer3}
				\hfill\par
				$A$是理论$M$的项或公式,求证:$A$的每个特别符号后面只能是$\Box$、$\tau$或字母.
			\end{exer}
			证明:根据构造的定义,特别符号后面的语句只能是项,而$\neg$、$\lor$和特别符号开头的语句是公式,故特别符号后面只能是是$\Box$、$\tau$或字母.

			\begin{exer}\label{exer4}
				\hfill\par
				$A$是理论$M$的项或公式,$B$是语句,求证:$AB$不是项也不是公式.
			\end{exer}
			证明:
			\par
			对$A$的符号数目用数学归纳法.
			\par
			$A$的符号数目为1时,$A$只能是单个字母,因此$AB$以字母开头,不可能是项或公式,故命题成立.
			\par
			假设$A$的符号数目小于$k$时,命题成立,考虑$A$的符号数目为k的情形:
			\par
			如果$A$以$\neg$开头,设故$A$为$\neg C$的形式,如果$AB$为$\neg D$的形式,则公式$D$为$CB$,根据归纳假设,矛盾.
			\par
			如果$A$以$\lor$或特别符号开头,同理可证.
			\par
			如果$A$以$\tau$开头,设$A$为$\tau_x(C)$,其中$C$为公式.假设$AB$为$\tau_y(D)$,其中$D$为公式,则令$u$\\为一个不同于x和y,并且不出现在$C$和$D$中的字母,根据替代规则\ref{CS3},$AB$和$\tau_u((u|y)D)$相同,$A$和$\tau_u((u|x)C)$相同,因此$((u|y)D)$和$((u|x)C)B$相同,根据归纳假设,矛盾.

			\begin{exer}\label{exer5}
				\hfill\par
				$A$是理论$M$的语句,$x$为字母,求证:如果$\tau_x(A)$是$M$的项,则$A$是$M$的公式.
			\end{exer}
			证明:设$\tau_x(A)$和$\tau_y(R)$相同,其中$R$为$M$的项,$y$为字母.由定义可知,$A$和$(y|x)R$相同.根据形成规则\ref{CF7},$A$是$M$的公式.

			\begin{exer}\label{exer6}
				\hfill\par
				$A$、$B$是理论$M$的语句,求证:如果$A$和$\Rightarrow AB$都是$M$的公式,则$B$是$M$的公式.
			\end{exer}
			证明:假设“$\Rightarrow AB$”为“$\Rightarrow CD$”的形式,其中$C$、$D$为公式.根据习题4,$C$和$A$相同,故$B$和$D$相同,得证.
		
		\section{定理(Théorèmes)}
			\begin{sign}
				\textbf{逻辑运算符(opérateurs logiques)}
				\par
				令$A$、$B$为语句,定义下列中序表达式的运算符:
				\par
				(1)“$\text{非}A$”表示“$\neg A$”(第一优先级);
				\par
				(2)“$A\text{或}B$”表示“$\lor AB$”(第四优先级);
				\par
				(3)“$A\Rightarrow B$”表示“$\Rightarrow AB$”(第五优先级).
			\end{sign}
		
			\begin{metadef}
				\textbf{显式公理(axiome explicite)}
				\par
				显式公理,是指理论的“显式公理”一栏列举的一系列公式.
			\end{metadef}

			\begin{metadef}
				\textbf{公理模式(schéma),隐式公理(axiome implicite)}
				\par
				公理模式,是指理论的“公理模式”一栏列举并且满足下列条件的规则:
				\par
				(1)将该规则适用到一个或数个项和/或公式上,可以生成公式;
				\par
				(2)令$T$为项,$x$为字母,$R$为根据该规则生成的公式,则$(T|x)R$也是根据该规则可以生成的公式.
				\par
				隐式公理是指根据公理模式生成的公式.
			\end{metadef}

			\begin{metadef}
				\textbf{公理(axiome)}
				\par
				公理是隐式公理和显式公理的总称.
			\end{metadef}

			\begin{metadef}
				\textbf{常数(constante)}
				\par
				显式公理中的字母,称为常数.
			\end{metadef}
			注:常数表示特定对象,其他字母表示不特定的对象.显式公理对常数的性质做出断言,隐式公理对不特定对象的性质做出断言.

			\begin{metadef}
				\textbf{证明文本(texte démonstratif),证明(démonstration)}
				\par
				理论$M$的证明文本,是指一个辅助构造和一个公式有限序列,并且序列其中每个公式都满足下列条件之一:
				\par
				(1)该公式是理论$M$的显式公理;
				\par
				(2)该公式是将理论$M$的一个公理模式适用到辅助构造中的项及公式,得到的公式;
				\par
				(3)设该公式为$R$,在序列中,存在$R$之前的两个公式$S$和$T$,其中$T$是“$S\Rightarrow R$”.
				\par
				其中,证明文本中的公式序列称为证明.
			\end{metadef}

			\begin{metadef}
				\textbf{定理(théorème)}
				\par
				理论$M$的证明文本的证明中的各公式,均称为定理.
			\end{metadef}

			\begin{metadef}
				\textbf{真公式(relation vraie),假公式(relation fausse)}
				\par
				如果理论$M$的公式$A$是定理,则称$A$为真,如果公式$\neg A$是定理,则称$A$为假.
			\end{metadef}

			\begin{metadef}
				\textbf{矛盾的理论(théorie contradictoire)}
				\par
				如果理论$M$中存在一个公式$A$,既是真又是假,则称理论M有矛盾.
			\end{metadef}

			\begin{Ccor}\label{Ccor1}
				\hfill\par
				理论$M$的公理都是定理.
			\end{Ccor}
			证明:建立仅包含公理$A$的公式序列,根据定理的定义,A是定理.

			\begin{C}\label{C1}
				\textbf{三段论}
				\par
				令$A$、$B$为理论$M$的公式,如果$A$和$A\Rightarrow B$是理论$M$的定理,则$B$是理论$M$的定理.
			\end{C}
			证明:设包含$A$的证明为$R_1$、$R_2$、$\cdots$、$R_n$,设包含$A\Rightarrow B$的证明为$S_1$、$S_2$、$\cdots$、$S_n$,则公式序列,$R_1$、$R_2$、$\cdots$、$R_n$、$S_1$、$S_2$、$\cdots$、$S_n$、$B$是一个证明.故$B$是定理.

			\begin{sign}
				\textbf{理论的替代(substitution dans une théorie)}
				\par
				令$A_1$、$A_2$、$\cdots$、$A_n$为理论$M$所有的显式公理,$T$为理论$M$的项,$x$为字母,则用$(T|x)$\\表示这样的理论:
				\par
				它的特殊符号和公理模式都与$M$相同,而其中的显式公理为$(T|x)A_1$、$(T|x)A_2$、$\cdots$、$(T|x)A_n$.
			\end{sign}

			\begin{C}\label{C2}
				\hfill\par
				令$A$为理论$M$的定理,$T$为理论$M$的项,$x$为字母,则$(T|x)A$是理论$(T|x)M$的定理.
			\end{C}
			证明:
			\par
			设包含$A$的证明为为$R_1$、$R_2$、$\cdots$、$R_n$,考虑公式序列$(T|x)R_1$、$(T|x)R_2$、$\cdots$、\\$(T|x)R_n$.
			\par
			若$R_n$是$M$的显式公理,则$(T|x)R_n$是$(T|x)M$的显式公理.
			\par
			若$R_n$是$M$的隐式公理,则$(T|x)R_n$是$(T|x)M$由同一个公理模式产生的隐式公理.
			\par
			若$R_n$是$M$的其他定理,则包含$A$的证明存在$R_i$和$R_i\Rightarrow R_n$,故$(T|x)R_i$、$(T|x)(Ri\Rightarrow Rn)$都在公式序列$(T|x)R_1$、$(T|x)R_2$、$\cdots$、$(T|x)R_n$中,根据替代规则\ref{CS5},$(T|x)(R_i\Rightarrow R_n)$即\\$(T|x)R_i\Rightarrow (T|x)R_n$.因此$(T|x)R_n$是$(T|x)M$的定理.

			\begin{C}\label{C3}
				\hfill\par
				令$A$为理论$M$的定理,$T$为理论$M$的项,$x$为字母,如果$x$不是理论M的常数,则$(T|x)A$\\是理论$M$的定理.
			\end{C}				
			证明:由于$M$的显式公理不包括字母$x$,故$(T|x)M$和$M$相同.根据证明规则\ref{C2},$(T|x)A$\\是$M$的定理.

			\begin{metadef}
				\textbf{更强的理论(théorie plus forte),等价的理论(théories \\équivalentes)}
				\par
				如果理论$M$的所有特殊符号、显式公理、公理模式分别都是理论$M'$的特殊符号、定理、公理模式,则称理论$M'$比理论$M$强.如果理论$M$比理论$M'$强,并且理论$M'$比理论$M$强,则称$M$和$M'$等价.
			\end{metadef}
			注:在原书中,“强”这个概念包括与自身相等的情况,即一个理论比自身强.
			
			\begin{C}\label{C4}		
				\hfill\par
				如果理论$M'$比理论$M$强,则理论$M$的定理,也是理论$M'$的定理.
			\end{C}		
			证明:
			\par
			设理论$M$中包含定理R的证明为$R_1$、$R_2$、$\cdots$、$R_n$.
			\par
			$R_1$为$M$公理,因此也是$M'$的公理,故命题对$R_1$成立.
			\par
			设命题对$R_1$、$R_2$、$\cdots$、$R_{k-1}$成立,对于$M$的定理$R_k$,如果$R_k$是$M$的公理,则$R_k$是$M'$\\的公理.如果$R_k$不是$M$的公理,则在对$R_1$、$R_2$、$\cdots$、$R_{k-1}$中,存在两个$M$的定理$R_i$和$R_i\Rightarrow R_k$,根据归纳假设,它们也都是$M'$的定理,故$R_k$也是$M'$的定理.

			\begin{C}\label{C5}		
				\hfill\par
				令$A_1$、$A_2$、$\cdots$、$A_n$为理论$M$的显式公理,$a_1$、$a_2$、$\cdots$、$a_h$为理论$M$的常数,$T_1$、$T_2$、$\cdots$、$T_h$为理论$M$的项,如果$(T_1|a_1)(T_2|a_2)\cdots(T_h|a_h)A_i$(其中$i=1, 2, \cdots, n$)都是另一个理论$M'$的定理,而且理论$M$的所有特殊符号和公理模式分别都是理论$M'$的特殊符号和公理模式,则对理论$M$的任何定理$A$,$(T_1|a_1)(T_2|a_2)\cdots(T_h|a_h)A$都是理论$M'$的定理.
			\end{C}		
			证明:根据证明规则\ref{C2},$(T_1|a_1)(T_2|a_2)\cdots(T_h|a_h)A$ 是理论$(T_1|a_1)(T_2|a_2)\cdots(T_h|a_h)M$的定理.由于$M'$比$(T_1|a_1)(T_2|a_2)\cdots(T_h|a_h)M$强,根据证明规则\ref{C4}得证.

			\begin{exer}\label{exer7}
				\hfill\par
				理论$M$的显式公理为$A_1$、$A_2$、$\cdots$、$A_n$,常数为$a_1$、$a_2$、$\cdots$、$a_h$.
				\par
				(1)理论$M'$的常数和公理模式与$M$相同,显式公理为为$A_1$、$A_2$、$\cdots$、$A_{n-1}$,如果$M'$\\和$M$不等价,则称$A_n$独立于其他公理.求证:当且仅当$A_n$不是$M'$的定理时,$A_n$独立.
				\par
				(2)理论$M''$的常数和公理模式与$M$相同,对于$M$的项$T_1$、$T_2$、$\cdots$、$T_h$是$M$,对$i=1$、$2$、$\cdots$、$n-1$,$(T_1|a_1)(T_2|a_2)\cdots(T_n|a_n)A_i$都是$M''$的定理,而$(\text{非}(T_1|a_1)(T_2|a_2)\cdots(T_n|a_n)A_n)$\\是$M''$的定理,求证:要么在$M$中$A_n$独立于其他公理,要么$M''$有矛盾.
			\end{exer}
			证明:
			\par
			(1)$M$强于$M'$.如果$A_n$不是$M'$的定理,则$M'$不比$M$强,因此$M'$和$M$不等价.反过来,若$M'$不比$M$强,则$M$必有一个显式公理不是$M'$的定理,该显式公理只能是$A_n$.
			\par
			(2)假设$A_n$是$M'$的定理,根据证明规则\ref{C5},$(T_1|a_1)(T_2|a_2)\cdots(T_n|a_n)A_n$是$M''$的定理,因此$M''$有矛盾.

		\section{逻辑理论(Théories logiques)}
			\begin{CScor}\label{CScor5}
				\hfill\par
				下列规则均为公理模式:
				\par
				(1)令$A$为公式,则$(A\text{或}A)\Rightarrow A$是公理.
				\par
				(2)令$A$、$B$为公式,则$A\Rightarrow(A\text{或}B) $是公理.
				\par
				(3)令$A$、$B$为公式,则$(A\text{或}B)\Rightarrow(B\text{或}A)$ 是公理.
				\par
				(4)令$A$、$B$、$C$为公式,则$(A\Rightarrow B)\Rightarrow((C\text{或}A)\Rightarrow(C\text{或}B))$是公理.
			\end{CScor}
			证明:根据替代规则\ref{CS5}可证.
			
			\begin{Sch}\label{Sch1}
				\hfill\par
				令$A$为公式,则$(A\text{或}A)\Rightarrow A$是公理.
			\end{Sch}
			注:形式语言的表述是$\lor\neg\lor AAA$.
			
			\begin{Sch}\label{Sch2}
				\hfill\par
				令$A$、$B$为公式,则$A\Rightarrow(A\text{或}B)$ 是公理.
			\end{Sch}
			注:形式语言的表述是$\lor\neg A\lor AB$.
			
			\begin{Sch}\label{Sch3}			
				\textbf{析取交换律}
				\par
				令$A$、$B$为公式,则$(A\text{或}B)\Rightarrow(B\text{或}A)$是公理.
			\end{Sch}			
			注:形式语言的表述是$\lor\neg\lor AB\lor BA$.
			
			\begin{Sch}\label{Sch4}			
				\hfill\par
				令$A$、$B$、$C$为公式,则$(A\Rightarrow B)\Rightarrow((C\text{或}A)\Rightarrow(C\text{或}B))$是公理.
			\end{Sch}
			注:形式语言的表述是$\lor\neg\lor\neg AB\lor\neg\lor CA \lor CB$.
			
			\begin{metadef}
				\textbf{逻辑理论(théorie logique)}
				\par
				包含公理模式\ref{Sch1}、公理模式\ref{Sch2}、公理模式\ref{Sch3}、公理模式\ref{Sch4}的理论,称为逻辑理论.
			\end{metadef}
			
			\begin{Ccor}\label{Ccor2}
				\hfill\par
				如果逻辑理论$M$有矛盾,则$M$中的任何一个公式均为其定理.
			\end{Ccor}
			证明:设$A$和$\text{非}A$均为$M$的定理,对任意公式$B$,根据公理模式\ref{Sch2},$\text{非}A\Rightarrow \text{非}A\text{或}B$,根据证明规则\ref{C1},$\text{非}A\text{或}B$,即$A\Rightarrow B$,又因为$A$是定理,根据根据证明规则\ref{C1},$B$是定理.
			
			\begin{C}\label{C6}			
				\textbf{蕴涵的传递性}
				\par
				令$A$、$B$、$C$为逻辑理论$M$的公式,$A\Rightarrow B$和$B\Rightarrow C$是$M$的定理,则$A\Rightarrow C$也是$M$的定理.
			\end{C}
			证明:
			\par
			根据公理模式\ref{Sch4},$(B\Rightarrow C)\Rightarrow ((\text{非}A\text{或}B)\Rightarrow (\text{非}A\text{或}C))$,即$(B\Rightarrow C)\Rightarrow ((A\Rightarrow B)\Rightarrow (A\Rightarrow C))$.
			\par
			由于$B\Rightarrow C$,根据证明规则\ref{C1}$(A\Rightarrow B)\Rightarrow (A\Rightarrow C)$,又因为$A\Rightarrow B$,根据证明规则\ref{C1},$A\Rightarrow C$.
			
			\begin{C}\label{C7}	
				\hfill\par
				令$A$、$B$为逻辑理论$M$的公式,则$B\Rightarrow (A\text{或}B)$是$M$的定理.
			\end{C}
			证明:根据公理模式\ref{Sch2},$B\Rightarrow (B\text{或}A)$是定理,根据公理模式\ref{Sch3},$(B\text{或}A)\Rightarrow (A\text{或}B)$,根据证明规则\ref{C6},$B\Rightarrow (A\text{或}B)$.

			\begin{C}\label{C8}				
				\textbf{排中律之一}
				\par
				令$A$为逻辑理论$M$的公式,则$A\Rightarrow A$是$M$的定理.
			\end{C}
			证明:根据公理模式\ref{Sch1},$(A\text{或}A)\Rightarrow A$,根据\ref{Sch2},$A\Rightarrow(A\text{或}A)$,根据证明规则6,$A\Rightarrow A$.
			
			\begin{C}\label{C9}				
				\hfill\par
				令$A$为逻辑理论$M$的公式,$B$为$M$的定理,则$A\Rightarrow B$是$M$的定理.
			\end{C}
			证明:根据证明规则\ref{C7},$B\Rightarrow (\text{非}A\text{或}B)$,即$B\Rightarrow (A\Rightarrow B)$.
			\par
			又因为$B$是定理,根据证明规则\ref{C1},$A\Rightarrow B$是定理.

			\begin{C}\label{C10}				
				\textbf{排中律之二}
				\par
				令$A$为逻辑理论$M$的公式,则“$A\text{或}(\text{非}A)$”是$M$的定理.
			\end{C}
			证明:根据证明规则\ref{C8},$\text{非}A\text{或}A$,根据公理模式\ref{Sch2},$\text{非}A\text{或}A\Rightarrow A\text{或}(\text{非}A)$,根据证明规则\ref{C1},$A\text{或}(\text{非}A)$.
			
			\begin{C}\label{C11}				
				\hfill\par
				令$A$为逻辑理论$M$的公式,则$A\Rightarrow(\text{非}(\text{非}A))$是$M$的定理.
			\end{C}
			证明:根据证明规则\ref{C8},$(\text{非}A)\text{或}(\text{非}(\text{非}A))$,即$A\Rightarrow (\text{非}(\text{非}A))$.

			\begin{C}\label{C12}				
				\textbf{逆否命题和原命题等价之一}
				\par
				令$A$、$B$为逻辑理论$M$的公式,则$(A\Rightarrow B)\Rightarrow (\text{非}B\Rightarrow \text{非}A)$是$M$的定理.
			\end{C}
			证明:
			\par
			根据证明规则\ref{C11},$B\Rightarrow \text{非}(\text{非}B)$;
			\par
			根据公理模式\ref{Sch4},$(B\Rightarrow \text{非}(\text{非}B))\Rightarrow ((\text{非}A\text{或}B)\Rightarrow (\text{非}A\text{或}\text{非}(\text{非}B))$;
			\par
			根据证明规则\ref{C6},$(\text{非}A\text{或}B)\Rightarrow (\text{非}A\text{或}\text{非}(\text{非}B))$;
			\par
			根据公理模式\ref{Sch3},$(\text{非}A\text{或}\text{非}(\text{非}B))\Rightarrow (\text{非}(\text{非}B)\text{或}\text{非}A)$;
			\par
			根据证明规则\ref{C1},$(\text{非}A\text{或}B)\Rightarrow (\text{非}(\text{非}B)\text{或}\text{非}A)$,即$(A\Rightarrow B)\Rightarrow (\text{非}B\Rightarrow \text{非}A)$.

			\begin{C}\label{C13}	
				\hfill\par
				令$A$、$B$、$C$为逻辑理论$M$的公式,$A\Rightarrow B$是$M$的定理,则$(B\Rightarrow C)\Rightarrow (A\Rightarrow C)$是$M$的定理.
			\end{C}
			证明:
			\par
			根据证明规则\ref{C12},$(A\Rightarrow B)\Rightarrow ((\text{非}B)\Rightarrow (\text{非}A))$;
			\par
			根据证明规则\ref{C1},$(\text{非}B)\Rightarrow (\text{非}A)$;
			\par
			根据公理模式\ref{Sch4},$((\text{非}B)\Rightarrow (\text{非}A))\Rightarrow ((C\text{或}\text{非}B)\Rightarrow (C\text{或}\text{非}A))$;
			\par
			根据证明规则\ref{C1},$(C\text{或}\text{非}B)\Rightarrow (C\text{或}\text{非}A)$;
			\par
			根据公理模式\ref{Sch3},$(B\Rightarrow C)\Rightarrow (C\text{或}\text{非}B)$,
			$(C\text{或}\text{非}A)\Rightarrow (A\Rightarrow C)$;
			\par
			根据证明规则\ref{C6},$(B\Rightarrow C)\Rightarrow (A\Rightarrow C)$.
			
			\begin{C}\label{C14}
				\textbf{辅助假设,演绎定理}
				\par	
				令$A$为逻辑理论$M$的公式,$M'$为$M$加上公理$A$组成的理论,如果$B$是$M'$的定理,则$A\Rightarrow B$是$M$的定理.
			\end{C}
			证明:
			令公式$B_1$、$B_2$、$\cdots$、$B_n$为包含B的证明.
			\par
			$B_1$为$M$的公理或者$A$本身,如果$B_1$为$M$的公理,根据补充证明规则\ref{Ccor1}、证明规则\ref{C9},$A\Rightarrow B_1$;如果$B_1$为$A$本身,根据证明规则\ref{C8},$A\Rightarrow B_1$.故命题对$B_1$成立.
			设对任意$k<i-1$,有$A\Rightarrow B_k$,以下证明$A\Rightarrow B_i$:
			\par
			若$B_i$是$M'$的公理,则$B_i$为$M$的公理或者$A$本身,如果$B_i$为$M$的公理,根据补充证明规则\ref{Ccor1}、证明规则\ref{C9},$A\Rightarrow B_i$;如果$B_i$为$A$本身,根据证明规则\ref{C8},$A\Rightarrow B_i$.
			\par
			若$B_i$不是$M'$的公理,则之前存在两个公式$B_j$和$B_j\Rightarrow B_i$,它们均为$M$的定理;
			\par
			根据归纳假设,$A\Rightarrow B_j$和$A\Rightarrow (B_j\Rightarrow B_i)$都是$M$的定理;
			\par
			根据证明规则\ref{C13},$(B_j\Rightarrow B_i)\Rightarrow (A\Rightarrow B_i)$是$M$的定理;
			\par
			根据证明规则\ref{C6},$A\Rightarrow (A\Rightarrow B_i)$,即$\text{非}A\text{或}(A\Rightarrow B_i)$是$M$的定理.
			\par
			根据公理模式\ref{Sch3},$(A\Rightarrow B_i)\text{或}\text{非}A$是$M$的定理.
			\par
			根据公理模式\ref{Sch2},$\text{非}A\Rightarrow (A\Rightarrow B_i)$是$M$的定理.
			\par
			根据公理模式\ref{Sch4},$((A\Rightarrow B_i)\text{或}\text{非}A)\Rightarrow((A\Rightarrow B_i)\text{或}(A\Rightarrow B_i))$是$M$的定理;
			\par
			根据证明规则\ref{C1},$(A\Rightarrow B_i)\text{或}(A\Rightarrow B_i)$是$M$的定理,根据公理模式\ref{Sch1}、证明规则\ref{C1},$A\Rightarrow B_i$是$M$的定理.
			\par
			综上,得证.
			
			\begin{C}\label{C15}
				\textbf{反证法}
				\par
				令$A$为逻辑理论$M$的公式,$M'$为$M$加上公理“$\text{非}A$”组成的理论,如果$M'$有矛盾,则$A$\\是$M$的定理.
			\end{C}
			证明:
			\par
			由于$M'$有矛盾,根据补充证明规则\ref{Ccor1},$A$是$M'$的定理.
			\par
			根据证明规则\ref{C14},$\text{非}A\Rightarrow A$是$M$的定理.
			\par
			根据公理模式\ref{Sch4},$(A\text{或}\text{非}A)\Rightarrow (A\text{或}A)$是$M$的定理.
			\par
			根据证明规则\ref{C10},“$A\text{或}A$”是$M$的定理.
			\par
			根据公理模式\ref{Sch1},A是$M$的定理.
			
			\begin{C}\label{C16}
				\hfill\par
				令$A$为逻辑理论$M$的公式,则$(\text{非}(\text{非}A))\Rightarrow A$是$M$的定理.
			\end{C}
			证明:假设“$\text{非}(\text{非}A)$”为真,而$A$为假,即“$\text{非}A$”为真.“$\text{非}A$”和“$\text{非}(\text{非}A)$”同时为真,矛盾,得证.
			\par
			注:
			\par
			上述证明系运用证明规则\ref{C14}和证明规则\ref{C15}做出的简化表述.完整的表述是:
			\par
			将“$\text{非}(\text{非}A$)”作为公理加入$M$得到理论$M'$,再加入公理“$\text{非}A$”得到理论$M''$,此时理论$M'$'中,“$\text{非}(\text{非}A)$”同时为真也为假,根据证明规则\ref{C15},$A$是理论$M'$的定理,根据证明规则则\ref{C14},$(\text{非}(\text{非}A))\Rightarrow A$是理论$M$的定理.
			\par
			以下证明中,如果运用证明规则\ref{C14}或证明规则\ref{C15},也同样使用简化表述.

			\begin{C}\label{C17}
				\textbf{逆否命题\text{与}原命题等价之二}
				\par
				令$A$、$B$为逻辑理论$M$的公式,则$(\text{非}B\Rightarrow \text{非}A)\Rightarrow (A\Rightarrow B)$是$M$的定理.
			\end{C}
			证明:假设$\text{非}B\Rightarrow \text{非}A$、$A$为真,$B$为假,即$\text{非}B$为真,则$\text{非}A$为真,矛盾,得证.

			\begin{C}\label{C18}
				\textbf{分情况讨论}
				\par
				令$A$、$B$、$C$为逻辑理论$M$的公式,如果$A\text{或}B$、$A\Rightarrow C$、$B\Rightarrow C$都是$M$的定理,则$C$是\\$M$的定理.
			\end{C}
			证明:根据公理模式\ref{Sch4},$(A\text{或}B)\Rightarrow (A\text{或}C)$、$(C\text{或}A)\Rightarrow (C\text{或}C)$,再根据公理模式\ref{Sch1}、公理模式\ref{Sch3}得证.

			\begin{C}\label{C19}
				\textbf{辅助常数}
				\par
				令$x$为字母,$A$、$B$为逻辑理论$M$的公式,其中,$x$不是论$M$的常数,$B$也不包含$x$,并且存在$M$的项$T$,使$(T|x)A$为定理,那么,令$M'$为$M$加上公理$A$组成的理论,如果$B$是$M'$的定理,则$B$是$M$的定理.
			\end{C}
			证明:根据证明规则\ref{C14},$A\Rightarrow B$是$M$的定理,根据证明规则\ref{C3},$(T|x)(A\Rightarrow B)$是$M$的定理.由于$B$不包含$x$,根据替代规则\ref{CS3},$(T|x)(A\Rightarrow B)$和$((T|x)A)\Rightarrow B$相同,又因为$(T|x)A$是\\$M$的定理,因此$B$是$M$的定理.
			\par
			注:只要$(\exists x)R$(即$(\tau_x(R)|x)R$)是定理,就可以运用添加辅助常数的方法.

			\begin{sign}
				\textbf{合取(conjonction)}
				\par
				令$A$、$B$为语句,$x$为字母,则用“$A\text{与}B$”表示“$\text{非}((\text{非}A)\text{或}(\text{非}B))$”(第四优先级).
			\end{sign}

			\begin{CS}\label{CS6}
				\hfill\par
				令$A$、$B$、$T$为语句,$x$为字母,则$(T|x)(A\text{与}B)$和$((T|x)A\text{与}(T|x)B)$相同.
			\end{CS}
			证明:$A\text{与}B$即$\text{非}((\text{非}A)\text{或}(\text{非}B))$,根据替代规则\ref{CS5}可证.

			\begin{CF}\label{CF9}
				\hfill\par
				如果$A$、$B$是理论M的公式,则$A\text{与}B$也是理论$M$的公式.
			\end{CF}
			证明:根据形成规则\ref{CF1}、形成规则\ref{CF2}可证.

			\begin{C}\label{C20}
				\hfill\par
				令$A$、$B$为逻辑理论$M$的定理,则“$A\text{与}B$”也是$M$的定理.
			\end{C}
			证明:假设“$A\text{与}B$”为假,则“$\text{非}\text{非}((\text{非}A)\text{或}(\text{非}B))$”,根据证明规则\ref{C16},\\“$\text{非}A\text{或}\text{非}B$”,即“$A\Rightarrow \text{非}B$”,又因为$A$为真,故“$\text{非}B$”为真,和$B$为真矛盾,得证.

			\begin{C}\label{C21}			
				\hfill\par
				令$A$、$B$为逻辑理论$M$的公式,则$(A\text{与}B)\Rightarrow A$和$(A\text{与}B)\Rightarrow B$是$M$的定理.
			\end{C}
			证明:
			\par
			根据公理模式\ref{Sch2}、证明规则\ref{C7},$\text{非}A\Rightarrow (\text{非}A\text{或}\text{非}B)$,$\text{非}B\Rightarrow (\text{非}A\text{或}\text{非}B)$.
			\par
			根据证明规则\ref{C11},$(\text{非}A\text{或}\text{非}B)\Rightarrow \text{非}(A\text{与}B)$.
			\par
			根据证明规则\ref{C6},$\text{非}A\Rightarrow \text{非}(A\text{与}B)$,$\text{非}B\Rightarrow \text{非}(A\text{与}B)$.
			\par
			根据证明规则\ref{C17},$(A\text{与}B)\Rightarrow A$,$(A\text{与}B)\Rightarrow B$.
			
			\begin{sign}
				\textbf{等价(équivalence)}
				\par
				令$A$、$B$为语句,$x$为字母,则用“$A\Leftrightarrow B$”表示“$(A\Rightarrow B)\text{与}(B\Rightarrow A)$ ”(第五优先级).
			\end{sign}
			
			\begin{CS}\label{CS7}
				\hfill\par
				令$A$、$B$、$T$为语句,$x$为字母,则$(T|x)(A\Leftrightarrow B)$和$((T|x)A\Leftrightarrow (T|x)B)$相同.
			\end{CS}
			证明:$A\Leftrightarrow B$即$(A\Rightarrow B)\text{与}(B\Rightarrow A)$,根据替代规则\ref{CS5}、替代规则\ref{CS6}可证.

			\begin{CF}\label{CF10}
				\hfill\par
				如果$A$、$B$是理论$M$的公式,则$A\Leftrightarrow B$也是理论$M$的公式.
			\end{CF}
			证明:根据形成规则\ref{CF5}、形成规则\ref{CF9}可证.

			\begin{Ccor}\label{Ccor3}
				\textbf{等价的反身性}
				\par
				令$A$为逻辑理论$M$的公式,则$A\Leftrightarrow A$是$M$的定理.
			\end{Ccor}			
			证明:根据证明规则\ref{C8}可证.

			\begin{C}\label{C22}
				\textbf{等价的对称性和传递性}
				\par
				(1)令$A$、$B$为逻辑理论$M$的公式,如果$A\Leftrightarrow B$是$M$的定理,则$B\Leftrightarrow A$是$M$的定理.
				\par
				(2)令$A$、$B$、$C$为逻辑理论M的公式,如果$A\Leftrightarrow B$和$B\Leftrightarrow C$是$M$的定理,则$A\Leftrightarrow C$是$M$的定理.
			\end{C}
			证明:
			\par
			(1)根据证明规则\ref{C21},$A\Rightarrow B$、$B\Rightarrow A$,根据证明规则\ref{C20},$(B\Rightarrow A)\text{与}(A\Rightarrow B)$,即$B\Leftrightarrow A$.
			\par
			(2)根据证明规则\ref{C21}、证明规则\ref{C6}可证.
			
			\begin{Ccor}\label{Ccor4}
				\hfill\par
				令$A$为逻辑理论$M$的定理,则$B\Leftrightarrow (A\text{与}B)$和$B\Leftrightarrow (B\text{与}A)$是$M$的定理.
			\end{Ccor}
			证明:假设B为真,根据证明规则\ref{C20},$A\text{与}B$、$B\text{与}A$均为真,根据证明规则\ref{C9},$B\Rightarrow (A\text{与}B)$、$B\Rightarrow (B\text{与}A)$.再根据证明规则\ref{C21}可证.
		
			\begin{C}\label{C23}
				\hfill\par
				令$A$、$B$、$C$为逻辑理论$M$的公式,如果$A\Leftrightarrow B$是$M$的定理,则下列公式都是$M$的定理:
				\par
				(1)$(\text{非}A)\Leftrightarrow (\text{非}B)$;
				\par
				(2)$(A\Rightarrow C)\Leftrightarrow (B\Rightarrow C)$;
				\par
				(3)$(C\Rightarrow A)\Leftrightarrow (C\Rightarrow B)$;
				\par
				(4)$(A\text{与}C)\Leftrightarrow (B\text{与}C)$;
				\par
				(5)$(A\text{或}C)\Leftrightarrow (B\text{或}C)$.
			\end{C}
			证明:
			\par
			(1)	因为$A\Leftrightarrow B$,根据证明规则\ref{C21}、根据证明规则\ref{C1},$A\Rightarrow B$、$B\Rightarrow A$.根据证明规则\ref{C12},$(A\Rightarrow B)\Rightarrow (\text{非}B\Rightarrow \text{非}A)$、$(B\Rightarrow A)\Rightarrow (\text{非}A\Rightarrow \text{非}B)$.根据证明规则\ref{C20}可证.
			\par
			(2)	因为$A\Leftrightarrow B$,根据证明规则\ref{C21}、根据证明规则\ref{C1},$A\Rightarrow B$、$B\Rightarrow A$.若$A\Rightarrow C$,根据证明规则\ref{C6},$B\Rightarrow C$.反过来,若$B\Rightarrow C$,根据证明规则\ref{C6},$A\Rightarrow C$.根据证明规则\ref{C20}可证.
			\par			
			(3)类似证明规则\ref{C23}(2)可证.
			\par
			(4)	因为$A\Leftrightarrow B$,根据证明规则\ref{C21}、根据证明规则\ref{C1},$A\Rightarrow B$、$B\Rightarrow A$.根据证明规则\ref{C21},$(A\text{与}C)\Rightarrow C$、$(A\text{与}C)\Rightarrow A$.根据证明规则\ref{C6},$(A\text{与}C)\Rightarrow B$.根据证明规则\ref{C20},$(A\text{与}C)\Rightarrow (B\text{与}C)$.同理可证$(B\text{与}A)\Rightarrow (A\text{与}C)$.根据证明规则\ref{C20}可证.
			\par
			(5)	因为$A\Leftrightarrow B$,根据证明规则\ref{C21}、根据证明规则\ref{C1},$A\Rightarrow B$、$B\Rightarrow A$.根据公理模式\ref{Sch4},$(C\text{或}A)\Rightarrow (C\text{或}B)$、$(C\text{或}B)\Rightarrow (C\text{或}A)$.根据公理模式\ref{Sch3}、证明规则\ref{C6},$(A\text{或}C)\Rightarrow (B\text{或}C)$、$(B\text{或}C)\Rightarrow (A\text{或}C)$.根据证明规则\ref{C20}可证.

			\begin{C}\label{C24}
				\hfill\par
				令$A$、$B$、$C$为逻辑理论$M$的公式,则下列公式都是$M$的定理:
				\par
				(1)$(\text{非}(\text{非}A))\Leftrightarrow A$;
				\par
				(2)$(A\Rightarrow B)\Leftrightarrow ((\text{非}B)\Rightarrow (\text{非}A))$;
				\par
				(3)$(A\text{与}A)\Leftrightarrow A$;
				\par
				(4)$(A\text{与}B)\Leftrightarrow (B\text{与}A)$;
				\par
				(5)$(A\text{与}(B\text{与}C))\Leftrightarrow ((A\text{与}B)\text{与}C)$;
				\par
				(6)$(A\text{或}B)\Leftrightarrow (\text{非}((\text{非}A)\text{与}(\text{非}B)))$;
				\par
				(7)$(A\text{或}A)\Leftrightarrow A$;
				\par
				(8)$(A\text{或}B)\Leftrightarrow (B\text{或}A)$;
				\par
				(9)$(A\text{或}(B\text{或}C))\Leftrightarrow ((A\text{或}B)\text{或}C)$;
				\par
				(10)$(A\text{与}(B\text{或}C))\Leftrightarrow ((A\text{与}B)\text{或}(A\text{与}C))$;
				\par
				(11)$(A\text{或}(B\text{与}C))\Leftrightarrow ((A\text{或}B)\text{与}(A\text{或}C))$;
				\par
				(12)$(A\text{与}(\text{非}B))\Leftrightarrow (\text{非}(A\Rightarrow B))$;
				\par
				(13)$(A\text{或}B)\Leftrightarrow ((\text{非}A)\Rightarrow B)$.
			\end{C}
			证明:
			\par
			(1)根据证明规则\ref{C11}、证明规则\ref{C16}、证明规则\ref{C20}可证.
			\par
			(2)根据证明规则\ref{C12}、证明规则\ref{C17}、证明规则\ref{C20}可证.
			\par
			(3)根据证明规则\ref{C20}、证明规则\ref{C21}可证.
			\par
			(4)根据公理模式\ref{Sch3}、、证明规则\ref{C20},$\text{非}A\text{或}\text{非}B\Leftrightarrow \text{非}B\text{或}\text{非}A$,根据证明规则\ref{C23}(1)可证.
			\par
			(5)若$A\text{与}(B\text{与}C)$,根据证明规则\ref{C21},$A$、$B$、$C$均为真,根据证明规则\ref{C20},$(A\text{与}B)\\\text{与}C$.反之亦然.根据证明规则\ref{C20}可证.
			\par
			(6)根据证明规则\ref{C24}(1),$\text{非}\text{非}A\Leftrightarrow A$,$\text{非}\text{非}B\Leftrightarrow B$.根据证明规则\ref{C23}(5),\\“$\text{非}\text{非}A\text{或}\text{非}\text{非}B\Leftrightarrow A\text{或}B$”.根据证明规则\ref{C23},$\text{非}\text{非}(\text{非}\text{非}A\text{或}\text{非}\text{非}B)\Leftrightarrow A\text{或}B$,即$(A\text{或}B)\\\Leftrightarrow (\text{非}((\text{非}A)\text{与}(\text{非}B)))$.
			\par
			(7)根据公理模式\ref{Sch1}、公理模式\ref{Sch2}、证明规则\ref{C20}可证.
			\par
			(8)	根据公理模式\ref{Sch3}、证明规则\ref{C20}可证.
			\par
			(9)根据证明规则\ref{C24}(1)、证明规则\ref{C24}(2)、证明规则\ref{C23}(5),$(A\text{或}B)\text{或}C\Leftrightarrow \text{非}C\Rightarrow (\text{非}A\Rightarrow B)$,$A\text{或}(B\text{或}C)\Leftrightarrow \text{非}A\Rightarrow (\text{非}C\Rightarrow B)$.
			\par
			若$\text{非}C\Rightarrow (\text{非}A\Rightarrow B)$、$\text{非}A$、$\text{非}C$为真,根据证明规则\ref{C6},$B$为真,即$(\text{非}C\Rightarrow (\text{非}A\Rightarrow B))\Rightarrow (\text{非}A\Rightarrow (\text{非}C\Rightarrow B))$.
			\par
			反过来,若$\text{非}A\Rightarrow (\text{非}C\Rightarrow B)$、$\text{非}C$、$\text{非}A$为真,根据证明规则\ref{C6},$B$为真,即$(\text{非}A\Rightarrow (\text{非}C\Rightarrow B))\Rightarrow (\text{非}C\Rightarrow (\text{非}A\Rightarrow B))$.
			\par
			综上,根据证明规则\ref{C20}得证.
			\par
			(10)假设$A\text{与}(B\text{或}C)$,根据证明规则\ref{C21},$A$、$B\text{或}C$为真,即$\text{非}B\Rightarrow C$.假设$A\Rightarrow \text{非}B$,则$\text{非}B$为真,根据证明规则\ref{C6},C为真,根据证明规则\ref{C20},A\text{与}C为真,即$A\text{与}(B\text{或}C)\\\Rightarrow ((A\Rightarrow \text{非}B)\Rightarrow (A\text{与}C))$,即$A\text{与}(B\text{或}C)\Rightarrow ((A\text{与}B)\text{或} (A\text{与}C))$.
			\par
			反过来,若$(A\text{与}B)\text{或}(A\text{与}C)$,即$(\text{非}A\text{或}\text{非}B)\Rightarrow (A\text{与}C)$,根据证明规则\ref{C21},\\$(\text{非}A\text{或}\text{非}B)\Rightarrow A$、$(\text{非}A\text{或}\text{非}B)\Rightarrow C$.假设$\text{非}A$为真,根据公理模式\ref{Sch1},$(\text{非}A\text{或}\text{非}B)$,所以A为真,矛盾.故A为真.由于$(\text{非}A\text{或}\text{非}B)\Rightarrow C$,假设$\text{非}B$,则C为真,即$\text{非}B\Rightarrow C$为真,根据证明规则\ref{C23}(5)、证明规则\ref{C24}(1),$B\text{或}C$.由于A、B\text{或}C为真,根据证明规则\ref{C20},$A\text{与}(B\text{或}C)$.
			\par
			综上,根据证明规则\ref{C20}得证.
			\par
			(11)根据证明规则\ref{C21},$(B\text{与}C)\Rightarrow B$,根据公理模式\ref{Sch4},$A\text{或}(B\text{与}C)\Rightarrow A\text{或}B$,同理$A\text{或}(B\text{与}C)\Rightarrow A\text{或}C$,根据证明规则\ref{C20},$A\text{或}(B\text{与}C)\Rightarrow (A\text{或}B)\text{与}(A\text{或}C)$.
			\par
			反过来,假设$(A\text{或}B)\text{与}(A\text{或}C)$,根据证明规则\ref{C21},$A\text{或}B$、$A\text{或}C$,即$\text{非}A\Rightarrow B$、$\text{非}A\\\Rightarrow C$,根据证明规则\ref{C20},$\text{非}A\Rightarrow (B\text{与}C)$,即$A\text{或}(B\text{与}C)$.因此,$(A\text{或}B)\text{与}(A\text{或}C)\Rightarrow A\text{或}\\(B\text{与}C)$,得证.
			\par
			(12)根据证明规则\ref{C24}(1)、证明规则\ref{C23}(2)可证.
			\par
			(13)根据证明规则\ref{C24}(1)、证明规则\ref{C23}(5)可证.

			\begin{C}\label{C25}
				\hfill\par
				(1)如果$A$为逻辑理论$M$的定理,$B$为$M$的公式,则$(A\text{与}B)\Leftrightarrow B$是$M$的定理.
				\par
				(2)	如果$(\text{非}A$)为逻辑理论$M$的定理,$B$为$M$的公式,则$(A\text{或}B)\Leftrightarrow B$是$M$的定理.
			\end{C}
			证明:
			\par
			(1)根据证明规则\ref{C21},$(A\text{与}B)\Rightarrow B$.根据补充证明规则\ref{Ccor4},$B\Rightarrow (A\text{与}B)$,故$(A\text{与}B)\\\Leftrightarrow B$是逻辑理论$M$的定理.
			\par
			(2)	根据证明规则\ref{C25}(1),$\text{非}A\text{与}\text{非}B\Leftrightarrow \text{非}B$,根据证明规则\ref{C23}(1),$\text{非}(\text{非}A\text{与}\text{非}B)\\\Leftrightarrow \text{非}(\text{非}B)$,根据证明规则\ref{C24}(1)、证明规则\ref{C23}(1),$(A\text{或}B)\Leftrightarrow B$.
			
			\begin{Ccor}\label{Ccor5}
				\hfill\par
				令$A$、$B$、$C$、$D$为逻辑理论$M$的公式,则以下公式都是$M$的定理:
				\par
				(1)$((A\Rightarrow B)\text{与}(C\Rightarrow D))\Rightarrow ((A\text{或}C)\Rightarrow (B\text{或}D))$;
				\par
				(2)$((A\Leftrightarrow B)\text{与}(C\Leftrightarrow D))\Rightarrow ((A\text{或}C)\Leftrightarrow (B\text{或}D))$;
				\par
				(3)$((A\Rightarrow B)\text{与}(C\Rightarrow D))\Rightarrow ((A\text{与}C)\Rightarrow (B\text{与}D))$;
				\par
				(4)$((A\Leftrightarrow B)\text{与}(C\Leftrightarrow D))\Rightarrow ((A\text{与}C)\Leftrightarrow (B\text{与}D))$;
				\par
				(5)$(A\Rightarrow B)\Rightarrow ((A\text{或}C)\Rightarrow (B\text{或}C))$;
				\par
				(6)$(A\Leftrightarrow B)\Rightarrow ((A\text{或}C)\Leftrightarrow (B\text{或}C))$;
				\par
				(7)$(A\Rightarrow B)\Rightarrow ((A\text{与}C)\Rightarrow (B\text{与}C))$;
				\par
				(8)$(A\Leftrightarrow B)\Rightarrow ((A\text{与}C)\Leftrightarrow (B\text{与}C))$;
				\par
				(9)$A\Leftrightarrow (A\text{或}A)$;
				\par
				(10)$(A\Rightarrow C)\Rightarrow ((B\Rightarrow C)\Rightarrow ((A\text{或}B)\Rightarrow C))$;
				\par
				(11)$(A\text{或}B)\Leftrightarrow (B\text{或}A)$;
				\par
				(12)$(A\Rightarrow B)\Rightarrow ((A\Rightarrow C)\Rightarrow (A\Rightarrow (B\text{与}C)))$;
				\par
				(13)$(A\text{与}\text{非}A)\Rightarrow B$;
				\par
				(14)$B\text{或}(A\text{与}\text{非}A)\Leftrightarrow B$;
				\par
				(15)$B\text{或}(A\text{与}\text{非}A\text{与}C)\Leftrightarrow B$;
				\par
				(16)$(A\text{与}B\Rightarrow C)\Leftrightarrow (A\text{与}\text{非}C\Rightarrow \text{非}B)$.
			\end{Ccor}
			证明:
			\par
			(1)$假设((A\Rightarrow B)\text{与}(C\Rightarrow D))$,根据证明规则\ref{C21},$A\Rightarrow B$、$C\Rightarrow D$,根据公理模式\ref{Sch4}、公理模式\ref{Sch2}、公理模式\ref{Sch3},$(A\text{或}C)\Rightarrow (C\text{或}A)$、$(C\text{或}A)\Rightarrow (C\text{或}B)$、$(C\text{或}B)\Rightarrow (B\text{或}C)$、$(B\text{或}C)\Rightarrow (B\text{或}D)$,根据证明规则\ref{C6}可证.
			\par
			(2)根据补充证明规则\ref{C5}(1)、证明规则\ref{C20}可证.
			\par
			(3)假设$(A\Rightarrow B)\text{与}(C\Rightarrow D)$,根据证明规则\ref{C21},$A\Rightarrow B$、$C\Rightarrow D$,根据证明规则\ref{C12},$\text{非}B\Rightarrow \text{非}A$,$\text{非}D\Rightarrow \text{非}C$,根据公理模式\ref{Sch4}、公理模式\ref{Sch2}、公理模式\ref{Sch3}、证明规则\ref{C6},$(\text{非}B\text{或}\text{非}D)\Rightarrow (\text{非}A\text{或}\text{非}C)$,根据证明规则\ref{C12},$(A\text{与}C)\Rightarrow (B\text{与}D)$.
			\par
			(4)根据补充证明规则\ref{C5}(3)、证明规则\ref{C20}可证.
			\par
			(5)根据证明规则\ref{C8},$C\Rightarrow C$,根据证明规则\ref{C20}、证明规则\ref{C21},$(A\Rightarrow B)\text{与}(C\Rightarrow C)\Leftrightarrow (A\Rightarrow B)$,根据补充证明规则\ref{Ccor5}(1)可证.
			\par
			(6)根据证明规则\ref{C8}、证明规则\ref{C20},$C\Leftrightarrow C$,根据证明规则\ref{C20}、证明规则\ref{C21},$(A\Leftrightarrow B)\text{与}(C\Leftrightarrow C)\Leftrightarrow (A\Leftrightarrow B)$,根据补充证明规则\ref{Ccor5}(2)可证.
			\par
			(7)根据证明规则\ref{C8},$C\Rightarrow C$,根据证明规则\ref{C20}、证明规则\ref{C21},$(A\Rightarrow B)\text{与}(C\Rightarrow C)\Leftrightarrow (A\Rightarrow B)$,根据补充证明规则\ref{Ccor5}(3)可证.
			\par
			(8)根据证明规则\ref{C8}、证明规则\ref{C20},$C\Leftrightarrow C$,根据证明规则\ref{C20}、证明规则ref{C21},$(A\Leftrightarrow B)\text{与}(C\Leftrightarrow C)\Leftrightarrow (A\Leftrightarrow B)$,根据补充证明规则\ref{Ccor5}(4)可证.
			\par
			(9)根据公理模式\ref{Sch1}、公理模式\ref{Sch2}、证明规则\ref{C20}可证.
			\par
			(10)根据证明规则\ref{C14}、证明规则\ref{C18}可证.
			\par
			(11)根据公理模式\ref{Sch3}、证明规则\ref{C20}可证.
			\par
			(12)假设$A\Rightarrow B$、$A\Rightarrow C$、$A$为真,则$B$、$C$均为真,根据证明规则\ref{C20},$B\text{与}C$,得证.
			\par
			(13)根据证明规则\ref{C11}、证明规则\ref{C16},$((A\text{与}\text{非}A)\Rightarrow B)\Leftrightarrow A\text{或}\text{非}A\text{或}B$.根据证明规则\ref{C10}、公理模式\ref{Sch2},$A\text{或}\text{非}A\text{或}B$,得证.
			\par
			(14)根据公理模式\ref{Sch1},$B\Rightarrow B\text{或}(A\text{与}\text{非}A)$,根据证明规则\ref{C18}、补充证明规则\ref{Ccor5}(13),$B\text{或}(A\text{与}\text{非}A)\Rightarrow B$,得证.
			\par
			(15)	根据公理模式\ref{Sch1},$B\Rightarrow B\text{或}(A\text{与}\text{非}A\text{与}C)$.根据证明规则\ref{C21},$(A\text{与}\text{非}A\text{与}C)\Rightarrow (A\text{与}\text{非}A)$,根据补充证明规则\ref{Ccor5}(14),$B\text{或}(A\text{与}\text{非}A\text{与}C)\Rightarrow B$,得证.
			\par
			(16)	根据证明规则\ref{C24}(1)、证明规则\ref{C23}(5),$(A\text{与}B\Rightarrow C)\Leftrightarrow \text{非}A\text{或}\text{非}B\text{或}C$、\\$(A\text{与}\text{非}C\Rightarrow \text{非}B)\Leftrightarrow \text{非}A\text{或}C\text{或}\text{非}B$,根据证明规则\ref{C24}(8)、证明规则\ref{C24}(9)可证.
			
			\begin{exer}\label{exer8}
				\hfill\par
				令$A$、$B$、$C$为逻辑理论$M$的公式,求证:
				\par
				(1)$A\Rightarrow (B\Rightarrow A)$;
				\par
				(2)$(A\Rightarrow B)\Rightarrow ((B\Rightarrow C)\Rightarrow (A\Rightarrow C))$;
				\par
				(3)$A\Rightarrow (\text{非}A\Rightarrow B)$;
				\par
				(4)$(A\text{或}B)\Leftrightarrow ((A\Rightarrow B)\Rightarrow B)$;
				\par
				(5)$(A\Leftrightarrow B)\Leftrightarrow ((A\text{与}B)\text{或}(\text{非}A\text{与}\text{非}B))$;
				\par
				(6)$\text{非}((\text{非}A)\Leftrightarrow B)\Rightarrow (A\Leftrightarrow B)$;
				\par
				(7)$(A\Rightarrow (B\text{或}(\text{非}C)))\Leftrightarrow ((C\text{与}A)\Rightarrow B)$;
				\par
				(8)$(A\Rightarrow (B\text{或}C))\Leftrightarrow (B\text{或}(A\Rightarrow C))$;
				\par
				(9)$(A\Rightarrow B)\Rightarrow ((A\Rightarrow C)\Rightarrow (A\Rightarrow (B\text{与}C)))$;
				\par
				(10)$(A\Rightarrow C)\Rightarrow ((B\Rightarrow C)\Rightarrow ((A\text{或}B)\Rightarrow C))$;
				\par
				(11)$(A\Rightarrow B)\Rightarrow ((A\text{与}C)\Rightarrow (B\text{与}C))$;
				\par
				(12)$(A\Rightarrow B)\Rightarrow ((A\text{或}C)\Rightarrow (B\text{或}C))$;
			\end{exer}
			证明:
			\par
			(1)假设$A$为真,根据证明规则\ref{C9},$B\Rightarrow A$,故$A\Rightarrow (B\Rightarrow A)$.
			\par
			(2)假设$A\Rightarrow B$、$B\Rightarrow C$,根据证明规则\ref{C6},$A\Rightarrow C$,故$(A\Rightarrow B)\Rightarrow ((B\Rightarrow C)\Rightarrow (A\Rightarrow C))$.
			\par
			(3)假设$A$、$\text{非}\text{非}A$为真,则“$\text{非}\text{非}A\text{或}B$”为真,即$\text{非}A\Rightarrow B$,因此$A\Rightarrow (\text{非}A\Rightarrow B)$.
			\par
			(4)假设$A\text{或}B$、$A\Rightarrow B$,又因为$B\Rightarrow B$,根据证明规则\ref{C18},$(A\text{或}B)\Rightarrow ((A\Rightarrow B)\Rightarrow B)$;
			假设$(A\Rightarrow B)\Rightarrow B$,即$\text{非}(\text{非}A\text{或}B)\text{或}B$.根据证明规则\ref{C23},$\text{非}(\text{非}A\text{或}B)\Leftrightarrow \text{非}(\text{非}A\text{或}\text{非}\text{非}B)$,即$(A\text{与}\text{非}B)\text{或}B$,根据证明规则\ref{C21},$A\text{与}\text{非}B\Rightarrow A$,根据公理模式\ref{Sch2}、证明规则\ref{C6} $A\text{与}\text{非}B\Rightarrow A\text{或}B$.根据证明规则\ref{C7},$B\Rightarrow (A\text{或}B)$,根据证明规则\ref{C18},$((A\Rightarrow B)\Rightarrow B)\Rightarrow (A\text{或}B)$,得证.
			\par
			(5)若$A\Leftrightarrow B$,则$((A\text{与}B)\text{或}(\text{非}A\text{与}\text{非}B))\Leftrightarrow A\text{或}\text{非}A$,根据证明规则\ref{C10},\\$((A\text{与}B)\text{或}(\text{非}A\text{与}\text{非}B))$.反过来,若$(A\text{与}B)\text{或}(\text{非}A\text{与}\text{非}B)$,假设$A$为真,根据证明规则\ref{C25}(2),$B\text{或}(\text{非}A\text{与}\text{非}B)$,即$\text{非}(A\text{或}B)\text{或}B$,即$(A\text{或}B)\Rightarrow B$.又因为$A$为真,根据公理模式\ref{Sch2},$A\text{或}B$,因此$B$为真,因此,$A\Rightarrow B$,同理可证$B\Rightarrow A$,故$A\Leftrightarrow B$.
			\par
			(6)$(\text{非}A)\Rightarrow B$即$(\text{非}\text{非}A)\text{或}B$,根据证明规则\ref{C16}、证明规则\ref{C23}(5),$(\text{非}\text{非}A)\text{或}B\Leftrightarrow A\text{或}B$.而$B\Rightarrow (\text{非}A)$,即$\text{非}B\text{或}\text{非}A$.
			\par
			根据证明规则\ref{C20},$((\text{非}A)\Leftrightarrow B)\Leftrightarrow (A\text{或}B)\text{与}(\text{非}A\text{或}\text{非}B)$.根据证明规则\ref{C23}(10),$\text{非}((\text{非}A)\Leftrightarrow B)\Leftrightarrow (A\text{与}\text{非}A)\text{或}(A\text{与}\text{非}B)\text{或}(B\text{与}\text{非}A)\text{或}(B\text{与}\text{非}B)$,进而,根据证明规则\ref{C23}(10)、公理模式\ref{Sch2}、证明规则\ref{C24}(4),$\text{非}((\text{非}A)\Leftrightarrow B)\Leftrightarrow ((\text{非}A\text{或}B)\text{与}(\text{非}B\text{或}A))$,即\\“$\text{非}((\text{非}A)\Leftrightarrow B)\Leftrightarrow (A\Leftrightarrow B)$”.
			\par
			(7)$A\Rightarrow (B\text{或}(\text{非}C))$即$\text{非}A\text{或}(B\text{或}\text{非}C)$,$(C\text{与}A)\Rightarrow B$即$(\text{非}\text{非}(\text{非}C\text{或}\text{非}A))\text{或}B$,根据证明规则\ref{C24}(1)、证明规则\ref{C24}(8)、证明规则\ref{C24}(9)可证.
			\par
			(8)$A\Rightarrow (B\text{或}C)$即$\text{非}A\text{或}(B\text{或}C)$,$B\text{或}(A\Rightarrow C)$即$B\text{或}(\text{非}A\text{或}C)$,、证明规则\ref{C24}(8)、证明规则\ref{C24}(9)可证.
			\par
			(9)	即补充证明规则\ref{Ccor5}(12).
			\par
			(10)	即补充证明规则\ref{Ccor5}(10).
			\par
			(11)	假设$A\Rightarrow B$,$A\text{与}C$,根据证明规则\ref{C21},$A$、$C$为真,则$B$为真,根据证明规则\ref{C20},$B\text{与}C$,得证.
			\par
			(12)	根据公理模式\ref{Sch4}、公理模式\ref{Sch3}可证.
			\par
	
			\begin{exer}\label{exer9}
				\hfill\par
				$A$为逻辑理论$M$的公式,$A\Leftrightarrow \text{非}A$是$M$的定理,求证:$M$存在矛盾.
			\end{exer}
			证明:
			\par
			$A\Rightarrow \text{非}A$,即$\text{非}A\text{或}\text{非}A$,根据公理模式\ref{Sch1},“$\text{非}A”$为真.
			\par
			$\text{非}A\Rightarrow A$,即$\text{非}\text{非}A\text{或}$真,根据证明规则\ref{C24}(1)、证明规则\ref{C23}(5),$A\text{或}A$,根据公理模式\ref{Sch1},$A$为真.
			\par
			故$M$存在矛盾.
	
			\begin{exer}\label{exer10}
				\hfill\par
				令$A_1$、$A_2$、$\cdots$、$A_n$为逻辑理论$M$的公式,求证:
				\par
				(1)要证明$A_1\text{或}A_2\text{或}\cdots\text{或}A_n$,只需要在$M$添加显式公理$\text{非}A_1$、$\text{非}A_2$、$\cdots$、$\text{非}A_{n-1}$得到的理论$M'$中,证明$A_n$即可.
				\par
				(2)若$A_1\text{或}A_2\text{或}\cdots\text{或}A_n$是$M$的定理,要证明$A$是$M$的定理,只需要证明$A_1\Rightarrow A$、$A_2\Rightarrow A$、$\cdots$、$A_n\Rightarrow A$即可.
			\end{exer}
			证明:
			\par
			(1)对$n$用数学归纳法:
			\par
			n=2时,根据证明规则\ref{C14},“$\text{非}\text{非}A_1\text{或}A_2$”是$M$的定理,根据证明规则\ref{C24}(1)、证明规则\ref{C23}(5),$A_1\text{或}A_2$是$M$的定理.
			\par
			假设命题对$n=i$成立,当$n=i+1$时,令添加公理$\text{非}A_1$、$\text{非}A_2$、$\cdots$、$\text{非}A_{i-1}$得到的理论为$M'$,再添加公理$\text{非}A_i$得到的理论为$M''$,根据证明规则\ref{C14}、证明规则\ref{C24}(1)、证明规则\ref{C23}(5),若$M''$中$Ai_1$为真,则$M'$中$A_i\text{或}A_{i+1}$为真,因此$M$中的$A_1\text{或}A_2\text{或}\cdots\text{或}A_{i-1}\text{或}(A_i\text{或}A_{i+1}$\\为真,根据证明规则\ref{C24}(9),可证.
			\par
			(2)	对$n$用数学归纳法,根据证明规则\ref{C18}可证.
	
			\begin{exer}\label{exer11}
				\hfill\par
				$A$、$B$为逻辑理论$M$的公式,令$A|B$表示$(\text{非}A\text{或}\text{非}B)$,求证:
				\par
				(1)$\text{非}A\Leftrightarrow (A|A)$;
				\par
				(2)$(A\text{或}B)\Leftrightarrow (A|A)|(B|B)$;
				\par
				(3)$(A\text{与}B)\Leftrightarrow (A|B)|(A|B)$;
				\par
				(4)$(A\Rightarrow B)\Leftrightarrow (A|(B|B))$.
			\end{exer}
			证明:
			\par
			(1)	根据补充证明规则\ref{Ccor5}(9)可证.
			\par
			(2)	根据习题\ref{exer11}(1),$(\text{非}\text{非}A)\text{或}(\text{非}\text{非}B)\Leftrightarrow (A|A)|(B|B)$,根据证明规则\ref{C23}(5)、证明规则\ref{C24}(1)可证.
			\par
			(3)	根据习题\ref{exer11}(1),$\text{非}(A|B)\Leftrightarrow (A|B)|(A|B)$,即$(A\text{与}B)\Leftrightarrow (A|B)|(A|B)$.
			\par
			(4)	根据习题\ref{exer11}(1),$(A|\text{非}B)\Leftrightarrow (A|(B|B))$,根据证明规则\ref{C23}(5)、证明规则\ref{C24}(1)可证.
			\par
			注:
			\par
			习题\ref{exer11}中的连接词“$|$”仅在该题中使用,和替代记号“$|$”不可混淆.
			\par
			习题\ref{exer11}表明,连接词“$|$”具有完全性.	
	
			\begin{exer}\label{exer12}
				\hfill\par
				令$A_1$、$A_2$、$\cdots$、$A_n$为逻辑理论$M$的显式公理,求证:当且仅当符号和公理模式与$M$相同、显式公理为$A_1$、$A_2$、$\cdots$、$A_{n-1}$、$\text{非}A_n$的理论没有矛盾时,$A_n$是独立的显式公理.
			\end{exer}			
			证明:
			\par
			令理论$M'$为符号和公理模式与M相同、显式公理为$A_1$、$A_2$、$\cdots$、$A_{n-1}$的理论,理论\\$M''$为符号和公理模式与M相同、显式公理为$A_1$、$A_2$、$\cdots$、$A_{n-1}$、$\text{非}A_n$的理论.
			\par
			假设$A_n$不独立,则$M'$与$M$等价,故$A_n$是$M'$的定理.由于$M''$比$M'$强,根据证明规则\ref{C4},$A_n$是$M''$的定理,因此$M''$有矛盾.
			\par
			假设$M''$有矛盾,根据证明规则\ref{C15},$A_n$是$M'$的定理,故$M'$和$M$等价,即$A_n$不独立.
			综上,得证.
			
		\section{量词理论(Théories quantifiées)}
		
			\begin{sign}
				\textbf{量词(quantificateur)}
				\par
				令$A$、$R$为语句,$x$为字母,则用“$(\exists x)R$”表示“$(\tau_x(R)|x)R$”,用“$(\forall x)R$”表示\\“$\text{非}((\exists x)(\text{非}R))$”.
			\end{sign}

			\begin{CS}\label{CS8}
				\hfill\par
				令$R$为语句,$x$、$x'$为字母,如果$R$不包含$x'$,则$(\exists x)R$、$(\forall x)R$分别和$(\exists x')R'$、$(\forall x')R'$相同,其中$R'$为$(x'|x)R$.
			\end{CS}
			证明:根据替代规则\ref{CS1},$(\exists x)R$和$(\tau_x(R)|x')R'$相同,根据替代规则\ref{CS3},$\tau_x(R)$和$\tau x'(R')$相同,故$(\exists x)R$和$(\exists x' )R'$相同.同理并结合替代规则\ref{CS5},可证$(\forall x)R$和$(\forall x' )R'$相同.
				
			\begin{CS}\label{CS9}
				\hfill\par
				令$R$、$U$为语句,$x$、$y$为字母,如果$U$不包含$x$,则$(U|y)(\exists x)R$、$(U|y)(\forall x)R$分别和\\$(\exists x)R'$、$(\forall x)R'$相同,其中$R'$为$(U|y)R$.
			\end{CS}
			证明:根据替代规则\ref{CS2},$(U|y)(\exists x)R$和$(T|x)(U|y)R$相同,其中$T$为$(U|y)\tau_x(R)$,根据替代规则\ref{CS4},$(T|x)(U|y)R$和$(\exists x)R'$相同.同理可证$(U|y)(\forall x)R$和$(\forall x)R'$相同.	

			\begin{CF}\label{CF11}
				\hfill\par
				如果$R$是理论$M$的公式,$x$为字母,则$(\exists x)R$和$(\forall x)R$也是理论$M$的公式.
			\end{CF}
			证明:根据形成规则\ref{CF3}、形成规则\ref{CF8}、形成规则\ref{CF2},可证.

			\begin{C}\label{C26}
				\hfill\par
				令$R$为逻辑理论$M$的公式,$x$为字母,则$(\forall x)R\Leftrightarrow(\tau_x(\text{非}R)|x)R$是$M$的定理.
			\end{C}
			证明:根据替代规则5,$(\forall x)R$即“$\text{非}\text{非}(\tau_x(\text{非}R)|x)R$”,得证.

			\begin{C}\label{C27}
				\hfill\par
				令$R$为逻辑理论$M$的定理,$x$为不是常数的字母,则$(\forall x)R$是$M$的定理.
			\end{C}
			证明:根据证明规则\ref{C26},$(\forall x)R\Leftrightarrow(\tau_x(\text{非}R)|x)R$,得证.

			\begin{C}\label{C28}
				\hfill\par
				令$R$为逻辑理论$M$的公式,$x$为字母,则$(\text{非}((\forall x)R))\Leftrightarrow(\exists x)(\text{非}R)$是$M$的定理.
			\end{C}
			证明:$(\text{非}((\forall x)R))$即“$\text{非}\text{非}(\tau_x(\text{非}R)|x)(\text{非}R)$”,得证.

			\begin{CScor}\label{CScor6}
				\hfill\par
				“令$R$为公式,$x$为字母,$T$为项,则$(T|x)R\Rightarrow(\exists x)R$是公理”是公理模式.
			\end{CScor}
			证明:
			\par
			以下证明$(U|y)((T|x)R\Rightarrow(\exists x)R)$也是该规则产生的公式:
			\par
			若$U$不包含$x$且$x$、$y$不同,根据替代规则\ref{CS2}、替代规则\ref{CS9},$(T|x)(U|y)R\Rightarrow(\exists x)(U|y)R$和明$(U|y)((T|x)R\Rightarrow(\exists x)R)$相同.
			\par
			若$U$包含$x$或$x$、$y$相同,则令$R'$为$(x'|x)R$,其中$x'$为与$y$不同的字母且$U$不包含$x'$,根据替代规则\ref{CS1}、替代规则\ref{CS8}及上述结论,得证.

			\begin{Sch}\label{Sch5}
				\hfill\par
				令$R$为公式,$x$为字母,$T$为项,则$(T|x)R\Rightarrow(\exists x)R$是公理.
			\end{Sch}

			\begin{metadef}
				\textbf{量词理论(théorie quantifiée)}
				\par
				包含公理模式$5$的逻辑理论,称为量词理论.
			\end{metadef}

			\begin{C}\label{C29}
				\hfill\par
				令$R$为量词理论$M$的公式,$x$为字母,则“$\text{非}((\exists x)R)\Leftrightarrow(\forall x)(\text{非}R)$”是$M$的定理.
			\end{C}
			证明:
			\par
			考虑其他规则相同但不包含显式公理的理论$M_0$:
			\par
			由于$R\Leftrightarrow\text{非}\text{非}R$,根据证明规则\ref{C3},$(\exists x)R\Rightarrow(\tau_x(R)|x)(\text{非}\text{非}R)$、($\exists x)(\text{非}\text{非}R)\Rightarrow\\(\tau_x(\text{非}\text{非}R)|x)R$.
			\par
			根据公理模式\ref{Sch5},$(\tau_x(R)|x)(\text{非}\text{非}R)\Rightarrow(\exists x)(\text{非}\text{非}R)$、$(\tau_x(\text{非}\text{非}R)|x)R\Rightarrow(\exists x)R$.
			\par
			故$(\exists x)R\Leftrightarrow(\exists x)(\text{非}\text{非}R)$.
			\par
			又因为$(\exists x)(\text{非}\text{非}R)\Leftrightarrow\text{非}\text{非}(\exists x)(\text{非}\text{非}R)$,后者即$\text{非}(\forall x)(\text{非}R)$,因此$\text{非}((\exists x)R))\Leftrightarrow\\(\forall x)(\text{非}R)$.
			\par
			由于$M$强于$M_0$,因此上述结论对理论$M$也成立.
			\par
			注:如果已知条件中不包括任何含常数的定理,可以用这种方法.从而,在证明过程中,可以运用“字母不是常数”的条件.

			\begin{C}\label{C30}
				\hfill\par
				令$R$为量词理论$M$的公式,$T$为$M$的项,$x$为字母,则$(\forall x)R\Rightarrow(T|x)R$是$M$的定理.
			\end{C}
			证明:根据公理模式\ref{Sch5},$(T|x)(\text{非}R)\Rightarrow(\exists x)\text{非}R$,即$\text{非}(T|x)R\Rightarrow\text{非}(\tau_x(\text{非}R)|x)R$,因此$\tau_x(\text{非}R)|x)R\Rightarrow(T|x)R$,根据证明规则\ref{C26},得证.

			\begin{Ccor}\label{Ccor6}
				\hfill\par
				$x$不是量词理论$M$的常数,则当且仅当$R$为$M$的定理时,$(\forall x)R$为$M$的定理.
			\end{Ccor}
			证明:根据证明规则\ref{C30}、证明规则\ref{C27}可证.

			\begin{C}\label{C31}
				\hfill\par
				令$R$、$S$为量词理论$M$的公式,$x$为字母,并且$x$不是量词理论$M$的常数.如果$R\Rightarrow S$是\\$M$的定理,则$(\forall x)R\Rightarrow(\forall x)S$和$(\exists x)R\Rightarrow(\exists x)S$也是$M$的定理;如果$R\Leftrightarrow S$是$M$的定理,则$(\forall x)R\Leftrightarrow(\forall x)S$和$(\exists x)R\Leftrightarrow(\exists x)S$也是$M$的定理.
			\end{C}
			证明:
			\par
			如果$R\Rightarrow S$,假设$(\forall x)R$为真,根据证明规则\ref{C30},R为真,因此S为真,根据证明规则\ref{C27},$(\forall x)S$为真,故$(\forall x)R\Rightarrow(\forall x)S$.
			\par
			如果$R\Rightarrow S$,则$\text{非}S\Rightarrow\text{非}R$,根据上述结论故$(\forall x)(\text{非}R)
			\Rightarrow(\forall x)(\text{非}S)$,根据证明规则\ref{C29},$\text{非}(\exists x(S))\Rightarrow \text{非}(\exists x(R))$,故$(\exists x)R\Rightarrow (\exists x)S$.
			\par
			根据上述结论可证,如果$R\Leftrightarrow S$是量词理论$M$的定理,则$(\forall x)R\Leftrightarrow(\forall x)S$和$(\exists x)R\Leftrightarrow(\exists x)S$.

			\begin{C}\label{C32}
				\hfill\par
				令$R$、$S$为量词理论$M$的公式,$x$为字母,则$(\forall x)(R\text{与}S)\Leftrightarrow ((\forall x)R)\text{与}((\forall x)S)$,$(\exists x)\\(R\text{或}S)\Leftrightarrow ((\exists x)R)\text{或}((\exists x)S)$是$M$的定理.
			\end{C}
			证明:
			\par
			考虑其他规则相同但不包含显式公理的理论$M_0$:
			\par
			若$(\forall x)(R\text{与}S)$为真,根据补充证明规则\ref{Ccor4},$R\text{与}S$为真,即$R$、$S$为真,根据补充证明规则\ref{Ccor4},$((\forall x)R)\text{与}((\forall x)S)$为真.反之亦然.根据证明规则\ref{C29},$(\exists x) (R\text{或}S)\Leftrightarrow ((\exists x)R)\text{或}\\((\exists x)S)$.
			\par
			由于$M$强于$M_0$,因此上述结论对理论M也成立.

			\begin{C}\label{C33}
				\hfill\par
				令$R$、$S$为量词理论$M$的公式,$x$为字母,并且$R$不包含$x$,则$(\forall x)(R\text{或}S)\Leftrightarrow (R\text{或}(\forall x)S)$\\和$(\exists x)(R\text{与}S)\Leftrightarrow (R\text{与}(\exists x)S)$是$M$的定理.
			\end{C}
			证明:
			\par
			考虑其他规则相同但不包含显式公理的理论$M_0$:
			\par
			$R\text{或}S$即$\text{非}R\Rightarrow S$,将$\text{非}R$作为公理添加到$M_0$,则$S$是定理,根据证明规则\ref{C27},$(\forall x)S$是定理,因此,$\text{非}R\Rightarrow (\forall x)S$,即$R\text{或}(\forall x)S$.反之亦然.
			\par
			根据证明规则\ref{C29},$(\exists x)(R\text{与}S)\Leftrightarrow (R\text{与}(\exists x)S)$.
			\par
			由于$M$强于$M_0$,因此上述结论对理论$M$也成立.

			\begin{C}\label{C34}
				\hfill\par
				令$R$为量词理论M的公式,x、y为字母,则以下公式都是$M$的定理:
				\par
				(1)$(\exists x)(\exists y)R\Leftrightarrow (\exists y)(\exists x)R$;
				\par
				(2)$(\forall x)(\forall y)R\Leftrightarrow (\forall y)(\forall x)R$;
				\par
				(3)$(\exists x)(\forall y)R\Rightarrow (\forall y)(\exists x)R$;
			\end{C}
			证明:
			\par
			考虑其他规则相同但不包含显式公理的理论$M_0$:
			\par
			若$(\forall x)(\forall y)R$,根据证明规则\ref{C30},$R$为真,根据证明规则\ref{C27},$(\forall y)(\forall x)R$,反之依然,即第一式成立.
			\par
			根据证明规则\ref{C29},可证第二式.
			\par
			根据证明规则\ref{C31},证明规则\ref{C30},$(\exists x)(\forall y)R\Rightarrow (\exists x)R$,若$(\exists x)(\forall y)R$,则$(\exists x)R$,根据证明规则\ref{C27},$(\forall y)(\exists x)R$,第三式得证.
			\par
			由于$M$强于$M_0$,因此上述结论对理论$M$也成立.

			\begin{sign}
				\textbf{类别量词(quantificateur typique)}
				\par
				令$A$、$R$为语句,$x$为字母,则用“$(\exists_Ax)R$”表示“$(\exists x)(A\text{与}R)$”,用“$(\forall_Ax)R$”表示“$\text{非}((\exists_Ax)(\text{非}R))$”.
			\end{sign}
			注:原书很少使用类别量词.
			
			\begin{CS}\label{CS10}
				\hfill\par
				令$A$、$R$为语句,$x$、$x'$为字母,如果$R$不包含$x'$,则$(\exists_Ax)R$、$(\forall_Ax)R$分别和$(\exists_{A'}x')R'$、$(\forall_{A'}x')R'$相同,其中$R'$为$(x'|x)R$,$A'$为$(x'|x)A$.
			\end{CS}
			证明:根据替代规则\ref{CS8}、替代规则\ref{CS5}、替代规则\ref{CS6}可证.

			\begin{CS}\label{CS11}
				\hfill\par
				令$A$、$R$、$U$为语句,$x$、$y$为字母,如果$U$不包含$x$,则$(U|y)(\exists_Ax)R$、$(U|y)(\forall_Ax)R$分别和$(\exists_{A'}x)R'$、$(\forall_{A'}x)R'$相同,其中$R'$为$(U|y)R$,$A'$为$(x'|x)A$.
			\end{CS}
			证明:根据替代规则\ref{CS9}、替代规则\ref{CS5}、替代规则\ref{CS6}可证.

			\begin{CF}\label{CF12}
				\hfill\par
				如果$A$、$R$是理$M$的公式,$x$为字母,则$(\exists_Ax)R$和$(\forall_Ax)R$也是理论$M$的公式.
			\end{CF}
			证明:根据形成规则\ref{CF11}、形成规则\ref{CF9}、形成规则\ref{CF2},可证.

			\begin{C}\label{C35}
				\hfill\par			
				令$A$、$R$为量词理论$M$的公式,$x$为字母,则$(\forall_Ax)R\Leftrightarrow (\forall x)(A\Rightarrow R)$是$M$的定理.
			\end{C}
			证明:
			\par
			考虑其他规则相同但不包含显式公理的理论$M_0$:
			\par
			$(\forall_Ax)R$即$\text{非}(\exists x)(A\text{与}\text{非}R)$,又因为$(A\text{与}\text{非}R)\Leftrightarrow (\text{非}(A\Rightarrow R))$,根据证明规则\ref{C31},\\$(\forall_Ax)R\Leftrightarrow \text{非}(\exists x)(\text{非}(A\Rightarrow R))$,即$(\forall_Ax)R\Leftrightarrow (\forall x)(A\Rightarrow R)$.
			\par
			由于$M$强于$M_0$,因此上述结论对理论$M$也成立.

			\begin{C}\label{C36}
				\hfill\par			
				令$A$、$R$为量词理论$M$的公式,$x$为字母,$M'$为量词理论$M$加上公理$A$组成的量词理论,如果$x$不是$M$的常数,并且$R$是量词理论$M'$的定理,则$(\forall_Ax)R$是$M$的定理.
			\end{C}
			证明:在理论$M$中,$A\Rightarrow R$,根据证明规则\ref{C27}、证明规则\ref{C35}可证.
	
			\begin{C}\label{C37}
				\hfill\par			
				令$A$、$R$为量词理论$M$的公式,$x$为字母,$M'$为理论$M$加上公理$A\text{与}(\text{非}R)$组成的理论,如果$x$不是$M$的常数,并且$M'$有矛盾,则$(\forall_Ax)R$是量词理论$M$的定理.				
			\end{C}
			证明:$M'$即$M$加入公理$(\text{非}(A\Rightarrow \text{非}\text{非}R))$得到的理论,因此$A\Rightarrow (\text{非}\text{非}R)$是$M$的定理,故$A\Rightarrow R$,根据证明规则\ref{C27}、证明规则\ref{C35}可证.
			
			\begin{C}\label{C38}
				\hfill\par			
				令$A$、$R$为量词理论$M$的公式,$x$为字母,则$(\text{非}(\exists_Ax(R)))\Leftrightarrow (\forall_Ax)(\text{非}R)$和$(\text{非}(\forall_Ax(R)))\\\Leftrightarrow (\exists_Ax)(\text{非}R)$都是$M$的定理.
			\end{C}
			证明:类似证明规则\ref{C29}可证.
			
			\begin{C}\label{C39}
				\hfill\par			
				令$A$、$R$、$S$为量词理论$M$的公式,$x$为字母,并且$x$不是量词理论$M$的常数.如果$A\Rightarrow (R\Rightarrow S)$是$M$的定理,则$(\forall_Ax)R\Rightarrow (\forall_Ax)S$和$(\exists_Ax)R\Rightarrow (\exists_Ax)S$也是$M$的定理;如果$A\Rightarrow (R\Leftrightarrow S)$是$M$的定理,则$(\forall_Ax)R\Leftrightarrow (\forall_Ax)S$和$(\exists_Ax)R\Leftrightarrow (\exists_Ax)S$也是$M$的定理.				
			\end{C}
			证明:类似证明规则\ref{C31}可证.			

			\begin{C}\label{C40}
				\hfill\par
				令$A$、$R$、$S$为量词理论$M$的公式,$x$为字母,则$(\forall_Ax)(R\text{与}S)\Leftrightarrow((\forall_Ax)R)\text{与}((\forall_Ax)S)$和\\$(\exists_Ax) (R\text{或}S)\Leftrightarrow ((\exists_Ax)R)\text{或}((\exists_Ax)S)$是$M$的定理.				
				\end{C}
			证明:类似证明规则\ref{C32}可证.			
			\begin{C}\label{C41}
				\hfill\par			
				令$A$、$R$、$S$为量词理论$M$的公式,$x$为字母,并且$R$不包含$x$,则$(\forall_Ax)(R\text{或}S)\\\Leftrightarrow (R\text{或}(\forall_Ax)S)$和$(\exists_Ax)(R\text{与}S)\Leftrightarrow (R\text{与}(\exists_Ax)S)$是$M$的定理.				
			\end{C}
			证明:类似证明规则\ref{C33}可证.			
			
			\begin{C}\label{C42}
				\hfill\par			
				令$A$、$B$、$R$为量词理论$M$的公式,$x$、$y$为字母,则以下公式都是$M$的定理:
				\par
				(1)$(\exists_Ax)(\exists_By)R\Leftrightarrow (\exists_By)(\exists_Ax)R$;
				\par
				(2)$(\forall_Ax)(\forall_By)R\Leftrightarrow (\forall_By)(\forall_Ax)R$;
				\par
				(3)$(\exists_Ax)(\forall_By)R\Rightarrow (\forall_By)(\exists_Ax)R$;	
			\end{C}
			证明:类似证明规则\ref{C34}可证.			

			\begin{exer}\label{exer13}
				\hfill\par
				令$A$、$B$为量词理论$M$的公式,$x$为字母,且$A$不包含$x$,求证:$(\forall x)(A\Rightarrow B)\Leftrightarrow (A\Rightarrow (\forall x)B)$.				
			\end{exer}
			证明:根据证明规则\ref{C33}可证.
			
			\begin{exer}\label{exer14}
				\hfill\par
				令$A$、$B$为量词理论$M$的公式,$x$为字母且不是$M$的常数,并且$A$不包含$x$,如果$B\Rightarrow A$是$M$的定理,求证:$(\exists x)B\Rightarrow A$是$M$的定理.
			\end{exer}
			证明:根据证明规则\ref{C31},$(\exists x)B\Rightarrow (\exists x)A$.由于$A$不包含$x$,故$(\exists x)A$和$A$相同,得证.

			\begin{exer}\label{exer15}
				\hfill\par
				令$A$为量词理论$M$的公式,$x$、$y$为字母,求证:$(\forall x)(\forall y)A\Rightarrow (\forall x)(x|y)A$、$(\exists x)(x|y)A\\\Rightarrow (\exists x)(\exists y)A$是$M$的定理.
			\end{exer}
			证明:
			\par
			考虑其他规则相同但不包含显式公理的理论$M_0$:
			\par
			根据证明规则\ref{C30}、证明规则\ref{C31},$(\forall x)(\forall y)A\Rightarrow (\forall x)(x|y)A$,根据公理模式\ref{Sch5}、证明规则\ref{C31},$(\exists x)(x|y)A\Rightarrow (\exists x)(\exists y)A$.
			\par
			由于$M$强于$M_0$,因此上述结论对理论$M$也成立.

			\begin{exer}\label{exer16}
				\hfill\par
				令$A$、$B$为量词理论$M$的公式,$x$为字母,求证:
				\par
				(1)$(\forall x)(A\text{或}B)\Rightarrow ((\forall x)A\text{或}(\exists x)B)$;
				\par
				(2)$((\forall x)A\text{与}(\exists x)B)\Rightarrow (\exists x)(A\text{与}B)$.				
			\end{exer}
			证明:
			\par
			(1)根据公理模式\ref{Sch4}、公理模式\ref{Sch5},$(A\text{或}B)\Rightarrow (A\text{或}(\exists x)B)$,根据证明规则\ref{C31}、证明规则\ref{C33}可证.
			\par
			(2)根据习题\ref{exer16}(1)、证明规则\ref{C12},$(\forall x)A\text{与}(\exists x)B)\Rightarrow \text{非}(\forall x)(A\text{或}B)$,根据证明规则\ref{C29}可证.			
			
			\begin{exer}\label{exer17}
				\hfill\par
				令$A$、$B$为量词理论$M$的公式,$x$、$y$为字母,且$B$不包含$x$、$A$不包含$y$,求证:$(\forall x)(\forall y)\\(A\text{与}B)\Rightarrow ((\forall x)A\text{与}(\forall y)B)$.			
			\end{exer}
			证明:根据证明规则\ref{C31}、证明规则\ref{C33}可证.			
			
			\begin{exer}\label{exer18}
				\hfill\par
				令$A$、$R$为量词理论$M$的公式,$x$为字母,求证:$(\exists_Ax)R\Rightarrow (\exists x)R$,$(\forall x)R\Rightarrow (\forall_Ax)R$.
			\end{exer}
			证明:根据证明规则\ref{C31}、证明规则\ref{C21},$(\exists_Ax)R\Rightarrow (\exists x)R$.根据证明规则\ref{C12},$(\forall x)R\Rightarrow (\forall_Ax)R$.			
			
			\begin{exer}\label{exer19}
				\hfill\par
				令$A$、$R$为量词理论$M$的公式,$x$为字母且不是$M$的常数.求证:如果$R\Rightarrow A$是$M$的定理,则$(\exists x)R\Leftrightarrow (\exists_ Ax)R$是$M$的定理.如果$(\text{非}R)\Rightarrow A$是$M$的定理,则$(\forall x)R\Leftrightarrow (\forall_Ax)R$是$M$\\的定理.特别是,如果$A$是$M$的定理,则$(\exists x)R\Leftrightarrow (\exists_Ax)R$、$(\forall x)R\Leftrightarrow (\forall_Ax)R$都是$M$的定理.
			\end{exer}
			证明:
			\par
			如果$R\Rightarrow A$是$M$的定理,则$R\Leftrightarrow (A\text{与}R)$,根据证明规则\ref{C31},$(\exists x)R\Leftrightarrow (\exists_Ax)R$.
			\par
			如果$(\text{非}R)\Rightarrow A$是$M$的定理,则$(\text{非}R)\Leftrightarrow (A\text{与}\text{非}R)$,根据证明规则\ref{C31}、证明规则\ref{C12},$(\forall x)R\Leftrightarrow (\forall_Ax)R$.
			如果$A$是$M$的定理,则$R\Rightarrow A$、$(\text{非}R)\Rightarrow A$,根据上述结论,$(\exists x)R\Leftrightarrow (\exists_Ax)R$、$(\forall x)R\Leftrightarrow (\forall_Ax)R$都是$M$的定理.			
			
			\begin{exer}\label{exer20}
				\hfill\par
				令$A$、$R$为量词理论$M$的公式,$T$是$M$的项,$x$为字母,如果$(T|x)A$是$M$的定理,求证:$(T|x)R\Rightarrow (\exists_Ax)R$、$(\forall_Ax)R\Rightarrow (T|x)R$是$M$的定理.				
			\end{exer}
			证明:由于$(T|x)A$是$M$的定理,因此$(T|x)R\Rightarrow (T|x)(R\text{与}A)$.根据公理模式\ref{Sch5},$(T|x)\\(R\text{与}A)\Rightarrow (\exists_Ax)R$,故$(T|x)R\Rightarrow (\exists_Ax)R$.根据证明规则\ref{C12},$(\forall_Ax)R\Rightarrow (T|x)R$.


		\section{等式理论(Théories égalitaires)}
			\begin{sign}
				\textbf{等式(égalité)}
				\par
				令$T$、$U$为语句,$x$为字母,则“用$T=U$”表示“$=TU$”(第三优先级);用“$T\neq U$”表示“$\text{非}(T=U)$”(第三优先级).
			\end{sign}
			
			\begin{CScor}\label{CScor7}
				\hfill\par
				下列规则均为公理模式:
				\par
				(1)	令$x$为字母,$T$和$U$为项,$R$为公式,则$(T=U)\Rightarrow((T|x)R\Leftrightarrow(U|x)R)$是公理.
				\par
				(2)令$R$和$S$为公式,$x$为字母,则$(\forall x)(R\Leftrightarrow S)\Rightarrow(\tau_x(R)=\tau_x(S))$是公理.
			\end{CScor}
			证明:
			\par
			令原公式为$A$,以下证明$(V|y)A$也是该规则产生的公式:
			\par
			若$V$不包含$x$且$x$、$y$不同,根据替代规则\ref{CS2}、替代规则\ref{CS5},可证明(1),根据替代规则\ref{CS4}、替代规则\ref{CS7},可证明(2).
			\par
			若$V$包含$x$或$x$、$y$相同,则令$R'$为$(x'|x)R$,其中$x'$为与y不同的字母且$U$不包含$x'$,根据替代规则\ref{CS1}、替代规则\ref{CS8}及上述结论可证.

			\begin{Sch}\label{Sch6}:
				\hfill\par
				令$x$为字母,$T$和$U$为项,$R$为公式,则$(T=U)\Rightarrow((T|x)R\Leftrightarrow(U|x)R)$是公理.
			\end{Sch}
			注:即相等的量有相同的性质.
			
			\begin{Sch}\label{Sch7}:
				\hfill\par
				令$R$和$S$为公式,$x$为字母,则$(\forall x)(R\Leftrightarrow S)\Rightarrow(\tau_x(R)=\tau_x(S))$是公理.
			\end{Sch}

			\begin{metadef}
				\textbf{等式理论(théorie égalitaire)}
				\par
				等式理论是指,定义了2元特别符号“$=$”,并且包含公理模式$6$、公理模式$7$的量词理论.
			\end{metadef}

			\begin{C}\label{C43}
				\hfill\par
				令$x$为字母,$T$和$U$为等式理论$M$的项,$R$为$M$的公式,则$((T=U)\text{与}(T|x)R)\Leftrightarrow((T=U)\text{与}(U|x)R)$是$M$的定理.
			\end{C}
			证明:
			\par
			假设$((T=U)\text{与}(T|x)R)$,则$T=U$.
			\par
			由于$(T|x)R$,根据公理模式\ref{Sch6},$(U|x)R$.
			\par
			反之亦然.

			注:下文中提及的所有定理、补充定理,都是基于前文已提及的特别符号、公理模式、显式公理组成的理论.它们也适用于更强的理论.
			\begin{theo}\label{theo1}
				\textbf{等式的反身性}
				\par
				x=x.
			\end{theo}
			证明:
			\par
			设理论为$M$,考虑其他规则相同但不包含显式公理的理论$M_0$:
			\par
			对任意公式R,根据证明规则\ref{C27},$\forall x(R\Leftrightarrow R)$,根据公理模式\ref{Sch7},$\tau_x(R)=\tau_x(R)$,即\\$(\tau_x(R)|x)(x=x)$,令R为$\text{非}(x=x)$,根据证明规则\ref{C26},$(\forall x)(x=x)$,根据证明规则\ref{C30},$x=x$.
			\par
			由于$M$强于$M_0$,因此上述结论对理论$M$也成立.

			\begin{Ccor}\label{Ccor7}
				\textbf{同一律}
				\par
				令$T$为等式理论M的项,则$T=T$.
			\end{Ccor}
			证明:
			\par
			考虑其他规则相同但不包含显式公理的理论$M_0$:
			根据定理\ref{theo1}、证明规则\ref{C27},$(\forall x)(x=x)$,根据证明规则\ref{C30},$T=T$.
			由于$M$强于$M_0$,因此上述结论对理论$M$也成立.
			
			\begin{Ccor}\label{Ccor8}			
				\hfill\par
				令$T$为等式理论M的项,则$(\exists x)(x=T)$,$(\exists x)(T=x)$.
			\end{Ccor}
			证明:根据补充证明规则\ref{Ccor7}、替代规则\ref{CS9}、公理模式\ref{Sch5}可证.
			
			\begin{theo}\label{theo2}
				\textbf{等式的对称性}
				\par
				$(x=y)\Leftrightarrow(y=x)$.
			\end{theo}
			证明:假设$x=y$,根据公理模式\ref{Sch6},$(x|y)(y=x)\Leftrightarrow(y|y)(y=x)$,即$(x=x)\Leftrightarrow(y=x)$,根据定理\ref{theo1},$y=x$.反之亦然.
			\begin{theo}\label{theo3}
				\textbf{等式的传递性}			
				\par
				$((x=y)\text{与}(y=z))\$Rightarrow(x=z)$.
			\end{theo}
			证明:假设$x=y$、$y=z$,根据公理模式\ref{Sch6},$(x=y)\Rightarrow((x=z)\Leftrightarrow(y=z))$,因此,$(x=z)\Leftrightarrow(y=z)$,因此,$x=z$.

			\begin{C}\label{C44}
				\hfill\par
				令$T$、$U$、$V$为等式理论$M$的项,$x$为字母,则$(T=U)\Rightarrow((T|x)V=(U|x)V)$是$M$的定理.
			\end{C}
			证明:
			\par
			令$y$、$z$为和$x$不同的字母,且不出现在$T$、$U$、$V$中.
			\par
			假设$y=z$,则$(y|z)((y|x)V=(z|x)V)\Leftrightarrow((y|x)V=(z|x)V)$,即$((y|x)V=(y|x)V)\Leftrightarrow((y|x)V=(z|x)V)$,根据定理\ref{theo1},$(y|x)V=(y|x)V$,故$(y|x)V=(z|x)V$.
			\par
			因此,$(y=z)\Rightarrow((y|x)V=(z|x)V)$,进而$(T|y)(U|z)((y=z)\Rightarrow((y|x)V=(z|x)V))$,即$(T=U)\Rightarrow((T|x)V=(U|x)V)$.

			\begin{metadef}
				\textbf{单一公式(relation univoque),唯一(il existe au plus un)}
				\par
				令$R$为等式理论$M$的公式,$x$、$y$、$z$为不同的字母,并且$R$不包含$y$和$z$,如果$(\forall y)(\forall z)\\(((y|x)R)\text{与}((z|x)R))\Rightarrow (y=z)$是$M$的定理,则称在$M$中,$R$是$x$上的单一公式 ,或称在$M$\\中,满足$R$的$x$是唯一的.
			\end{metadef}

			\begin{metadef}
				\textbf{函数性公式(relation fonctionnelle),有且仅有一个(il existe un et un seul)}
				\par
				令$R$为等式理论$M$的公式,$x$为字母,如果在$M$中,$R$是$x$上的单一公式,并且$(\exists x)R$是\\$M$的定理,则称在$M$中,$R$是$x$上的函数性公式,或称在$M$中,有且仅有一个$x$满足$R$.
			\end{metadef}

			\begin{C}\label{C45}
				\hfill\par
				令$R$为等式理论$M$的公式,$x$为字母并且不是$M$的常数.如果在$M$中,$R$是$x$上的单一公式,则$R\Rightarrow (x=\tau_x(R))$是$M$的定理.反之,如果存在等式理论$M$的项$T$,使$R\Rightarrow (x=T)$为$M$的定理,则在$M$中,$R$是$x$上的单一公式.
			\end{C}
			证明:
			\par
			当$R\Rightarrow (x=\tau_x(R))$时,假设$R$为真,根据公理模式\ref{Sch5},$(\tau_x(R)|x)R$,因此$R\text{与}(\tau_x(R)|x)R$,由于$R$是$x$上的单一公式,因此$R\text{与}(\tau_x(R)|x)R\Rightarrow (x=\tau_x(R))$.根据证明规则\ref{C30}可证.
			\par
			反过来,当$R\Rightarrow (x=T)$时,令$y$、$z$是和$x$不同且不出现在$R$中的字母,则$(y|x)R\Rightarrow (y=T)$、$(z|x)R\Rightarrow (z=T)$.$假设(\forall y)(\forall z)(((y|x)R)\text{与}((z|x)R))$,根据证明规则\ref{C30},$(y|x)R)\text{与}\\((z|x)R$,因此$y=T$、$z=T$,根据\ref{theo3},得证.
			
			\begin{C}\label{C46}
				\hfill\par	
				令$R$为等式理论$M$的公式,$x$为字母并且不是等式理论$M$的常数.如果在$M$中,$R$是$x$上的函数性公式,则$R\Leftrightarrow (x=\tau_x(R))$是$M$的定理.反之,如果存在等式理论$M$的项$T$,使$R\Leftrightarrow (x=T)$为理论$M$的定理,则在$M$中,$R$是$x$上的函数性公式.
			\end{C}			
			证明:
			\par
			当$R$是$x$上的函数性公式时,根据证明规则\ref{C45},$R\Rightarrow (x=\tau_x(R))$.同时,根据公理模式\ref{Sch5},$(\exists x)R$.根据公理模式\ref{Sch6},$(x=\tau_x(R))\Rightarrow (R\Leftrightarrow (\exists x)R)$.当$x=\tau_x(R)$时,$R\Leftrightarrow (\exists x)R$,又因为$(\exists x)R$,因此$R$为真.综上,前一部分得证.
			\par
			反之,如果$R\Leftrightarrow (x=T)$,根据证明规则\ref{C45},$R$是$x$上的单一公式.又因为$(T|x)R\Leftrightarrow (T=T)$,根据定理\ref{theo1},$(T|x)R$,根据公理模式\ref{Sch5},$(\exists x)R$,得证.
		
			\begin{C}\label{C47}
				\hfill\par
				令$R$、$S$为等式理论$M$的公式,$x$为字母并且不是等式理论$M$的常数.如果在$M$中,$R$是\\$x$上的函数性公式,则$(\tau_x(R)|x)S\Leftrightarrow (\exists x)(R\text{与}S)$.
			\end{C}
			证明:
			\par
			根据证明规则\ref{C46},$R\Leftrightarrow (x=\tau_x(R))$.
			\par			根据证明规则\ref{C43},$(x=\tau_x(R))\text{与}S\Leftrightarrow (x=\tau_x(R))\text{与}(\tau_x(R)|x)S$.
			\par
			因此,$R\text{与}(\tau_x(R)|x)S\Leftrightarrow R\text{与}S$.
			\par
			由于$(\tau_x(R)|x)S$不包含$x$,根据证明规则\ref{C33},$(\exists x)R\text{与} (\tau_x(R)|x)S\Leftrightarrow (\exists x)(R\text{与}S)$.
			\par
			由于$R$是$x$上的函数性公式,故$(\exists x)R$,因此$(\tau_x(R)|x)S\Leftrightarrow (\exists x)(R\text{与}S)$.						

			\begin{Ccor}\label{Ccor9}	
				\hfill\par			
				令$T$为等式理论$M$的项,$x$为字母且$T$不包含$x$,则在$M$中,$x=T$是$x$上的函数性公式.
			\end{Ccor}
			证明:
			\par
			考虑其他规则相同但不包含显式公理的理论$M_0$:
			\par
			根据证明规则\ref{C46}可证.
			\par
			由于$M$强于$M_0$,因此上述结论对理论$M$也成立.
			
			\begin{Ccor}\label{Ccor10}
				\hfill\par		
				令$R$为等式理论$M$的公式,$T$是$M$的项,$x$为字母且$T$不包含$x$,则$(\exists x)(x=T\text{与}R)\Leftrightarrow (T|x)R$.
			\end{Ccor}
			证明:
			\par
			考虑其他规则相同但不包含显式公理的理论$M_0$:
			\par
			根据补充证明规则\ref{Ccor9},$x=T$是$x$的函数性公式;根据证明规则\ref{C47},$(\tau_x(x=T)|x)R\Leftrightarrow (\exists x)(x=T\text{与}R)$.
			\par
			根据证明规则\ref{C46},$(x=T)\Leftrightarrow (x=\tau_x(x=T))$.根据补充证明规则\ref{Ccor8},$(\exists x)(x=T)$,即$\tau_x(x=T)=T$.根据公理模式\ref{Sch6},$(\tau_x(x=T)|x)R\Leftrightarrow (T|x)R$.得证.
			\par
			由于$M$强于$M_0$,因此上述结论对理论$M$也成立.

			\begin{exer}\label{exer21}
				\hfill\par
				令$M$为等式理论,求证:在$M$中,$x=y$是$x$上的函数性公式.
			\end{exer}
			证明:根据补充证明规则\ref{Ccor9}可证.

			\begin{exer}\label{exer22}
				\hfill\par
				令$R$为等式理论$M$的公式,$x$、$y$是不同的字母.求证:$(\exists x)(x=y\text{与}R)\Leftrightarrow (y|x)R$.
			\end{exer}
			证明:根据补充证明规则\ref{Ccor10}可证.

			\begin{exer}\label{exer23}
				\hfill\par
				令$R$、$S$为等式理论$M$的公式,$T$为$M$的项,$x$、$y$为字母,$y$不是$M$的常数,$T$不包含$x$.令理论$M'$为$M$添加显式公理$S$形成的理论.在理论$M'$中,$R$是$x$上的函数性公式,$(T|y)S$是理论$M$的定理.求证:在$M$中,$(T|y)R$是$x$上的函数性公式.
			\end{exer}	
			证明:
			\par
			在理论$M$中,根据证明规则\ref{C14},$S\Rightarrow (\exists x)R$.根据证明规则\ref{C3}、替代规则\ref{CS5},$(T|y)S\Rightarrow (T|y)(\exists x)R$.
			\par
			根据替代规则\ref{CS9},$(T|y)(\exists x)R即(\exists x)(T|y)R$,故$(\exists x)(T|y)R$.
			\par
			令$u$、$v$为不同于$x$、$y$且不出现在$R$和$T$中的字母,在理论$M$中,根据证明规则\ref{C14},$S\Rightarrow (\forall u)(\forall v)(((u|x)R)\text{与}((v|x)R))\Rightarrow (u=v)$.
			\par
			根据证明规则\ref{C3}、替代规则\ref{CS9}、替代规则\ref{CS2}、替代规则\ref{CS5},$(T|y)S\Rightarrow (\forall u)(\forall v)\\(((u|x)(T|y)R)\text{与}((v|x)(T|y)R))\Rightarrow (u=v)$.
			\par
			因此,$(\forall u)(\forall v)(((u|x)(T|y)R)\text{与}((v|x)(T|y)R))\Rightarrow (u=v)$.
			\par
			综上,得证.

			\begin{exer}\label{exer24}
				\hfill\par
				令$R$、$S$为等式理论$M$的公式,$x$为字母且不是$M$的常数.若$R$是$x$上的函数性公式,$R\\\Leftrightarrow S$是定理,求证:$S$是$x$上的函数性公式.
			\end{exer}	
			证明:
			\par
			根据证明规则\ref{C31},$(\exists x)R\Leftrightarrow (\exists x)S$,又因为$R$是$x$上的函数性公式,故$(\exists x)S$.
			\par
			令$y$、$z$为不同且不同于$x$的字母,并且$R$不包含$y$和$z$,又因为$R$是$x$上的函数性公式,所以$(\forall y)(\forall z)(((y|x)R)\text{与}((z|x)R))\Rightarrow (y=z)$.
			\par
			根据证明规则\ref{C3}、替代规则\ref{CS7},$(y|x)R\Leftrightarrow (y|x)S$,$(z|x)R\Leftrightarrow (z|x)S$.
			\par
			根据补充证明规则\ref{Ccor5}(4),$(y|x)R\text{与}(z|x)R\Leftrightarrow (y|x)S\text{与}(z|x)S$.根据证明规则\ref{C31},\\$(\forall y)(\forall z)(((y|x)R)\text{与}((z|x)R))\Leftrightarrow (\forall y)(\forall z)(((y|x)S)\text{与}((z|x)S))$,因此$(\forall y)(\forall z)(((y|x)S)\text{与}\\((z|x)S))\Rightarrow (y=z)$.

			\begin{exer}\label{exer25}
				\hfill\par
				令$R$、$S$、$T$为等式理论$M$的公式,$x$为字母,若$R$是$x$上的函数性公式,求证:下列公式是$M$的定理:
				\par
				(1)$(\text{非}(\exists x)(R\text{与}S))\Leftrightarrow ((\exists x)R\text{与}(\text{非}S))$;
				\par
				(2)$(\exists x)(R\text{与}(S\text{与}T))\Leftrightarrow ((\exists x)(R\text{与}S)\text{与}(\exists x)(R\text{与}T))$;
				\par
				(3)$(\exists x)(R\text{与}(S\text{或}T))\Leftrightarrow ((\exists x)(R\text{与}S)\text{与}(\exists x)(R\text{或}T))$.			
			\end{exer}	
			证明:
			\par
			(1)	根据证明规则\ref{C47},$\text{非}(\exists x)(R\text{与}S)\Leftrightarrow \text{非}(\tau_x(R)|x)S$、$(\exists x)R\text{与}(\text{非}S)\Leftrightarrow\\(\tau_x(R)|x)(\text{非}S)$,根据替代规则\ref{CS5}可证.
			\par
			(2)	根据证明规则\ref{C47},$(\exists x)(R\text{与}(S\text{与}T))\Leftrightarrow (\tau_x(R)|x)(S\text{与}T)$、$(\exists x)(R\text{与}S)\Leftrightarrow\\(\tau_x(R)|x)S$、$(\exists x)(R\text{与}T)\Leftrightarrow (\tau_x(R)|x)T$,根据替代规则\ref{CS6}可证.
			\par
			(3)	根据证明规则\ref{C47},$(\exists x)(R\text{与}(S\text{或}T))\Leftrightarrow (\tau_x(R)|x)(S\text{或}T)$、$(\exists x)(R\text{与}S)\Leftrightarrow\\(\tau_x(R)|x)S$、$(\exists x)(R\text{与}T)\Leftrightarrow (\tau_x(R)|x)T$,根据替代规则\ref{CS5}可证.			
			
			\begin{exer}\label{exer26}
				\hfill\par
				求证:$(\exists x)R\Rightarrow R$不是公理模式.
			\end{exer}	
			证明:
			\par
			令$x$、$y$为不同的字母,$R$是包含$x$、$y$的公式.
			\par
			根据替代规则\ref{CS5},$(y|x)((\exists x)R\Rightarrow R)$即$(\exists x)R\Rightarrow (y|x)$R.
			\par
			假设该公式具有$(\exists z)R'\Rightarrow R'$的形式,则该公式是$\lor$开头的平衡语句,因此,它能唯一的表示为$\lor BC$的形式,即$(\exists x)R$和$(\exists z)R'$相同,$(y|x)R$和$R'$相同.即$(\exists x)R$和$(\exists z)(y|x)R$相同.
			\par
			假设$z$和$x$相同,则$(\tau_x(R)|x)R$和$(y|x)R$相同,但$\tau_x(R)$至少包含两个字符,二者字符数量不同,矛盾.故$z$和$x$不同.
			\par
			由于$(\exists z)(y|x)R$不包含$z$,因此$(\exists x)R$不包含$z$,即$R$不包含$z$.故$(\tau_x(R)|x)R$和$(y|x)R$相同,但$\tau_x(R)$显然至少包含两个字符,二者字符数量不同,同样矛盾.得证.
			\par
			注:习题\ref{exer26}涉及尚未介绍的“平衡片段唯一性”的知识.			
			
			\begin{exer}\label{exer27}
				\hfill\par
				求证:$(R\Leftrightarrow S)\Rightarrow (\tau_x(R)=\tau_x(S))$不是公理模式.
			\end{exer}	
			证明:
			\par
			令$x$、$y$为不同的字母,$R$、$S$是包含$x$、$y$的公式.
			\par
			根据替代规则\ref{CS5}、替代规则\ref{CS7},$(y|x)((R\Leftrightarrow S)\Rightarrow (\tau_x(R)=\tau_x(S)))$即$((y|x)R\Leftrightarrow (y|x)S)\\\Rightarrow (\tau_x(R)=\tau_x(S))$.
			\par
			假设该公式具有$(R'\Leftrightarrow S')\Rightarrow (\tau_z(R')=\tau_z(S'))$的形式,则$R'$、$S'$、$\tau_z(R')$、$\tau_z(S')$分别和$(y|x)R$、$(y|x)S$、$\tau_x(R)$、$\tau_x(S)$相同.即$\tau_z((y|x)R)$、$\tau_z((y|x)S)$分别和$\tau_x(R)$、$\tau_x(S)$相同.
			\par
			若$z$和$x$相同,由于$(y|x)R$、$(y|x)S$不包含$x$,而$R$、$S$包含$x$,$\tau_x((y|x)R)$、$\tau_x((y|x)S)$没有连线,而$\tau_x(R)$、$\tau_x(S)$有连线,矛盾.
			\par
			因此$z$和$x$不同,由于$\tau_z((y|x)R)$、$\tau_z((y|x)S)$不包含$z$,因此$(y|x)R$、$(y|x)S$不包含$z$.故\\$\tau_z((y|x)R)$、$\tau_z((y|x)S)$没有连线,而$\tau_x(R)$、$\tau_x(S)$有连线,同样矛盾.得证.
			\par
			注:习题\ref{exer27}证明使用习题\ref{exer26}的结论,同样涉及尚未介绍的“平衡片段唯一性”的知识.			

		\section{项和公式的性质(Caractérisation des termes et des relations)}
			\begin{gramdef}
				\textbf{单词(mot),单词幺半群(monoïde de mots)}
				\par
				令$S$为理论的符号集合,在$S$上按照下列规则构建的自由幺半群$L_0(S)$,称为单词幺半群,其元素称为单词:
				\par
				第一,其元素为所有有限个符号组成的序列$s_0s_1\cdots s_n$(也可以记作$(s_i)_{i \in [0, n]}$);
				\par
				第二,元素$A$和$B$的乘法运算,为序列$A$和序列$B$连接成的序列,记作$AB$;
				\par
				第三,其单位元为零个元素组成的序列.
			\end{gramdef}

			\begin{gramdef}
				\textbf{长度(longueur)}
				\par
				单词$A$的长度,为$A$包含的符号集合的元素数目,记作$l(A)$.
			\end{gramdef}

			\begin{gramdef}
				\textbf{权重(poid)}
				\par				
				建立符号集合到非负整数集的映射.定义单词$A$的权重为$A$包含的各符号对应的数值之和,记作$n(A)$.
			\end{gramdef}
	
			\begin{gramcor}\label{gramcor1}
				A、B为单词,则l(AB)=l(A)+l(B).
			\end{gramcor}
			证明:根据定义可证.
			
			\begin{gramcor}\label{gramcor2}
				A、B为单词,则n(AB)=n(A)+n(B).
			\end{gramcor}
			证明:根据定义可证.
			
			\begin{gramdef}
				\textbf{单词的片段(segment d'un mot),单词的真片段(segment propre d'un mot),单词的头(segment initial d'un mot),单词的真头(segment initial propre d'un mot),单词的尾(segment final d'un mot),单词的真尾(segment final propre d'un mot)}
				\par
				$A$、$A'$、$B$、$A''$均为单词,如果$A=A'BA''$,则称$B$为$A$的片段,如果$B\neq A$,则称$B$为$A$\\的真片段.如果$A'=\emptyset$,则称$B$为$A$的头,此时,如果$B\neq A$,则称$B$为$A$的真头.如果$A''=\emptyset$,则称$B$为$A$的尾,此时,如果$B\neq A$,则称$B$为$A$的真尾.	
			\end{gramdef}			
			
			\begin{gramdef}
				\textbf{不相交的单词片段(segments disjoint d'un mot),相交的单词片段\\(segments d'un mot qui rencontrent)}
				\par
				令$A$、$B$、$C$、$D$、$E$、$F$为单词,如果$A=BCDEF$,则称$C$和$E$为$A$的不相交的片段,反之,则称两个片段相交.				
			\end{gramdef}
			
			\begin{gramdef}
				\textbf{有意义的序列(suit significative),有意义的单词(mot significatif)}
				\par
				单词群$L_0(S)$的单词序列$A_1$、$A_2$、$\cdots$、$A_n$,当序列中每个元素$A_i$($0\leq i\leq n$)均满足下列条件之一时,称为有意义的序列:
				\par
				(1)$A_i$仅包含一个权重为$0$的符号;
				\par
				(2)在$A_i$之前有$p$个元素$A_{i_1}$、$A_{i_2}$、$\cdots$、$A_{i_p}$,以及一个权重为$p$的字符$f$,使$A_i$=\\$fA_{i_1}A_{i_2}\cdots A_{i_p}$.
				\par
				有意义的序列中的各元素,称为有意义的单词.				
			\end{gramdef}

			\begin{gramthe}\label{gramthe1}
				\hfill\par
				如果$A_1$、$A_2$、$\cdots$、$A_p$都是有意义的单词,符号$f$是权重为$p$的元素,则$fA_1A_2\cdots A_p$为有意义的单词.
			\end{gramthe}
			证明:将$A_1$、$A_2$、$\cdots$、$A_p$所在的有意义的序列合并在一起,然后加入$fA_1A_2\cdots A_p$,仍能得到一个有意义的序列,得证.
			
			\begin{gramdef}
				\textbf{平衡单词(mot équilibré)}
				\par
				单词$A$,同时满足下列两个条件的,称为平衡单词:
				\par
				(1)$l(A)=n(A)+1$;
				\par
				(2)对$A$的任何真头$B$,$l(B)\leq n(B)$.
			\end{gramdef}
			
			\begin{gramthe}\label{gramthe2}
				\hfill\par			
				$A$是平衡单词,对任意$0\leq k<l(A)$,存在唯一的从$A$的第$k+1$个符号开始的平衡片段$S$.
			\end{gramthe}			
			证明:
			\par
			根据定义,任何平衡单词的真头,不可能是平衡单词,故唯一性成立.
			\par
			下面证明存在性:
			\par
			令$A=BC$,其中$l(B)=k$,则$l(C)=l(A)-l(B)\geq n(A)+1-n(B)=n(C)+1$.
			\par
			令$C_q$为$C$的前$q$个字符组成的序列,则$l(C_0)=n(C_0)=0$.
			\par
			设$i$是使$l(C_i)\leq n(C_i)$成立的最大非负整数,即$l(C_{i+1})=i+1\geq n(C_{i+1})+1$,则$n(C_{i+1})+1\leq i+1$,$i+\leq n(C_i)+1$,$n(C_i)+1\leq n(C{i+1})+1$,因此三个式子中的等号全部成立,故$l(C_{i+1})=n(C_{i+1})+1$,所以$C_{i+1}$即为所求片段.			
			
			\begin{gramthe}\label{gramthe3}
				\hfill\par
				任何平衡单词$A$都可以写成$fA_1A_2\cdots A_p$的形式,其中$A_1$、$A_2$、$\cdots$、$A_p$均为平衡单词,且$n(f)=p$.	
			\end{gramthe}			
			证明:
			\par
			设$A$的第一个符号是$f$.根据语法定理\ref{gramthe2},$A$的其他符号可以分为$p$个平衡单$A_1$、$A_2$、$\cdots$、$A_p$均为平衡单词.
			\par
			又因为$l(A)=1+l(A_1)+l(A_2)+\cdots+l(A_p)$,
			\par
			故$l(A)=1+p+n(A_1)+n(A_2)+\cdots+n(A_p)$,
			\par
			进而$l(A)=1+p+n(A)-n(f)$.
			\par
			又因为$l(A)=1+n(A)$,
			\par
			所以$n(f)=p$,得证.
								
			\begin{gramthe}\label{gramthe4}
				\hfill\par
				当且仅当一个单词平衡时,该单词有意义.			
			\end{gramthe}			
			证明:
			\par
			充分性:
			\par
			令$A$为有意义的单词,设其属于单词序列$A_1$、$A_2$、$\cdots$、$A_n$.$A_1$只能是权重为$0$的符号,显然是平衡单词.
			\par
			假设对任意$j<k$,$A_j$均为平衡单词,考虑单词$A_k$:
			\par
			若$A_k$是权重为$0$的符号,显然是平衡单词.
			\par
			若$A_k=fB_1B_2\cdots B_p$,其中$n(f)=p$,且$B_1$、$B_2$、$\cdots$、$B_p$均为序列之前的单词,即为平衡单词,则:
			\par
			由于$l(A_k)=1+l(B_1)+l(B_2)+\cdots+l(B_p)$,
			\par
			故$l(A_k)=1+p+ n(B_1)+n(B_2)+\cdots+n(B_p)$,
			\par
			因此,$l(A_k)=1+n(f)+n(B_1)+n(B_2)+\cdots+n(B_p)$,
			\par
			所以,$l(A_k)=1+n(A_k)$.
			\par
			同时,对于$A$的任何一个真头$C=fB_1B_2\cdots B_qD$,有:
			\par
			$l(C)= 1+l(f)+l(B_1)+l(B_2)+\cdots+l(B_q)+l(D)$,
			\par
			所以$l(C)\leq 1+q+n(f)+n(B_1)+n(B_2)+\cdots+n(B_q)+n(D)$,
			\par
			进而$l(C)\leq p+n(f)+n(B_1)+n(B_2)+\cdots+n(B_q)+n(D)$
			\par
			故$l(C)\leq n(C)$,故$A_k$是平衡单词.
			\par
			必要性:
			\par
			对单词的长度运用数学归纳法.长度为$1$的平衡单词,权重为$0$,显然为有意义的单词.
			\par
			设长度$i<l(A)$的平衡单词均有意义,根据语法定理\ref{gramthe3},$A$可以写成$f A_1A_2\cdots A_p$的形式,其中$A_1$、$A_2$、$\cdots$、$A_p$均为平衡单词,因此$A_1$、$A_2$、$\cdots$、$A_p$均有意义,根据语法定理\ref{gramthe1},$A$有意义.			
						
			\begin{gramthe}\label{gramthe5}
				\hfill\par
				$A$是有意义的单词,对任意$k\in [0, l(A)[$,存在唯一的从$A$的第$k+1$个符号开始的有意义的片段$S$.	
			\end{gramthe}			
			证明:根据语法定理\ref{gramthe4}、语法定理\ref{gramthe2}可证.
						
			\begin{gramthe}\label{gramthe6}
				\hfill\par			
				$A$是有意义的单词,则$A$可以唯一表示为$fA_1A_2\cdots A_p$的形式,其中$A_1$、$A_2$、$\cdots$、$A_p$均为有意义的单词,且$n(f)=p$.
			\end{gramthe}			
			证明:
			\par
			存在性:根据语法定理\ref{gramthe4}、语法定理\ref{gramthe3}可证.
			\par
			唯一性:根据语法定理\ref{gramthe2},A可以唯一表示为$fA_1A_2\cdots A_p$的形式,其中$A_1$、$A_2$、$\cdots$、$A_p$均为平衡单词,即都是有意义的单词.
			\par
			而$l(A)=1+l(A_1)+l(A_2)+\cdots+l(A_p)$,
			\par
			故$l(A)=1+p+n(A_1)+n(A_2)+\cdots+n(A_p)$,
			\par
			因此$l(A)=1+n(A)-n(f)+p$,
			\par
			又因为$l(A)=1+n(A)$,
			\par
			所以$n(f)=p$,唯一性得证.

			\begin{gramdef}
				\textbf{平衡语句(assemblage équilibré)}
				\par
				如果语句$A$去掉连线后产生的字符序列$A^*$是平衡单词,则称$A$为平衡语句.
			\end{gramdef}
			
			\begin{gramdef}
				\textbf{语句的片段(segment d'un assemblage),语句的真片段(segment propre d'un assemblage),语句的头(segment initial d'un assemblage),语句的真头(segment initial propre d'un assemblage),语句的尾(segment final d'un assemblage),语句的真尾(segment final propre d'un assemblage),不相交的语句片段(segments disjoint d'un assemblages),相交的语句片段(segments d'un assemblages qui rencontrent)}
				\par
				对于语句$A$,去掉连线后产生的字符序列$A^*$的任何(真)片段$S^*$,如果$A$在$S^*$的相应位置有连线,则在$S^*$内部添加相应连线后,形成的语句$S$,称为$A$的(真)片段.如果片段$S^*$是$A^*$的头(尾),则称相应的语句$S$为$A$的头(尾).如果语句$A$的两个片段,在去掉连线后产生的字符序列$A^*$中相应的片段相交(不相交),则称语句$A$的这两个片段相交(不相交).				
			\end{gramdef}
			
			\begin{gramthe}\label{gramthe7}
				\hfill\par
				对于理论$M$,令$S$为$M$的符号集合,并构建单词群$L_0(S)$,其中$n(\tau)=n(\neg)\\=1$,$n(\lor)=2$,$n(\Box)=0$,$n(x)=0$($x$为字母),$n(s)=m$($s$为特别符号,$m$为该特别符号的元).则理论$M$的公式和项均为平衡语句.			
			\end{gramthe}			
			证明:
			对包含A的构造为$A_1$、$A_2$、$\cdots$、$A_p$.用数学归纳法,根据语法定理\ref{gramthe1}可证.			
			
			\begin{gramdef}
				\textbf{先行语句(assemblage antécédent),完美平衡(parfaitement \\équilibré)}
				\par
				对于以$\neg$、$\lor$或特别符号开头的平衡语句$A$,将去掉连线后产生的平衡单词$A^*$表示为\\$f{A_1}^*{A_2}^*\cdots {A_p}^*$的形式,其中${A_1}^*$、${A_2}^*$、$\cdots$、${A_p}^*$均为平衡单词.在各平衡单词${A_1}^*$、${A_2}^*$、$\cdots$、${A_p}^*$内部,如果$A$的相应位置有连线,则相应恢复连线,得到的语句$A_1$、$A_2$、$\cdots$、$A_p$称为$A$的先行语句.
				\par
				此时,如果A和$fA_1A_2\cdots A_p$完全相同,则称$A$为完美平衡语句.
				\par
				对于以$\tau$开头的平衡语句$A$,将去掉连线后产生的平衡单词$A^*$表示为$\tau B^*$的形式,选择任意一个$B^*$中未包含的字母x,将与开头的$\tau$连线的所有$\Box$替换为$x$,然后按照$A$当中的其他连线位置重新恢复连线,得到的语句$B$称为$A$的先行语句.
				\par
				此时,如果$A$和$\tau_x(B)$完全相同,则称$A$为完美平衡语句.			
			\end{gramdef}
			
			\begin{gramthe}\label{gramthe8}
				\hfill\par			
				设$A$为理论$M$的平衡语句,当且仅当$A$满足下列条件之一时,$A$是$M$的项:
				\par
				第一,$A$是单个字母;
				\par
				第二,$A$是以$\tau$开头的完美平衡语句,且A的先行语句是$M$的公式.
				\par
				当且仅当$A$满足下列条件之一时,$A$是$M$的公式:
				\par
				第一,$A$是以$\neg$或$\lor$k开头的完美平衡语句,且先行语句都是$M$的公式;
				\par
				第二,$A$是以特别符号开头的完美平衡语句,且先行语句都是$M$的项.
			\end{gramthe}			
			证明:
			\par
			根据形成规则\ref{CF1}、形成规则\ref{CF2}、形成规则\ref{CF3}、形成规则\ref{CF4},可证得充分性.
			\par
			反过来,如果$A$是公式,则$A$是$\neg B$、$\lor BC$或$sD_1D_2\cdots D_n$的形式,因此$A$是完美平衡语句,且先行语句$B$、$C$是公式,$D_1$、$D_2$、$\cdots$、$D_n$是项.
			\par
			如果$A$是项,则$A$是单个字母或者以$\tau$开头.如果$A$以$\tau$开头,则$A$可以表示为$\tau_x(B)$的形式,因此$A$是完美平衡语句,且先行语句$B$是项.
			\par
			综上,必要性成立.

			\begin{exer}\label{exer28}
				\hfill\par
				令$S$为理论的符号集合,$A$为$L_0(S)$的单词,$B$、$C$是$A$的两个有意义的片段.求证:或者$B$是$C$的片段,或者$C$是$B$的片段,或者$B$和$C$不相交.
			\end{exer}
			证明:根据语法定理\ref{gramthe4},$B$、$C$是平衡单词.
			\par
			如果$B$和$C$相交:
			\par
			设$C$的开头在$B$的开头之后,令$C$开头符号为$f$,根据语法定理\ref{gramthe2},在平衡单词$B$中,以$f$\\开头的片段中,存在唯一的平衡片段$D$.因此$D$也是$C$的片段.在平衡单词$C$中,根据语法定理\ref{gramthe2},$D$和$C$相同,即$C$是$B$的片段.
			\par
			设$C$的开头在$B$的开头之前,同理可证$B$是$C$的片段.
			\par
			若$C$和$B$开头相同,则$C$是$B$的片段,或者$B$是$C$的片段.
			\par
			综上得证.
			
			\begin{exer}\label{exer29}
				\hfill\par
				令$S$为理论的符号集合,$A$为$L_0(S)$的有意义的单词,其形式为$A'BA''$,其中$B$有意义.求证:若$C$有意义,则$A'CA''$有意义.
			\end{exer}
			证明:
			\par
			若$A$长度为$1$,则$A'$、$A''$均为单位元,$A=B$,故$A'CA''=C$,命题成立.
			\par
			设命题对长度小于$k$的单词成立,对长度为$k$的单词,根据语法定理\ref{gramthe6},$A$可以表示为\\$fA_1A_2\cdots A_p$,其中$A_1$、$A_2$、$\cdots$、$A_p$均有意义.若其中与$B$相交的多于$1$个,根据语法定理\ref{gramthe2},矛盾.故只有$1$个与B相交,设其为$A_i$,根据习题\ref{exer28},$B$是$A_i$的真片段,即$A_i$可表示为${A'}_iB{A''}_i$,因此${A'}_iC{A''}_i$也是有意义的单词.因此,$A'CA''=fA_1A_2\cdots {A'}_iC{A''}_i\cdots A_p$也是有意义的单词.

			\begin{exer}\label{exer30}
				\hfill\par
				令$E$为集合,$f$为$E\times E$到$E$的映射.令$S=E\cup{f}$.令$n(f)=2$,对于$x\in E$,令$n(x)=0$:
				\par
				(1)令$M$为$L_0(S)$有意义的单词的集合,求证:存在$M$到$E$且满足下列条件的唯一映射v:
				\par
				第一,对所有$x\in E$,$v(x)=x$;
				\par
				第二,对任意两个有意义的单词$A$、$B$,$v(fAB)=f(v(A), v(B))$.
				\par
				(2)设$A=(s_i)_{i\in [0, n]}$为$L_0(S)$的单词,定义$A^*$为其子串,其下标序列$i——1$、$i_2$、$\cdot$、$s_{i_k}$是所有满足$S_i\neq f$的下标$i$按照递增顺序排序而成.若$A$、$B$为$L_0(S)$的单词,且满足$A^*=B^*$,则称$A$、$B$相似.求证:如果$f$满足结合律(即$f(f(x, y), z) = f (x, f(y, z))$),则对任意有意义的单词$A$、$B$,$v(A)=v(B)$.				
			\end{exer}
			证明:
			\par
			(1)根据定义,存在性成立,根据语法定理\ref{gramthe6},唯一性成立.
			\par
			(2)对任意有意义的单词$A$,若$A^*=A_1A_2\cdots A_n$,则称$fA_1fA_2\cdots fA_{n-1}A_n$为$A$的标准化单词,记作$A'$.则$A$存在唯一的标准化单词,且$A$和$A'$相似.
			\par
			对于$n=1$,显然$v(A)=v(A')$,设命题$v(A)=v(A')$对$n<k$成立,则对$n=k$:
			\par
			令$A=fBC$,其中$B^*=A_1A_2\cdots A_i$,$C^*=A_{i+1}A_{i+2}\cdots A_k$,则$v(A)=f(v(B)$, $v(C))= f(v(B'), v(C'))$.
			\par
			若$i=1$,则$v(A)=f(v(A_1), v(C'))=v(A')$显然成立.
			\par
			设$v(A)=v(A')$对$i<j$成立,则对$i=j$: 根据$f$的结合律,
			\par
			$v(A)=f(f(A_1, v(fA_2fA_3\cdots fA_{j-1}A_j)), v(fA_{j+1}fA_{j+2}\cdots fA_{k-1}A_k))=\\f(A_1, f(v(fA_2fA_3\cdots fA_{j-1}A_j), v(fA_{j+1}fA_{j+2}\cdots fA_{k-1}A_k))$.
			\par
			令$D=ffA_2fA_3\cdots fA_{j-1}A_jfA_{j+1}fA_{j+2}\cdots fA_{k-1}A_k$,因此$D'=fA_2\cdots fA_{n-1}A_n$,根据归纳假设,$v(D)=v(D')$,故$v(A)=f(A_1, v(D'))=v(A')$.得证.
			
			\begin{exer}\label{exer31}
				\hfill\par
				令$A$为理论$M$的项或公式.考虑下列语句序列:
				\par
				先写$A$,如果$A$是单个字母,则结束.否则,写下$A$的先行语句(如果$A$以$\tau$开头,则写下任何一个先行语句).如有一个或数个新写出的先行语句不是字母,则继续写这些先行语句的先行语句,直至新写出的语句全部是单个字母为止.
				\par
				(1)	求证:将上述语句序列的顺序颠倒,则形成一个构造.
				\par
				(2)	若平衡语句$B$是$A$的片段,且在语句$A$中,没有$B$内部和$B$外部之间的连线,求证:$B$是$M$的项或公式.
				\par
				(3)	若$B$是项(或公式),则将$A$中的$B$替代为另一个项(或公式).求证:若$A$是项(或公式),则得到的新语句是项(或公式).
			\end{exer}
			证明:
			\par
			(1)	根据定义可证.
			\par
			(2)	
			对语句$A$的长度用数学归纳法可知,该语句序列中存在语句$C$,和$B$的起始位置相同,
			\par
			根据语法定理\ref{gramthe2},$C$和$B$长度相同.同时,由于没有$B$内部和$B$外部之间的连线,故$B$中的$\Box$没有被替代掉,即$B$和$C$相同.因此$B$是$M$的项或公式.
			\par
			(3)	设$A$为$A'BA''$,以$C$替代A.
			\par
			若$A$的长度为$1$,则$A$和$B$相同,命题成立.
			\par
			设命题对长度小于$k$的语句成立,则当$A$的长度为k时,根据习题29\ref{exer29}证明过程,$A$的先行语句中,只有一个和$B$相交,且$B$为该先行语句的片段.
			\par
			设该语句为$A_p$,将$A_p$中的$B$替代为$C$得到$A_p'$,若$A_p'$为项(或公式),则$A_p'$相应为项(或公式).因此,若$A$为项(或公式),则$A$的先行公式$A_p$替代为$A_p'$,根据语法定理\ref{gramthe8},得到的$A'CA''$也是项(或公式).

			\begin{exer}\label{exer32}
				\hfill\par
				令$A$为理论$M$的语句,$T$为$M$的项,$x$为字母.求证,若$(T|x)A$为项(或公式),则$A$为项(或公式).
			\end{exer}
			证明:根据习题\ref{exer31}(3)可证.
			
			\begin{exer}\label{exer33}
				\hfill\par
				理论$M$的公式,如果以特别符号开头,则称该公式逻辑上不可约.令$R_1$、$R_2$、$\cdots$、$R_n$为$M$中逻辑上不可约的不同公式.对于$M$的语句序列$A_1$、$A_2$、$\cdots$、$A_n$,如果其中每个语句$A_i$都满足下列条件之一,则称该序列为基于令$R_1$、$R_2$、$\cdots$、$R_n$的逻辑构造:
				\par
				第一,$A_i$是公式$R_1$、$R_2$、$\cdots$、$R_n$中的一个;
				\par
				第二,在$A_i$之前有一个语句$A_j$使$A_i$为$\neg A_j$;
				\par
				第三,在$A_i$之前有两个语句$A_j$、$A_k$,使$A_i$为$\lor A_jA_k$.
				\par
				(1)求证:基于$R_1$、$R_2$、$\cdots$、$R_n$的逻辑构造的每个语句均为理论$M$的公式.
				\par
				(2)基于$R_1$、$R_2$、$\cdots$、$R_n$的逻辑构造的每个语句,称为$R_1$、$R_2$、$\cdots$、$R_n$的逻辑构造公式.求证:若$R$、$S$为$R_1$、$R_2$、$\cdots$、$R_n$的逻辑构造公式,则$\neg R$、$\lor RS$、$\Rightarrow RS$、$R\text{与}S$、$R\Leftrightarrow S$均为为$R_1$、$R_2$、$\cdots$、$R_n$的逻辑构造公式.
				\par
				(3)令R为M的公式.考虑下列语句序列:先写$R$,如果$R$逻辑上不可约,则结束.否则,写下$R$的先行语句.如有一个或数个先行语句不是逻辑上不可约的公式,则然后继续写这些先行语句的先行语句,直至新写出的语句全部是逻辑上不可约的公式为止.若$R_1$、$R_2$、$\cdots$、$R_n$为上述语句序列中不同的逻辑上不可约公式,则称$R_1$、$R_2$、$\cdots$、$R_n$为$R$的逻辑成分.求证:$R$为其各逻辑成分的逻辑构造公式,且若从其逻辑成分中去掉一个公式,则$R$不是其剩余公式的逻辑构造公式.
				\par
				(4)令$R$为$M$的公式,令$R_1$、$R_2$、$\cdots$、$R_n$为$M$中逻辑上不可约的不同公式,并且:
				\par
				第一,$R$为$R_1$、$R_2$、$\cdots$、$R_n$的逻辑构造公式;
				\par
				第二,从$R_1$、$R_2$、$\cdots$、$R_n$中去掉任何一个公式,$R$不是其剩余公式的逻辑构造公式,
				\par
				求证:$R$的逻辑成分是为$R_1$、$R_2$、$\cdots$、$R_n$.
			\end{exer}
			证明:
			\par
			(1)对于$A_1$,其为公式$R_1$、$R_2$、$\cdots$、$R_n$中的一个,显然为公式.设命题对$i<k$成立,则对$A_k$,其为公式$R_1$、$R_2$、$\cdots$、$R_n$中的一个,或为$\neg A_j$,或为$\lor A_jA_k$,根据形成规则\ref{CF1}、形成规则\ref{CF2},$A_k$为公式.得证.
			\par
			(2)类似形成规则\ref{CF1}、形成规则\ref{CF2}、形成规则\ref{CF5}、形成规则\ref{CF9}、形成规则\ref{CF10}的证明,可证.
			\par
			(3)	将语句序列的顺序颠倒,根据定义,$R$为其各逻辑成分的逻辑构造公式.下面证明,去掉$R_n$后,$R$不是剩余公式的逻辑构造公式:
			\par
			若$R$长度为$2$,则$R$本身逻辑上不可约,仅有一个逻辑成分,去掉后则无法产生逻辑构造公式,显然成立.
			\par
			设待证命题对于小于$k$的公式成立,则对于长度为$k$的公式,若$R$本身逻辑上不可约,显然成立;若$R$为$\neg A$的形式,$R$和$A$的逻辑成分相同,假设$R$是剩余公式的逻辑构造公式,则该逻辑构造包含$A$,与$A$不是剩余公式的逻辑构造公式矛盾;若$R$为$\lor BC$的形式,则$B$、$C$必有一个公式的逻辑成分包含$R_n$,设该公式为$B$,假设$R$是剩余公式的逻辑构造公式,则该逻辑构造包含$B$ 、$C$,与$B$不是剩余公式的逻辑构造公式矛盾.综上,得证.
			\par
			(4)	按照(3)写下公式序列.若存在$R_i$($1\leq i\leq n$)不在公式序列中,则去掉$R_i$,$R$仍是剩余公式的逻辑构造公式,矛盾.得证.
			
			\begin{exer}\label{exer34}
				\hfill\par
				令$R_1$、$R_2$、$\cdots$、$R_n$为理论$M$中逻辑上不可约的公式,序列$A_1$、$A_2$、$\cdots$、$A_n$为基于$R_1$、$R_2$、$\cdots$、$R_n$的逻辑构造.
				\par
				先将每个$R_i$对应符号$0$或$1$,然后按照下列规则将每个$A_i$对应$0$或$1$:
				\par
				第一,若存在$j$使$A_i$与$R_j$相同,则$A_i$与$R_j$对应相同的符号;
				\par
				第二,若存在$j$使$A_i$与$\neg A_j$相同,而$A_j$对应$0$(或$1$),则$A_i$对应$1$(或$0$);
				\par
				第三,若存在$j$、$k$使$A_i$与$\lor A_jA_k$相同,在$A_j$ 、$A_k$均为$1$的情况下,$A_i$为$1$,其他情况下$A_i$为$0$.
				\par
				(1)求证:根据上述方法,每个$A_i$有且仅有一个对应的符号.
				\par
				(2)令$R$为$R_1$、$R_2$、$\cdots$、$R_n$的逻辑构造公式,求证:$R$对应的符号,与选择哪一个基于$R_1$、$R_2$、$\cdots$、$R_n$且包含$R$的逻辑构造无关.
				\par
				(3)令$R$、$S$为$R_1$、$R_2$、$\cdots$、$R_n$的逻辑构造公式,$R$和$\Rightarrow RS$对应的符号均为$0$,求证:$S$对应的符号为$0$.
				\par
				(4)设理论$M$的公理仅包含公理模式\ref{Sch1}、公理模式\ref{Sch2}、公理模式\ref{Sch3}、公理模式\ref{Sch4}生成的公理,$R$是$M$的定理,$R_1$、$R_2$、$\cdots$、$R_n$是$R$的逻辑成分.求证:不论各公式$R_1$、$R_2$、$\cdots$、$R_n$对应的符号是$0$还是$1$,$R$对应的符号均为$0$.
				\par
				(5)令$R$为理论$M$逻辑上不可约的公式.求证:$R$、$\text{非}R$都不是$M$的定理,并且,$M$没有矛盾.
				\par
				(6)令$R_1$、$R_2$、$\cdots$、$R_n$为理论$M$中逻辑上不可约的公式.令$S_1$、$S_2$、$\cdots$、$S_p$为所有$(R_1'\text{或}R_2'\text{或}\cdots\text{或}R_n')$形式的公式,其中${R'}_i$为$R_i$或者“$\text{非}R_i$”其中之一.令$T_1$、$T_2$、$\cdots$、$T_q$为所有$S_{i_1}\text{与}S_{i_2}\text{与}\cdots\text{与}S_{i_r}$形式的公式,其中$i_1$、$i_2$、$\cdots$、$i_r$为严格递增的下标序列.令$T_0$\\为$M$中的定理“$R_1\text{或}\text{非}R_1$”,求证:令$R_1$、$R_2$、$\cdots$、$R_n$的一切逻辑构造公式,与且仅与$T_0$、$T_1$、$T_2$、$\cdots$、$T_q$中的一个公式等价.
				\par
				(7)令$R$为理论$M$的公式,$R_1$、$R_2$、$\cdots$、$R_n$为其逻辑成分.求证:当且仅当$R$为定理时,不论各公式$R_1$、$R_2$、$\cdots$、$R_n$对应的符号是$0$还是$1$,$R$对应的符号是$0$.
			\end{exer}
			证明:
			\par
			(1)根据定义,存在性成立;根据语法定理\ref{gramthe6},唯一性成立.
			\par
			(2)若$R$长度为$2$,则$R$只能是$R_1$、$R_2$、$\cdots$、$R_n$其中之一,其对应的符号不取决于逻辑构造,故命题对长度为$2$的$R$成立.
			\par
			设公式对长度小于$k$的$R$成立,对于长度为$k$的$R$:
			\par
			若$R$为$R_1$、$R_2$、$\cdots$、$R_n$之一,命题显然成立;若$R$为$\neg A_j$或$\lor A_j A_k$,由于$A_j$(以及$A_k$)对应的符号不取决于逻辑构造;另一方面,根据语法定理\ref{gramthe6},$R$的表示形式唯一,故$R$对应的符号也不取决于逻辑构造.
			\par
			(3)	根据习题\ref{exer34}(1)、习题\ref{exer34}(2),$S$对应的符号唯一.$R$对应$0$,则$\neg R$对应$1$,若$S$对应1,则$\Rightarrow RS$对应$1$,矛盾,因此$S$对应$0$.
			\par
			(4)	考虑包含$R$的证明:若$R$为公理,对于公理模式\ref{Sch1}、公理模式\ref{Sch2}、公理模式\ref{Sch3}、公理模式\ref{Sch4},无论$A$、$B$、$C$对应的符号是$0$还是$1$,$R$对应的符号均为$0$.若$R$不是公理,则根据习题\ref{exer34}(3)可证.
			\par
			(5)	$R$逻辑上不可约,故$R$、$\text{非}R$的逻辑成分仅有公式$R$.$\text{非}R$对应的符号因$R$对应的符号而变,根据习题\ref{exer34}(4),$R$、$\text{非}R$都不是定理.另一方面,如果$M$有矛盾,即存在定理$S$和定理$\text{非}S$,根据习题\ref{exer34}(4),$S$、$\text{非}S$对应的符号均为$0$.但$S$对应的符号为$0$,$\text{非}S$对应的符号为1.根据习题\ref{exer34}(1)、\ref{exer34}(2),$S$对应的符号唯一,矛盾.因此$M$没有矛盾.
			\par
			(6)唯一性:
			\par
			若某个$S_{i_1}\text{与}S_{i_2}\text{与}\cdots\text{与}S_{i_r}$和$T_0$等价,则$S_{i_1}\text{与}S_{i_2}\text{与}\cdots\text{与}S_{i_r}$为定理,因此,$S_{i_1}$为定理,根据习题\ref{exer34}(4),$S_i1$对应的符号恒为$0$,但设$S_{i_1}$为$R_1'\text{或}R_2'\text{或}\cdots\text{或}R_n')$,令其中各公式${R'}_i$对应的符号均为$1$,则$S_{i_1}$对应的符号为$1$,根据习题\ref{exer34}(1)、习题\ref{exer34}(2),矛盾.
			\par
			若存在$(S_{i_1}\text{与}S_{i_2}\text{与}\cdots\text{与}S_{i_r})\Leftrightarrow (S_{j_1}\text{与}S_{j_2}\text{与}\cdots\text{与}S_{j_{r'}})$,则序列$i_1$、$i_2$、$\cdots$、$i_r$和$j_1$、$j_2$、$\cdots$、$j_{r'}$必不完全相同,设$j_m$不在$i_1$、$i_2$、$\cdots$、$i_r$之中,则$(S_{i_1}\text{与}S_{i_2}\text{与}\cdots\text{与}S_{i_r})\Rightarrow S_{j_m}$,设$S_{j_m}$\\为$R_1'\text{或}R_2'\text{或}\cdots\text{或}R_n'$,令其中各公式${R'}_i$对应的符号均为$1$,则$S_{j_m}$对应的符号为$1$,而$S_{i_1}$、\\$S_{i_2}$、$\cdots$、$S_{i_r}$对应的符号均为$0$,因此$S_{i_1}\text{与}S_{i_2}\text{与}\cdots\text{与}S_{i_r}$对应的符号为$0$,故\\$(S_{i_1}\text{与}S_{i_2}\text{与}\cdots\text{与}S_{i_r})\Rightarrow S_{j_m}$对应的符号为$1$,根据习题\ref{exer34}(1)、习题\ref{exer34}(2),矛盾.
			\par
			综上,唯一性得证.
			\par
			存在性:
			\par
			先证引理:
			\par
			引理1:对于$R_1$、$R_2$、$\cdots$、$R_n$构成的所有$S_1$、$S_2$、$\cdots$、$S_p$($p=2^n$),$\text{非}S_1\text{或}\text{非}S_2\text{或}\cdots\\\text{或}\text{非}S_p$为定理.
			\par
			令$U_n$为$n$个逻辑上不可约的不同公式构成的“$\text{非}S1\text{或}\text{非}S2\text{或}\text{非}\cdots\text{或}\text{非}Sp$”($p=2^n$).对$n=1$,$R_1\text{与}\text{非}R_1$为定理,显然命题成立.设公式对$n<k$成立,对于$n=k$,$U_k$等价于$(U_{k-1}\text{与}R_k)\text{或}(U_{k-1}\text{与}\text{非}R_n)$,等价于“$R_n\text{与}\text{非}R_n$”,显然命题成立,引理1得证.
			\par
			引理2:对于$R_1$、$R_2$、$\cdots$、$R_n$构成的所有$S_1$、$S_2$、$\cdots$、$S_p$($p=2^n$),对任意$i\neq j$,$S_i\text{或}S_j$均为定理.
			\par
			由于$i\neq j$,$S_i\text{或}S_j$中必有$Ri$和$\text{非}Ri$,故引理2得证.
			\par
			引理3:对于$R_1$、$R_2$、$\cdots$、$R_n$构成的所有$T_0$、$T_1$、$T_2$、$\cdots$、$T_q$,则对任意$i$,$R_i$、$\text{非}R_i$分别等价于其中某个$T_m$、$T_{m'}$.
			\par
			设$S_1$、$S_2$、$\cdots$、$S_p$中所有包含$R_i$的公式为$S_{i_1}$、$S_{i_2}$、$\cdots$、$S_{i_r}$,又因为$U_{n-1}$为真,故\\$(S_{i_1}\text{与}S_{i_2}\text{与}\cdots\text{与}S_{i_r})\Leftrightarrow(\text{非}(U_{n-1})\text{或}R_i)$,因此$(S_{i_1}\text{与}S_{i_2}\text{与}\cdots\text{与}S_{i_4})\Leftrightarrow(U_{n-1}\Rightarrow R_i)$,故\\$(S_{i_1}\text{与}S_{i_2}\text{与}\cdots\text{与}S_{i_4})\Leftrightarrow R_i$.同理可证存在等价于$\text{非}R_i$的某个$T_{m'}$.引理3得证.
			\par
			引理4:对于$R_1$、$R_2$、$\cdots$、$R_n$构成的所有$S_1$、$S_2$、$\cdots$、$S_p$($p=2^n$),则对任意$i$,$S_1\text{与}S_2\text{与}\cdots\text{与}S_i)\Leftrightarrow \text{非}(S_{i+1}\text{与}S_{i+2}\text{与}\cdots\text{与}S_p$.
			\par
			根据引理1,$\text{非}S_1\text{或}\text{非}S_2\text{或}\text{非}\cdots\text{或}\text{非}S_p$,则$\text{非}(S_1\text{与}S2\text{与}\cdots\text{与}S_i)\text{或}\\\text{非}(S_{i+1}\text{与}S_{i+2}\text{与}\cdots\text{与}S_p)$,故$(S_1\text{与}S_2\text{与}\cdots\text{与}S_i)\Rightarrow \text{非}(S_{i+1}\text{与}S_{i+2}\text{与}\cdots\text{与}S_p)$.
			\par
			反过来,考虑$(S_1\text{与}S_2\text{与}\cdots\text{与}S_i)\text{或}(S_{i+1}\text{与}S_{i+2}\text{与}\cdots\text{与}S_p)$,用分配律展开,根据引理2,各项均为真,故$(S_1\text{与}S_2\text{与}\cdots\text{与}S_i)\text{或}(S_{i+1}\text{与}S_{i+2}\text{与}\cdots\text{与}S_p)$,因此$\text{非}(S_{i+1}\text{与}S_{i+2}\text{与}\cdots\text{与}S_p)\\\Rightarrow (S_1\text{与}S_2\text{与}\cdots\text{与}S_i)$,引理4得证.
			\par
			考虑任何一个基于$R_1$、$R_2$、$\cdots$、$R_n$的逻辑构造$A_1$、$A_2$、$\cdots$、$A_m$,$A_1$必为$R_i$的形式,根据引理3,存在性对公式$A_1$成立.
			\par
			设存在性对公式$A_1$、$A_2$、$\cdots$、$A_{i-1}$成立,对于$A_i$:
			\par
			如果$A_i$为$R_j$的形式,根据引理3,存在性成立.
			\par
			如果$A_i$是$\neg A_j$的形式,若$A_j\Leftrightarrow T_0$,根据引理1,$A_i\Leftrightarrow (S_1\text{与}S_2\text{与}\cdots\text{与}S_p)$;在其他情况下,令$A_j\Leftrightarrow (S_{i_1}\text{与}S_{i_2}\text{与}\cdots\text{与}S_{i_r})$,设剩余公式为$S_{j_1}$、$S_{j_2}$、$\cdots$、$S_{j_{r'}}$,则根据引理4,$A_i\Leftrightarrow \text{非}A_j\Leftrightarrow \text{非}(S_{i_1}\text{与}S_{i_2}\text{与}\cdots\text{与}S_{i_r})\Leftrightarrow (S_{j_1}\text{与}S_{j_2}\text{与}\cdots\text{与}S_{j_{r'}})$.以上两种情况下,存在性均成立.
			\par
			如果$A_i$是$\lor A_jA_k$的形式,若$A_j\Leftrightarrow T_0$或者$A_k\Leftrightarrow T_0$,则$A_i\Leftrightarrow T_0$;在其他情况下,令$A_j\Leftrightarrow (S_{i_1}\text{与}S_{i_2}\text{与}\cdots\text{与}S_{i_r})$,$A_k\Leftrightarrow (S_{j_1}\text{与}S_{j_2}\text{与}\cdots\text{与}S_{j_{r'}})$,若$A_j$包含的各项与$A_k$包含的各项没有相同的,则$A_j\text{或}A_k$用分配律展开,各项均为真,故$A_i\Leftrightarrow T_0$,若$A_j$包含的各项与$A_k$包含的各项中有相同项$S_{l_1}$、$S_{l_2}$、$\cdots$、$S_{l_{r''}}$为相同项,则$A_i\Leftrightarrow S_{l_1}\text{与}S_{l_2}\text{与}\cdots\text{与}S_{l_{r''}}$.以上两种情况下,存在性均成立.
			\par
			综上,存在性成立.
			\par
			(7)	根据习题\ref{exer34}(4),必要性成立.
			\par
			若$R$对应的符号恒为$0$,根据习题\ref{exer34}(6),存在唯一的$T_m$,使$R\Leftrightarrow T_m$.如果$R\Leftrightarrow T_0$,则$R$为定理.否则,存在$S_{i_1}$、$S_{i_2}$、$\cdots$、$S_{i_r}$,使$R\Leftrightarrow (S_{i_1}\text{与}S_{i_2}\text{与}\cdots\text{与}S_{i_r})$,则\\$S_{i_1}\text{与}S_{i_2}\text{与}\cdots\text{与}S_{i_r}$对应的符号恒为$0$,$\text{非}(S_{i_1}\text{与}S_{i_2}\text{与}\cdots\text{与}S_{i_r})$即\\$(\text{非}S_{i_1}\text{或}\text{非}S_{i_2}\text{或}\text{非}\cdots\text{或}\text{非}S_{i_r})$,故其对应的符号恒为$1$,因此$S_{i_1}$、$S_{i_2}$、$\cdots$、$S_{i_r}$对应的符号均恒为$0$.但设$S_{i_1}$为$R_1'\text{或}R_2'\text{或}\cdots\text{或}R_n'$,令其中各公式${R'}_i$对应的符号均为$1$,则$S_{i_1}$对应的符号为1,根据习题\ref{exer34}(1)、习题\ref{exer34}(2),矛盾.
			\par
			注:
			\par
			习题\ref{exer34}(6)表明主合取范式的存在性和唯一性(同理可证主析取范式的存在性和唯一性).
			\par
			习题\ref{exer34}(4)、(7)表明逻辑理论的可靠性.
			\par
			习题\ref{exer34}(5)表明逻辑理论的一致性.
			
			\begin{exer}\label{exer35}
				\hfill\par
				令$R_1$、$R_2$、$\cdots$、$R_n$为理论$M$中逻辑上不可约的公式,将每个$R_i$对应符号$0$、$1$或$2$,对于基于$R_1$、$R_2$、$\cdots$、$R_n$的逻辑构造的一切公式,按照下列规则确定其对应的符号:$\neg 0=1$,$\neg 1=0$,$\neg 2=2$,$\lor 00=\lor 01=\lor 02=\lor 10=\lor 20=\lor 22=0$,$\lor 11=1$,$\lor 12=\lor 21=2$.
				\par
				(1)	求证:令$R$为$R_1$、$R_2$、$\cdots$、$R_n$的逻辑构造公式,求证:$R$对应的符号不取决于基于$R_1$、$R_2$、$\cdots$、$R_n$且包含$R$的逻辑构造.
				\par
				(2)	设理论$M$的公理仅包含公理模式\ref{Sch2}、公理模式\ref{Sch3}、公理模式\ref{Sch4}生成的公理,$R$是$M$的定理.求证:不论$R$的各逻辑成分对应的符号是$0$、$1$还是$2$,$R$对应的符号都是$0$.同时,如果$S$为逻辑上不可约的公式,并且对应的符号为$2$,则$(S\text{或}S)\Rightarrow S$对应的符号为$2$,进而,$M$不可能等价于一个符号与$M$相同并且仅包含公理模式\ref{Sch1}、公理模式\ref{Sch2}、公理模式\ref{Sch3}、公理模式\ref{Sch4}生成的公理的理论.
				\par
				(3)	对于仅包含公理模式公理模式\ref{Sch1}、公理模式\ref{Sch3}、公理模式\ref{Sch4}生成的公理的理论,适用下列规则:$\neg 0=1$,$\neg 1=0$,$\neg 2=2$,$\lor 00=\lor 01=\lor 10=\lor 02=\lor 20=0$,$\lor 11=1$,$\lor 12=\lor 21=1$;$\lor 22=2$;对于仅包含公理模式公理模式\ref{Sch1}、公理模式\ref{Sch2}、公理模式\ref{Sch4}生成的公理的理论,适用下列规则:$\neg 0=1$,$\neg 1=2$,$\neg 2=0$,$\lor 00=\lor 01=\lor 10=\lor 02=\lor 21=0$,$\lor 11=\lor 12=1$;$\lor 22=2$.求证与(1)、(2)类似的结论.
				\par
				(4)	对于仅包含公理模式公理模式\ref{Sch1}、公理模式\ref{Sch2}、公理模式\ref{Sch3}生成的公理的理论,适用下列规则:$\neg 0=1$,$\neg 1=0$,$\neg 2=3$,$\neg 3=0$,$\lor 00=\lor 01=\lor 10=\lor 02=\lor 20=\lor 03=\lor 30=\lor 23=\lor 32=0$,$\lor 11=1$,$\lor 12=\lor 21=\lor 22=2$,$\lor 13=\lor 31=\lor 33=3$.求证与(1)、(2)类似的结论.
			\end{exer}
			证明:
			\par
			(1)类似习题\ref{exer34}(2)可证.
			\par
			(2)第一点:类似习题\ref{exer34}(4)可证.
			\par
			第二点:若公理模式\ref{Sch1}生成的公理是$M$的定理,则$(S\text{或}S)\Rightarrow S$对应的符号为$0$.又因为$(S\text{或}S)\Rightarrow S$对应的符号为2,类似习题\ref{exer34}(1)、习题\ref{exer34}(2),$S$对应的符号唯一,矛盾.
			\par
			(3)	类似习题\ref{exer35}(1)、习题\ref{exer35}(2)可证.
			\par
			(4)	类似习题\ref{exer35}(1)、习题\ref{exer35}(2)可证.
			注:习题\ref{exer35}表明逻辑理论的独立性.

	\chapter{集合论(Théorie des ensembles)}
		\section{集合化公式(Relations collectivisantes)}

			\begin{metadef}
				\textbf{集合(ensemble)}
				\par				
				在包含二元特别符号$\in$的理论中,项又称集合.
			\end{metadef}
			注:集合论中,项与集合为同义词,万物皆为集合.

			\begin{sign}
				\textbf{属于(appartenance)}
				\par
				令$T$、$U$为语句,$x$为字母,则用“$T\in U$”表示“$\in TU$”(第三优先级);“$T\notin U$”表示“$\text{非}(T\in U)$”(第三优先级).
			\end{sign}

			\begin{de}
				\textbf{元素(élément)}
				\par
				如果$A\in B$,则称$A$为$B$的元素.
			\end{de}

			\begin{de}
				\textbf{包含于(contenu dans),包含(contenir),子集(partie/sous-ensemble)}
				\par
				如果不包含字母$z$的公式$(\forall z)((z\in x)\Rightarrow(z\in y))$为真,则称$x$包含于$y$,$y$包含$x$,或者$x$是$y$的子集,记作$x\subset y$(第三优先级).
			\end{de}

			\begin{sign}
				\textbf{非子集(non partie/non sous-ensemble)}
				\par
				“$\text{非}(x\subset y)$”记作$x\not\subset y$(第三优先级).
			\end{sign}

			\begin{CS}\label{CS12}
				\hfill\par
				令$T$、$U$、$V$为语句,$x$为字母,则$(V|x)(T\subset U)$和$(V|x)T\subset(V|x)U$相同.
			\end{CS}
			证明:根据替代规则\ref{CS9}、替代规则\ref{CS5}可证.

			\begin{CF}\label{CF13}
				\hfill\par
				令$T$、$U$为包含2元特别符号$\in$的理论$M$的项,则$T\subset U$是$M$的公式.
			\end{CF}
			证明:根据形成规则\ref{CF8},可证.

			\begin{theo}\label{theo4}
				\textbf{子集的反身性}
				\par
				$x\subset x$.
			\end{theo}
			证明:令$z$为不是常数的字母,则$(z\in x)\Rightarrow (z\in x)$,根据证明规则\ref{C27}可证.

			\begin{theo}\label{theo5}
				\textbf{子集的传递性}
				\par
				$((x\subset y)\text{与}(y\subset z))\Rightarrow (x\subset z)$.
			\end{theo}
			证明:令$u$为不是常数的字母,根据证明规则\ref{C30},$(u\in x)\Rightarrow(u\in y)$、$(u\in y)\Rightarrow(u\in z)$,故$(u\in x)\Rightarrow(u\in z)$.根据证明规则\ref{C31}可证.

			\begin{cor}\label{cor1}
				\hfill\par
				$(\forall x)(\forall y)((x\subset y)\text{与}(y\subset x)\Rightarrow (x=y))\Leftrightarrow (\forall y)(y=\tau_x((\forall z)(z\in x\Leftrightarrow z\in y)))$.
			\end{cor}
			证明:
			\par
			由于待证公式不包含字母,故可令$x$、$y$为不同字母且都不是常数.
			\par
			令$R$为公式$(\forall z)(z\in x\Leftrightarrow z\in y)$,根据证明规则\ref{C32},$(\forall x)(\forall y)((x\subset y)\text{与}(y\subset x)\Rightarrow (x=y))\Leftrightarrow (\forall x)(\forall y)(R\Rightarrow (x=y))$.
			\par
			假设$(\forall x)(\forall y)(R\Rightarrow (x=y))$,则$R\Rightarrow (x=y)$.同时,根据公理模式\ref{Sch6},$(x=y)\Rightarrow R$,故$R\Leftrightarrow x=y$.根据公理模式\ref{Sch7},$\tau_x(R)=\tau_x(x=y)$.根据定理\ref{theo1},$y=y$,根据补充证明规则\ref{Ccor8},$(\exists x)(x=y)$,即$\tau_x(x=y)=y$,根据定理\ref{theo3},$y=\tau_x(R)$,根据证明规则\ref{C27},$(\forall y)(y=\tau_x((\forall z)(z\in x\Leftrightarrow z\in y)))$.故$(\forall x)(\forall y)((x\subset y)\text{与}(y\subset x)\Rightarrow (x=y))\Rightarrow (\forall y)(y=\tau_x((\forall z)(z\in x\Leftrightarrow z\in y)))$.
			\par
			反过来,令$x'$、$x''$是与x、y、z不同的字母,假设$(\forall y)(y=\tau_x((\forall z)(z\in x\Leftrightarrow z\in y)))$.
			\par
			若$(\forall x')(\forall x'')((x'|x)R\text{与}(x''|x)R)$,根据证明规则\ref{C30},$(x'|x)R$、$(x''|x)R$,因此$(x'|x)R\Leftrightarrow (x''|x)R$,根据证明规则\ref{C27},$(\forall y)((x'|x)R\Leftrightarrow (x''|x)R)$,根据公理模式\ref{Sch7},$\tau_y((x'|x)R)=\\\tau_y((x''|x)R)$,进而,$x'=\tau_y((x'|x)R)$、$x''=\tau_y((x'|x)R)$,根据定理\ref{theo3},$x'=x''$,即$R$是$y$上的单一公式.
			\par
			根据证明规则\ref{C45},$R\Rightarrow (y=\tau_y(R))$.假设$(\forall x)(\forall y)((x\subset y)\text{与}(y\subset x))$,根据证明规则\ref{C32},$(\forall x)(\forall y)R$,根据证明规则\ref{C27},$R$为真,故$y=\tau_y(R)$.又因为$(\forall x)(x=\tau_y(R))$,根据证明规则\ref{C27},$x=\tau_y(R)$,根据定理\ref{theo3},$x=y$,故$(\forall y)(y=\tau_x((\forall z)(z\in x\Leftrightarrow z\in y)))\Rightarrow (\forall x)(\forall y)((x\subset y)\text{与}(y\subset x)\Rightarrow (x=y))$.
			\par
			注:左边为本文的外延公理,右边是外延公理的另一种表述,包含相同元素的集合相等.本补充定理表明,外延公理的两种表述方式是等价的.

			\begin{ex}\label{ex1}
				\textbf{外延公理}
				\par
				$(\forall x)(\forall y)((x\subset y)\text{与}(y\subset x)\Rightarrow(x=y))$.
			\end{ex}

			\begin{cor}\label{cor2}
				\hfill\par
				$(\forall z)((z\in x)\Leftrightarrow(z\in y))\Rightarrow(x=y)$.
			\end{cor}
			证明:根据显式公理\ref{ex1}可证.

			\begin{C}\label{C48}
				\hfill\par
				令$R$为包含$2$元特别符号$\in$和显式公理\ref{ex1}的等式理论$M$的公式,$x$、$y$为不同的字母,并且$R$不包含$y$,则在$M$中,$(\forall x)((x\in y)\Leftrightarrow R)$是$y$上的单一公式.
			\end{C}
			证明:令$z$、$z'$为不同于$x$且不出现在$R$中的字母.假设$(\forall x)((x\in z)\Leftrightarrow R)\text{与}(\forall x)((x\in z')\Leftrightarrow R)$,则$(\forall x)(((x\in z)\Leftrightarrow R)\text{与}(x\in z')\Leftrightarrow R))$,因此$(\forall x)((x\in z)\Leftrightarrow (x\in z'))$,即$(z\subset z')\text{与}(z'\subset z)$,根据显式公理\ref{ex1},$z=z'$,得证.

			\begin{sign}
				\textbf{集合化(collectivisante)}
				\par
				如果公式$R$不包含$y$,$x$、$y$是不同的字母,则语句$(\exists y)(\forall x)((x\in y)\Leftrightarrow R)$记作$Coll_xR$.
			\end{sign}
			注:如果$Coll_xR$为真,则意味着满足$R$的$x$的集合存在.

			\begin{metadef}
				\textbf{集合化公式(relation collectivisante),满足公式的元素集合\\(ensemble de éléments tels que une relation)}
				\par
				在包含2元特别符号$\in$ 的理论$M$中,如果$R$不包含$y$,$x$、$y$是不同的字母,如果$Coll_xR$是定理,则称在$M$中,$R$为$x$上的集合化公式,称不含$y$的项$\tau_y ((\forall x)((x\in y)\Leftrightarrow R))$为满足$R$的$x$\\的集合,记作$\{x|R\}$.
			\end{metadef}

			\begin{cor}\label{cor3}
				\hfill\par
				$Coll_x(x\in y)$.
			\end{cor}
			证明:$(x\in y)\Leftrightarrow (x\in y)$,得证.

			\begin{cor}\label{cor4}
				\textbf{罗素悖论,所有不属于自身的项不能组成集合}
				\par
				$\text{非}Coll_(x\notin x)$.
			\end{cor}
			证明:
			令$y$为不是常数的字母,由于$y\notin y\text{或}y\in y$,故$\text{非}((y\in y)\Leftrightarrow (y\notin y))$,根据公理模式\ref{Sch5},$(\exists x)(\text{非}((x\in y)\Leftrightarrow (x\notin x)))$,因此,$\text{非}(\forall x)((x\in y)\Leftrightarrow (x\notin x))$,根据证明规则27,$(\forall y)(\text{非}(\forall x)((x\in y)\Leftrightarrow (x\notin x)))$,因此,$\text{非}(\exists y)(\forall x)((x\in y)\Leftrightarrow (x\notin x))$.

			\begin{C}\label{C49}
				\hfill\par
				令$R$为包含$2$元特别符号$\in$和显式公理\ref{ex1}的等式理论$M$的公式,$x$、$y$为不同的字母,并且$R$不包含$y$,如果在$M$中,$R$是$x$上的集合化公式,则在$M$中,公式$(\forall x)((x\in y)\Leftrightarrow R)$是$y$上的函数性公式.
			\end{C}
			证明:根据证明规则\ref{C48}可证.

			\begin{Ccor}\label{Ccor11}
				\hfill\par
				令$R$为包含$2$元特别符号$\in$和显式公理\ref{ex1}的等式理论$M$的公式,$x$为字母,如果在$M$中,$R$是$x$上的集合化公式,则$(x\in \{x|R\})\Leftrightarrow R$是$M$的定理.
			\end{Ccor}
			证明:根据定义,$Coll_xR$和$(\forall x)(x\in \{x|R\})\Leftrightarrow R)$相同,根据证明规则\ref{C30}可证.

			\begin{Ccor}\label{Ccor12}
				\hfill\par
				令$R$、$S$为包含$2$元特别符号$\in$和显式公理\ref{ex1}的等式理论$M$的公式,$x$为字母,如果在$M$\\中,$R$是$x$上的集合化公式,且$R\Leftrightarrow S$,则$S$也是$x$上的集合化公式.
			\end{Ccor}
			证明:由于$R\Leftrightarrow S$,因此$Collx_R\Leftrightarrow Coll_xS$,得证.

			\begin{C}\label{C50}
				\hfill\par
				令$R$、$S$为包含$2$元特别符号$\in$和显式公理\ref{ex1}的等式理论$M$的公式,$x$为字母,如果在$M$\\中,$R$和$S$都是$x$上的集合化公式,则$(\forall x)(R\Rightarrow S)\Leftrightarrow \{x|R\}\subset \{x|S\}$,$(\forall x)(R\Leftrightarrow S)\Leftrightarrow \{x|R\}=\{x|S\}$.
			\end{C}
			证明:根据补充证明规则\ref{Ccor11}、显式公理\ref{ex1}可证.

			\begin{ex}\label{ex2}
				\textbf{配对公理}
				\par
				$(\forall x)(\forall y) Coll_z(z=x\text{或}z=y)$.
			\end{ex}
			注:本公理表明,任何两个项均可构成集合.

			\begin{de}
				\textbf{二元集合(ensemble à deux éléments),仅有一个元素的集合(ensemble réduit à un élément),仅有两个元素的集合(ensemble réduit à deux éléments)}
				\par
				$\{z|z=x\text{或}z=y\}$称为二元集合,记作$\{x, y\}$.$\{x, x\}$称为仅有一个元素的集合,记作$\{x\}$.如果$x\neq y$,则$\{x, y\}$称为仅有两个元素的集合.
			\end{de}

			\begin{de}
				\textbf{二元子集(partie à deux éléments)}
				\par
				如果某个二元集合是另一个集合的子集,则称其为二元子集.
			\end{de}

			\begin{cor}\label{cor5}
				\hfill\par
				$\{x, y\}=\{y, x\}$.
			\end{cor}
			证明:根据证明规则\ref{C50}可证.

			\begin{cor}\label{cor6}
				\hfill\par
				$x\in \{y\}\Leftrightarrow x=y$.
			\end{cor}
			证明:根据补充证明规则\ref{Ccor12},$x\in \{y\}\Leftrightarrow (x=y\text{或}x=y)$,得证.

			\begin{cor}\label{cor7}
				\hfill\par
				$\{z|z=x\}=\{x\}$.
			\end{cor}
			证明:$\{x\}$即$\{z|z=x\text{或}z=x\}$,根据证明规则\ref{C50}、公理模式\ref{Sch1}可证.
			
			\begin{cor}\label{cor8}
				\hfill\par
				$(\{x\}\subset X)\Leftrightarrow (x\in X)$.
			\end{cor}
			证明:
			\par
			令$z$为不是常数的字母,$\{x\}\subset X$即$(\forall z)((z\in \{x\})\Rightarrow (z\in X))$,根据补充定理\ref{cor6},$(\forall z)((z\in \{x\})\Rightarrow (z\in X))\Leftrightarrow (\forall z)((z=x)\Rightarrow (z\in X))$.若$\{x\}\subset X$,则$(\forall z)((z=x)\Rightarrow (z\in X))$,故$(x=x)\Rightarrow (x\in X)$,因此$x\in X$.
			\par
			反过来,假设$x\in X$,则$(z=x)\Rightarrow (z\in X)$,进而$(\forall z)((z=x)\Rightarrow (z\in X))$.得证.

			\begin{cor}\label{cor9}
				\hfill\par
				$(\{x\}=\{y\})\Leftrightarrow (x=y)$.
			\end{cor}
			证明:
			\par
			令$z$为不是常数的字母,根据补充定理\ref{cor7}、补充证明规则\ref{Ccor7},$z\in \{x\}\Leftrightarrow z=x$,$z\in \{y\}\Leftrightarrow z=y$.由于$\{x\}=\{y\}$,根据公理模式\ref{Sch6},$z\in \{x\}\Leftrightarrow z\in \{y\}$,则$(z=x)\Leftrightarrow (z=y)$.根据证明规则\ref{C27},$(\forall z)((z=x)\Leftrightarrow (z=y))$,根据证明规则\ref{C30},$(x=x)\Leftrightarrow (x=y)$,根据定理\ref{theo1},$x=y$,故$(\{x\}=\{y\})\Rightarrow (x=y)$.
			\par
			反过来,根据证明规则\ref{C44},$(x=y)\Rightarrow (\{x\}=\{y\})$.

			\begin{CScor}\label{CScor8}
				\hfill\par
				“令$R$为公式,$x$、$y$、$X$、$Y$为不同字母,并且$R$不包含$X$和$Y$,则$(\forall y)(\exists X)(\forall x)(R\Rightarrow(x\in X))\Rightarrow(\forall Y)Coll_x((\exists y)((y\in Y\text{与}R))$是公理”是公理模式.
			\end{CScor}
			证明:根据替代规则\ref{CS8}可证.

			\begin{Sch}\label{Sch8}
				\textbf{搜集和并集公理模式}
				\par
				令$R$为公式,$x$、$y$、$X$、$Y$为不同字母,并且$R$不包含$X$和$Y$,则$(\forall y)(\exists X)(\forall x)(R\Rightarrow(x\in X))\Rightarrow(\forall Y)Coll_x((\exists y)((y\in Y\text{与}R))$是公理.
			\end{Sch}
			注:该公理模式的含义是,对任何一个$y$值,能使公式$R$(含参数$x$、$y$)成立的$x$值,都属于某个集合,则对任何一个集合,它的所有元素$y$值对应的能使公式$R$成立的$x$值,构成一个集合.

			\begin{C}\label{C51}
				\textbf{分类定理}
				\par
				令$P$为包含$2$元特别符号$\in$ 、显式公理\ref{ex1}、显式公理\ref{ex2}和公理模式\ref{Sch8}的等式理论$M$的公式,$x$为字母,$A$为不包含$x$的项,则在$M$中,“$P\text{与}x\in A$”是$x$上的集合化公式.
			\end{C}
			证明:
			\par
			考虑其他规则相同但不包含其他显式公理的理论$M_0$,则$M_0$不包含任何常数:
			\par
			令$R$为公式$P\text{与}x=y$,其中$y$是不同于$x$且不出现在$P$、$A$中的字母.根据证明规则\ref{C27},$(\forall x)(R\Rightarrow x\in \{y\})$,即$(\{y\}|X) (R\Rightarrow x\in X)$.
			\par
			由于x、y是不同字母,根据公理模式\ref{Sch5}、证明规则\ref{C27},$(\forall y)(\exists X)(\forall x)(R\Rightarrow (x\in X))$,由于$A$不包含$x$、$y$,根据公理模式\ref{Sch8}、证明规则\ref{C30},$Coll_x((\exists y)((y\in A\text{与}R))$.根据证明规则\ref{C43},$(y\in Y\text{与}R)\Leftrightarrow (x=y\text{与}x\in A\text{与}R)$,由于$y$不出现在$P$、$A$中,根据证明规则\ref{C33},$(\exists y)(y\in Y\text{与}R)\Leftrightarrow ((\exists y)(x=y))\text{与}x\in A\text{与}R)$.根据补充证明规则\ref{Ccor8},$(\exists y)(x=y)$,因此$(\exists y)(y\in Y\text{与}R)\Leftrightarrow (x\in A\text{与}R)$即$Coll_x(x\in A\text{与}R)$.
			\par
			由于$M$强于$M_0$,因此上述结论对理论$M$也成立.

			\begin{C}\label{C52}
				\hfill\par
				令$R$为包含$2$元特别符号$\in$ 、显式公理\ref{ex1}、显式公理\ref{ex2}和公理模式\ref{Sch8}的等式理论$M$的公式,$x$为字母,$A$为不包含$x$的项,如果$R\Rightarrow (x\in A)$是$M$的定理,则在$M$中,$R$是$x$上的集合化公式.				
			\end{C}
			证明:$R\Rightarrow (x\in A)$,因此$R\Leftrightarrow R\text{与}(x\in A)$,根据证明规则\ref{C51}可证.

			\begin{Ccor}\label{Ccor13}
				\hfill\par
				令$R$为包含$2$元特别符号$\in$ 、显式公理\ref{ex1}、显式公理\ref{ex2}和公理模式\ref{Sch8}的等式理论$M$的定理,$x$为字母,则$(\text{非}Coll_x(R))$为真.				
			\end{Ccor}
			证明:
			\par
			如果$M$存在矛盾,根据补充证明规则\ref{Ccor2},任何公式均为真.
			\par
			如果$M$不存在矛盾,假设$Coll_x(R)$为真,根据补充证明规则\ref{Ccor11},$(x\in \{x|R\})\Leftrightarrow R$,即$x\in \{x|R\}$,因此对任意公式$R'$,$R'\Rightarrow x\in \{x|R\}$,根据证明规则\ref{C52},$R'$是$x$上的集合化公式.与补充定理\ref{cor4}矛盾.故“$\text{非}Coll_x(R)$”.

			\begin{cor}\label{cor10}
				\textbf{所有的项不能组成集合}
				\par
				(1)$\text{非}(\exists y)(\forall x)(x\in y)$;
				\par
				(2)$(\forall y)(\exists x)(x\notin y)$.
			\end{cor}
			证明:根据补充证明规则\ref{Ccor13}可证.

			\begin{C}\label{C53}
				\hfill\par
				令$T$、$A$为包含$2$元特别符号$\in$ 、显式公理\ref{ex1}、显式公理\ref{ex2}和公理模式\ref{Sch8}的等式理论$M$的项,$x$和$y$为不同的字母,如果$T$不包含$y$,$A$不包含$x$和$y$,则在$M$中,$(\exists x)(y=T\text{与}x\in A)$是$y$上的集合化公式.
			\end{C}
			证明:令$R$为公式$y=T$,则$(\forall y)((y=T)\Rightarrow (y\in \{T\}))$,令$X$为不同于$y$且不出现在$R$的字母,则$(\forall x)(\exists X)(\forall y)((y=T)\Rightarrow (y\in X))$.根据公理模式\ref{Sch8},$(\forall A)Colly((\exists x)((x\in A\text{与}y=T))$,根据证明规则\ref{C30},$(\exists x)((x\in A\text{与}y=T)$是$y$上的集合化公式.

			\begin{metadef}
				\textbf{形式为项的对象集合(ensemble des objets de la forme d'un terme)}
				\par
				令$T$、$A$为包含$2$元特别符号$\in$ 、显式公理\ref{ex1}、显式公理\ref{ex2}和公理模式\ref{Sch8}的等式理论$M$的项,$x$和$y$为不同的字母,如果$T$不包含$y$,$A$不包含$x$和$y$,则称$\{y|(\exists x)(y=T\text{与}x\in A)\}$为“对于$x\in A$形式为$T$的对象集合”.
			\end{metadef}

			\begin{cor}\label{cor11}
				\textbf{补集的存在性}				
				\par
				$x\notin A\text{与}x\in X$是$x$上的集合化公式.
			\end{cor}
			证明:根据证明规则\ref{C51}可证.

			\begin{de}
				\textbf{(complémentaire d'un ensemble)}
				\par
				如果$A\subset X$,则称$\{x|x\notin A\text{与}x\in X\}$为$A$的补集,记作$\complement_XA$或$X-A$.
			\end{de}

			\begin{cor}\label{cor12}
				\hfill\par
				如果$A\subset X$,则$\complement_X(\complement_XA)=A$.
			\end{cor}
			证明:根据补充证明规则\ref{Ccor12},$x\notin \{x|x\notin A\text{与}x\in X\}\Leftrightarrow (x\in X\Rightarrow x\in A)$,又因为$x\in X\Rightarrow x\in A$,所以$((x\in X\Rightarrow x\in A)\text{与}x\in X)\Leftrightarrow x\in A$,进而$\complement_X(\complement_XA)=A$.

			\begin{cor}\label{cor13}
				\hfill\par
				如果$A\subset X$,$B\subset X$,则$(A\subset B)\Leftrightarrow (\complement_XB\subset \complement_XA)$.
			\end{cor}
			证明:令$x$为字母,$A\subset B\Leftrightarrow (x\in A)\Rightarrow (x\in B)$,因此$A\subset B\Leftrightarrow ((x\notin B)\Rightarrow (x\notin A))$,故$A\subset B\Leftrightarrow ((x\notin B)\text{与}(x\in X)\Rightarrow (x\notin A)\text{与}(x\in X))$,因此$A\subset B\Leftrightarrow \complement_XB\subset \complement_XA)$.
			
			\begin{Ccor}\label{Ccor14}
				\hfill\par
				令$R$为包含$2$元特别符号$\in$ 、显式公理\ref{ex1}、显式公理\ref{ex2}和公理模式\ref{Sch8}的等式理论$M$的定理,$x$为字母,则$Coll_x(\text{非}R)$为真.
			\end{Ccor}
			证明:
			\par
			考虑其他规则相同但不包含其他显式公理的理论$M_0$,则$M_0$不包含任何常数:
			\par
			先证$((x\in y)\Leftrightarrow (\text{非}R))\Leftrightarrow (x\notin y)$.
			\par
			若$(x\notin y)$,则$(x\in y)\Leftrightarrow (\text{非}R)$即$((x\notin y)\text{或}\text{非}R)\text{与}((x\in y)\text{或}R)$,显然为真.反过来,若$(x\in y)\Leftrightarrow (\text{非}R)$,即$(x\notin y)\Leftrightarrow R)$,因此$x\notin y$.
			\par
			由于“$\text{非}(x\in Y\text{与}x\notin Y)$”,故$(\forall x)(x\notin \complement_YY)$,则$(\exists X)(\forall x)(x\notin X)$,即$(\exists X)(\forall x)(x\notin X)\Leftrightarrow (\exists X)(\forall x)((x\in y)\Leftrightarrow (\text{非}R))$,得证.
			\par
			由于$M$强于$M_0$,因此上述结论对理论$M$也成立.

			\begin{theo}\label{theo6}
				\textbf{空集的存在性和唯一性}
				\par
				$(\forall x)(x\notin X)$是$X$上的函数性公式.
			\end{theo}
			证明:
			\par
			$(\forall x)(x\notin X)$即$(\forall x)\text{非}(x\in X)$,而$(\forall x)\text{非}(x\in X)\Rightarrow (\forall x)\text{非}(x\in X)\text{或}(\forall T)(\forall x)(x\in T)$.根据证明规则\ref{C33},$(\forall x)\text{非}(x\in X)\text{或}(\forall T)(\forall x)(x\in T)\Leftrightarrow (\forall T)(\forall x)(\text{非}(x\in X)\text{或}(x\in T))$,即$(\forall T)(X\subset T)$.因此,$(\forall x)(x\notin X)\Rightarrow (\forall T)(X\subset T)$.
			\par
			若$(\forall x)(x\notin Y)\text{与}(\forall x)(x\notin Z)$,则$(\forall T)(Y\subset T)$、$(\forall T)(Z\subset T)$.因此$Y\subset Z$、$Z\subset Y$.根据显式公理\ref{ex1},$Y=Z$,即$(\forall x)(x\notin X)$是$X$上的单一公式.
			\par
			另一方面,由于“$\text{非}(x\in Y\text{与}x\notin Y)$”,故$(\forall x)(x\notin \complement_YY)$,则$(\exists X)(\forall x)(x\notin X)$,得证.

			\begin{de}
				\textbf{空集(ensemble vide),非空(n'est pas l'ensemble vide)}
				\par
				不包含$X$和$x$的项$\tau_X((\forall x)(x\notin X))$,称为空集,记作$\varnothing$.
				\par
				如果$E\neq \varnothing$,则称$E$非空.
			\end{de}
			注:用形式语言表示,空集是$\overline{\overline{\tau\neg\neg\neg\in\overline{\tau\neg\neg\in\Box}\Box}\Box}$.

			\begin{cor}\label{cor14}
				\hfill\par
				$(\forall x)(x\notin X)\Leftrightarrow (X=\varnothing)$,$(\exists x)(x\in X)\Leftrightarrow (X\neq \varnothing)$.
			\end{cor}
			证明:根据定理\ref{theo6}、证明规则\ref{C46}可证.

			\begin{cor}\label{cor15}
				\hfill\par
				$(\forall x)(x\notin \varnothing)$.
			\end{cor}
			证明:根据补充定理\ref{cor14}(1)、补充证明规则\ref{Ccor7}可证.

			\begin{cor}\label{cor16}
				\hfill\par
				(1)$\{x\}\neq \varnothing$;
				\par
				(2)$\{\{x\}\}\neq \{\varnothing\}$ ;
				\par
				(3)$\{\{\{x\}\}\}\neq \{\{\varnothing\}\}$.
			\end{cor}
			证明:
			\par
			(1)根据补充定理\ref{cor6},$x\in \{x\}$,根据公理模式\ref{Sch5},$(\exists y)(y\in \{x\})$,根据补充定理\ref{cor14}(2)可证.
			\par
			(2)根据补充定理\ref{cor9}、补充定理\ref{cor16}(1)可证.
			\par
			(3)根据补充定理\ref{cor9}、补充定理\ref{cor16}(2)可证.

			\begin{cor}\label{cor17}
				\hfill\par
				(1)$(\exists x)(\exists y)(x\neq y)$;
				\par
				(2)$(\exists x)(\exists y)(\exists z)(x\neq y\text{与}y\neq z\text{与}x\neq z)$;
				\par
				(3)$(\exists x)(\exists y)(\exists z)(\exists t)(x\neq y\text{与}y\neq z\text{与}x\neq z\text{与}x\neq t\text{与}y\neq t\text{与}z\neq t)$.
			\end{cor}
			证明:根据补充定理\ref{cor16}、补充定理\ref{cor9},$\{\{\{\varnothing\}\}\}$、$\{\{\varnothing\}\}$、$\{\varnothing\}$、$\varnothing$互不相等,得证.

			\begin{cor}\label{cor18}
				\hfill\par
				(1)$(\forall X)(\varnothing\subset X)$.
				\par
				(2)$X\subset \varnothing\Rightarrow X=\varnothing$.
				\par
				(3)$X\subset \{x\}\Leftrightarrow (X=\{x\})\text{或}(X= \varnothing)$.
				\par
				(4)如果$(\forall y)(y\in X\Rightarrow y=x)$,则$(X=\{x\})\text{或}(X= \varnothing)$.
			\end{cor}
			证明:
			\par
			(1)	令$z$为不是常数的字母,根据补充定理\ref{cor15},$z\notin \varnothing$,因此$(z\in \varnothing)\Rightarrow (z\in y)$,故$(\forall z)((z\in \varnothing)\Rightarrow (z\in y))$,得证.
			\par
			(2)	假设$X\not\subset\varnothing$,则$(\exists x)(x\in X)$,设$y\in X$,则$y\in \varnothing$,和补充定理\ref{cor15}矛盾,得证.
			\par
			(3)	根据定理\ref{theo4}、补充定理\ref{cor18}(1),$(X=\{x\})\text{或}(X= \varnothing)\Rightarrow X\subset \{x\}$.反过来,假设$X\subset \{x\}\text{与}X\neq \varnothing$,根据补充定理\ref{cor6},$(\forall z)((z\in X)\Rightarrow (z=x))$.由于$X\neq \varnothing$,根据补充定理\ref{cor14}(2),$(\exists x)(x\in X)$.添加辅助常数$x$、$X$以及公理$x\in X$.根据补充定理\ref{cor8},$\{x\}\subset X$,根据显式公理\ref{ex1},$X=\{x\}$,根据证明规则\ref{C19}得证.
			\par
			(4)	根据补充定理\ref{cor6},$y=x\Leftrightarrow y\in \{x\}$,因此$(\forall y)(y\in X\Rightarrow y=x)\Leftrightarrow X\subset \{x\}$,根据补充定理\ref{cor18}(3)得证.

			\begin{exer}\label{exer36}
				\hfill\par
				求证:$(x=y)\Leftrightarrow (\forall X)((x\in X)\Leftrightarrow (y\in X))$.
			\end{exer}
			证明:
			\par
			根据公理模式\ref{Sch6}、证明规则\ref{C27},$(x=y)\Rightarrow (\forall X)((x\in X)\Leftrightarrow (y\in X))$.
			反过来,假设$(\forall X)((x\in X)\Leftrightarrow (y\in X))$,根据证明规则\ref{C31},$(\{x\}|X)((x\in X)\Rightarrow (y\in X))$,即$(x\in \{x\})\Rightarrow (y\in \{x\})$,因此$y\in \{x\}$,根据补充定理\ref{cor6},$x=y$.

			\begin{exer}\label{exer37}
				\hfill\par
				求证:
				\par
				(1)$\varnothing\neq \{x\}$.
				\par
				(2)$(\exists x)(\exists y)(x\neq y)$.
			\end{exer}
			证明:
			\par
			(1)	即补充定理\ref{cor16}(1).
			\par
			(2)	即补充定理\ref{cor17}(1).

			\begin{exer}\label{exer38}
				\hfill\par
					如果$A\subset X$、$B\subset X$,求证:$(B\subset \complement_XA)\Leftrightarrow (\complement_XB\subset A)$、$(A\subset \complement_XB)\Leftrightarrow (\complement_XA\subset B)$.
			\end{exer}
			证明:根据补充定理\ref{cor12}、补充定理\ref{cor13}可证.

			\begin{exer}\label{exer39}
				\hfill\par
				求证:$X\subset \{x\}\Leftrightarrow (X=\{x\})\text{或}(X= \varnothing)$.
			\end{exer}
			证明:即补充定理\ref{cor18}(3).

			\begin{exer}\label{exer40}
				\hfill\par
				求证:$\varnothing=\tau_X(\tau_x(x\in X)\notin X)$.
			\end{exer}
			证明:
			由于待证公式不包含$X$,故令$X$为不是常数的字母.
			\par
			$\tau_x(x\in X)\notin X即\text{非}(\tau_x(x\in X)\in X)$,即$\text{非}(\exists x)(x\in X)$,根据证明规则\ref{C29},$\tau_x(x\in X)\notin X\Leftrightarrow (\forall x)(x\notin X)$,根据证明规则\ref{C27},$(\forall X)(\tau_x(x\in X)\notin X\Leftrightarrow (\forall x)(x\notin X))$,根据公理模式\ref{Sch7},得证.			
			
			\begin{exer}\label{exer41}
				\hfill\par
				令$M$为包含特别符号$\in$的等式理论,并有显式公理1':$(\forall y)(y=\tau_x((\forall z)(z\in x\Leftrightarrow z\in y)))$,求证:显式公理\ref{ex1}是M的定理.
			\end{exer}
			证明:根据补充定理\ref{cor1}可证.			

		\section{有序对(Couples)}
			\begin{theo}\label{theo7}
				\hfill\par
				$\{\{x\},\{x, y\}\}=\{\{x'\},\{x', y'\}\}\Leftrightarrow (x=x'\text{与}y=y')$.
			\end{theo}
			证明:
			\par
			根据证明规则\ref{C44},$(x=x'\text{与}y=y')\Rightarrow \{\{x\},\{x, y\}\}=\{\{x'\},\{x', y'\}\}$.
			\par
			反过来,假设$\{\{x\},\{x, y\}\}=\{\{x'\},\{x', y'\}\}$,若$x\neq x'$,根据补充定理\ref{cor9},$\{x\}\neq \{x'\}$,则$\{x\}=\{x', y'\}$,根据证明规则\ref{C50},$(\forall x)((z=x)\Leftrightarrow (z=x')\text{或}(z=y))$,因此$x=x'$,矛盾.故$x=x'$,同理可证$y=y'$.

			\begin{sign}
				\textbf{有序对的记号(signe d'un couple),三元组的记号(signe d'un triplet),四元组的记号(signe d'un quadlet)}
				\par
				$\{\{x\}, \{x, y\}\}$记作$(x, y)$.$((x, y), z)$记作$(x, y, z)$.$((x, y, z), t)$记作$(x, y, z, t)$.
			\end{sign}

			\begin{de}
				\textbf{有序对(couple)}
				\par
				如果$(\exists x)(\exists y)(z=(x, y))$,则称$z$为有序对.
			\end{de}

			\begin{de}
				\textbf{三元组(triplet)}
				\par
				如果$(\exists x)(\exists y)(\exists z)(u=(x, y, z))$,则称$u$为三元组.
			\end{de}

			\begin{de}
				\textbf{四元组(triplet)}
				\par
				如果$(\exists x)(\exists y)(\exists z)(\exists t)(u=((x, y, z, t))$,则称$u$为三元组.
			\end{de}

			\begin{cor}\label{cor19}
				\hfill\par
				$(x, y)\text{为有序对}$.
			\end{cor}
			证明:根据定理\ref{theo1},$(x, y)=(x, y)$,根据公理模式\ref{Sch5},$(\exists u)(\exists v)((u, v)=(x, y))$,得证.

			\begin{cor}\label{cor20}
				\textbf{有序对的第一元素和第二元素存在且唯一}
				\par
				$(\exists y)(z=(x, y))$是$x$上的函数性公式,$(\exists x)(z=(x, y))$是$y$上的函数性公式.
			\end{cor}
			证明:根据定理\ref{theo7}可证.

			\begin{de}
				\textbf{有序对的第一元素(première coordonnée d'un couple/première projection d'un couple),有序对的第二元素(seconde coordonnée d'un couple/seconde projection d'un couple)}
				\par
				令$z$为有序对,则不包含x、y的项$\tau_x((\exists y)(z=(x, y)))$称为$z$的第一元素,记作$pr_1z$,不包含x、y的项$\tau_y((\exists x)(z=(x, y)))$称为$z$的第二元素,记作$pr_2z$.
			\end{de}

			\begin{sign}
				\textbf{用有序对表示字母(remplacer le lettre par couple)}
				\par
				如果某个字母$z$只能是有序对,则可以选择两个不会引起混淆的新字母,令第一个新字母是$x$,第二个新字母是$y$,此时,可用$x$表示$pr_1z$,用$y$表示$pr_2z$,在其他情况下用$(x, y)$表示$z$.此时,公式中的“$(x, y)\text{为有序对}\text{与}$”可以省略.
			\end{sign}
			注:
			\par
			使用该记号的例子,如:
			\par
			$\{z|(z\text{为有序对})\text{与}R\}$可以表示为$\{(x, y)|R\}$;
			\par
			$(X_{pr_1k}\cap Y_{pr_2k})_{(k\text{为有序对})\text{与}k \in K}$可以表示为$(X_i\cap Y_j)_{(i, j)\in K}$,
			\par
			在本文中,公式记号过于复杂的情况下,会使用这种表述方式,
			
			\begin{cor}\label{cor21}
				\hfill\par
				(1)$(pr_1z=x)\Leftrightarrow (\exists y)(z=(x, y))$;
				\par
				(2)$(pr_2z=y)\Leftrightarrow (\exists x)(z=(x, y))$.				
			\end{cor}
			证明:根据补充定理\ref{cor20}、证明规则\ref{C46}可证.

			\begin{cor}\label{cor22}
				\hfill\par
				(1)$(z=(x, y))\Leftrightarrow (z\text{为有序对})\text{与}(pr_1z=x)\text{与}(pr_2z=y)$.
				\par
				(2)$(z=(x, y))\text{与}x\in X\text{与}y\in Y\Leftrightarrow (z\text{为有序对})\text{与}(pr_1z\in X)\text{与}(pr_2z\in Y)$.
				\par
				(3)$(z=(pr_1z, pr_2z))\Leftrightarrow (z\text{为有序对})$.
				\par
				(4)$pr_1(x, y)=x$,$pr_2(x, y)=y$.
			\end{cor}
			证明:
			\par
			(1)	根据补充定理\ref{cor21}(1)、补充定理\ref{cor21}(2), $(z\text{为有序对})\text{与}(pr_1z=x)\text{与}(pr_2z=y))\Leftrightarrow\\(\exists x')(\exists y')(\exists x'')(\exists y'')(z=(x', y')\text{与}z=(x, y'')\text{与}z=(x'', y))$.
			\par
			根据定理\ref{theo7},$(z=(x', y')\text{与}z=(x, y'')\text{与}z=(x'', y))$等价于$(z=(x, y)\text{与}x=x'\text{与}x=x''\text{与}y=y'\text{与}y=y'')$,进而等价于$(z=(x, y))\text{与}(\exists x')(\exists y')(\exists x'')(\exists y'')(x=x'\text{与}x=x''\text{与}y=y'\text{与}y=y'')$,根据定理\ref{theo1},$(\exists x')(\exists y')(\exists x'')(\exists y'')(x=x'\text{与}x=x''\text{与}y=y'\text{与}y=y'')$,则\\$(z\text{为有序对})\text{与}(pr_1z=x)\text{与}(pr_2z=y)\Leftrightarrow (z=(x, y))$.
			\par
			(2)根据补充定理\ref{cor20},$(\exists y)(z=(x, y))$是$x$上的函数性公式,$(\exists x)(z=(x, y))$是$y$上的函数性公式,根据补充证明规则\ref{Ccor10},$(pr_1z=x)\text{与}(x\in X)\Leftrightarrow (pr_1z\in X)$、$(pr_2z=y)\text{与}(y\in Y)\Leftrightarrow (pr_2z\in Y)$.因此,$(z=(x, y))\text{与}x\in X\text{与}y\in Y\Leftrightarrow (z\text{为有序对})\text{与}(pr_1z=x)\text{与}(pr_2z=y)\text{与}x\in X\text{与}y\in Y$,进而等价于$(z\text{为有序对})\text{与}(pr_1z\in X)\text{与}(pr_2z\in Y)$,得证.
			\par
			(3)	根据补充定理\ref{cor22}(1),$(z=(pr_1z, pr_2z))\Leftrightarrow (z\text{为有序对})\text{与}(pr_1z=pr_1z)\text{与}(pr_2z=pr_2z)$,得证.
			\par
			(4)	根据补充定理\ref{cor22}(1)可证.

			\begin{Ccor}\label{Ccor15}
				\hfill\par
				令$R$为包含$2$元特别符号$\in$ 、显式公理\ref{ex1}、显式公理\ref{ex2}和公理模式\ref{Sch8}的等式理论$M$的公式,$x$、$y$为不同字母,$z$为与$x$、$y$不同的字母且$R$不包含$z$,则:
				\par
				(1)$(\exists x)(\exists y)(z=(x, y)\text{与}R)\Leftrightarrow (z\text{为有序对})\text{与}(pr_1z|x)(pr_2z|y)R$.
				\par
				(2)$(\exists x)(\exists y)R\Leftrightarrow (\exists z)((z\text{为有序对})\text{与}(pr_1z|x)(pr_2z|y)R)$.
				\par
				(3)$(\forall x)(\forall y)R\Leftrightarrow ((\forall z)((z\text{为有序对})\Rightarrow (pr_1z|x)(pr_2z|y)R))$.
			\end{Ccor}
			证明:
			\par
			(1)根据补充定理\ref{cor22}(1),$(\exists x)(\exists y)(z=(x, y)\text{与}R)\Leftrightarrow (\exists x)(\exists y)((z\text{为有序对})\text{与}(pr_1z=x)\text{与}(pr_2z=y)\text{与}R)$.
			\par
			根据证明规则\ref{C33},$(\exists x)(\exists y)(z=(x, y)\text{与}R)\Leftrightarrow (z\text{为有序对})\text{与}(\exists x)(\exists y)((pr_1z=x)\text{与}(pr_2z\\=y)\text{与}R)$.
			\par
			根据证明规则\ref{C47},$(pr_2z|y)R\Leftrightarrow (\exists y)((\exists x)(z=(x, y))\text{与}R)$,根据补充定理\ref{cor21}(2),\\$(pr_2z|y)R\Leftrightarrow (\exists y)(R\text{与}(pr_2z=y))$.
			\par
			根据替代规则\ref{CS9},$(pr_1z|x)(pr_2z|y)R\Leftrightarrow (\exists y) (pr_1z|x) (R\text{与}(pr_2z=y))$,即等价于\\$(\exists y)((pr_1z|x)R\text{与}(pr_2z=y))$,根据证明规则\ref{C47}、补充定理\ref{cor21}(1),$(pr_1z|x)(pr_2z|y)R\Leftrightarrow\\(\exists y)((\exists x)(R\text{与}(pr_1z=x))\text{与}(pr_2z=y))$,即等价于$(\exists x)(\exists y)((pr_1z=x)\text{与}(pr_2z=y)\text{与}R)$.
			\par
			综上,得证.
			\par
			(2)	根据补充证明规则\ref{Ccor15}(1),$(\exists z)((z\text{为有序对})\text{与}(pr_1z|x)(pr_2z|y)R)\Leftrightarrow (\exists z)(\exists x)(\exists y)\\(z=(x, y)\text{与}R)$,根据证明规则\ref{C33},其等价于$(\exists z)(\exists x)(\exists y)(z=(x, y))\text{与}(\exists x)(\exists y)R$.
			根据定理\ref{theo1},$(\exists z)(\exists x)(\exists y)(z=(x, y))$,因此其等价于$(\exists x)(\exists y)R$.
			\par
			(3)	令$R'$为$\text{非}R$,根据补充证明规则\ref{Ccor15}(2),$\text{非}(\exists x)(\exists y)R'\Leftrightarrow \text{非}(\exists z)((z\text{为有序对})\text{与}\\(pr_1z|x)(pr_2z|y)R')$,即$(\forall x)(\forall y)(\text{非}\text{非}R)\Leftrightarrow (\forall z)(\text{非}(z\text{为有序对})\text{或}(pr_1z|x)(pr_2z|y)\text{非}\text{非}R)$,\\因此$(\forall x)(\forall y)R\Leftrightarrow ((\forall z)((z\text{为有序对})\Rightarrow (pr_1z|x)(pr_2z|y)R))$.

			\begin{theo}\label{theo8}
				\textbf{两个集合的乘积的存在性}
				\par
				$(\forall X)(\forall Y)(\exists Z)(\forall z)((z\in Z)\Leftrightarrow (\exists x)(\exists y)(z=(x, y)\text{与}x\in X\text{与}y\in Y)$.
			\end{theo}
			证明:
			\par
			由于待证公式不含y,故令y为不是常数的字母.
			\par
			根据证明规则\ref{C53},$(\exists x)(z=(x, y)\text{与}x\in X)$在$z$上是集合化公式,令$A_y=\{z|(\exists x)(z=(x, y)\text{与}x\in X)\}$,则$(\forall z)((z\in A_y)\Leftrightarrow (\exists x)(z=(x, y)\text{与}x\in X))$,根据公理模式\ref{Sch5}、证明规则\ref{C27},$(\forall y)(\exists x)(\forall z)((z=(x, y)\text{与}x\in X)\Rightarrow (z\in A))$,根据公理模式\ref{Sch8},$(\exists y)((y\in Y)\text{与}(\exists x)(z=(x, y)\text{与}x\in X))$是$z$上的集合化公式,又因为$(\exists y)((y\in Y)\text{与}(\exists x)(z=(x, y)\text{与}\\x\in X))\Leftrightarrow (\exists x)(\exists y)((z=(x, y))\text{与}x\in X\text{与}y\in Y)$,因此$(\forall X)(\forall Y)(\exists Z)(\forall z)((z\in Z)\Leftrightarrow (\exists x)(\exists y)(z=(x, y)\text{与}x\in X\text{与}y\in Y)$.

			\begin{de}
				\textbf{两个集合的乘积(produit de deux ensembles)}
				\par
				$\{z|(\exists x)(\exists y)((z=(x, y))\text{与}x\in X\text{与}y\in Y))\}$称为$X$和$Y$的乘积,记作$X\times Y$.
			\end{de}
			
			\begin{theo}\label{theo9}
				\hfill\par
				如果$A'\neq \varnothing$、$B'\neq \varnothing$,则$(A'\times B')\subset (A\times B)\Leftrightarrow (A'\subset A)\text{与}(B'\subset B)$.
			\end{theo}
			证明:
			\par
			令$z$为不是常数的字母.
			\par
			$(A'\times B')\subset (A\times B)$等价于$(\forall z)(z\in (A'\times B')\Rightarrow z\in (A\times B))$,进而等价于$(\forall z)((\exists x)(\exists y)\\(z=(x, y)\text{与}x\in A'\text{与}y\in B')\Rightarrow (\exists x)(\exists y)(z=(x, y)\text{与}x\in A\text{与}y\in B))$.
			\par
			根据补充定理\ref{cor22}(2),$(A'\times B')\subset (A\times B)\Leftrightarrow ((z\text{为有序对})\text{与}pr_1z\in A'\text{与}pr_2z\in B'\Rightarrow (z\text{为有序对})\text{与}pr_1z\in A\text{与}pr_2z\in B))$.
			\par
			若$(A'\subset A)\text{与}(B'\subset B)$,则$pr_1z\in A'\Rightarrow pr_1z\in A$、$pr_2z\in B'\Rightarrow pr_2z\in B$,根据补充证明规则\ref{Ccor5}(3),$(A'\times B')\subset (A\times B)$.
			\par
			反过来,若$(A'\times B')\subset (A\times B)$,假设$x\in A'$,由于$B'\neq \varnothing$,故$(\exists y)(y\in B')$,即$\tau_y(y\in B')\in B'$,则$(x, \tau_y(y\in B))\in (A'\times B')$,因此$(x, \tau_y(y\in B))\in (A\times B)$,故$x\in A$,所以,$A'\subset A$,同理$B'\subset B$.

			\begin{cor}\label{cor23}
				\hfill\par
				(1)$z\in X\times Y\Leftrightarrow (z\text{为有序对})\text{与}(pr_1z\in X)\text{与}(pr_2z\in Y)$.
				\par
				(2)$(x, y)\in X\times Y\Leftrightarrow (x\in X)\text{与}(y\in Y)$.
				\par
				(3)$X\times Y=X'\times Y'\Leftrightarrow X=X'\text{与}Y=Y'$.
			\end{cor}
			证明:
			\par
			(1)	根据补充定理\ref{cor22}(2)可证.
			\par
			(2)	根据补充定理\ref{cor23}(1)、补充定理\ref{cor19}可证.
			\par
			(3)	根据定理\ref{theo9}可证.

			\begin{theo}\label{theo10}
				\hfill\par
				$A\times B=\varnothing\Leftrightarrow (A=\varnothing)\text{或}(B=\varnothing)$.
			\end{theo}
			证明:假设$A\times B\neq \varnothing$,根据补充定理\ref{cor14}(2),$(\exists x)(x\in (A\times B))$ ,即$\tau_x(x\in (A\times B))\in X\times Y\Rightarrow (pr_1(\tau_x(x\in (A\times B)))\in X)\text{与}(pr_2(\tau_x(x\in (A\times B)))\in Y)$,则$A\neq \varnothing$、$B\neq \varnothing$.
			\par
			反过来,若$A\neq \varnothing$、$B\neq \varnothing$,则$\tau_x(x\in A)\in A$、$\tau_y(y\in B)\in B$,故$(\tau_x(x\in A), \tau_y(y\in B))\in (A\times B)$,因此$A\times B\neq \varnothing$.得证.

			\begin{cor}
				\hfill\par
				(1)$(x , y, z)=(x', y', z')\Leftrightarrow x=x'\text{与}y=y'\text{与}z=z'$.
				\par
				(2)$(x , y, z, t)=(x', y', z', t')\Leftrightarrow x=x'\text{与}y=y'\text{与}z=z'\text{与}t=t'$.
			\end{cor}
			证明:根据定理\ref{theo7}可证.

			\begin{exer}\label{exer42}
				\hfill\par
				令$R$为公式,$x$、$y$为不同字母,$z$为与$x$、$y$不同的字母且$R$不包含$z$.求证:
				\par
				(1)$(\exists x)(\exists y)R\Leftrightarrow (\exists z)((z\text{为有序对})\text{与}(pr_1z|x)(pr_2z|y)R)$;
				\par
				(2)$(\forall x)(\forall y)R\Leftrightarrow ((\forall z)((z\text{为有序对})\Rightarrow (pr_1z|x)(pr_2z|y)R))$.
			\end{exer}
			证明:
			\par
			(1)即补充证明规则\ref{Ccor15}(2);
			\par
			(2)即补充证明规则\ref{Ccor15}(3).

		
		\section{对应(Correspondances)}
			\begin{de}
				\textbf{图(graphe)}
				\par
				如果$(\forall z)(z\in G\Rightarrow (z\text{为有序对}))$,则称$G$为图.
			\end{de}

			\begin{cor}\label{cor25}
				\hfill\par
				$G'\subset G$,$G$为图,则$G'$为图.
			\end{cor}
			证明:令$z$为不是常数的字母.由于$G$是图,故$(\forall z)(z\in G\Rightarrow (z\text{为有序对}))$.又因为$G'\subset G$,故$(\forall z)(z\in G'\Rightarrow z\in G)$,根据证明规则\ref{C27}、证明规则\ref{C30},$(\forall z)(z\in G'\Rightarrow (z\text{为有序对}))$,得证.

			\begin{cor}\label{cor26}
				\hfill\par
				$X\times Y$为图.
			\end{cor}
			证明:令$z$为不是常数的字母,根据补充定理\ref{cor23}(1),$z\in X\times Y\Leftrightarrow (z\text{为有序对})\text{与}\\(pr_1z\in X)\text{与}(pr_2z\in Y)$,因此,$z\in X\times Y\Rightarrow (z\text{为有序对})$,进而$(\forall z)(z\in X\times Y\Rightarrow (z\text{为有序对}))$,得证.

			\begin{de}
				\textbf{通过图对应(correspond par graphe)}
				\par
				令$G$为图,如果$(x, y)\in G$,则称$y$通过$G$对应$x$.
			\end{de}

			\begin{metadef}
				\textbf{生成图的公式(relation qui admet un graphe),公式的图(graphe de relation)}
				\par
				包含$2$元特别符号$\in$和显式公理\ref{ex1}、显式公理\ref{ex2}、公理模式\ref{Sch8}的等式理论$M$中,令$R$为公式,如果$(\exists G)((G\text{为图})\text{与}(\forall x)(\forall y)((x, y)\in G\Leftrightarrow R))$,则称$R$为对于$x$、$y$生成图$G$的公式.项\\$\tau_G((G\text{为图})\text{与}((x, y)\in G\Leftrightarrow R))$称为公式$R$关于$x$、$y$的图.
			\end{metadef}

			\begin{Ccor}\label{Ccor16}
				\hfill\par
				包含$2$元特别符号$\in$和显式公理\ref{ex1}、显式公理\ref{ex2}、公理模式\ref{Sch8}的等式理论$M$中,令$R$为对于$x$、$y$生成图$G$的公式,则:
				\par
				(1)$(x, y)\in G\Leftrightarrow R$.
				\par
				(2)如果$R$不包含$z$,则$z\in G\Leftrightarrow (z\text{为有序对})\text{与}(pr_1z|x)(pr_2z|y)R$.
			\end{Ccor}
			证明:
			\par
			(1)根据证明规则\ref{C21}可证.
			\par
			(2)根据补充证明规则\ref{Ccor16}(1),$(z\text{为有序对})\text{与}(pr_1z, pr_2z)\in G\Leftrightarrow (z\text{为有序对})\text{与}\\(pr_1z|x)(pr_2z|y)R$,根据补充定理\ref{cor22}(3),$(z=(pr_1z, pr_2z))\text{与}(pr_1z, pr_2z)\in G\Leftrightarrow \\(z\text{为有序对})\text{与}(pr_1z|x)(pr_2z|y)R$,根据证明规则\ref{C43},$(z=(pr_1z, pr_2z))\text{与}z\in G\Leftrightarrow \\(z\text{为有序对})\text{与}(pr_1z|x)(pr_2z|y)R$.
			\par
			由于$G$为图,因此$z\in G\Rightarrow (z\text{为有序对})$,进而$z\in G\Rightarrow z=(pr_1z, pr_2z)$,故$z\in G\Leftrightarrow (z\text{为有序对})\text{与}(pr_1z|x)(pr_2z|y)R$.

			\begin{cor}\label{cor27}
				\hfill\par
				$G_1$、$G_2$为图,则:
				\par
				(1)如果$(\forall x)(\forall y)((x, y)\in G_1\Leftrightarrow (x, y)\in G_2)$,则$G_1$=$G_2$.
				\par
				(2)如果$(\forall x)(\forall y)((x, y)\in G_1\Rightarrow (x, y)\in G_2)$,则$G_1\subset G_2$.
			\end{cor}
			证明:
			\par
			(1)令$z$为不是常数的字母.故$(pr_1z, pr_2z)\in G_1\Leftrightarrow (pr_1z, pr_2z)\in G_2$,根据补充定理\ref{cor22}(3),$(z=(pr_1z, pr_2z))\text{与}(pr_1z, pr_2z)\in G_1\Leftrightarrow (z=(pr_1z, pr_2z))\text{与}(pr_1z, pr_2z)\in G_2$,因此,$(z=(pr_1z, pr_2z))\text{与}z\in G_1\Leftrightarrow (z=(pr_1z, pr_2z))\text{与}z\in G_2$,故$z\in G_1\Leftrightarrow z\in G_2$,得证.
			\par
			(2)与补充定理\ref{cor27}(1)同理可证.

			\begin{Ccor}\label{Ccor17}
				\textbf{公式的图的唯一性}
				\par
				包含$2$元特别符号$\in$和显式公理\ref{ex1}、显式公理\ref{ex2}、公理模式\ref{Sch8}的等式理论$M$中,令$R$为关于$x$、$y$生成图的公式,则$R$关于$x$、$y$的图是唯一的.
			\end{Ccor}
			证明:设$G_1$、$G_2$均为$R$的图,$(x, y)\in G_1\Leftrightarrow R$,$(x, y)\in G_2\Leftrightarrow R$,因此$(x, y)\in G_1\Leftrightarrow (x, y)\in G_2$,根据补充定理\ref{cor27}(1)可证.

			\begin{Ccor}\label{Ccor18}
				\hfill\par
				包含$2$元特别符号$\in$和显式公理\ref{ex1}、显式公理\ref{ex2}、公理模式\ref{Sch8}的等式理论$M$中,$Z$为不同于$x$、$y$的字母,且$R$不包含$Z$,如果$(\exists Z)(\forall x)(\forall y)(R\Rightarrow (x, y)\in Z)$,则$R$为生成图的公式.
			\end{Ccor}
			证明:
			\par
			由于$(\exists Z)(\forall x)(\forall y)(R\Rightarrow (x, y)\in Z)$不包含$x$、$y$,故令$x$、$y$为不是常数的字母.
			\par
			添加辅助常数$Z$,令$(\forall x)(\forall y)(R\Rightarrow (x, y)\in Z)$为公理,则$R\Rightarrow (x, y)\in Z$.
			\par
			根据证明规则\ref{C51},$z\in Z\text{与}(z\text{为有序对})\text{与}(pr_1z|x)(pr_2z|y)R$是$z$上的集合化公式.令$G$为\\$\{z|z\in Z\text{与}(z\text{为有序对})\text{与}(pr_1z|x)(pr_2z|y)R\}$,则$z\in G\Leftrightarrow z\in Z\text{与}(z\text{为有序对})\text{与}\\(pr_1z|x)(pr_2z|y)R$,即$(z\in G\Rightarrow (z\text{为有序对}))$,因此,$G$为图.
			\par
			进而,$(x, y)\in G\Leftrightarrow (x, y)\in Z\text{与}((x, y)\text{为有序对})\text{与}(pr_1(x, y)|x)(pr_2(x, y)|y)R$.根据补充定理\ref{cor20},$(x, y)\text{为有序对}$,根据补充定理\ref{cor22}(4),$pr_1(x, y)=x$、$pr_2(x, y)=y$.因此,\\$(pr_1(x, y)|x)(pr_2(x, y)|y)R\Leftrightarrow R$,故$(x, y)\in G\Leftrightarrow (x, y)\in Z\text{与}R$,又因为$R\Rightarrow (x, y)\in Z$,因此$(x, y)\in G\Leftrightarrow R$.
			\par
			综上,根据证明规则\ref{C19},得证.

			\begin{Ccor}\label{Ccor19}
				\hfill\par
				包含$2$元特别符号$\in$和显式公理\ref{ex1}、显式公理\ref{ex2}、公理模式\ref{Sch8}的等式理论$M$中,$T$为不包含$x$、$y$的项,且字母$x$、$y$不是$M$的常数,如果$(R\Rightarrow (x, y)\in T)$,则$R$为生成图的公式.
			\end{Ccor}
			证明:由于$R\Rightarrow (x, y)\in T$且$x$、$y$不是常数,根据证明规则\ref{C27},因此$(\forall x)(\forall y)(R\Rightarrow (x, y)\in T)$,由于$T$不包含$x$、$y$,令$Z$为不同于$x$、$y$的字母且$R$不包含$Z$,根据替代规则\ref{CS9},$(\exists Z)(\forall x)(\forall y)(R\Rightarrow (x, y)\in Z)$,根据补充证明规则\ref{Ccor18}可证.

			\begin{Ccor}\label{Ccor20}
				\hfill\par
				包含$2$元特别符号$\in$和显式公理\ref{ex1}、显式公理\ref{ex2}、公理模式\ref{Sch8}的等式理论$M$中,$A$、$B$为项,$x$、$y$为不同的字母,且$A$、$B$均不包含$x$、$y$,则“$x\in A\text{与}y\in B$”为生成图的公式.
			\end{Ccor}
			证明:根据补充定理\ref{cor23},$(x\in A\text{与}y\in B)\Rightarrow (x, y)\in A\times B$,根据补充证明规则\ref{Ccor19}可证.
			
			\begin{theo}\label{theo11}
				\textbf{图的第一射影和第二射影存在且唯一}
				\par
				$G$为图,则有且仅有一个$A$满足$(\exists y)((x, y)\in G)\Leftrightarrow x\in A$,有且仅有一个$B$满足$(\exists x)\\((x, y)\in G)\Leftrightarrow y\in B$.
			\end{theo}
			证明:根据证明规则\ref{C53},$(\exists z)(x=pr_1z\text{与}z\in G)$为集合化公式.令$A=\{x|(\exists z)(x=pr_1z\text{与}(z\text{为有序对})\text{与}z\in G)\}$,根据补充证明规则\ref{Ccor11},$(\exists z)(x=pr_1z\text{与}(z\text{为有序对})\text{与}z\in G)\Leftrightarrow x\in A$.
			\par
			由于$G$为图,故$z\in G\Rightarrow (z\text{为有序对})$,因此$(z\text{为有序对})\text{与}z\in G\Leftrightarrow z\in G$.
			\par
			根据补充定理\ref{cor21}(1),$(\exists z)(x=pr_1z\text{与}z\in G)\Leftrightarrow (\exists z)((\exists y)(z=(x, y))\text{与}z\in G)$,后者即$(\exists z)(\exists y)(z=(x, y)\text{与}z\in G)$,即$(\exists y)(\exists z)(z=(x, y)\text{与}z\in G)$.
			\par
			根据补充证明规则\ref{Ccor10},$(\exists z)(z=(x, y)\text{与}z\in G)\Leftrightarrow (x, y)\in G$.
			综上,$x\in A\Leftrightarrow (\exists y)(x, y)\in G$.$A$的存在性得证.
			\par
			$A$的唯一性,根据显式公理\ref{ex1}可证.同理可证$B$的存在性和唯一性.

			\begin{de}
				\textbf{图的第一射影(première projection d'un graphe),图的定义域(ensemble de définition d'un graphe),图的第二射影(seconde projection d'un graphe),值域(ensemble des valeurs d'un graphe)}
				\par
				$G$为图,则$\{x|(\exists y)((x, y)\in G))\}$称为$G$的第一射影,或称为$G$的定义域,记作$pr_1G$;$\{y|\\(\exists x)((x, y)\in G))\}$称为$G$的第二射影,或称为$G$的值域,记作$pr_2G$.
			\end{de}

			\begin{Ccor}\label{Ccor21}
				\hfill\par
				包含$2$元特别符号$\in$和显式公理\ref{ex1}、显式公理\ref{ex2}、公理模式\ref{Sch8}的等式理论$M$中,$R$为不包含字母G的公式,如果$(\exists y)R$或者$(\exists x)R$是$M$的定理,则“$\text{非}(\exists G)((G\text{为图})\text{与}(\forall x)(\forall y)((x, y)\in G\Leftrightarrow R))$”是$M$的定理.
			\end{Ccor}
			证明:
			\par
			若$(\exists y)R$,假设$(\exists G)((G\text{为图})\text{与}(\forall x)(\forall y)((x, y)\in G\Leftrightarrow R))$,令$G$为该图,则$(\exists y)\\((x, y)\in G)\Leftrightarrow (\exists y)R$,因此$(\exists y)((x, y)\in G)$,根据定理\ref{theo11},$(\exists A)(\forall x)(x\in A)$,与补充定理\ref{cor10}矛盾.同理可证$(\exists x)R$为真的情况.

			\begin{cor}\label{cor28}
				\hfill\par				
				(1)$\text{非}(\exists G)((G\text{为图})\text{与}(\forall x)(\forall y)((x, y)\in G\Leftrightarrow x=y))$;
				\par
				(2)$\text{非}(\exists G)((G\text{为图})\text{与}(\forall x)(\forall y)((x, y)\in G\Leftrightarrow x\in y))$;
				\par
				(3)$\text{非}(\exists G)((G\text{为图})\text{与}(\forall x)(\forall y)((x, y)\in G\Leftrightarrow x\subset y))$;
				\par
				(4)$\text{非}(\exists G)((G\text{为图})\text{与}(\forall x)(\forall y)((x, y)\in G\Leftrightarrow x=\{y\}))$.
			\end{cor}
			证明:根据补充证明规则\ref{Ccor21}可证.

			\begin{cor}\label{cor29}
				\hfill\par
				$(x, y)\in G\Rightarrow x\in pr_1G$,$(x, y)\in G\Rightarrow y\in pr_2G$.
			\end{cor}
			证明:假设$(x, y)\in G$,则$(\exists y)((x, y)\in G)$,根据补充证明规则\ref{Ccor11},$x\in pr_1G$.
			\par
			同理可证$y\in pr_2G$.

			\begin{cor}\label{cor30}
				\hfill\par
				$G$为图,则$G\subset pr_1G\times pr_2G$.
			\end{cor}
			证明:令$x$、$y$为不是常数的字母.设$(x, y)\in G$.根据补充定理\ref{cor29},$x\in pr_1G$,$y\in pr_2G$.根据补充定理\ref{cor22}(1),$(x, y)\in pr_1G\times pr_2G$.即$(x, y)\in G\Rightarrow (x, y)\in pr_1G\times pr_2G$,根据补充定理\ref{cor27}(2)得证.				

			\begin{cor}\label{cor31}
				\hfill\par
				(1)$\varnothing$为图.
				\par
				(2)$G$为图,如果$pr_1G= \varnothing\text{或}pr_2G=\varnothing$,则$G=\varnothing$.
				\par
				(3)$pr_1\varnothing = \varnothing$,$pr_2\varnothing =\varnothing$.
				\par
				(4)	如果$X\neq \varnothing$,$Y\neq \varnothing$,则$pr_1(X\times Y)=X$,$pr_2(X\times Y)=X$.
			\end{cor}
			证明:
			\par
			(1)	根据补充定理\ref{cor15}可证.
			\par
			(2)	根据补充定理\ref{cor30},$G\subset \varnothing$,根据补充定理\ref{cor18}(2)得证.
			\par
			(3)	令$x$、$y$为不是常数的字母,根据补充定理\ref{cor15},$(x, y)\notin \varnothing$,因此“$\text{非}(\exists y)((x, y)\in \varnothing)$”,即$x\notin pr_1\varnothing$,根据补充定理\ref{cor14}(1)可证.
			\par
			(4)$\{x|(\exists y)((x, y)\in X\times Y))\}=\{x|(\exists y)(x\in X\text{与}y\in Y))\}$,等于$\{x|x\in X\text{与}(\exists y)(y\in Y)\}$,等于$\{x|x\in X\}$,得证.	
			
			\begin{cor}\label{cor32}
				\hfill\par
				$G_1$、$G_2$为图,且$G_1\subset G_2$,则p$r_1G_1\subset pr_1G_2$,$pr_2G_1\subset pr_2G_2$.
			\end{cor}
			证明:令$x$、$y$为不是常数的字母,由于$G_1\subset G_2$,因此$(x, y)\in G_1\Rightarrow (x, y)\in G_2$.
			\par
			根据证明规则\ref{C31},$(\exists y)((x, y)\in G_1\Rightarrow (\exists y)((x, y)\in G_2$,$(\exists x)((x, y)\in G_1\Rightarrow (\exists x)((x, y)\\\in G_2$,根据证明规则\ref{C50}可证.				

			\begin{de}
				\textbf{对应(correspondance),对应的图(graphe d'une correspondance)、出发域(ensemble de départ)、到达域(ensemble d'arrivée),通过一个对应对应\\(correspond par une correspondance)、对应的定义域(ensemble de définition d'une correspondance/domaine d'une correspondance)、对应的值域(ensemble des valeurs d'une correspondance/image d'une correspondance)}
				\par
				$G$为图,如果$pr_1G\subset A$且$pr_2G\subset B$,则称三元组$(G, A, B)$为$A$到$B$的对应,称$G$为\\$(G, A, B)$的图,$A$为$(G, A, B)$的出发域,$B$为$(G, A, B)$的到达域.如果$(x, y)\in G$,则称$y$通过$(G, A, B)$对应$x$,$pr_1G$称为$(G, A, B)$的定义域,$pr_2G$称为$(G, A, B)$的值域.
			\end{de}

			\begin{de}
				\textbf{元素到元素的公式(relation entre un élément et un élément)、公式定义的对应(correspondance définie par la relation)}
				\par
				如果$R$为对于$x$、$y$生成图$G$的公式,$A$、$B$满足$pr_1G\subset A$、$pr_2G\subset B$,则称$R$为对于$x$、$y$从$A$的元素到$B$的元素的公式,$(G, A, B)$称为$R$定义的对于$x$、$y$的$A$到$B$的对应.
			\end{de}
			
			\begin{cor}\label{cor33}
				\hfill\par
				$G$为图,则$(x\in X\text{与}(x, y)\in G)$为对于$x$、$y$生成图$G'$的公式,并且$pr_2G'$=$\{y|(\exists x) (x\in X\text{与}(x, y)\in G))$.
			\end{cor}
			证明:令$G'=\{z|pr_1z\in X\text{与}z\in G\}$,则$G'\subset G$,根据补充定理\ref{cor25},$G'$为图.根据补充定理\ref{cor22}(4),$(x, y)\in G'\Leftrightarrow (x\in X\text{与}(x, y)\in G)$,根据公理模式\ref{Sch5},$(x\in X\text{与}(x, y)\in G)$为对于$x$、$y$生成图的公式.又因为$(x, y)\in G'\Leftrightarrow (x\in X\text{与}(x, y)\in G)$,因此$pr_2G'=\{y|(\exists x) (x\in X\text{与}(x, y)\in G))$.
			
			\begin{de}
				\textbf{在图下的像(image par une image),在对应下的像(image par une image correspond)}
				\par
				$G$为图,则$\{y|(\exists x)(x\in X\text{与}(x, y)\in G)\}$称为$X$在$G$下的像,记作$G\langle X\rangle$.令$F$为对应,且$F$的图为$G$,则$X$在$G$下的像也称为$X$在$F$下的像.
			\end{de}
			
			\begin{cor}\label{cor34}
				\hfill\par
				$G$为图,则$G\langle X \rangle \subset pr_2G$.
			\end{cor}
			证明:由于$z\in G\langle X \rangle \Leftrightarrow (\exists x)(x\in X\text{与}(x, y)\in G)$,$z\in pr_2G\Leftrightarrow (\exists x)((x, y)\in G))$,根据证明规则\ref{C31},$z\in G\langle X \rangle \Rightarrow z\in pr_2G$,得证.
			
			\begin{cor}\label{cor35}
				\hfill\par
				$G$为图,则$G\langle pr_1G\rangle=pr_2G$.
			\end{cor}
			证明:根据公理模式\ref{Sch5},$(x, y)\in G\Rightarrow (\exists x)((x, y)\in G)$,因此,$(x\in \{y|(\exists x)((x, y)\in G))\}\text{与}(x, y)\in G)\Leftrightarrow (x, y)\in G$,得证.				
			
			\begin{cor}\label{cor36}
				\hfill\par
				$G$为图,则$G\langle\varnothing\rangle= \varnothing$.
			\end{cor}
			证明:根据补充定理\ref{cor15},$x\notin \varnothing$,则$\text{非}(x\in \varnothing\text{与}(x, y)\in G)$,因此$\text{非}(\exists x)(x\in \varnothing\text{与}(x, y)\in G)$,即$(\exists x)(x\in \varnothing\text{与}(x, y)\in G)\Leftrightarrow x\in \varnothing$,故$G\langle\varnothing\rangle=\varnothing$.

			\begin{theo}\label{theo12}
				\hfill\par
				$G$为图,则$X\subset Y\Rightarrow G\langle X \rangle \subset G\langle Y \rangle $.
			\end{theo}
			证明:根据证明规则\ref{C50}可证.

			\begin{theo}\label{theo13}
				\hfill\par
				$G$为图,$pr_1G\subset A$,则$G\langle A \rangle =pr_2G$.
			\end{theo}
			证明:根据定理\ref{theo12}、补充定理\ref{cor32}、补充定理\ref{cor34}、显式公理\ref{ex1}可证.

			\begin{de}
				\textbf{切割(coupe)}
				\par
				令$G$为图,$x$为字母,则称$G\langle\{x\}\rangle$为$G$对$x$的切割,也可以记作$G(x)$.令$F$为$A$到$B$的对应,其图为$G$,则$G\langle\{x\}\rangle$也称为$F$对$x$的切割,记作$F\langle\{x\}\rangle$或$F(x)$.
			\end{de}
					
			\begin{cor}\label{cor37}
				\hfill\par
				$G$为图,则$y\in G\langle \{x\} \rangle \Leftrightarrow (x, y)\in G$.			
			\end{cor}
			证明:$y\in G\langle \{x\} \rangle \Leftrightarrow (\exists x')(x'\in \{x\}\text{与}(x', y)\in G))$,即$y\in G\langle \{x\} \rangle \Leftrightarrow (\exists x')(x'=x\text{与}(x', y)\in G))$,根据证明规则\ref{C43},$y\in G\langle \{x\} \rangle \Leftrightarrow (\exists x')(x'=x\text{与}(x, y)\in G))$,因此,$y\in G\langle \{x\} \rangle \Leftrightarrow (\exists x')(x'=x)\text{与}(x, y)\in G)$.根据补充证明规则\ref{Ccor8},$(\exists x')(x'=x)$,因此,$y\in G\langle \{x\} \rangle \Leftrightarrow (x, y)\in G$.
			
			\begin{cor}\label{cor38}
				\hfill\par
				$G$、$G'$均为图,则:
				\par
				(1)$G\subset G'\Leftrightarrow (\forall x)(G\langle x \rangle \subset G'\langle x \rangle )$.
				\par
				(2)$G\subset G'\Rightarrow G\langle A \rangle \subset G'\langle A \rangle $.
			\end{cor}
			证明:
			\par
			(1)根据补充定理\ref{cor37}可证.
			\par
			(2)由于$G\subset G'$,因此$(x, y)\in G\Rightarrow (x, y)\in G'$,故$x\in A\text{与}(x, y)\in G\Rightarrow x\in A\text{与}(x, y)\in G'$,因此$(\exists x)(x\in A\text{与}(x, y)\in G)\Rightarrow (\exists x)(x\in A\text{与}(x, y)\in G')$,得证.			

			\begin{cor}\label{cor39}
				\textbf{逆图是图}
				\par
				(1)$(z\text{为有序对})\text{与}(pr_2z, pr_1z)\in G)$是$z$上的集合化公式.
				\par
				(2)$\{z|(z\text{为有序对})\text{与}(pr_2z, pr_1z)\in G)\}$为图.
			\end{cor}
			证明:
			\par
			(1)如果$(z\text{为有序对})\text{与}(pr_2z, pr_1z)\in G)$,根据补充定理\ref{cor29},$pr_2z\in pr_1G$,$pr_1z\in pr_2G$,故$z\in pr_2G\times pr_1G$,根据证明规则\ref{C52}可证.
			\par
			(2)根据定义可证.

			\begin{de}
				\textbf{逆图(graphe réciproque)}
				\par
				$G$为图,则$\{z|(z\text{为有序对})\text{与}(pr_2z, pr_1z)\in G)\}$称为$G$的逆图,记作$G^{-1}$.
			\end{de}

			\begin{cor}\label{cor40}
				\hfill\par
				$G$为图,则$(x, y)\in G\Leftrightarrow (y, x)\in G^{-1}$.				
			\end{cor}
			证明:根据定义,$(y, x)\in G^{-1}\Leftrightarrow ((y, x)\text{为有序对})\text{与}(x, y)\in G$,根据补充定理\ref{cor19},\\$(y, x)\text{为有序对}$,得证.
			
			\begin{cor}\label{cor41}
				\hfill\par
				$\varnothing^{-1}=\varnothing$.
			\end{cor}
			证明:令$z$为不是常数的字母,则$z\in \varnothing^{-1}\Leftrightarrow (z\text{为有序对})\text{与}(pr_2z, pr_1z)\in \varnothing$,$z\notin \varnothing^{-1}\Leftrightarrow (z不\text{为有序对})\text{或}(pr_2z, pr_1z)\notin \varnothing$,故$(\forall z)z\notin \varnothing^{-1}$,因此$\varnothing^{-1}=\varnothing$.			
			
			\begin{cor}\label{cor42}
				\hfill\par
				$G$为图,则$pr_1G^{-1}=pr_2G$,$pr_2G^{-1}=pr_1G$.
			\end{cor}
			证明:$pr_1G^{-1}$即$\{y|(\exists x)((x, y)\in G ^{-1}))\}$,因此,$pr_1G^{-1}=\{y|(\exists x)((y, x)\in G))\}$,即\\$pr_1G^{-1}=pr_2G$,同理可证$pr_2G^{-1}=pr_1G$.
			
			\begin{cor}\label{cor43}
				\hfill\par
				$(X\times Y) ^{-1}=Y\times X$.
			\end{cor}
			证明:令$z$为不是常数的字母,$z\in (X\times Y)^{-1}\Leftrightarrow \{z|(z\text{为有序对})\text{与}(pr_2z,pr_1z)\in (X\times Y)\}$.根据补充定理\ref{cor23}(1),$z\in (X\times Y)\Leftrightarrow (z\text{为有序对})\text{与}(pr_1z\in X)\text{与}(pr_2z\in Y)$,因此,$z\in (X\times Y)^{-1}\Leftrightarrow \{z|(z\text{为有序对})\text{与}pr_2z\in X\text{与}pr_1z\in Y\}$.根据补充定理\ref{cor23}(1),得证.
			
			\begin{cor}\label{cor44}
				\hfill\par
				$G$为图,则$(G^{-1})^{-1}=G$.
			\end{cor}
			证明:令$z$为不是常数的字母,$z\in (G^{-1})^{-1}\Leftrightarrow (z\text{为有序对})\text{与}(pr_2z,pr_1z)\in G^{-1})$.根据补充定理\ref{cor40},$(pr_2z,pr_1z)\in G^{-1}\Leftrightarrow (pr_1z,pr_2z)\in G$,根据补充定理\ref{cor22}(3),$z\in (G^{-1})^{-1}\Leftrightarrow (z=(pr_1z,pr_2z))\text{与}(pr_1z,pr_2z)\in G)$.根据证明规则\ref{C43},$z\in (G^{-1})^{-1}\Leftrightarrow (z=(pr_1z,pr_2z))\text{与}z\in G)$,根据补充定理\ref{cor22}(3),$z\in (G^{-1})^{-1}\Leftrightarrow (z\text{为有序对})\text{与}z\in G$,又因为$z\in G\Rightarrow (z\text{为有序对})$,因此,$z\in (G^{-1})^{-1}\Leftrightarrow z\in G$,得证.
			
			\begin{cor}\label{cor45}
				\hfill\par
				$G_1$、$G_2$为图,则:
				\par
				(1)$G_1\subset G_2\Leftrightarrow {G_1}^{-1}\subset {G_2}^{-1}$.
				\par
				(2)如果$G_1\subset G_2$,则$(G_2-G_1)^{-1}= {G_2}^{-1}-{G_1}^{-1}$.
			\end{cor}
			证明:
			\par
			(1)根据补充定理\ref{cor40}可证.
			\par
			(2)根据定义,$(x, y)\in (G_2-G_1)^{-1}\Leftrightarrow (y, x)\in G_2\text{与}(y, x)\notin G_1$,得证.

			\begin{de}
				\textbf{在图下的原像(image réciproque par une image)}
				\par
				$G$为图,$G^{-1}\langle X \rangle $称为$X$在$G$下的原像.
			\end{de}

			\begin{de}
				\textbf{对称图(graphe symétrique)}
				\par
				对于图$G$,如果$G=G^{-1}$,则称$G$是对称图.
			\end{de}
			
			\begin{cor}\label{cor46}
				\textbf{逆对应是对应}
				\par
				如果$(G, A, B)$为$A$到$B$的对应,则$(G^{-1}, B, A)$为$B$到$A$的对应.
			\end{cor}
			证明:根据定义和补充定理\ref{cor39},$G$、$G^{-1}$为图,$pr_1G\subset A且pr_2G\subset B$,根据补充定理\ref{cor42},$pr_1G^{-1}\subset B$,$pr_2G^{-1}\subset A$,因此$(G^{-1}, B, A)$为$B$到$A$的对应.

			\begin{de}
				\textbf{逆对应(correspondance réciproque),在对应下的原像(image réciproque par une corespondance)}
				\par
				令$F$为$A$到$B$的对应,且$F$的图为$G$,则$B$到$A$的对应$(G^{-1}, B, A)$,称为$F$的逆对应,记作$F^{-1}$.$X$在$G$下的原像,也称为$X$在$F$下的原像.
			\end{de}

			\begin{cor}\label{cor47}
				\textbf{图的复合是图}
				\par
				$G$、$G'$为图,则$(\exists y)((x, y)\in G\text{与}(y, z)\in G')\Rightarrow (x, z)\in pr_1G\times pr_2G'$,且$(\exists y)((x, y)\in G\text{与}(y, z)\in G')$为生成图的公式.
			\end{cor}
			证明:根据补充定理\ref{cor28},$(\exists y)((x, y)\in G\text{与}(y, z)\in G')\Rightarrow (x\in pr_1G\text{与}z\in pr_2G')$,根据补充定理\ref{cor23}(1),$(\exists y)((x, y)\in G\text{与}(y, z)\in G')\Rightarrow (x, z)\in pr_1G\times pr_2G'$.根据补充证明规则\ref{Ccor18},可知其为生成图的公式.

			\begin{de}
				\textbf{图的复合(composée de deux graphes)}
				\par
				令$G$、$G'$为图,则公式$(\exists y)((x, y)\in G\text{与}(y, z)\in G')$关于$x$、$z$的图,称为$G$和$G'$的复合,记作$G'\circ G$或者$G'G$.
			\end{de}

			\begin{theo}\label{theo14}
				\hfill\par
				$G$、$G'$为图,则$(G'\circ G)^{-1}=G^{-1}\circ {G'}^{-1}$.
			\end{theo}
			证明:令$z$、$x$为不是常数的字母,根据补充定理\ref{cor39},$(x, y)\in G\text{与}(y, z)\in G'\Leftrightarrow (z, y)\in {G'}^{-1}\text{与}(y, x)\in {G'}^{-1}$,即$(\exists y)((x, y)\in G\text{与}(y, z)\in G')\Leftrightarrow (\exists y)((z, y)\in {G'}^{-1}\text{与}(y, x)\in {G'}^{-1})$,根据补充证明规则\ref{Ccor16}(1),$(x, z)\in G'\circ G\Leftrightarrow (z, x)\in G^{-1}\circ {G'}^{-1}$,根据补充定理\ref{cor39},$(z, x)\in (G'\circ G)^{-1}\Leftrightarrow (z, x)\in G^{-1}\circ {G'}^{-1}$,根据补充定理\ref{cor27}(1)可证.

			\begin{theo}\label{theo15}
				\textbf{图的复合的结合律}
				\par
				$G_1$、$G_2$、$G_3$为图,则$(G_3\circ G_2)\circ G_1=G_3\circ (G_2\circ G_1)$.
			\end{theo}
			证明:$(x, t)\in (G_3\circ G_2)\circ G_1\Leftrightarrow (\exists y)((x, y)\in G_1\text{与}(\exists z)((y, z)\in G_2\text{与}(z, t)\in G_3))$,根据证明规则\ref{C33},$(x, t)\in (G_3\circ G_2)\circ G_1\Leftrightarrow (\exists y)(\exists z)((x, y)\in G_1\text{与}(y, z)\in G_2\text{与}(z, t)\in G_3)$.同理可得,$(x, t)\in G_3\circ (G_2\circ G_1)\Leftrightarrow (\exists z)(\exists y)((x, y)\in G_1\text{与}(y, z)\in G_2\text{与}(z, t)\in G_3)$.根据补充定理\ref{cor27}(1)可证.			

			\begin{theo}\label{theo16}
				\hfill\par
				$G$、$G'$为图,则$G'\circ G\langle A \rangle =G'\langle G\langle A \rangle\rangle$.
			\end{theo}
			证明:令$z$为不是常数的字母,根据证明规则\ref{C33},$z\in G'\circ G\langle A \rangle \Leftrightarrow (\exists y)(\exists x) (x\in A\text{与}\\(x, y)\in G\text{与}(y, z)\in G')$,故$z\in G\circ G'\langle A \rangle \Leftrightarrow (\exists y)(y\in G\langle A \rangle \text{与}(y, z)\in G')$,即$z\in G\circ G'\langle A \rangle \Leftrightarrow z\in G'\langle G\langle A \rangle\rangle$.得证.			

			\begin{cor}\label{cor48}
				\hfill\par
				$G$、$G'$为图,则$pr_1(G'\circ G)=G^{-1}\langle pr_1G'\rangle$.
			\end{cor}
			证明:令$x$为不是常数的字母,则$x\in pr_1(G\circ G')\Leftrightarrow (\exists z)((x, z)\in G'\circ G))$,即$(\exists z)(\exists y)\\((x, y)\in G\text{与}(y, z)\in G')$.同时,$x\in G^{-1}\langle pr_1G'\rangle\Leftrightarrow (\exists y)(y\in pr_1G'\text{与}(y, x)\in G^{-1})$,即$x\in G^{-1}\langle pr_1G'\rangle\Leftrightarrow (\exists y)((\exists z)((y, z)\in G')\text{与}(y, x)\in G^{-1})$,因此,$x\in G^{-1}\langle pr_1G'\rangle\Leftrightarrow (\exists z)(\exists y)((x, y)\in G\text{与}(y, z)\in G')$,得证.			
			
			\begin{cor}\label{cor49}
				\hfill\par
				$G$、$G'$为图,则$pr_2(G'\circ G)=G'\langle pr_2G\rangle$.
			\end{cor}
			证明:令$x$为不是常数的字母,则$x\in pr_2(G'\circ G)\Leftrightarrow (\exists z)((z, x)\in G\circ G'))$,即$(\exists z)(\exists y)\\((z, y)\in G'\text{与}(y, x)\in G)$.同时,$x\in G'\langle pr_2G\rangle\Leftrightarrow (\exists y)(y\in pr_2G'\text{与}(y, x)\in G)$,即$x\in G'\langle pr_2G\rangle\Leftrightarrow (\exists y)(\exists z)((z, y)\in G'\text{与}(y, x)\in G)$,得证.			
			
			\begin{cor}\label{cor50}
				\hfill\par
				$G$为图,则$X\subset pr_1G\Leftrightarrow X\subset G^{-1}\langle G\langle X \rangle \rangle$.
			\end{cor}
			证明:
			假设$X\subset pr_1G$,令$x$为不是常数的字母,则$x\in X\Rightarrow (\exists y)((x, y)\in G)$,因此$x\in X\Rightarrow (\exists y)( x\in X\text{与}(x, y)\in G)$,根据公理模式\ref{Sch5},$x\in X\text{与}(x, y)\in G\Rightarrow (\exists z)(z\in X\text{与} (z, y)\in G)$,故$x\in X\Rightarrow (\exists z)(z\in X\text{与} (z, y)\in G)$.
			\par
			同时,根据补充定理\ref{cor40},$(x, y)\in G\Leftrightarrow (y, x)\in G^{-1}$,则$x\in X\Rightarrow (\exists y)((y, x)\in G^{-1})$.因此,$x\in X\Rightarrow (\exists y)((\exists z)(z\in X\text{与} (z, y)\in G)\text{与}(y, x)\in G^{-1})$.进而,$X\subset pr_1G\Rightarrow X\subset G^{-1}\langle G\langle X \rangle \rangle$.
			反过来,根据补充定理\ref{cor34},$G^{-1}\langle G\langle X \rangle \rangle \subset pr_2G^{-1}$,根据补充定理\ref{cor42},$G^{-1}\langle G\langle X \rangle \rangle \subset pr_1G$.故$X\subset G^{-1}\langle G\langle X \rangle \rangle \Rightarrow X\subset pr_1G$.
			综上,得证.
			
			\begin{cor}\label{cor51}
				\hfill\par
				(1)$G_1$、$G_2$、$G_1'$、$G_2'$为图,$G_1\subset G_2$,$G_1'\subset G_2'$,则$G_1'\circ G_1\subset G_2'\circ G_2$.
				\par
				(2)$G$、$G'$为图,则$pr_1(G'\circ G)\subset pr_1G$,$pr_2(G'\circ G)\subset pr_2G'$.
			\end{cor}
			证明:
			\par
			(1)由于$G_1\subset G_2$,$G_1'\subset G_2'$,则$(x, y)\in G_1\Rightarrow (x, y)\in G_2$,$(y, z)\in G_1'\Rightarrow (y, z)\in G_2'$,因此$(\exists y)((x, y)\in G_1\text{与}(y, z)\in G_1')\Rightarrow (\exists y)((x, y)\in G_2\text{与}(y, z)\in G_2')$,根据证明规则\ref{C50}得证.
			\par
			(2)$(\exists z)(\exists y)((x, y)\in G\text{与}(y, z)\in G')\Leftrightarrow (\exists y)((x, y)\in G)\text{与}(\exists z)(\exists y)(y, z)\in G'$.因此$(\exists z)(\exists y)((x, y)\in G\text{与}(y, z)\in G')\Rightarrow (\exists y)((x, y)\in G)$.根据证明规则\ref{C50},$pr_1(G'\circ G) \subset pr_1G$得证.同理可证$pr_2(G'\circ G) \subset pr_2G'$.
			
			\begin{cor}\label{cor52}
				\hfill\par
				$G$为图,则:
				\par
				(1)$G\circ \varnothing=\varnothing$,$\varnothing\circ G=\varnothing$.
				\par
				(2)当且仅当$G= \varnothing$时,$G^{-1}\circ G=\varnothing$.
			\end{cor}
			证明:
			\par
			(1)令$y$为不是常数的字母,则$(x, z)\in G\circ \varnothing\Leftrightarrow (\exists y)((x, y)\in \varnothing\text{与}(y, z)\in G)$,$\text{非}(x, y)\in \varnothing$,因此$\text{非}(x, z)\in G\circ \varnothing$,根据补充定理\ref{cor14}(1),$G\circ \varnothing=\varnothing$.同理,$\varnothing\circ G=\varnothing$.
			\par
			(2)$G= \varnothing$时,根据补充定理\ref{cor52}(1),$G^{-1}\circ G=\varnothing$.反过来,如果$G^{-1}\circ G=\varnothing$,令$x$、$y$为不是常数的字母,则$(x, x)\in \varnothing \Leftrightarrow (\exists y)((x, y)\in G\text{与}(x, y)\in G)))$.假设$z\in G$,则$z\text{为有序对}$,故$(pr_1z, pr_2z) \notin G$,矛盾,因此$z\notin G$,故$G=\varnothing$.
			\begin{cor}\label{cor53}
				\textbf{对应的复合是对应}
				\par
				令$F$为$A$到$B$的对应,其图为$G$,$F'$为$B$到$C$的对应,其图为$G'$,则$(G'\circ G, A, C)$为$A$到$C$\\的对应.
			\end{cor}
			证明:根据补充定理\ref{cor51}(2),$pr_1(G'\circ G)\subset pr_1G$,$pr_2(G'\circ G)\subset pr_2G'$.而$pr_1G\subset A$、$pr_2G'\subset C$,因此$pr_1(G'\circ G)\subset A$、$pr_2(G'\circ G)\subset C$,得证.

			\begin{de}
				\textbf{对应的复合(composée de deux correspondances)}
				\par
				令$F$为$A$到$B$的对应,其图为$G$,$F'$为$B$到$C$的对应,其图为$G'$,则称$(G'\circ G, A, C)$为$F$和\\$F'$的复合,记作$F'\circ F$或者$F'F$.
			\end{de}
		
			\begin{cor}\label{cor54}
				\hfill\par
				$F$、$F'$为对应,则:
				\par
				(1)$F'\circ F\langle X \rangle =F'\langle F\langle X \rangle \rangle$.
				\par
				(2)$(F'\circ F)^{-1}=F^{-1}\circ {F'}^{-1}$.
			\end{cor}
			证明:
			\par
			(1)根据定理\ref{theo16}可证.
			\par
			(2)根据定理\ref{theo14}可证.
			
			\begin{cor}\label{cor55}
				\textbf{对角集合的存在性}
				\par
				$(z\text{为有序对})\text{与}pr_1z=pr_2z\text{与}pr_1z\in A$是$z$上的集合化公式.
			\end{cor}
			证明:如果$(z\text{为有序对})\text{与}pr_1z=pr_2z\text{与}pr_1z\in A$,根据补充定理\ref{cor29},$pr_2z\in A$,$pr_1z\in A$,故$z\in A\times A$,根据证明规则\ref{C52}可证.
			
			\begin{de}
				\textbf{对角集合(ensemble de la diagonale)}
				\par
				$\{z|(z\text{为有序对})\text{与}pr_1z=pr_2z\text{与}pr_1z\in A \}$称为$A\times A$的对角集合,记作$\Delta_A$.
			\end{de}
			
			\begin{cor}\label{cor56}
				\hfill\par
				$\Delta_A\subset A\times A$.
			\end{cor}
			证明:令$z$为不是常数的字母.根据补充定理\ref{cor23}(1),$z\in A\times A\Leftrightarrow (z\text{为有序对})\text{与}(pr_1z\\\in A)\text{与}(pr_2z\in A)$.而$(z\text{为有序对})\text{与}pr_1z=pr_2z\text{与}pr_1z\in A\Rightarrow \text{与}(pr_1z\in A)\text{与}(pr_2z\in A)$,得证.
			
			\begin{cor}\label{cor57}
				\hfill\par
				$\Delta_A$为图.
			\end{cor}
			证明:根据定义可证.
			
			\begin{cor}\label{cor58}
				\hfill\par
				$pr_1\Delta_A=A$,$pr_2\Delta_A=A$.
			\end{cor}
			证明:令$x$为不是常数的字母,则$x\in pr_1\Delta_A\Leftrightarrow (\exists y)((x, y)\in \Delta_A)$,即$x\in pr_1\Delta_A\Leftrightarrow (\exists y)(x=y\text{与}x\in A)$,因此$x\in pr_1\Delta_A\Leftrightarrow x\in A$,前半部分得证.同理可证后半部分.
			
			\begin{cor}\label{cor59}
				\hfill\par
				$(x, y)\in \Delta_A\Leftrightarrow (x=y)\text{与}(x\in A)$,$(x, x)\in \Delta_A\Leftrightarrow (x\in A)$.
			\end{cor}
			证明:根据补充定理\ref{cor19}、补充定理\ref{cor22}(4)可证.
			
			\begin{cor}\label{cor60}
				\hfill\par
				$G$为图,则:
				\par
				(1)如果$pr_1G\subset A$,则$G\circ \Delta_A=G$.
				\par
				(2)如果$pr_2G\subset B$,则$\Delta_B\circ G=G$.
			\end{cor}
			证明:
			\par
			(1)	令$x$、$z$为不是常数的字母,$(\exists y)((x, y)\in \Delta_A\text{与}(y, z)\in G)\Leftrightarrow (\exists y)(x\in A\text{与}x=y\text{与}(y, z)\in G)$,因此$(\exists y)((x, y)\in \Delta_A\text{与}(y, z)\in G)\Leftrightarrow x\in A\text{与}(\exists y)(x=y)\text{与}(x, z)\in G$.根据补充定理\ref{cor29},$(x, z)\in G\Rightarrow x\in pr_1G$,又因为$pr_1G\subset A$,因此$(x, z)\in G\Rightarrow x\in A$.而根据补充证明规则\ref{Ccor8},$(\exists y)(x=y)$.因此$(\exists y)((x, y)\in \Delta_A\text{与}(y, z)\in G)\Leftrightarrow (x, z)\in G$,根据证明规则\ref{C50}可证.
			\par
			(2)与补充定理\ref{cor60}(1)同理可证.
			
			\begin{cor}\label{cor61}
				\hfill\par
				${\Delta_A}^{-1}=\Delta_A$
			\end{cor}
			证明:令$z$为不是常数的字母,根据证明规则\ref{C44},$(z\text{为有序对})\text{与}pr_1z=pr_2z\text{与}pr_1z\in A\Leftrightarrow (z\text{为有序对})\text{与}pr_1z=pr_2z\text{与}pr_2z\in A$,根据证明规则\ref{C50}得证.

			\begin{de}
				\textbf{恒等对应(correspondance identique)}
				\par
				对应$(\Delta_A, A, A)$称为恒等对应,记作$Id_A$.
			\end{de}
			
			\begin{cor}\label{cor62}
				\hfill\par
				如果$F$为$A$到$B$的对应,则$F\circ Id_A=F$,$Id_B\circ F=F$.
			\end{cor}
			证明:根据补充定理\ref{cor60}(1)、补充定理\ref{cor60}(2)可证.

			\begin{de}
				\textbf{函数图(graphe fonctionnel),函数(fonction),函数的定义域(ensemble de définition d'une fonction)、函数的值域(ensemble des valeurs d'une fonction),映射(application),函数的值(valeur de fonction)}
				\par
				如果图F满足$(\forall x)(\forall y)(\forall z)((x, y)\in F\text{与}(x, z)\in F\Rightarrow (y=z))$,则称$F$为函数图.如果$F$为函数图,且$pr_2F\subset B$,则对应$(F, pr_1F, B)$称为函数,$pr_1F$称为函数$(F, pr_1F, B)$的定义域;$pr_2F$称为函数$(F, pr_1F, B)$的值域.函数$(F, A, B)$也可称为$A$到$B$的映射.如果$f$为函数,其图为$F$,且$x\in pr_1F$,则$\tau_y((x, y)\in F)$称为函数$f$对于$x$的值,记作$f(x)$、$f_x$、$F(x)$或\\$F_x$.
			\end{de}
			注:如果$x$不属于函数$f$的定义域,根据定义可知,$f(x)$表示论域的第一个对象,该对象除了与自身相等外,不具有任何其他性质.
			
			\begin{de}
				\textbf{族(famille),集族(famille d'ensembles),元素族(famille d'éléments),指标集(ensemble des indices),双族(famille double),子集族(famille de parties),空族(famille vide),非空族(famille non vide)}
				\par
				函数$(X, I, E)$的图也可以称为族或集族,或称为$E$的元素族,令i为不出现在$X$、$I$、
				$E$的任何一个字母,则族可以记作$(X_i)_{i\in I}(X_i\in E)$,在没有歧义的情况下也可以简记为\\$(X_i)_{i\in I}$,当$I$为$\{i|R\}$时,也可以记作$(X_i)R$.此时,$I$称为族的指标集.
				\par
				如果$I$可以表示为$A\times B$的形式,则称其为双族.
				\par
				如果族满足$(\forall x)(x\in E\Rightarrow x\subset F)$,则称其为$F$的子集族.			
				\par
				如果族的指标集为空集,则称其为空族.如果族的指标集非空,则称其为非空族.
			\end{de}
			注:在本文中,“非空集族”只表示该族的值域各元素均非空集;“非空族”则表示该族的指标集为空,以避免歧义.
			
			\begin{cor}\label{cor63}
				\hfill\par
				$F$为函数图,$x\in pr_1F$,则$(x, y)\in F$为$y$上的函数性公式.
			\end{cor}
			证明:根据定义可证.
			
			\begin{cor}\label{cor64}
				\textbf{函数的值的基本性质}
				\par
				(1)$f$为函数,其图为$F$,则$(x\in pr_1F)\text{与}(y=f(x))\Leftrightarrow (x, y)\in F$.
				\par
				(2)$x\in pr_1F\Leftrightarrow (x, f(x))\in F$.
				\par
				(3)$x\in pr_1F\Rightarrow f(x)\in pr_2F$.
			\end{cor}
			证明:
			\par
			(1)如果$(x\in pr_1F)$,根据证明规则\ref{C46}、补充定理\ref{cor63},$(y=f(x))\Leftrightarrow (x, y)\in F$.根据补充定理\ref{cor29},$(x, y)\in F\Rightarrow (x\in pr_1F)$.得证.
			\par
			(2)根据补充定理\ref{cor64}(1)可证.
			\par
			(3)根据补充定理\ref{cor64}(2)、补充定理\ref{cor29}可证.
			
			\begin{cor}\label{cor65}
				\hfill\par
				$F$为函数图,$(\forall x)(x\in pr_1F\Rightarrow f(x)\in A)$,则$pr_2F\subset A$.
			\end{cor}
			证明:令$x$为不是常数的字母,由于$x\in pr_1F\Rightarrow f(x)\in A$,根据补充定理\ref{cor64}(1),$(x, y)\in F\Rightarrow (y=f(x))\text{与}(f(x)\in A)$,其等价于$(y=f(x))\text{与}(y\in A)$,因此$(\exists x)((x, y)\in F)\Rightarrow y\in A$,得证.
			
			\begin{cor}\label{cor66}
				\hfill\par
				(1)$\varnothing$为函数图,$(\varnothing, \varnothing, A)$为函数.
				\par
				(2)$Id_A$为函数.
				\par
				(3)函数$f$为$(F, A, pr_2F)$,且$x\in A\Leftrightarrow f(x)=x$,则$f=Id_A$.
				\par
				(4)如果$X\subset A$,则$Id_A\langle X \rangle =X$.
			\end{cor}
			证明:
			\par
			(1)根据补充定理\ref{cor31}(1),$\varnothing$为图;根据补充定理\ref{cor31}(3),$pr_1\varnothing= \varnothing$,$pr_2\varnothing= \varnothing$,则$pr_2\varnothing\subset A$;令$x$、$y$、$z$为不是常数的字母,根据补充定理\ref{cor15},$(x, y)\in \varnothing$为假、$(x, z)\in \varnothing$为假,故$(x, y)\in F\text{与}(x, z)\in F\Rightarrow (y=z)$,因此$(\varnothing, \varnothing, A)$为函数.
			\par
			(2)根据补充定理\ref{cor57},$\Delta_A$为图.根据补充定理\ref{cor59},$pr_1\Delta_A=A$,$pr_2\Delta_A=A$.根据补充定理\ref{cor59},$(x, y)\in \Delta_A\Rightarrow x=y$,$(x, z)\in \Delta_A\Rightarrow x=z$,得证.
			\par
			(3)令$f$的图为$F$,则$(x, y)\in F\Leftrightarrow (x\in A\text{与}f(x)=y)$,又因为$x\in A\Leftrightarrow f(x)=x$,则$(x, y)\in F\Leftrightarrow (x\in A\text{与}x=y)$,根据补充定理\ref{cor59}、补充证明规则\ref{Ccor17},$F=\Delta_A$,根据补充定理\ref{cor58},$pr_2F=A$,得证.
			\par
			(4)$Id_A\langle X \rangle =\{y|(\exists x)(x\in X\text{与}(x, y)\in \Delta_A)\}$,等于$\{y|(\exists x)(x\in X\text{与}y\in A\text{与}x=y)\}$,等于$\{y|y\in X\}$,等于$X$,得证.			

			\begin{de}
				\textbf{恒等映射(application identique),恒等函数(fonction identique)}
				\par
				恒等对应又称恒等映射或恒等函数.
			\end{de}
			
			\begin{cor}\label{cor67}
				\hfill\par
				$f$、$g$为函数,其图分别为$F$、$G$,且$F\subset G$,则$(f\text{的定义域})\subset (g\text{的定义域})$.
			\end{cor}
			证明:根据补充定理\ref{cor32}可证.
			
			\begin{cor}\label{cor68}
				\hfill\par
				$f$、$g$为函数,其图分别为$F$、$G$,且$pr_1F=pr_1G$,$(\forall x)(x\in pr_1F\Rightarrow f(x)=g(x))$,则$F$=$G$.如果$(f\text{的到达域})=(g\text{的到达域})$,则$f=g$.
			\end{cor}
			证明:根据补充定理\ref{cor64}(1),$(x\in pr_1F)\text{与}(y=f(x))\Leftrightarrow (x, y)\in F$,$(x\in pr_1G)\text{与}(y=f(x))\Leftrightarrow (x, y)\in G$.$(x, y)\in F\Rightarrow (x\in pr_1F)\text{与}y=g(x)$,因此$(x, y)\in F\Rightarrow (x, y)\in G$,同理可证$(x, y)\in G\Rightarrow (x, y)\in F$,根据补充定理\ref{cor27}(1),$F=G$.进而,如果$(f\text{的到达域})=(g\text{的到达域})$,根据定义,$f=g$.
			
			\begin{cor}\label{cor69}
				\hfill\par
				$G$为图,则$G\text{为函数图}\Leftrightarrow (\forall X)(G\langle G^{-1}\langle X \rangle \rangle\subset X)$.
			\end{cor}
			证明:$(\forall X)(G\langle G^{-1}\langle X \rangle \rangle\subset X) \Leftrightarrow (\forall X)(\exists x)(\exists y)(x\in X\text{与}(y, x) \in G\text{与}(y, z)\in G\Rightarrow z\in X)$.如果$G$为函数图,则$(y, x)\in G\text{与}(y, z)\in G\Rightarrow z=x$,故$x\in X\text{与}(y, x)\in G\text{与}(y, z)\in G\Rightarrow z\in X$,因此$(\forall X)(G\langle G^{-1}\langle X \rangle \rangle\subset X)$.反过来,如果$(\forall X)(G\langle G^{-1}\langle X \rangle \rangle\subset X)$,当$(y, x)\in G\text{与}(y, z)\in G$时,$(\exists x)(x\in \{x\})\Rightarrow z\in \{x\}$,故$z=x$,因此$G$为函数图,得证.
			
			\begin{cor}\label{cor70}
				\hfill\par
				$f$为$A$到$B$的映射,则:
				\par
				(1)$x\in f^{-1}\langle X \rangle \Leftrightarrow f(x)\in X\text{与}x\in A$.
				\par
				(2)$f^{-1}\langle B \rangle = A$.
				\par
				(3)$x\in f^{-1}(y)\Leftrightarrow f(x)=y\text{与}x\in A$.
				\par
				(4)$x\in f^{-1}(f(x))\Leftrightarrow x\in A$.
				\par
				(5)$X\subset A$,则$x\in X\Rightarrow f(x)\in f\langle X \rangle $.
				\par
				(6)$x\in A\Rightarrow \{f(x)\}=f\langle \{x\} \rangle $.
				\par
				(7)$X\subset A\Rightarrow X\subset f^{-1}\langle f\langle X \rangle \rangle$.
				\par
				(8)$(X\subset(f\text{的值域}))\Rightarrow (X=f\langle f^{-1}\langle X \rangle \rangle)$.
			\end{cor}
			证明:
			\par
			(1)令$f$的图为$F$,则$x\in f^{-1}\langle X \rangle \Leftrightarrow (\exists y)(y\in X\text{与}(x, y) \in F)$,根据补充定理\ref{cor64}(1),等价于$(\exists y)(y\in X\text{与}(x\in A)\text{与}(y=f(x)))$,等价于$(\exists y)(y=f(x))\text{与}x\in A\text{与}f(x) \in X$.$x\in A \Rightarrow (\exists y)((x, y)\in F)$,根据补充定理\ref{cor64}(1),$x\in A \Rightarrow (\exists y)(y=f(x))$,因此$x\in f^{-1}\langle X \rangle \Leftrightarrow x\in A\text{与}f(x) \in X$.
			\par
			(2)由于$f$为$A$到$B$的映射,根据补充定理\ref{cor64}(3),$f(x)\in B$,根据补充定理\ref{cor70}(1),$x\in f^{-1}\langle B \rangle \\\Leftrightarrow x\in A$.
			\par
			(3)$x\in f^{-1}(y)$即$x\in f^{-1}\langle\{y\}\rangle$,根据补充定理\ref{cor70}(1)可证.
			\par
			(4)根据补充定理\ref{cor70}(3)可证.
			\par
			(5)令$f$的图为$F$,$f(x)\in f\langle X \rangle \Leftrightarrow (\exists z)(z\in X\text{与}(z, f(x))\in F)$.由于$x\in X$,因此$x\in A$,根据补充定理$\ref{cor64}$(2),$(x, f(x)) \in F$,因此$x\in X\text{与}(x, f(x)) \in F$,故$(\exists z)(z\in X\text{与}(z, f(x))\in F)$,得证.
			\par
			(6)$f\langle \{x\} \rangle =\{y|(\exists z)(z\in \{x\}\text{与}(z, y)\in F)\}$,等于$\{y|(\exists z)(z=x\text{与}(x, y)\in F)\}$,等于$\{y|(x, y)\in F)\}$,等于$\{y|y=f(x)\}$,即$\{f(x)\}$.
			\par
			(7)令$f$的图为$F$,则$u\in f^{-1}f\langle X \rangle \Leftrightarrow (\exists y)((\exists x)((x, y)\in F\text{与}x\in X)\text{与}(u, y)\in F)$,等价于$(\exists y)(\exists x)((x, y)\in F\text{与}x\in X\text{与}(u, y)\in F)$.同时,$X\subset A$,则$u\in X \Rightarrow (\exists y)((u, y)\in F)$,故$(\exists y)((u, y)\in F\text{与}u\in X)$,因此$(\exists y)(\exists x)((x, y)\in F\text{与}x\in X\text{与}(u, y)\in F)$,得证.
			\par
			(8)当$X\subset(f\text{的值域})$时,令$f$的图为$F$,则$u\in f\langle f^{-1}\langle X \rangle \rangle\Leftrightarrow (\exists x)(\exists y)((x, y)\in F\text{与}y\in X\text{与}(x, u)\in F)$.由于$F$为函数图,故$(\exists x)(\exists y)((x, y)\in F\text{与}y\in X\text{与}(x, u)\in F)\Rightarrow u=y$,因此$(\exists x)(\exists y)((x, y)\in F\text{与}y\in X\text{与}(x, u)\in F)\Rightarrow u\in X$.
			\par
			反过来,由于$X\subset(f\text{的值域})$,故$u\in X\Rightarrow (\exists x)((x, u)\in F)$,因此$u\in X\Rightarrow (\exists x)(\exists y)\\((x, y)\in F\text{与}y\in X\text{与}(x, u)\in F)$,得证.	

			\begin{de}
				\textbf{常数函数(fonction constante),常数映射(application constante)}
				\par
				令$f$为函数,且$(\forall x)(\forall x')(x\in (f的定义域)\text{与}x'\in (f的定义域)\Rightarrow f(x)=f(x'))$,则称$f$为常数函数或常数映射.
			\end{de}

			\begin{de}
				\textbf{不动点(élément invariant)}
				\par
				令$f$为映射,如果$x\in (f\text{的定义域})\text{与}f(x)=x$,则称$x$为$f$的不动点.
			\end{de}
			
			\begin{de}
				\textbf{重合(coïncident)}
				\par
				令$f$、$g$为函数,$E\subset f\text{的定义域}$,$E\subset g\text{的定义域}$,并且$(\forall x)((x\in E)\Rightarrow f(x)=g(x))$,则称$f$和$g$在$E$上重合.
			\end{de}
			
			\begin{de}
				\textbf{延拓(prolongement)}
				\par
				令$f=(F, A, B)$和$g=(G, C, D)$为函数,$F\subset G$,并且$f$和$g$在$A$上重合,则称$g$为$f$在$C$上的延拓.
			\end{de}

			\begin{cor}\label{cor71}
				\textbf{函数的限制是函数}
				\par
				$f$为函数,定义域为$A$,$X\subset A$,则$(x\in X\text{与}y=f(x))$为对于$x$、$y$生成图的公式,并且,其对于$x$、$y$生成的图$G$为函数图,且$pr_1G=X$.
			\end{cor}
			证明:令$f$的图为$F$,根据补充定理\ref{cor64}(1),$(x\in A\text{与}y=f(x))\Leftrightarrow (x, y)\in F$,根据补充证明规则\ref{Ccor18},$(x\in X\text{与}y=f(x))$为对于$x$、$y$生成图的公式.
			\par
			因此,$(x, y)\in G\Leftrightarrow x\in X\text{与}y=f(x)$,令$f$的图为$F$,根据补充定理\ref{cor64}(1),等价于$x\in X\text{与}(x, y)\in F$,则$pr_1G=\{x|x\in X\text{与}x\in A\}$,即$pr_1G=X$.对于$(x, y)\in G$、$(x, y')\in G$,有$(x, y)\in F$、$(x, y')\in F$,由于$F$为函数图,故$y=y'$,因此,$G$为函数图.

			\begin{de}
				\textbf{限制(restriction)}
				\par
				函数$f=(F, A, B)$,$X\subset A$,$(x\in X\text{与}y=f(x))$对于$x$、$y$生成的图为$G$,则称$(G, X, B)$\\为函数$f$在$X$上的限制,记作$f|X$.
			\end{de}

			\begin{cor}\label{cor72}
				\hfill\par
				$f$为函数,$x\subset (f\text{的定义域})$,则$f|X$和$f$在$X$上重合,并且,$f$为$f|X$在其定义域上的延拓.
			\end{cor}
			证明:令$f$的图为$F$,$f|X$的图为$G$,$x$、$y$为不是常数的字母.由于$(x\in X\text{与}y=f(x))\\\Rightarrow y=f(x)$,因此,$F\subset G$.根据补充定理\ref{cor64}(1)、补充证明规则\ref{Ccor16}(1),$(x\in X\text{与}y=f(x))\Leftrightarrow y=f|X(x)$,则$x\in X\Leftrightarrow f(x)=f|X(x)$,因此,$f|X$和$f$在$X$上重合.又因为$X$为$f|X$的定义域,因此,$f$为$f|X$在其定义域上的延拓.
			
			\begin{cor}\label{cor73}
				\hfill\par
				函数$f=(F, A, B)$,$X\subset A$,则:
				\par
				(1)$x\in X\Rightarrow f(x)=(f|X)(x)$.
				\par
				(2)$f|X=f\circ Id_X$.
			\end{cor}
			证明:
			\par
			(1)设$x\in X$,根据补充定理\ref{cor64}(1),$(x, f(x))\in F$.令$f|X$的图为$G$,则$y=f(x)\Leftrightarrow (x, y)\in G$,即$y=f(x)\Leftrightarrow y=f|X(x)$,因此$f(x)=f|X(x)$.
			\par
			(2)$(x, z)\in F\circ \Delta_X\Leftrightarrow (\exists y)((x, y)\in \Delta_X\text{与}(y, z)\in F)$,根据补充定理\ref{cor59},等价于$x\in X\text{与}(x, z)\in F$,根据补充定理\ref{cor64}(1),等价于$x\in X\text{与}y=f(x)$,得证.

			\begin{C}\label{C54}
				\hfill\par
				令$T$、$A$为包含$2$元特别符号$\in$ 、显式公理\ref{ex1}、显式公理\ref{ex2}和公理模式\ref{Sch8}的等式理论$M$的项,$x$、$y$为不同的字母,$A$不包含$x$,$A$、$T$均不包含$y$.$R$为公式$x\in A\text{与}y=T$.则其为对于$x$、$y$生成图的公式.设其生成的图为$F$,则$F$为函数图,$pr_1F=A$,$pr_2F=(\text{对于}x\in A\text{形式为}T\text{的对象集合})$,且$x\in A\Leftrightarrow F(x)=T$.
			\end{C}
			证明:
			\par
			考虑其他规则相同但不包含其他显式公理的理论$M_0$,则$M_0$不包含任何常数:
			\par
			令$B$为对于$x\in A$形式为$T$的对象集合,则$B$不包含$x$、$y$.$R\Rightarrow (x, y)\in A\times B$,由于$A$、$B$都不包含$x$、$y$,根据补充证明规则\ref{Ccor19},$R$为对于$x$、$y$生成图的公式.
			\par
			令$z$为不同于$M$中常数的字母,则根据补充证明规则\ref{Ccor16}(1),$(x, y)\in F\text{与}(x, z)\in F\Leftrightarrow x\in A\text{与}y=T\text{与}z=T$,因此,$(x, y)\in F\text{与}(x, z)\in F\Rightarrow (y=T)\text{与}(z=T)$,根据定理\ref{theo3},$(x, y)\in F\text{与}(x, z)\in F\Rightarrow y=z$,故$F$为函数图.
			\par
			由于$A$不包含$y$,因此$(\exists y)(x\in A\text{与}y=T)\Leftrightarrow x\in A\text{与}(\exists y)(y=T)$,其等价于$x\in A$,因此$pr_1F=A$.
			\par
			根据定义,$pr_2F=(\text{对于}x\in A\text{形式为}T\text{的对象集合})$.
			\par
			根据补充定理\ref{cor64}(1),$y=F(x)\Leftrightarrow x\in A\text{与}y=T$.假设$x\in A$,则$y=F(x)\Leftrightarrow y=T$,因此$F(x)=T$.故$x\in A\Rightarrow F(x)=T$.
			\par
			反过来,假设$F(x)=T$,则$x\in A$为真.故$x\in A\Leftrightarrow F(x)=T$.
			\par
			由于$M$强于$M_0$,因此上述结论对理论$M$也成立.

			\begin{metadef}
				\textbf{用项定义的函数(fonction par un terme)}
				\par
				令$T$、$A$、$C$为包含$2$元特别符号$\in$ 、显式公理\ref{ex1}、显式公理\ref{ex2}和公理模式\ref{Sch8}的等式理论$M$的项,$x$、$y$为不同的字母,$A$不包含$x$,$A$、$T$、$C$均不包含$y$.如果$(\text{对于}x\in A\text{形式为}T\text{的对象}\\\text{集合})\subset C$,公式$x\in A\text{与}y=T$对于$x$、$y$生成图的公式为F,则$(F, A, C)$称为用$T$定义的函数,记作$x\mapsto T (x\in A, T\in C)$,在没有歧义的情况下也可以简记为$x\mapsto T(x\in A)$、$(T)x\in A$、$x\mapsto T$或者$T$.
			\end{metadef}

			\begin{theo}\label{theo17}
				\textbf{函数的复合是函数}
				\par
				$f$为$A$到$B$的映射,$g$为$B$到$C$的映射,则$g\circ f$为$A$到$C$的映射.
			\end{theo}
			证明:令$f$、$g$的图分别为$F$、$G$,$x$、$y$、$z$均为不是常数的字母,则$(x, z)\in G\circ F\Leftrightarrow (\exists y)((x, y)\in F\text{与}(y, z)\in G)$,$(x, z')\in G\circ F\Leftrightarrow (\exists y)((x, y)\in F\text{与}(y, z')\in G)$.添加辅助常数$y$、$y'$,设$(x, y)\in F\text{与}(y, z)\in G$,$(x, y')\in F\text{与}(y', z')\in G$.则根据函数定义,$y=y'$,故$z=z'$.故$g\circ f$为函数.
			\par
			$f$的定义域$A=\{x|(\exists y)(x, y)\in F\}$,$g$的定义域为$B=\{y|(\exists z)(y, z)\in G\}$.令$g\circ f$的定义域$A'=\{x|(\exists z)((\exists y)((x, y)\in F\text{与}(y, z)\in G))\}$.则$(\exists z)((\exists y)((x, y)\in F\text{与}(y, z)\in G))\Rightarrow (\exists y)((x, y)\in F)$,故$A'\subset A$.
			另一方面,$(\exists x)(x, y)\in F\Rightarrow (\exists z)(y, z)\in G$,又因为$(x, y)\in F\Rightarrow (\exists x)(x, y)\in F$,因此,$(\exists y)((x, y)\in F)\Rightarrow (\exists z)((\exists y)((x, y)\in F\text{与}(y, z)\in G))$,故$A\subset A'$.
			根据补充定理\ref{cor51}(2),$pr_2G\circ F\subset C$.
			\par
			综上,得证.
	
			\begin{de}
				\textbf{单射(injection/application injective),满射(surjection/application surjective),双射(bijection/application bijective)}
				\par
				令$f$为$A$到$B$的映射.如果$(\forall x)(\forall y)(x\in A\text{与}y\in A\text{与}f(x)=f(y)\Rightarrow (x=y))$,则称$f$为$A$到$B$的单射;如果$f\langle A \rangle =B$,则称$f$为$A$到$B$的满射.如果$f$是$A$到$B$的单射,也是$A$\\到$B$的满射,则称$f$为$A$到$B$的双射.
			\end{de}

			\begin{de}
				\textbf{排列(permutation)}
				\par
				$A$到$A$的双射称为$A$的排列.
			\end{de}
			
			\begin{cor}\label{cor74}
				\hfill\par
				(1)函数$(\varnothing, \varnothing, A)$为单射,函数$(\varnothing, \varnothing, \varnothing)$为双射.
				\par
				(2)函数$(F, \{x\}, A)$为单射,函数$(F, \{x\}, \{y\})$为双射.
				\par
				(3)函数$Id_A$为双射.
				\par
				(4)如果$A\subset B$,则函数$(\Delta_A, A, B)$为单射.
				\par
				(5)$x\mapsto (x, x)(x\in A, (x, x)\in A\times A)$为单射.
			\end{cor}
			证明:根据定义可证.				

			\begin{de}
				\textbf{子集的规范映射(application canonique de partie),子集的规范单射\\(injection canonique de partie)}
				\par
				如果$A\subset B$,则映射$(\Delta_A, A, B)$称为$A$到$B$的规范映射或规范单射.	
			\end{de}

			\begin{de}
				\textbf{到两个集合的乘积的对角映射(application diagonale dans produit de deux ensembles)}
				\par
				映射$x\mapsto (x, x)(x\in A, (x, x)\in A\times A)$称为对角映射.
			\end{de}

			\begin{cor}\label{cor75}
				\hfill\par
				$f$为$A$到$B$的单射,$X\subset A$,则$f|X$为$X$到$B$的单射.
			\end{cor}
			证明:$X\subset A$,故$x\in X\Rightarrow x\in A$,根据定义可证.	

			\begin{theo}\label{theo18}
				\hfill\par
				$f$为$A$到$B$的映射,则$f^{-1}\text{为函数}\Leftrightarrow f\text{为双射}$.
			\end{theo}
			证明:
			\par
			令$f$的图为$F$,则$f^{-1}$的图为$F^{-1}$.令$x$、$y$、$z$为不是常数的字母.
			\par
			由于$f$为$A$到$B$的映射,故$pr_1F=A$,根据补充定理\ref{cor42},$pr_2F^{-1}=A$.
			\par
			根据补充定理\ref{cor35}、补充定理\ref{cor42},$f\text{为满射}\Leftrightarrow pr_1F^{-1}=B$.
			\par
			根据补充定理\ref{cor64}(1),$(\forall x)(\forall y)(\forall z)((y, x)\in F\text{与}(z, x)\in F\Rightarrow (y=z))\\\Leftrightarrow (\forall x)(\forall y)(\forall z)(y\in A\text{与}z\in A\text{与}x=f(y)\text{与}x=f (z)\Rightarrow (y=z))$.
			假设$(\forall x)(\forall y)(\forall z)(y\in A\text{与}z\in A\text{与}x=f(y)\text{与}x=f (z)\Rightarrow (y=z))$,则$(\forall y)(\forall z)(y\in A\text{与}z\in A\text{与}f(y)=f(y)\text{与}f(y)=f (z)\Rightarrow (y=z))$,进而$(\forall y)(\forall z)(y\in A\text{与}z\in A\text{与}f(y)=f (z)\Rightarrow (y=z))$.
			\par
			反过来,假设$(\forall y)(\forall z)(y\in A\text{与}z\in A\text{与}f(y)=f (z)\Rightarrow (y=z))$,由于$y\in A\text{与}z\in A\text{与}x=f(y)\text{与}x=f (z)\Rightarrow f (y)=f (z)$,故$(\forall x)(\forall y)(\forall z)(y\in A\text{与}z\in A\text{与}x=f(y)\text{与}x=f (z)\Rightarrow (y=z))$.
			\par
			综上,得证.

			\begin{de}
				\textbf{逆映射(application réciproque),反函数(fonction réciproque)}
				\par
				令$f$为$A$到$B$的双射,则称$f^{-1}$为$f$的逆映射或反函数.
			\end{de}

			\begin{de}
				\textbf{对合函数(fonction involutive)}
				\par
				令$f$为函数,如果$f^{-1}=f$,则称$f$为对合函数.
			\end{de}
			
			\begin{cor}\label{cor76}
				\hfill\par
				$f$为$A$到$B$的双射,则$f^{-1}$为$B$到$A$的双射.
			\end{cor}
			证明:根据补充定理\ref{cor44},$(f^{-1})^{-1}$的图与$f$的图相同,根据定理\ref{theo18}可证.
			
			\begin{cor}\label{cor77}
				\hfill\par
				$Id_A$的反函数是$Id_A$.
			\end{cor}
			证明:根据补充定理\ref{cor61}可证.
			
			\begin{cor}\label{cor78}
				\hfill\par
				$f$为$A$到$B$的双射,则($\forall X)(X\subset A\Rightarrow f^{-1}\langle f\langle X \rangle \rangle=X)$.
			\end{cor}
			证明:根据补充定理44,$(f)^{-1})^{-1}=f$,根据补充定理\ref{cor70}(8)可证.
			
			\begin{theo}\label{theo19}
				\textbf{左逆和右逆的存在性}
				\par
				$f$为$A$到$B$的映射,如果存在$B$到$A$的映射$r$(或$s$),使$r\circ f=Id_A$(或$f\circ s=Id_B$),则$f$为单射(或满射).
				\par
				反过来,如果$f$为满射,则存在$B$到$A$的映射$s$,使$f\circ s=Id_B$;如果$f$为单射,且$A\neq \varnothing$,则存在$B$到$A$的映射$r$,使$r\circ f=Id_A$.
			\end{theo}
			证明:
			\par
			如果$r\circ f$为$Id_A$,则$(x\in A\text{与}y\in A)\Rightarrow (r(f(x))=x\text{与}r(f(y))=y)$.根据公理模式\ref{Sch6}、补充定理\ref{cor29}, $f(x)=f(y)\Rightarrow r(f(x))=r(f(y))\text{与}x\in A\text{与}y\in A$,因此$f(x)=f(y)\Rightarrow x=y$,故$f$为单射.
			\par
			如果$f\circ s=Id_B$,则$f\langle s\langle B \rangle \rangle=B$,同时,根据定理\ref{theo12},$f\langle s\langle B \rangle \rangle\subset f\langle A \rangle $,又因为$f\langle A \rangle \subset f\langle B \rangle $,故$f\langle A \rangle =f\langle B \rangle $,因此$f$为满射.
			\par
			如果$f$为满射,则$f\langle A \rangle =B$,根据补充定理\ref{cor35},$x\in B\Leftrightarrow (\exists y)((y, x)\in F)$,根据补充定理\ref{cor64}(1),$(y, x)\in F\Leftrightarrow y\in A\text{与}x=f(y)$.令$T$为项$\tau_y(y\in A\text{与}x=f(y))$,根据证明规则\ref{C54},$x\in B\Leftrightarrow f(T)=x)$.令$s$为映射$x\mapsto T(x\in B, T\in A)$,则$x\in B\Leftrightarrow f(s(x))=x$,即$f\circ s=Id_B$.
			如果$f$为单射,且$A\neq \varnothing$,根据补充定理\ref{cor14}(2),$(\exists x)(x\in A)$,因此可以添加辅助常数$a$,令$a\in A$.令$f$的图为$F$,$R$为公式$(y, x)\in F\text{或}(y=a\text{与}x\in B-f\langle A \rangle )$.由于$(y, x)\in F\Rightarrow y\in A\text{与}x\in B$,因此$R\Rightarrow (x, y)\in B\times A$,因此,该公式为对于$x$、$y$生成图的公式.令其生成的图为$P$.
			\par
			$R\text{与}(z|y)R\Leftrightarrow ((y, x)\in F\text{与}(z, x)\in F)\text{或}(y=a\text{与}z=a\text{与}x\in B-f\langle A \rangle )\text{或}((y, x)\in F \text{与}z=a\text{与}x\in B-f\langle A \rangle )\text{或}((z, x)\in F \text{与}y=a\text{与}x\in B\text{与}x\notin f\langle A \rangle )$.$由于(z, x)\in F\Rightarrow x\in f\langle A \rangle )$,$(y, x)\in F\Rightarrow x\in f\langle A \rangle $,根据补充证明规则\ref{Ccor5}(13),$R\text{与}(z|y) R\Rightarrow y=z$,故$P$为函数图.
			\par
			$(\exists y)R\Leftrightarrow (\exists y)( (y, x)\in F)\text{或}(\exists y)(y=a\text{与}x\in B-f\langle A \rangle )$,等价于$(\exists y)((y, x)\in F)\text{或}x\in B-f\langle A \rangle $,等价于$x\in f\langle A \rangle \text{或}x\in B-f\langle A \rangle $,等价于$x\in B$,故$pr_1P=B$.$(\exists x)R\Leftrightarrow y\in A\text{或} (y=a\text{与}(\exists x)( x\in B-f\langle A \rangle ))$,又由于$y=a\Rightarrow y\in A$,故$(y=a\text{与}(\exists x)( x\in B-f\langle A \rangle ))\Rightarrow y\in A$,故$(\exists x)R\Leftrightarrow y\in A$,故$pr_2P=A$.由此可知,$(P, B, A)$是函数,令其为$r$,则$r\circ f(x)=x\Leftrightarrow (x, f(x))\in F\text{或}(x=a\text{与}f(x)\in B-f\langle A \rangle )$,$r\circ f(x)=x\Leftrightarrow x\in A\text{或}(x=a\text{与}f(x)\in B\text{与}f(x)\notin f\langle A \rangle )$.$由于a\in A\Rightarrow f(x)\in f\langle A \rangle $,根据补充证明规则\ref{Ccor5}(15),$r\circ f(x)=x\Leftrightarrow x\in A$,故$r\circ f=Id_A$.

			\begin{theo}\label{theo20}
				\hfill\par
				$f$为$A$到$B$的映射,$g$为$B$到$A$的映射,且$g\circ f=Id_A$,$f\circ g=Id_B$,则$f$和$g$均为双射,且$g=f^{-1}$.
			\end{theo}
			证明:根据定理\ref{theo19},f和g均为双射.
			\par
			令$f$的图为$F$,$g$的图为$G$,假设$(x, y)\in F$,则$y\in B$,$y=f(x)$.由于$g(f(x))=x$,因此$g(y)=x$,根据补充定理\ref{cor64}(1),$(y, x)\in G$.
			\par
			假设$(y, x)\in $G,同理可得$(x, y)\in F$.
			\par
			因此$(x, y)\in F\Leftrightarrow (y, x)\in G$,即$(y, x)\in F^{-1}\Leftrightarrow (y, x)\in G$,根据补充定理\ref{cor27}(2),$G=F^{-1}$.又因为$g$为$B$到$A$的映射,故$g=f^{-1}$.

			\begin{de}
				\textbf{回缩(rétractions),截面(section),左逆(inverse à gauche),右逆\\(inverse à droite)}
				\par
				令$f$为$A$到$B$的单射(或满射),如果$B$到$A$的映射$r$(或$s$)使使$r\circ f=Id_A$(或$f\circ s=Id_B$),则称$r$(或$s$)为$f$的回缩(或截面),或称为$f$的左逆(或右逆).
			\end{de}

			\begin{cor}\label{cor79}
				\hfill\par
				如果$g$是$f$的左逆,则$f$是$g$的右逆;如果$f$是$g$的右逆,则$g$是$f$的左逆.
			\end{cor}
			证明:根据定义可证.
						
			\begin{cor}\label{cor80}
				\hfill\par
				单射的左逆是满射,满射的右逆是单射.
			\end{cor}
			证明:根据补充定理\ref{cor79}、定理\ref{theo19}可证.
			
			\begin{cor}\label{cor81}
				\textbf{右逆的唯一性}
				\par
				令$f$为$A$到$B$的满射,$s$、$s'$都是f的右逆,如果$s\langle B \rangle =s'\langle B \rangle $,则$s=s'$.
			\end{cor}
			证明:
			\par
			如果$B=\varnothing$,根据补充定理\ref{cor31}(2),$s$、$s'$的图均为$\varnothing$,则$s=s'$.
			\par
			如果$B\neq \varnothing$,添加辅助常数$x$、$y$,使$x\in B$,$y\in B$,且$s(x)=s'(y)$.由于$f(s(x))=x$,$f(s'(y))=y$,因此$s(x)=s'(x)$.根据补充定理\ref{cor68},得证.

			\begin{theo}\label{theo21}
				\hfill\par
				令$f$为$A$到$B$的映射,$f'$为$B$到$C$的映射,$f''$=$f'\circ f$,则:
				\par
				(1)	如果$f$、$f'$为单射,则$f'\circ f$为单射;如果$r$、$r'$分别为$f$、$f'$的左逆,则$r\circ r'$是$f''$的左逆;
				\par
				(2)	如果$f$、$f'$为满射,则$f'\circ f$为满射;如果$s$、$s'$分别为$f$、$f'$的左逆,则$r\circ r'$是$f''$的左逆;
				\par
				(3)	如果$f''$为单射,则$f$为单射;如果$r''$是$f''$的左逆,则$r''\circ f'$是$f$的左逆;
				\par
				(4)	如果$f''$为满射,则$f'$为满射;如果$s''$是$f''$的右逆,则$f\circ s''$是$f'$的右逆;
				\par
				(5)	如果$f''$为满射,$f'$为单射,则$f$为满射;如果$s''$是$f''$的右逆,则$s''\circ f'$是$f$的右逆;
				\par
				(6)	如果$f''$为单射,$f$为满射,则$f'$为单射;如果$r''$是$f''$的左逆,则$f\circ r''$是$f$的左逆.
			\end{theo}
			证明:
			\par
			(1)如果$A=\varnothing$,则$f$、$f'$、$r$、$r'$均为$(\varnothing, \varnothing, \varnothing)$,显然成立.如果$A\neq \varnothing$,则$r\circ f=Id_A$,$r'\circ f'=Id_B$.$r\circ r'\circ f'\circ f=r\circ Id_B\circ f$,等于$r\circ f$,等于$Id_A$.
			\par
			此外,根据定理\ref{theo19},$f'\circ f$为单射.
			\par
			(2)类似定理\ref{theo21}(1)可证.
			\par
			(3)如果$A=\varnothing$,则$f$、$f'$、$r$、$r'$均为$(\varnothing, \varnothing, \varnothing)$,显然成立.如果$A\neq \varnothing$,$r''\circ f''=Id_A$,则$(r''\circ f') \circ f=Id_A$.此外,根据定理\ref{theo19},$f$为单射.
			\par
			(4)类似定理\ref{theo21}(3)可证.
			\par
			(5)$f''\circ s''=Id_C$,根据定理\ref{theo21}(4),$f'$为双射,则$f\circ (s''\circ f')=({f'}^{-1}\circ f')\circ f\circ (s''\circ f')$,等于${f'}^{-1}\circ f'' \circ s''\circ f'$,等于${f'}^{-1}\circ f'$,等于$Id_B$.此外,根据定理\ref{theo19},$f$为满射.
			\par
			(6)类似定理\ref{theo21}(5)可证.

			\begin{theo}\label{theo22}
				\textbf{函数唯一存在的条件}
				\par
				(1)	令$g$为$E$到$F$的满射,$f$为$E$到$G$的映射,则当且仅当$(\forall x)(\forall y)(x\in E\text{与}y\in E\text{与}\\g(x)=g(y)\Rightarrow f(x)=f(y))$时,存在$F$到$G$的映射$h$,使$f=h\circ g$;并且,$h$是唯一的,令$s$为$g$的右逆,则$h=f\circ s$.
				\par
				(2)	令$g$为$F$到$E$的单射,$f$为$G$到$E$的映射,则当且仅当$f(G)\subset g(F)$时,存在$G$到$F$的映射$h$,使$f=g\circ h$;并且,$h$是唯一的,令$r$为$g$的左逆,$h=r\circ f$.
			\end{theo}
			证明:
			\par
			(1)若$f=h\circ g$,则$(\forall x)(\forall y)(x\in E\text{与}y\in E\text{与}g(x)=g(y)\Rightarrow f(x)=f(y))$.同时,$s$为$g$的右逆,则$h=h\circ (g\circ s)$,因此$h=f\circ s$,故$h$是唯一的.
			\par
			反过来,假设$(\forall x)(\forall y)(x\in E\text{与}y\in E\text{与}g(x)=g(y)\Rightarrow f(x)=f(y))$,令$h=f\circ s$,则当$x\in E$时,$g(s(g(x)))=g(x)$,$s(g(x))\in E$,因此$f(s(g(x)))=f(x)$,即$h(g(x))=f(x)$.因此$f=h\circ g$.
			\par
			(2)若$f=g\circ h$,根据补充定理\ref{cor51}(2),$f(G)\subset g(F)$.同时,$r$为$g$的左逆,则$h=(r\circ g)\circ h$,因此$h=r\circ f$.
			\par
			反过来,假设$f(G)\subset g(F)$,令$h=r\circ f$.由于$f(G)\subset g(F)$,因此$(\exists x)(x\in G\text{与}y=f(x))\Rightarrow (\exists x)(x\in F\text{与}y=g(x))$,因此,$(\exists x)(x\in G\text{与}f(z)=f(x))\Rightarrow (\exists x)(x\in F\text{与}f(z)=g(x))$.如果$G=\varnothing$,则$f$为$(\varnothing, \varnothing, E)$,$h$为$(\varnothing, \varnothing, F)$,因此$f=h\circ g$.如果$G \neq\varnothing$,则添加辅助变量$z$使$z\in G$,故$(\exists x)(x\in F\text{与}f(z)=g(x))$,添加辅助变量$y$,使$f(z)=g(y)$,因此$g(h(z))=g(r(f(z)))$,进而等于$g(r(g(y)))$,进而等于$g(y)$,最后等于$f(z)$.因此,$f=g\circ h$.

			\begin{de}
				\textbf{二元函数(fonction de deux arguments)}
				\par
				如果函数$f$的定义域为图$G$,则称$f$为二元函数.当$(x, y)\in G$时,函数$f$对于$(x, y)$的值记作$f(x, y)$.
			\end{de}

			\begin{de}
				\textbf{偏映射(application partielle)}
				\par
				令$f$为二元函数,定义域为$D$,到达域为$C$,令$A_y=D^{-1}\langle\{y\}\rangle$,则映射$x\mapsto f(x, y)\\(x\in A-y, f(x, y)\in C)$称为$f$关于第二个参数值为$y$的偏映射,记作$f(., y)$、$f(, y)$或$f_y$.令$A_x=D\langle \{x\} \rangle $,则映射$y\mapsto f(x, y)(y\in B_x, f(x, y)\in C)$称为$f$关于第一个参数值为$x$的偏映射,记作$f(x, .)$、$f(x,)$或$f_x$.
			\end{de}

			\begin{cor}\label{cor82}
				\hfill\par
				(1)令$f$为二元函数,定义域为$D$,则$(x, y)\in D\Leftrightarrow f(., y)(x)=f(x, y)$,$(x, y)\in D\Leftrightarrow f(x, .)(y)=f(x, y)$.
				\par
				(2)	令$f$、$g$为二元函数,定义域均为$D$,如果$(x, y)\in D\Rightarrow f(., y)(x)=g(., y)(x)$,或者$(x, y)\in D\Rightarrow f(x, .)(y)=g(x, .)(y)$,则$f$和$g$在$D$上重合.
			\end{cor}
			证明:
			\par
			(1)根据证明规则\ref{C54}可证.
			\par
			(2)当$(x, y)\in D$时,如果$(x, y)\in D\Rightarrow f(., y)(x)=g(., y)(x)$,则$f(., y)(x)=g(., y)(x)$,根据补充定理\ref{cor82}(1),$f(x, y)=g(x, y)$.同理可证$(x, y)\in D\Rightarrow f(x, .)(y)=g(x, .)(y)$的情况.			

			\begin{de}
				\textbf{不依赖于参数(ne dépend pas de argument)}
				\par
				令$f$为二元函数,如果 $f(., y)$(或$f(x, .)$)为常数函数,则称$f$不依赖于第一个参数(或第二个参数).
			\end{de}
			
			\begin{cor}\label{cor83}
				\hfill\par
				令$g$为$B$到$C$的映射,则映射$z\mapsto g(pr_2z)(z\in A\times B, g(pr_2z)\in C)$不依赖于第一个参数.
			\end{cor}
			证明:如果$B\neq \varnothing$,若$y\in B$,则该映射关于第二个参数的偏映射是$x\mapsto g(y)(x\in A_y, g(y)\in C)$,若$y\notin B$,则该映射关于第二个参数的偏映射是$(\varnothing, \varnothing, C)$;如果$B=\varnothing$,则关于第二个参数的偏映射是$(\varnothing, \varnothing, C)$,均为常数函数,得证.

			\begin{de}
				\textbf{映射在乘积集合上的规范扩展(extension canonique de applications aux ensembles produits),映射的乘积(produit de applications)}
				\par
				令$u$为$A$到$C$的映射,$v$为$B$到$D$的映射,则称映射$z\mapsto (u(pr_1z), v(pr_2z))(z\in A\times B, \\(u(pr_1z), v(pr_2z))\in C\times D)$为$u$和$v$在乘积集合上的规范扩展,或称$u$和$v$的乘积,记作$u\times v$或$(u, v)$.
			\end{de}

			
			\begin{cor}\label{cor84}
				\hfill\par
				$u$为$A$到$C$的映射,$v$为$B$到$D$的映射,则$u$和$v$的乘积的值域为$u\langle A \rangle \times v\langle B \rangle $.
			\end{cor}
			证明:令$u$的图为$U$,$v$的图为$V$,则$u\langle A \rangle \times v\langle B \rangle =\{z|(z\text{为有序对})\text{与}(\exists x)(x, pr_1z)\\\in U\text{与}(\exists x)(x, pr_2z)\in V\}$.u和v的乘积的值域为$\{z|(\exists w)(w\in A\times B\text{与}z=\\(u(pr_1w), v(pr_2w)))\}$.
			\par
			由于$(z\text{为有序对})\text{与}(\exists x)(x, pr_1z)\in U\text{与}(\exists x)(x, pr_2z)\in V\Leftrightarrow (z\text{为有序对})\text{与}(\exists x)(\exists y)\\((x, pr_1z)\in U\text{与}(y, pr_2z)\in V)$,而$(\exists w)(w\in A\times B, z=(u(pr_1w), v(pr_2w)))\Leftrightarrow (z\text{为有序对})\text{与}\\(\exists w)((w\text{为有序对})\text{与}pr_1w\in A\text{与}pr_2w\in B\text{与}pr_1z=u(pr_1w)\text{与}pr_2z=v(pr_2w))$,等价于\\$(z\text{为有序对})\text{与}(\exists w)((w\text{为有序对})\text{与}(pr_1w, pr_1z)\in U\text{与}(pr_2w, pr_2z)\in U)$,根据补充证明规则\ref{Ccor15},得证.
			
			\begin{cor}\label{cor85}
				\hfill\par
				$u$、$v$为函数,如果$u$、$v$都是单射(或满射),则$u\times v$为单射(或满射).
			\end{cor}
			证明:如果u、v都是单射,根据定理\ref{theo7}可证.如果$u$、$v$都是满射,根据补充定理\ref{cor84}可证.
			
			\begin{cor}\label{cor86}
				\hfill\par
				$u$为$A$到$C$的映射,$v$为$B$到$D$的映射,$u'$为$C$到$E$的映射,$v'$为$D$到$F$的映射,则$u'\times v'\circ u \times v=u'\circ u\times v' \circ v$.
			\end{cor}
			证明:若$z\in A\times B$,则$((u'\times v')\circ (u\times v))(z)= (u'\times v')(u(pr_1z), v(pr_2z))$,等于\\$(u'(u(pr_1z), v'(v(pr_2z)))$,得证.
			
			\begin{cor}\label{cor87}
				\hfill\par
				$Id_A\times Id_B=Id_{A\times B}$.
			\end{cor}
			证明:若$z\in A\times B$,则$(Id_A\times Id_B)(z)= (Id_A(pr_1z), Id_B(pr_2z))$,即等于$z$,得证.
			
			\begin{cor}\label{cor88}
				\hfill\par
				$u$、$v$都是双射,则$u\times v$为双射,且其反函数为$u^{-1}\times v^{-1}$.
			\end{cor}
			证明:根据补充定理\ref{cor85},$u\times v$为双射.
			\par
			令$u$的定义域为$A$,$v$的定义域为$B$,根据补充定理\ref{cor86},$u^{-1}\times v^{-1}\circ u\times v=Id_A\times Id_B$,根据补充定理\ref{cor87}得证.

			\begin{exer}\label{exer43}
				\hfill\par
				求证:$x\in y$、$x\subset y$、$x=\{y\}$都不是对于$x$、$y$生成图的公式.
			\end{exer}
			证明:即补充定理\ref{cor28}(2)、补充定理\ref{cor28}(3)、补充定理\ref{cor28}(4).

			\begin{exer}\label{exer44}
				\hfill\par
				$G$为图,求证:$X\subset pr_1G\Leftrightarrow X\subset G^{-1}\langle G\langle X \rangle \rangle$.
			\end{exer}
			证明:即补充定理\ref{cor50}.
			
			\begin{exer}\label{exer45}
				\hfill\par
				$G$和$F$为图,求证:$pr_1H\subset pr_1G\Leftrightarrow H\subset H\circ G^{-1}\circ G$,并且$G\subset G\circ G^{-1}\circ G$.
			\end{exer}
			证明:
			\par
			如果$pr_1H\subset pr_1G$,设$(h, h')\in H$,则$h\in pr_1G$,故$(\exists g)(h, g)\in G$,则$(\exists g)(\exists i)((h, g)\in G\text{与}(i, g) \in G\text{与}(i, h') \in H)$,故$(h, h') \in H\circ G^{-1}\circ G$,即$H\subset H\circ G^{-1}\circ G$.
			\par
			由于$pr_1G\subset pr_1G$,故$G\subset G\circ G^{-1}\circ G$.
			
			\begin{exer}\label{exer46}
				\hfill\par
				$G$为图,求证:
				\par
				(1)	$G\circ \varnothing=\varnothing$,$\varnothing\circ G=\varnothing$;
				\par
				(2)	当且仅当$G= \varnothing$时,$G^{-1}\circ G=\varnothing$.
			\end{exer}
			证明:即补充定理\ref{cor52}.
			
			\begin{exer}\label{exer47}
				\hfill\par
				$G$为图,求证:
				\par
				(1)$(A\times B)\circ G=G^{-1}\langle A \rangle \times B$.
				\par
				(2)$G\circ (A\times B)=A\times G\langle B \rangle $.
			\end{exer}
			证明:
			\par
			(1)	令$x$、$y$、$z$为不是常数的字母,$(x, z)\in (A\times B)\circ G\Leftrightarrow (\exists y)((x, y)\in G\text{与}y\in A\text{与}z\in B)$,$(x, z)\in G^{-1}\langle A \rangle \times B\Leftrightarrow (\exists y)(y\in A\text{与}(x, y)\in G)\text{与}z\in B$,得证.
			\par
			(2)	令$x$、$y$、$z$为不是常数的字母,$(x, z)\in G\circ (A\times B)\Leftrightarrow (\exists y)(x\in A\text{与}y\in B\text{与}(y, z)\in G)$,$(x, z)\in A\times G\langle B \rangle \Leftrightarrow x\in A\text{与}(\exists y)(y\in B\text{与}(y, z)\in G)$,得证.
			
			\begin{exer}\label{exer48}
				\hfill\par
				对任意图$G$,用$G'$表示$pr_1G\times pr_2G-G$,求证:
				\par
				(1)$(G^{-1})'=(G')^{-1}$.
				\par
				(2)如果$pr_1G\subset A$,$pr_2G\subset B$,则$G\circ (G^{-1})'\subset (\Delta_B)'$,$(G^{-1})' \circ G\subset (\Delta_A)'$.
				\par
				(3)当且仅当$G\circ (G^{-1})'\circ G=\varnothing$时,$G=pr_1G\times pr_2G$.
			\end{exer}
			证明:
			\par
			(1)	根据补充定理\ref{cor42}、补充定理\ref{cor45}可证.
			\par
			(2)	令$x$、$y$、$z$为不是常数的字母,$(x, y)\in (G^{-1})'\Leftrightarrow (x\in A\text{与}y\in B\text{与}(y, x)\notin G)$,则$(x, z)\in G\circ (G^{-1})'\Leftrightarrow (\exists y)(x\in B\text{与}y\in A\text{与}(y, x)\notin G\text{与}(y, z)\in G)$,$(x, z)\in (\Delta_B)'\Leftrightarrow (x\in B\text{与}z\in B\text{与}x\neq z)$.由于$(y, z)\in G\Rightarrow z\in B$,$(y, x)\notin G\text{与}(y, z)\in G\Rightarrow x\neq z$,故$(x, z)\in G\circ (G^{-1})'\Rightarrow (x, z)\in (\Delta_B)'$,即$G\circ (G^{-1})'\subset (\Delta_B)'$.同理可证$(G^{-1})' \circ G\subset (\Delta_A)'$.
			\par
			(3)	如果$G=pr_1G\times pr_2G$,则$(G^{-1})' = \varnothing$,根据补充定理\ref{cor52}(1),$G\circ (G^{-1})'\circ G=\varnothing$.如果$G\circ (G^{-1})'\circ G=\varnothing$,则$(\exists y)(\exists z)((x, y)\in G\text{与}y\in pr_2G\text{与}z\in pr_1G\text{与}(z, y)\notin G\text{与}(z, t)\in G)$为假,即$(x, y)\notin G\text{或}y\notin pr_2G\text{或}z\notin pr_1G\text{或}(z, y)\in G\text{或}(z, t)\notin G)$,即$(x, y)\in G\text{与}y\in pr_2G\text{与}z\in pr_1G\text{与}(z, t)\in G\Rightarrow (z, y)\in G$,因此,$(\exists x)((x, y)\in G)\text{与}(\exists t)((z, t)\in G)\text{与}y\in pr_2G\text{与}z\in pr_1G\Rightarrow (z, y)\in G$,又因为$y\in pr_2G\Rightarrow (\exists x)((x, y)\in G)$,$z\in pr_1G\Rightarrow ((z, t)\in G)$,因此$y\in pr_2G\text{与}z\in pr_1G\Rightarrow (z, y)\in G$,因此$G=pr_1G\times pr_2G$.
			
			\begin{exer}\label{exer49}
				\hfill\par
				$G$为图,求证:$G\text{为函数图}\Leftrightarrow (\forall X)(G\langle G^{-1}\langle X \rangle \rangle\subset X)$.
			\end{exer}
			证明:即补充定理\ref{cor70}.
			
			\begin{exer}\label{exer50}
				\hfill\par
				令$F$是$A$到$B$的对应,$F'$是$B$到$A$的对应.如果$(\forall x)(x\in A\Rightarrow F'(F(x))=\{x\})$、$(\forall y)(y\in B\Rightarrow F(F'(y))=\{y\})$,求证:$F$、$F'$均为双射,且$F'$为$F$的逆映射.
			\end{exer}
			证明:令$G$、$G'$分别是$F$、$F'$的图,假设$(x, y)\in G$、$(x, y') \in G$,则$y\in F(x)$,因此$F'(y)\subset F'(F(x))$.又因为$F(F'(y))=\{y\}$,故$F'(y)\neq \varnothing$,因此$F'(y)=\{x\}$,同理$F'(y')=\{x\}$,因此$F'(F(y))= F'(F(y'))$,因此$y=y'$,故$G$为函数图,同时,对于$x\in A$,由于$F'(F(x))\neq \varnothing$,因此$F(x)\neq \varnothing$,故$pr_1G=A$,因此,$F$为映射,同理$F'$为映射.根据定理\ref{theo20}得证.
			
			\begin{exer}\label{exer51}
				\hfill\par
				$f$为$A$到$B$的映射,$g$为$B$到$C$的映射,$h$为$C$到$D$的映射,且$g\circ f$和$h\circ g$为双射,求证:$f$、$g$、$h$为双射.
			\end{exer}
			证明:根据定理\ref{theo21}(3)、定理\ref{theo21}(4),$g$为双射.根据定理\ref{theo21}(5)、定理\ref{theo21}(6),$f$和$h$\\也是双射.
			
			\begin{exer}\label{exer52}
				\hfill\par
				$f$为$A$到$B$的映射,$g$为$B$到$C$的映射,$h$为$C$到$A$的映射,求证:如果$h\circ g\circ f$、$g\circ f\circ h$、$f\circ h\circ g$之中两个满射一个单射,或者两个单射一个满射,则$f$、$g$、$h$都是双射.
			\end{exer}
			证明:
			\par
			假设$h\circ g\circ f$、$g\circ f\circ h$为满射,$f\circ h\circ g$为单射,根据定理\ref{theo15}、定理\ref{theo21}(3)、定理\ref{theo21}(4),$g$为双射、$h$为满射、$h\circ g$为双射、$g\circ f$为满射,根据定理\ref{theo21}(5)、定理\ref{theo21}(6),$h$、$f$为双射.
			\par
			假设$h\circ g\circ f$、$g\circ f\circ h$为单射,$f\circ h\circ g$为满射,根据定理\ref{theo15}、定理\ref{theo21}(3)、定理\ref{theo21}(4),$f\circ h$为双射、$f$为双射、$g\circ f$为单射、$h$为单射,根据定理\ref{theo21}(5)、定理\ref{theo21}(6),$g$、$h$为双射.
			
			\begin{exer}\label{exer53}
				\hfill\par
				试找到以下推理的错误:令$N$为自然数集,$A$为满足$n>2$且存在不等于$0$的自然数$x$、$y$、$z$使$x^n+y^n=z^n$成立的不等于$0$的自然数$n$的集合,则$A$不为空集(即费马大定理为假).设$B=\{A\}$,$C=\{N\}$,由于$B$、$C$均为仅有一个元素的集合,因此存在$B$到$C$的双射$f$.因此$f(A)=N$,如果$A=\varnothing$,则$f(\varnothing)=\varnothing$,故$N=\varnothing$,矛盾.
			\end{exer}
			答:$f(\varnothing)=\varnothing$推理错误.混淆了$f(\varnothing)$与$f\langle\varnothing\rangle$,即将函数的值和在对应下的像混淆.
			\par
			注:习题\ref{exer53}涉及尚未介绍的“自然数”知识.

		\section{\texorpdfstring{集族的并集和交集(Réunion et intersection d'une \\famille d'ensembles)}{集族的并集和交集(Réunion et intersection d'une famille d'ensembles)}}
						
			\begin{cor}\label{cor89}
				\textbf{并集定理}
				\par
				令$(X_i)_{i\in I}$为集族,则$(\exists i)(i\in I\text{与}x\in X_i)$是$x$上的集合化公式.
			\end{cor}
			证明:由于$(\forall x)((i\in I\text{与}x\in X_i)\Rightarrow (x\in X_i))$,根据公理模式\ref{Sch5},$(\forall i)(\forall x)(\exists Z)((i\in I\text{与}x\in X_i)\Rightarrow (x\in Z))$,根据公理模式\ref{Sch8}得证.
			
			\begin{de}
				\textbf{集族的并集(réunion d'une famille)}
				\par
				令$(X_i)_{i\in I}$为集族,集合$\{x|(\exists i)(i\in I\text{与}x\in X_i)\}$称为该集族的并集,记作$\bigcup\limits_{i\in I}X_i$.
				
			\end{de}
			
			\begin{cor}\label{cor90}
				\hfill\par
				$E$的子集族的并集,是$E$的子集.
			\end{cor}
			证明:如果$i\in I$,则$X_i\subset E$,因此$x\in X_i\Rightarrow x\in E$,则$(\exists i)(i\in I\text{与}x\in X_i)\Rightarrow x\in E$,得证.
			
			\begin{cor}\label{cor91}
				\hfill\par
				$\bigcup\limits_{i\in \varnothing}X_i=\varnothing$.
			\end{cor}
			证明:$i\in \varnothing$为假,故$(\exists i)(i\in \varnothing\text{与}x\in X_i)$为假,得证.
			
			\begin{cor}\label{cor92}
				\textbf{交集定理}
				\par
				对于集族$(X_i)_{i\in I}$,如果$I\neq \varnothing$,则$(\forall i)(i\in I\Rightarrow x\in X_i)$是$x$上的集合化公式.
			\end{cor}
			证明:设$a\in I$,则$(\forall i)(i\in I\Rightarrow x\in X_i) \Rightarrow x\in X_a$,根据证明规则\ref{C52}得证.
						
			\begin{cor}\label{cor93}
				\textbf{空族的交集不存在}
				\par
				$\text{非}Coll_x(\forall i)(i\in \varnothing\Rightarrow x\in X_i)$.
			\end{cor}
			证明:$i\in \varnothing$为假,因此$i\in \varnothing\Rightarrow x\in X_i$为真,根据证明规则\ref{C13}可证.
			
			\begin{cor}\label{cor94}
				\textbf{子集族的交集存在}
				\par
				$(X_i)_{i\in I}$为$F$的子集族,则$x\in F\text{与}(\forall i)(i\in I\Rightarrow x\in X_i)$是$x$上的集合化公式.
			\end{cor}
			证明:根据证明规则\ref{C51}可证.

			\begin{de}
				\textbf{集族的交集(intersection d'une famille);子集族的交集(intersection d'une famille)}
				\hfill\par
				令$(X_i)_{i\in I}$为集族,且$I\neq \varnothing$,则集合$\{x|(\forall i)(i\in I\Rightarrow x\in X_i)\}$称为该集族的交集,记作$\bigcap\limits_{i\in I}X_i$.如果$(X_i)_{i\in I}$为$F$的子集族,则集合$\{x|x\in F\text{与}(\forall i)(i\in I\Rightarrow x\in X_i)\}$称为该子集族的交集,同样记作$\bigcap\limits_{i\in I}X_i$.
			\end{de}
			
			\begin{cor}\label{cor95}
				\textbf{空子集族的交集}
				\par
				$(X_i)i\in \varnothing$为$F$的子集族,则$\bigcap\limits_i\in \varnothing X_i=F$.
			\end{cor}
			证明:根据定义可证.
			
			\begin{cor}\label{cor96}
				\hfill\par
				(1)$E$的子集族的交集,是$E$的子集.
				\par
				(2)如果集族同时是E的子集族,且指标集不为空集,则该集族的交集也是该子集族的交集.
			\end{cor}
			证明:
			\par
			(1)$x\in E\text{与}(\forall i)(i\in I\Rightarrow x\in X_i)\Rightarrow x\in E$,得证.
			\par
			(2)根据补充定理\ref{cor96}(1)可证.

			\begin{theo}\label{theo23}
				\textbf{并集和交集的交换律}
				\par
				$(X_i)_{i\in I}$为集族,$f$为$K$到$I$的满射,则$\bigcup\limits_{k\in K}X_{f(k)}=\bigcup\limits_{i\in I}X_i$;如果$I\neq \varnothing$,则$\bigcap\limits_{k\in K}X_{f(k)}=\bigcap\limits_{i\in I}X_i$.如果$(X_i)_{i\in I}$为子集族,在$I=\varnothing$的情况下,$\bigcap\limits_{k\in K}X_{f(k)}=\bigcap\limits_{i\in I}X_i$也为真.
			\end{theo}
			证明:
			\par
			对于并集:
			\par
			设$x\in \bigcup\limits_{i\in I}X_i$,则存在$i\in I$,使$x\in X_i$,由于$f\langle K\rangle=I$,因此存在$k\in K$,使$i=f(k)$,故$x\in X_{f(k)}$,因此$x\in \bigcup\limits_{k\in K}X_{f(k)}$.
			\par
			反过来,设$x\in \bigcup\limits_{k\in K}X_{f(k)}$,因此存在$k\in K$,使$x\in X_{f(k)}$,设$i=f(k)$,则$x\in X_i$,因此$x\in \bigcup\limits_{i\in I}X_i$.
			\par
			对于交集:
			\par
			设$x\in \bigcap\limits_{i\in I}X_i$,对任意$k\in K$,都有$f(k)\in I$,即$x\in X_{f(k)}$,因此$x\in \bigcap\limits_{k\in K}X_{f(k)}$.
			\par
			反过来,设$x\in \bigcap\limits_{k\in K}X_{f(k)}$,$i\in I$,则存在在$k\in K$,使$i=f(k)$,因此$x\in X_i$,故$x\in \bigcap\limits_{i\in I}X_i$.
			\par
			设$(X_i)_{i\in I}$为$F$的子集族,在$I=\varnothing$的情况下,$f$为$K$到$I$的满射,则$K=\varnothing$,因此$\bigcap\limits_{k\in K}X_{f(k)}\\=F$,$\bigcap\limits_{i\in I}X_i=F$,得证.

			\begin{theo}\label{theo24}
				$(X_i)_{i\in I}$为集族,$(\forall i)(\forall k)(i\in I\text{与}k\in I\Rightarrow X_i=X_k)$,则$(\forall a)(a\in I\Rightarrow \bigcup\limits_{i\in I}X_i=X_a)$,当$I\neq \varnothing$时,$(\forall a)(a\in I\Rightarrow \bigcap\limits_{i\in I}X_i=X_a)$.
			\end{theo}
			证明:
			令$f$为常数映射$i\mapsto a(i\in I, a\in \{a\})$,其为$I$到$\{a\}$的满射,根据定理\ref{theo23}可证.

			\begin{de}
				\textbf{多个集合的并集(réunion d'ensembles),多个集合的交集(intersection d'ensembles)}
				\hfill\par
				集族$Id_F$的并集,也称为$F$的并集,记作$\bigcup\limits_{X\in F}X$;当$F\neq \varnothing$时,集族$Id_F$的交集,称为$F$的交集,记作$\bigcap\limits_{X\in F}X$.
			\end{de}

			\begin{cor}\label{cor97}
				\hfill\par
				(1)如果$(X_i)_{i\in I}$、$(Y_i)_{i\in I}$为集族,且$(\forall i)(i\in I\Rightarrow X_i\subset Y_i)$,则$\bigcup\limits_{i\in I}X_i\subset \bigcup\limits_{i\in I}Y_i$;如果$I\neq \varnothing$,则$\bigcap\limits_{i\in I}X_i\subset \bigcap\limits_{i\in I}Y_i$.
				\par
				(2)如果$(X_i)_{i\in I}$为集族,且$J\subset I$,则$\bigcup\limits_{i\in J}X_i\subset \bigcup\limits_{i\in I}X_i$;如果$J\neq \varnothing$,则$\bigcap\limits_{i\in I}X_i\subset \bigcap\limits_{i\in J}X_i$.
			\end{cor}
			证明:
			\par
			(1)设$x\in \bigcup\limits_{i\in I}X_i$,则存在$i\in I$,使$x\in X_i$,因此$x\in Y_i$,故$x\in \bigcup\limits_{i\in I}Y_i$.
			\par
			另一方面,设$x\in \bigcap\limits_{i\in I}X_i$,则对任意$i\in I$,$x\in X_i$,因此$x\in Y_i$,故$x\in \bigcap\limits_{i\in I}Y_i$.得证.
			\par
			(2)	设$x\in \bigcup\limits_{i\in J}X_i$,则存在$i\in J$,使$x\in X_i$.由于$J\subset I$,因此$i\in I$,故$x\in \bigcup\limits_{i\in I}X_i$.
			\par
			另一方面,设$x\in \bigcap\limits_{i\in I}X_i$,则对任意$i\in I$,$x\in X_i$.由于$J\subset I$,因此对任意$i\in J$,故$x\in \bigcap\limits_{i\in J}X_i$.得证.
			
			\begin{theo}
				\textbf{并集和交集的结合律}\label{theo25}
				\par
				$(X_i)_{i\in I}$为集族,其指标集$I=\bigcup\limits_{l\in L}J_l$,则$\bigcup\limits_{i\in I}X_i=\bigcup\limits_{l\in L}(\bigcup\limits_{i\in J_I}X_i)$,如果$L\neq \varnothing$,并且对任意$l\in L$,$J_l\neq \varnothing$,则$\bigcap\limits_{i\in I}X_i=\bigcap\limits_{l\in L}(\bigcap\limits_{i\in J_I}X_i)$.
			\end{theo}
			证明:
			\par
			对于并集:
			\par
			设$x\in \bigcup\limits_{i\in I}X_i$,则存在$i\in I$,使$x\in X_i$,由于$I=\bigcup\limits_{l\in L}J_l$,因此存在$l\in L$,使$i\in J_l$,因此$x\in \bigcup\limits_{i\in J_I}X_i$,故$\bigcup\limits_{l\in L}(\bigcup\limits_{i\in J_I}X_i)$.
			\par
			反过来,设$x\in \bigcup\limits_{i\in I}X_i$,则存在$l\in L$,使$x\in \bigcup\limits_{i\in J_I}X_i$,因此存在$i\in J_l$,使$x\in X_i$,由于$I=\bigcup\limits_{l\in L}J_l$,因此$i\in I$,故$x\in \bigcup\limits_{i\in I}X_i$.
			\par
			对于交集:
			\par
			设$x\in \bigcap\limits_{i\in I}X_i$,则对任意$i\in I$,$x\in X_i$,对任意$l\in L$,由于$J_l\subset I$,因此$x\in \bigcap\limits_{i\in J_I}X_i$,故$x\in \bigcap\limits_{l\in L}(\bigcap\limits_{i\in J_I}X_i)$.
			\par
			反过来,设$x\in \bigcap\limits_{l\in L}(\bigcap\limits_{i\in J_I}X_i)$,则对任意$l\in L$,$x\in \bigcap\limits_{i\in J_I}X_i$,由于$I=\bigcup\limits_{l\in L}J_l$,因此对任意$i\in I$,存在$l\in L$,使$i\in J_l$,因此$x\in X_i$,故$x\in \bigcap\limits_{i\in I}X_i$.
			
			\begin{cor}\label{cor98}
				\hfill\par
				$(X_i)_{i\in I}$为子集族,其指标集$I=\bigcup\limits_{l\in L}J_l$,则$\bigcap\limits_{i\in I}X_i=\bigcap\limits_{l\in L}(\bigcap\limits_{i\in J_I}X_i)$.
			\end{cor}
			证明:
			\par
			设$(X_i)_{i\in I}$为$F$的子集族.
			\par
			如果$L\neq \varnothing$,并且对所有$l\in L$,$J_l\neq \varnothing$,根据定理\ref{theo25}、补充定理\ref{cor96}(2)可证.
			\par
			如果$L=\varnothing$,根据补充定理\ref{cor91},$I=\varnothing$,故$\bigcap\limits_{i\in I}X_i=F$,$\bigcap\limits_{l\in L}(\bigcap\limits_{i\in J_I}X_i)=F$.
			\par
			如果$L\neq \varnothing$,但存在$l$,使$J_L=\varnothing$,则$\bigcap\limits_{i\in J_I}X_i=F$.类似定理\ref{theo25}仍然可证.

			\begin{theo}\label{theo26}
				\hfill\par
				$(X_i)_{i\in I}$为$A$的子集族,$F$为$A$到$B$的对应,则$F\langle\bigcup\limits_{i\in I}X_i\rangle= \bigcup\limits_{i\in I}F\langle X_i\rangle $,$F\langle\bigcap\limits_{i\in I}X_i\rangle\subset \\\bigcap\limits_{i\in I}F\langle X_i\rangle$.
			\end{theo}
			证明:
			\par
			$(\exists x)(x\in \bigcup\limits_{i\in I}X_i\text{与}y\in F(x))\Leftrightarrow (\exists x)(\exists i)(i\in I\text{与}x\in X_i\text{与}y\in F(x))$,等价于$(\exists i)(i\in I\text{与}y\in F(X_i))$,因此$y\in \bigcup\limits_{i\in I}F\langle X_i\rangle$.
			\par
			对任意$i\in I$,$\bigcap\limits_{i\in I}X_i\subset X_i$,根据定理\ref{theo12},$F\langle\bigcup\limits_{i\in I}X_i\rangle\subset F\langle X_i\rangle $,故$F\langle\bigcap\limits_{i\in I}X_i\rangle \subset \bigcap\limits_{i\in I}F\langle X_i\rangle$.

			\begin{theo}\label{theo27}
				\hfill\par
				$f$为$A$到$B$的映射,$(Y_i)_{i\in I}$为$B$的子集族,则$f^{-1}\langle\bigcap\limits_{i\in I}Y_i\rangle= \bigcap\limits_{i\in I}f^{-1}\langle Y_i\rangle$.
			\end{theo}
			证明:设$x\in \bigcap\limits_{i\in I}f^{-1}\langle Y_i\rangle $,根据补充定理\ref{cor70}(1),$x\in A$,且对任意$i\in I$,$f(x)\in Y_i$,因此$f(x)\in \bigcap\limits_{i\in I}Y_i$,故$x\in f^{-1}\langle\bigcap\limits_{i\in I}Y_i\rangle$.另一方面,根据定理\ref{theo26},$f^{-1}\langle\bigcap\limits_{i\in I}Y_i\rangle\subset \bigcap\limits_{i\in I}f^{-1}\langle Y_i\rangle$.得证.

			\begin{theo}\label{theo28}
				\hfill\par
				$(X_i)_{i\in I}$为$A$的子集族,$f$为$A$到$B$的单射,且$I\neq \varnothing$,则$f\langle\bigcap\limits_{i\in I}X_i\rangle=\bigcap\limits_{i\in I}f\langle X_i\rangle$.
			\end{theo}
			证明:设$f$的图为$F$,令$i=(\Delta_f\langle A \rangle , f\langle A \rangle , B)$,$g=(F, A, f\langle A \rangle )$.则$g$为双射,$i$为单射,$f=i\circ g$.令$h$为$g$的逆映射,则对任意$X\subset A$,$f\langle X \rangle =h^{-1}\langle X \rangle $,根据定理\ref{theo27}得证.

			\begin{theo}\label{theo29}
				\hfill\par
				设$(X_i)_{i\in I}$为$E$的子集族,则$\complement_E(\bigcup\limits_{i\in I}X_i)= \bigcap\limits_{i\in I}(\complement_EX_i)$,$\complement_E(\bigcap\limits_{i\in I}X_i)=\bigcup\limits_{i\in I}(\complement_EX_i)$.
			\end{theo}
			证明:设$x\in \complement_E(\bigcup\limits_{i\in I}X_i)$,则$x\in E$,且对任意$i\in I$,$x\notin X_i$,$x\in \complement_EX_i$,因此$x\in \bigcap\limits_{i\in I}(\complement_EX_i)$.
			\par
			反过来,设$x\in \bigcap\limits_{i\in I}(\complement_EX_i)$,根据补充定理\ref{cor96}(1),$x\in E$,同时,若$x\in \bigcup\limits_{i\in I}X_i$,则存在$i$,使$x\in X_i$,与$x\in \bigcap\limits_{i\in I}(\complement_EX_i)$矛盾,故$x\notin \bigcup\limits_{i\in I}X_i$,因此,$x\in \complement_E(\bigcup\limits_{i\in I}X_i)$.故$\complement_E(\bigcup\limits_{i\in I}X_i)= \bigcap\limits_{i\in I}(\complement_EX_i)$.根据补充定理\ref{cor12},$\complement_E(\bigcap\limits_{i\in I}X_i)= \bigcup\limits_{i\in I}(\complement_EX_i)$.

			\begin{de}
				\textbf{两个集合的并集与交集(réunion et intersection de deux ensembles),迹(trace)}
				\par
				$\bigcup\limits_{X\in \{A, B\}}X$称为$A$和$B$的并集,记作$A\cup B$,$\bigcap\limits_{X\in \{A, B\}}X$称为$A$和$B$的交集,记作$A\cap B$.$A\cap B$又称$A$在$B$上的迹.
			\end{de}
			
			\begin{de}
				\textbf{三元集合(ensemble à trois éléments),四元集合(ensemble à quatre éléments)}
				\par
				$\{x, y\}\cup\{z\}$称为三元集合,记作$\{x, y, z\}$;$\{x, y, z\}\cup\{t\}$称为四元集合,记作$\{x, y, z, t\}$.
			\end{de}
			
			\begin{cor}\label{cor99}
				\hfill\par
				$A\cup B=\{x|x\in A\text{或}x\in B\}$,$A\cap B=\{x|x\in A\text{与}x\in B\}$.
			\end{cor}
			证明:$A\cup B=\{x|(\exists i)(i\in \{A, B\}\text{与}x\in i)\}$,等于$\{x|(\exists i)((i=A\text{与}x\in A)\text{或}(i=B\text{或}x\in B))\}$,等于$\{x| x\in A\text{或}x\in B\}$.同理可证$A\cap B=\{x|x\in A\text{与}x\in B\}$.			
			
			\begin{cor}\label{cor100}
				\hfill\par
				(1)$A\cup B=B\cup A$;
				\par
				(2)$A\cap B=B\cap A$;
				\par
				(3)$(A\cup B)\cup C=A\cup (B\cup C)$;
				\par
				(4)$(A\cap B)\cap C=A\cap (B\cap C)$;
				\par
				(5)$(A\cup B)\cap C=(A\cap B)\cup (B\cap C)$;
				\par
				(6)$(A\cap B)\cup C=(A\cup B)\cap (B\cup C)$;
				\par
				(7)$\complement_E(A\cup B)=(\complement_EA)\cap (\complement_EB)$;
				\par
				(8)$\complement_E(A\cap B)=(\complement_EA)\cup (\complement_EB)$;
				\par
				(9)$A\cup \complement_EA=E$;
				\par
				(10)$A\cap \complement_EA=\varnothing$.
			\end{cor}
			证明:根据补充定理\ref{cor99}可证.
						
			\begin{cor}\label{cor101}
				\hfill\par
				(1)$A\subset B\Leftrightarrow A\cup B=B$.
				\par
				(2)$A\subset B\Leftrightarrow A\cap B=A$.
			\end{cor}
			证明:根据补充定理\ref{cor99}可证.
			
			\begin{cor}\label{cor102}
				\hfill\par
				令$G$为$E$到$F$的对应,$A$和$B$是$E$的子集,则$G(A\cup B)=G(A)\cup G(B)$,$G(A\cap B)\subset G(A)\cap G(B)$.
			\end{cor}
			证明:根据定理\ref{theo26}可证.
						
			\begin{cor}\label{cor103}
				\hfill\par
				令$f$为$F$到$E$的映射,$A$和$B$是$E$的子集,则$f^{-1}(A\cap B)\subset f^{-1}(A)\cap f^{-1}(B)$.
			\end{cor}
			证明:根据定理\ref{theo27}可证.
			
			\begin{cor}\label{cor104}
				\hfill\par
				令$G$为图,$X\subset pr_1G$,$pr_2G\subset B$,则$pr_1(G\cap X\times B)=X$.
			\end{cor}
			证明:$(x, y)\in (G\cap X\times B)\Leftrightarrow (x, y)\in G\text{与}x\in X\text{与}y\in B$,由于$pr_2G\subset B$,故$(x, y)\in G\Rightarrow y\in B$,因此$(x, y)\in (G\cap X\times B)\Leftrightarrow (x, y)\in G\text{与}x\in X$,故$(\exists y)((x, y)\in (G\cap X\times B))\Leftrightarrow (\exists y)((x, y)\in G)\text{与}x\in X$,又因为$X\subset pr_1G$,故$(\exists y)((x, y)\in (G\cap X\times B))\Leftrightarrow x\in X$,得证.

			\begin{de}
				\textbf{通过子集导出的函数(fonction déduite par passage au le sous-ensemble),通过子集导出的映射(application déduite par passage au le sous-ensemble)}
				\par
				令函数$f=(F, A, B)$,$X\subset A$,则$(F\cap X\times B, X, B)$称为$f$通过$A$的子集$X$导出的函数,或称为$f$通过$A$的子集$X$导出的映射.如果$pr_2F\subset Y$,则$(F, A, Y)$称为$f$通过$B$的子集$Y$导出的函数,或称为$f$通过$B$的子集$Y$导出的映射.如果$(pr_2F\cap X\times B)\subset Z$,则$(F\cap X\times B, X, Z)$称为$f$通过$A$的子集$X$和$B$的子集$Z$导出的函数,或称为$f$通过$A$的子集$X$和$B$的子集$Z$导出的映射.
			\end{de}
						
			\begin{cor}\label{cor105}
				\hfill\par
				令函数$f=(F, A, B)$,$X\subset A$,则$(f\text{通过}A\text{的子集}X\text{导出的函数})=f|X$,并且$f$通过$A$的子集$X$导出的函数和$f$在$X$上重合.
			\end{cor}
			证明:设$f|X$的图为$G$,根据补充定理\ref{cor64}(1),$x\in X\text{与}y=f(x)\Leftrightarrow (x, y)\in G$,$x\in A\text{与}y=f(x)\Leftrightarrow (x, y)\in F$.根据补充定理\ref{cor23}(2),$x\in X\text{与}y\in B\Leftrightarrow (x, y)\in X\times B$.根据补充定理\ref{cor64}(3),$x\in A\Rightarrow f(x)\in B$.因此$(x, y)\in (F\cap X\times B)\Leftrightarrow x\in A\text{与}y=f(x)\text{与}x\in X\text{与}y\in B$,等价于$x\in X\text{与}y=f(x)$,得证.
			
			\begin{cor}\label{cor106}
				\hfill\par
				令函数$f=(F, A, B)$,$X\subset A$,$pr_2F\subset Y$,$(pr_2F\cap X\times B)\subset Z$,如果$f$为单射,则$f$通过$A$的子集$X$导出的函数、$f$通过$B$的子集$Y$导出的函数、$f$通过$A$的子集$X$和$B$的子集$Z$导出的函数均为单射.
			\end{cor}
			证明:根据补充定理\ref{cor105}、补充定理\ref{cor72}可证.

			\begin{theo}\label{theo30}
				\hfill\par
				$f$为$A$到$B$的映射,$Y\subset B$,则$f^{-1}(B-Y)=f^{-1}(B)-f^{-1}(Y)$.
			\end{theo}
			证明:设$x\in f^{-1}(B-Y)$,根据补充定理\ref{cor70}(1),$x\in A$,且$f(x)\in B-Y$.因此$f(x)\in B$,且$f(x)\notin Y$,根据补充定理\ref{cor70}(1),$x\in(f^{-1}(B)-f^{-1}(Y))$.反过来,$x\in f^{-1}(B)-f^{-1}(Y)$,根据补充定理\ref{cor70}(1),$x\in $A,$f(x)\in B$,且$f(x)\notin Y$,因此$f(x)\in(B-Y)$,根据补充定理\ref{cor70}(1),$x\in f^{-1}(B-Y)$.
			
			\begin{theo}\label{theo31}
				\hfill\par
				$f$为$A$到$B$的单射,$X\subset A$,则$f(A-X)=f-(A)-f(X)$.
			\end{theo}
			证明:设$f$的图为$F$,令$i=(\Delta_{f\langle A \rangle }, f\langle A \rangle , B)$,$g=(F, A, f\langle A \rangle )$.则$g$为双射,$i$为单射,$f=i\circ g$.令$h$为$g$的逆映射,则对任意$X\subset A$,$f\langle X \rangle =h^{-1}\langle X \rangle $,根据定理\ref{theo30}得证.

			\begin{de}
				\textbf{覆盖(recouvrement),更细的覆盖(recouvrement plus fin)}
				\par
				令$(X_i)_{i\in I}$为集族,如果$E=\bigcup\limits_{i\in I}X_i$,则称$(X_i)_{i\in I}$为$E$的覆盖.
				\par
				如果$(X_i)_{i\in I}$和$(Y_k)_{k\in K}$都是$E$的覆盖,并且$(\forall k)(k\in K\Rightarrow (\exists i)(Y_k\subset X_i))$,则称$(Y_k)_{k\in K}$\\为比$(X_i)_{i\in I}$更细的覆盖.
			\end{de}
			注:
			\par
			从上下文来看,原书对覆盖的定义有误.
			\par
			在原书中,“更细”这个概念包括与自身相等的情况,即一个覆盖比自身更细.
			
			
			\begin{cor}\label{cor107}
				\hfill\par
				覆盖$R$比覆盖$R'$更细,覆盖$R'$比覆盖$R''$更细,则覆盖$R$比覆盖$R''$更细.
			\end{cor}
			证明:根据定义可证.

			\begin{cor}\label{cor108}
				\hfill\par
				$(X_i)_{i\in I}$和$(X_i)_{i\in J}$都是$E$的覆盖,且$J\subset I$,则$(X_i)_{i\in J}$比$(X_i)_{i\in I}$更细.
			\end{cor}
			证明:根据定义可证.
						
			\begin{cor}\label{cor109}
				\hfill\par
				$I\neq \varnothing$,$K\neq \varnothing$,$(X_i)_{i\in I}$和$(Y_k)_{k\in K}$都是$E$的覆盖,则$(X_i\cap Y_k)_{(i, k)\in I\times K}$也是$E$的覆盖,并且比$(X_i)_{i\in I}$和$(Y_k)_{k\in K}$更细.
			\end{cor}
			证明:设$x\in E$,则存在$i$、$k$,使$x\in X_i$,$x\in Y_k$,则$x\in (X_i\cap Y_k)$,且$(i, k)\in I\times K$,故$x\in (X_i\cap Y_k)_{(i, k)}\in I\times K$.如果$k\in K$,则$X_i\cap Y_k\subset Y_k$,故$(X_i\cap Y_k)_{(i, k)}\in I\times K$比$(Y_k)_{k\in K}$更细,同理可证比$(X_i)_{i\in I}$更细.
			
			\begin{cor}\label{cor110}
				\hfill\par
				$I\neq \varnothing$,$K\neq \varnothing$,$(X_i)_{i\in I}$、$(Y_k)_{k\in K}$、$(Z_l)_{l\in L}$都是$E$的覆盖,如果对任意$l\in L$,均存在$i$、$k$,使$Z_l\subset X_i$,$Z_l\subset Y_k$,则$(Z_l)_{l\in L}$是比$(X_i\cap Y_k)_{(i, k)\in I\times K}$更细的覆盖.
			\end{cor}
			证明:对任意$l\in L$,均存在$i$、$k$,使$Z_l\subset X_i$,$Z_l\subset Y_k$,则$Z_l\subset X_i\cap Y_k$,得证.

			\begin{de}
				\textbf{覆盖的像(image du recouvrement),覆盖的原像(image réciproque du recouvrement)}
				\par
				如果$(X_i)_{i\in I}$为$A$的覆盖,$f$为$A$到$B$的满射,则$(f\langle X_i\rangle)_{i\in I}$称为$(X_i)_{i\in I}$在$f$下的像.
				\par
				如果$(X_i)_{i\in I}$是$A$的覆盖,$g$是$C$到$A$的映射,则$(g^{-1}\langle X_i\rangle)_{i\in I}$称为$(X_i)_{i\in I}$在$g$下的原像.
			\end{de}
				
			\begin{cor}\label{cor111}
				\textbf{覆盖和像和原像都是覆盖}
				\par
				如果$(X_i)_{i\in I}$为$A$的覆盖,$f$为$A$到$B$的满射,$g$为$C$到$A$的映射,则$(X_i)_{i\in I}$在$f$下的像是$B$\\的覆盖,$(X_i)_{i\in I}$在$g$下的原像是$C$的覆盖.
			\end{cor}
			证明:根据定理\ref{theo26}可证.

			\begin{de}
				\textbf{覆盖的乘积(produit des recouvrements)}
				\par
				$(X_i)_{i\in I}$为$E$的覆盖,$(Y_k)_{k\in K}$为$F$的覆盖,则$(X_i\times Y_k)_{(i, k)\in I\times K}$称为$E$的覆盖$(X_i)_{i\in I}$和$F$的覆盖$(Y_k)_{k\in K}$的乘积.
			\end{de}
			
			\begin{cor}\label{cor112}
				\hfill\par
				$(X_i)_{i\in I}$为$E$的覆盖,$(Y_k)_{k\in K}$为$F$的覆盖,则两个覆盖的乘积是$E\times F$的覆盖.
			\end{cor}
			证明:设$z\in E\times F$,则$pr_1z\in E$,$pr_2z\in F$.存在$i\in I$,使$pr_1z\in X_i$,存在$k\in K$,使$pr_2z\in Y_k$.因此$z\in X_i\times Y_k$,得证.
			
			\begin{theo}\label{theo32}
				\hfill\par
				(1)$(X_i)_{i\in I}$是$E$的覆盖,$f$、$g$是两个定义域为$E$的函数,如果对任意$i\in I$,$f$和$g$均在$X_i$上重合,则$f$和$g$在E上重合.
				\par
				(2)$(X_i)_{i\in I}$是$E$的覆盖,$(f_i)_{i\in I}$是集族,其中,对任意$i\in I$,$f_i$均为函数,且定义域为$X_i$,到达域为$F$.如果$(\forall i)(\forall k)((i, k)\in (I\times I)\Rightarrow (f_i\text{和}f_k\text{在}x_i\cap X_k\text{上重合})$),则存在唯一的函数$f$,以$E$为定义域,以$F$为到达域,且当$i\in I$时,是$f_i$在$E$上的延拓.
			\end{theo}
			证明:
			\par
			(1)	设$x\in E$,则存在$i\in I$,使$x\in X_i$,故$f(x)=g(x)$,得证.
			\par
			(2)	令$f_i$的图为$G_i$,$G=\bigcup\limits_{i\in I}G_i$,设$(x, y)\in G$,$(x, y')\in G$,则存在$i\in I$、$k\in I$,使$(x, y)\in G_i$,$(x, y')\in G_k$.因此$x\in X_i\cap X_k$,$y=f_i(x)$,$y'=f_k(x)$,因此$y=y'$,故$G$为函数图.
			又因为$pr_1G=(\exists y)((x, y)\in \bigcup\limits_{i\in I}G_i)$,即$(\exists y)(\exists i)(i\in I\text{与}(x, y)\in G_i)$,即$(\exists i)(i\in I\text{与}(\exists y)(x, y)\in G_i)$,即$\bigcup\limits_{i\in I}pr_1G_i$,即$\bigcup\limits_{i\in I}X_i$,即E.令$(G, E, F)$为函数$f$,当$i\in I$时,$(x, y) \in G_i\Rightarrow (x, y) \in G$,因此当$x\in X_i$时,$f_i(x)=f(x)$,即f是$f_i$在$E$上的延拓.因此,函数$f$即为所求.
			同时,根据定理\ref{theo32}(1),$f$具有唯一性.得证.
			\par
			注:因原书对覆盖的定义更正,本定理也做相应更正.			

			\begin{de}
				\textbf{不相交的集合(ensembles disjoint),相交的集合(ensembles qui rencontrent),两两不相交的集合(ensembles mutuellement disjoint/ensembles deux à deux disjoint)}
				\par
				如果$A\cap B=\varnothing$,则称$A$和$B$不相交.如果$A\cap B\neq \varnothing$,则称$A$和$B$相交.对于集族$(X_i)_{i\in I}$,如果$(\forall i)(\forall k)(i\in I\text{与}k\in I\text{与}i\neq k\Rightarrow X_i\cap X_k=\varnothing)$,则称该集族两两不相交.
			\end{de}
						
			\begin{cor}\label{cor113}
				\hfill\par
				$f$为$A$到$B$的映射,$(Y_i)_{i\in I}$为$B$的子集族且两两不相交,则$(f^{-1}\langle Y_i\rangle)_{i\in I}$为$A$的子集族且两两不相交.
			\end{cor}
			证明:根据定理\ref{theo27}可证.		
			
			\begin{theo}\label{theo33}
				\hfill\par
				$(X_i)_{i\in I}$是两两不相交的集族,$(f_i)_{i\in I}$是函数族,且定义域为$X_i$,到达域为$F$,则存在唯一的函数$f$,以$\bigcup\limits_{i\in I}X_i$为定义域,以$F$为到达域,且当$i\in I$时,是$f_i$在$\bigcup\limits_{i\in I}X_i$上的延拓.
			\end{theo}
			证明:根据定理\ref{theo32}(2)可证.

			\begin{de}
				\textbf{划分(parition)}
				\par
				如果一个两两不相交的集族是$E$的覆盖,则称其为$E$的划分.
			\end{de}
						
			\begin{cor}\label{cor114}
				\hfill\par
				(1)$E$的划分的并集是$E$.
				\par
				(2)“$\Delta_G\text{为}E\text{的划分}$”是$G$上的集合化公式.
			\end{cor}
			证明:
			\par
			(1)根据定义可证.
			\par
			(2)如果$\Delta_G$为$E$的划分,根据补充定理\ref{cor114}(1),对任意$x\in G$,$x\subset E$,故$G\subset \mathcal{P}(E)$,根据证明规则\ref{C52}可证.
			\par
			注:集合$\{G|\Delta_G\text{为}E\text{的划分}\}$的与严肃数目.为$E$的划分数目(仅指标集不同的划分,为同一个划分).
			\par
			
			\begin{cor}\label{cor115}
				\hfill\par
				$f$为$E$到$F$的映射,则集族$(f^{-1}(y)){y\in f\langle E \rangle }$是$E$的划分.
			\end{cor}
			证明:设$x\in f^{-1}(y_1)$、$x\in f^{-1}(y_2)$,根据补充定理\ref{cor70}(3),$f(x)=y_1$,$f(x)=y_2$,因此$y_1=y_2$,故集族$(f^{-1}(y))y\in f\langle E \rangle $两两不相交.对任意$x\in f^{-1}(y)$,根据补充定理\ref{cor70}(3),$x\in E$,
			\par
			反过来,如果$x\in E$,根据补充定理\ref{cor70}(4),$x\in f^{-1}(f(x))$,因此$\bigcup\limits_{y\in f\langle E \rangle }(f^{-1}(y))=E$.				
						
			\begin{theo}\label{theo34}
				\hfill\par
				对于集族$(X_i)_{i\in I}$,存在满足下列性质的$X$:存在两两不相交的集族$(X'i)_{i\in I}$,使$X=\bigcup\limits_{i\in I}({X'}_i)_{i\in I}$,并且,对任意$i\in I$,存在$X_i$到${X'}_i$的双射.
			\end{theo}
			证明:令$A=\bigcup\limits_{i\in I}X_i$.由于$(z\text{为有序对})\text{与}pr_1z\in X_i\text{与}pr_2z=i \Rightarrow z\in A\times I$,因此其为集合化公式,令${X'}_i$为$\{z|(z\text{为有序对})\text{与}pr_1z\in X_i\text{与}pr_2z=i\}$,如果$i\in I$,则$x\mapsto (x, i)(x\in X_i)$为$X_i$到${X'}_i$的双射,且当$i\neq k$时,假设$z\in ({X'}_i\cap X_k')$,则$pr_1z=i$,$pr_1z=k$,矛盾,故${X'}_i\cap X_k'=\varnothing$,因此,$({X'}_i)_{i\in I}$两两不相交.

			\begin{de}
				\textbf{集族的和(somme d'une famille),到和的规范映射(application canonique dans somme)}
				\par
				$(X_i)_{i\in I}$为集族,则称$\bigcup\limits_{i\in I}(X_i\times \{i\})_{i\in I}$为其和.对任意$i\in I$,映射$x\mapsto (x, i)(x\in X_i)$称为$X_i$到$(X_i)_{i\in I}$的和的规范映射.
			\end{de}

			\begin{de}
				\textbf{集合和单元素集合的和(somme d'un ensemble et un ensemble à un seul élément),将元素添加到集合得到的集合(ensemble obtenu par adjonction d'un élément à un ensemble)}
				\par
				如果$a\notin X$,则$X\cup\{a\}$称为$X$和$\{a\}$的和,或称为将$a$添加到$X$得到的集合.
			\end{de}
						
			\begin{theo}\label{theo35}
				\hfill\par
				$(X_i)_{i\in I}$为两两不相交的集族,$A$为其并,$S$为其和,则存在$A$到$S$的双射.
			\end{theo}
			证明:对任意$i\in I$,$x\mapsto (x, i)(x\in X_i)$为$X_i$到$X_i\times \{i\}$的双射,根据定理\ref{theo33},存在唯一的函数$f$,以$A$为定义域,以$A\times I$为到达域,且当$i\in I$时,是$x\mapsto (x, i)(x\in X_i)$在$A$上的延拓.$f$即为$A$到$S$的双射.
			
			\begin{cor}\label{cor116}
				\hfill\par
				(1)$(X_i)_{i\in I}$为集族,$A$为其并,$S$为其和,则存在$S$到$A$的满射,存在$A$到$S$的单射.
				\par
				(2)$(X_i)_{i\in I}$为集族,$S$为其和,对任意$i\in I$,令$f_i$为$X_i$到$S$的规范映射,$Y_i=f_i\langle X_i\rangle$,则$(Y_i)_{i\in I}$为$S$的划分.
				\par
				(3)$(X_i)_{i\in I}$为集族,对任意$i\in I$,令$f_i$为$X_i$到$S$的规范映射,$Y_i=f_i\langle X_i\rangle$,则对任意$i\in I$,$x\mapsto (x, i)(x\in X_i)$为$X_i$到$Y_i$的双射.
				\par
				(4)$(X_i)_{i\in \varnothing}$的和为$\varnothing$.
				\par
				(5)$(\varnothing)_{i\in I}$的和为$\varnothing$.
			\end{cor}
			证明:
			\par
			(1)	$z\mapsto pr_1z(z\in S)$为$S$到$A$的满射.其右逆为$A$到$S$的单射.
			\par
			(2)	根据定义可证.
			\par
			(3)	根据定义可证.
			\par
			(4)	根据定义可证.
			\par
			(5)	根据定义可证.

			\begin{exer}\label{exer54}
				\hfill\par
				$G$为图,求证以下三个公式等价:
				\par
				公式一:$G$为函数图;
				\par
				公式二:$G^{-1}(X\cap Y)=G^{-1}(X)\cap G^{-1}(Y)$;
				\par
				公式三:$X\cap Y=\varnothing\Rightarrow G^{-1}(X)\cap G^{-1}(Y)=\varnothing$.
			\end{exer}
			证明:
			\par
			($G^{-1}(X\cap Y)= G^{-1}(X)\cap G^{-1}(Y))\Leftrightarrow ((\exists y)((x, y)\in G\text{与}y\in X)\text{与}(\exists y)((x, y)\in G\text{与}y\in Y)\Leftrightarrow (\exists y) ((x, y)\in G\text{与}y\in X\text{与}y\in Y))$.如果$G$为函数图,则等价于$f(x)\in X\text{与}x\in pr_1G\text{与}f(x)\in Y\Leftrightarrow f(x)\in X\text{与}x\in pr_1G\text{与}f(x)\in Y$,故公式一$\Rightarrow$公式二.
			\par
			$X\cap Y=\varnothing$,则$G^{-1}(X\cap Y)=\varnothing$,如果$G^{-1}(X\cap Y)=G^{-1}(X)\cap G^{-1}(Y)$,则$G^{-1}(X)\cap G^{-1}(Y)=\varnothing$,即公式二$\Rightarrow$公式三.
			\par
			如果$X\cap Y=\varnothing\Rightarrow G^{-1}(X)\cap G^{-1}(Y)=\varnothing$,设$(x, y)\in G$、$(x, y')\in G$,假设$y\neq y'$,则$\{y\}\cap \{y'\}=\varnothing$,则$G^{-1}(\{y\})\cap G^{-1}(\{y'\})=\varnothing$,但$x\in G^{-1}(\{y\})$、$x\in G^{-1}(\{y'\})$,矛盾,故公式三$\Rightarrow$公式一.
			
			\begin{exer}\label{exer55}
				\hfill\par
				$G$为图,求证:
				\par
				(1)$G(X)=pr_2(G\cap(X\times pr_2G))$;
				\par
				(2)$G(X)=G(X\cap pr_1G)$.
			\end{exer}
			证明:
			\par
			(1)$pr_2(G\cap(X\times pr_2G))=\{y|(\exists x)((x, y)\in G\text{与}x\in X\text{与}y\in pr_2G)\}$,根据补充定理\ref{cor29},$(x, y)\in G\Rightarrow y\in pr_2G$,故其等于$\{y|(\exists x)((x, y)\in G\text{与}x\in X)\}$,即$G(X)$.
			\par
			(2)$G(X\cap pr_1G) =\{y|(\exists x)((x, y)\in G\text{与}x\in X\text{与}x\in pr_1G)\}$,根据补充定理\ref{cor29},$(x, y)\in G\Rightarrow x\in pr_1G$,故其等于$\{y|(\exists x)((x, y)\in G\text{与}x\in X)\}$,即$G(X)$.
			
			\begin{exer}\label{exer56}
				\hfill\par
				求证:如果$Y\cap Y'=\varnothing$,则$(Y'\times Z)\circ (X\times Y)=\varnothing$;如果$Y\cap Y'\neq \varnothing$,则$(Y'\times Z)\circ (X\times Y)=X\times Z$.
			\end{exer}
			证明:$(x, z)\in (Y'\times Z)\circ (X\times Y)\Leftrightarrow (\exists y)(x\in X\text{与}y\in Y\text{与}y\in Y'\text{与}z\in Z)$,根据补充定理\ref{cor23}(2),等价于$(x\in X\text{与}z\in Z)\text{与}(\exists y)(y\in (Y\cap Y'))$,等价于$(x, z)\in X\times Z\text{与}(\exists y)(y\in (Y\cap Y'))$.得证.
			
			\begin{exer}\label{exer57}
				\hfill\par
				$(G_i)_{i\in I}$为图族,求证:对任意$X$,$(\bigcup\limits_{i\in I}G_i)\langle X \rangle =\bigcup\limits_{i\in I}G_i\langle X \rangle $,对任意x,$(\bigcap\limits_{i\in I}G_i)\langle \{x\} \rangle =\bigcap\limits_{i\in I}G_i\langle \{x\} \rangle $,并给出图$G$、$H$的例子,使$G\langle X \rangle \cap H\langle X \rangle \neq (G\cap H)\langle X \rangle $.
			\end{exer}
			证明:
			\par
			对于并集:
			\par
			若$y\in (\bigcup\limits_{i\in I}G_i)\langle X \rangle $,则存在$x\in X$,使$(x, y)\in \bigcup\limits_{i\in I}G_i$,故存在$i\in I$,使$(x, y)\in G_i$,因此$y\in G_i\langle X \rangle $,故$y\in \bigcup\limits_{i\in I}G_i\langle X \rangle $.
			\par
			反过来,若$y\in \bigcup\limits_{i\in I}G_i\langle X \rangle $,则存在$i\in I$,使$y\in G_i\langle X \rangle $,故存在$x\in X$,使$(x, y)\in G_i$,因此$(x, y)\in \bigcup\limits_{i\in I}G_i$,故$y\in (\bigcup\limits_{i\in I}G_i)\langle X \rangle $.
			\par
			对于交集:
			\par
			若$y\in (\bigcap\limits_{i\in I}G_i)\langle \{x\} \rangle $,则$(x, y)\in \bigcap\limits_{i\in I}G_i$,故对任意$i\in I$,$(x, y)\in G_i$,因此$y\in G_i\langle \{x\} \rangle $,故$y\in \bigcap\limits_{i\in I}G_i\langle \{x\} \rangle $.
			\par
			反过来,若$y\in \bigcap\limits_{i\in I}G_i\langle \{x\} \rangle $,则对任意$i\in I$,$y\in G_i\langle \{x\} \rangle $,故$(x, y)\in G_i$,则$(x, y)\in \bigcap\limits_{i\in I}G_i$,因此$y\in (\bigcap\limits_{i\in I}G_i)\langle \{x\} \rangle $.
			\par
			设$a$、$b$、$c$互不相等,令$G=\{(b, a), (a, b), (b, c)\}$,$H=\{(b, b), (b, a), (b, c)\}$,$X=\{a, b\}$,则$G\langle X \rangle \cap H\langle X \rangle =\{a, b, c\}$,$(G\cap H)\langle X \rangle =\{a, c\}$.
			
			\begin{exer}\label{exer58}
				\hfill\par
				$(G_i)_{i\in I}$为图族,$H$为图,求证:$(\bigcup\limits_{i\in I}G_i)\circ H=\bigcup\limits_{i\in I}(G_i\circ H)$,$H\circ (\bigcup\limits_{i\in I}G_i)=\bigcup\limits_{i\in I}(H\circ G_i)$.
			\end{exer}
			证明:
			\par
			若$(x, z)\in (\bigcup\limits_{i\in I}G_i)\circ H$,则存在$y$,使$(x, y)\in H$,$(y, z)\in \bigcup\limits_{i\in I}G_i$,故对任意$i\in I$,$(y, z)\in G_i$,因此$(x, z)\in G_i\circ H$,故$(x, z)\in \bigcup\limits_{i\in I}(G_i\circ H)$.
			\par
			反过来,若$(x, z)\in \bigcup\limits_{i\in I}(G_i\circ H)$,则对任意$i\in I$,$(x, z)\in G_i\circ H$,则存在$y$,使$(x, y)\in H$,$(y, z)\in G_i$,故$(y, z)\in \bigcup\limits_{i\in I}G_i$,因此$(x, z)\in (\bigcup\limits_{i\in I}G_i)\circ H$.
			\par
			同理可证$H\circ (\bigcup\limits_{i\in I}G_i)=\bigcup\limits_{i\in I}(H\circ G_i)$.
			
			\begin{exer}\label{exer59}
				\hfill\par
				$G$、$H$、$H'$为图,求证:当且仅当$(\forall H)(\forall H')((H\cap H')\circ G=(H\circ G)\cap(H'\circ G))$时,$G$为函数图.
			\end{exer}
			证明:
			\par
			如果$G$为函数图,设$(x, z)\in (H\cap H')\circ G$,则存在$y$,使$(x, y)\in G$,$(y, z)\in H\cap H'$,因此$(x, z)\in (H\circ G)\cap(H'\circ G)$.
			\par
			反过来,如果$(x, z)\in (H\circ G)\cap(H'\circ G)$,则存在$y$、$y'$,使$(x, y)\in G$,$(y, z)\in H$,$(x, y')\in G$,$(y', z)\in H'$.由于$G$为函数图,因此$y=y'$,故$(x, z)\in (H\cap H')\circ G$.
			\par
			如果$(H\cap H')\circ G=(H\circ G)\cap(H'\circ G)$,设$(x, y) \in G$、$(x, y') \in G$,令$H=\{(y, z)\}$,$H'=\{(y', z)\}$,如果$y\neq y'$,则$H\cap H'=\varnothing$,故$(H\cap H')\circ G=\varnothing$.但$(x, z)\in (H\circ G)\cap(H'\circ G)$,矛盾.故$y=y'$,因此$G$为函数图.
			
			\begin{exer}\label{exer60}
				\hfill\par
				$G$、$H$、$K$为图,求证:$(H\circ G)\cap K\subset (H\cap (K\circ G^{-1}))\circ (G\cap (H^{-1}\circ K))$.
			\end{exer}
			证明:若$(x, z)\in (H\circ G)\cap K$,则$(x, z)\in K$,且存在$y$,使$(x, y)\in G$、$(y, z)\in H$.则$(x, y)\in H^{-1}\circ K$,因此$(x, y)\in (G\cap(H^{-1}\circ K))$.同时,$(y, z)\in K\circ G^{-1}$,因此$(y, z)\in (H\cap(K\circ G^{-1}))$.故$(x, z)\in (H\cap (K\circ G^{-1}))\circ (G\cap(H^{-1}\circ K))$.得证.
		
			\begin{exer}\label{exer61}
				\hfill\par
				$H=(X_i)_{i\in I}$和$G=(Y_k)_{k\in K}$都是$E$的覆盖,
				\par
				(1)如果$G$是$E$的划分,$H$是比$G$更细的覆盖,且对任意$k\in K$,$Y_k\neq \varnothing$.求证:对任意$k\in K$,存在$i\in I$,使$X_i\subset Y_k$.
				\par
				(2)写出$E$的两个覆盖$H$和$G$,$H$是比$G$更细的覆盖,但(1)中的性质不成立.
				\par
				(3)写出$E$的两个划分$H$和$G$,对任意$k\in K$,存在$i\in I$,使$X_i\subset Y_k$,但$H$并不是比$G$更细的覆盖.
			\end{exer}
			证明:
			\par
			(1)对任意$k\in K$,设$x\in Y_k$,故存在$i\in I$,使$x\in X_i$,由于$H$是比$G$更细的覆盖,因此存在$k' \in K$,使$X_i\subset Y_k'$.假设$k\neq k'$,由于$G$是$E$的划分,故$Y_k'\cap Y_k=\varnothing$,矛盾,因此$k=k'$,故$X_i\subset Y_k$.
			\par
			(2)设$a$、$b$互不相等,$E=\{a, b\}$,$H=\Delta_{\{E\}}$,$G=\Delta_{\{\{a\}, E\}}$.
			\par
			(3)设$a$、$b$、$c$、$d$互不相等,$E=\{a, b, c, d\}$,$H=\Delta_{\{\{a\}, \{d\}, \{b, c\}\}}$,$G=\Delta_{\{\{a, b\}, \\\{c, d\}\}}$.

		\section{集族的乘积(Produit d'une famille d'ensembles)}

			\begin{ex}\label{ex3}
				\textbf{幂集公理}
				\par
				$(\forall X)Coll_Y(Y\subset X)$.
			\end{ex}
			
			\begin{de}
				\textbf{幂集(ensemble des parties)}
				\par
				$\{Y|Y\subset X\}$称为$X$的幂集,记作$\mathcal{P}(X)$或$2^X$.
			\end{de}

			\begin{cor}\label{cor117}
				\hfill\par
				(1)$ \mathcal{P}(\varnothing)=\{\varnothing\}$;
				\par 
				(2)$ \mathcal{P}(\{x\})=\{\varnothing, \{x\}\}$.
			\end{cor}
			证明:
			\par
			(1)根据补充定理\ref{cor18}(2)可证.
			\par
			(2)根据补充定理\ref{cor18}可证.

			\begin{cor}\label{cor118}
				\hfill\par
				$(X\subset Y)\Rightarrow( \mathcal{P}(X)\subset \mathcal{P}(Y))$.
			\end{cor}
			证明:设$x\in \mathcal{P}(X)$,则$x\subset X$,故$x\subset Y$,因此$x\in \mathcal{P}(Y)$,得证.
			
			\begin{de}
				\textbf{在子集上的规范扩展(extension canonique aux ensembles de parties),在子集上的逆扩展(extension réciproque aux ensembles de parties)}
				\par
				令$F$为$A$到$B$的对应,函数$X\mapsto F\langle X \rangle (X\in \mathcal{P}(A), F\langle X \rangle \in \mathcal{P}(B))$称为$F$在子集上的规范扩展,记作$\hat{F}$.函数$Y\mapsto F^{-1}\langle Y \rangle (Y\in \mathcal{P}(A), F^{-1}\langle Y \rangle \in \mathcal{P}(A))$,称为$F$在子集上的逆扩展.
			\end{de}
			
			\begin{cor}\label{cor119}
				\hfill\par
				(1)令$F$为$A$到$B$的对应,$F'$为$B$到$C$的对应,则$F'\circ F$在子集上的规范扩展为$\hat{F'}\circ \hat{F}$ .
				\par
				(2)令$F$为$A$到$B$的双射,则$F^{-1}$在子集上的规范扩展为$(\hat{F})^{-1}$.
				\par
				(3)$Id_A$在子集上的规范扩展为$Id_{\mathcal{P}(A)}$.			
			\end{cor}
			证明:
			\par
			(1)	根据定理\ref{theo16},$F'\circ 	F\langle X \rangle =F'\langle F\langle X \rangle \rangle$,且定义域均为$\mathcal{P}(A)$、到达域均为$\mathcal{P}(C)$,得证.
			\par
			(2)	根据补充定理\ref{cor78}可证.
			\par
			(3)	根据补充定理\ref{cor66}(4)可证.
									
			\begin{theo}\label{theo36}
				\hfill\par
				(1)设$f$为$E$到$F$的满射,则$\hat{f}$是$\mathcal{P}(E)$到$\mathcal{P}(F)$的满射.
				\par
				(2)设$f$为$E$到$F$的单射,则$\hat{f}$是$\mathcal{P}(E)$到$\mathcal{P}(F)$的单射.
			\end{theo}
			证明:
			\par
			(1)	设$s$是$f$的右逆,则$f\circ s=Id_F$,根据补充定理\ref{cor119}(1),$\hat{f}\circ \hat{s}=Id_{\mathcal{P}(F)}$,得证.
			\par
			(2)	如果$E=\varnothing$,根据补充定理\ref{cor18}(2),$\mathcal{P}(E) =\{\varnothing\}$,根据补充定理\ref{cor74}(2),$\hat{f}$是单射.如果$E\neq \varnothing$,设$r$是$f$的左逆,则$r\circ f=Id_E$,根据补充定理\ref{cor119}(1),$\hat{r}\circ \hat{f} =Id \mathcal{P}(E)$,则$\hat{f}$是单射.
			
			\begin{cor}\label{cor120}
				\textbf{所有从一个集合到另一个集合的映射的图能够组成集合}
				\par
				$(G\text{为图})\text{与}(pr_1G=E)\text{与}(pr_2G\subset F)$是$G$上的集合化公式.
			\end{cor}
			证明:$(G\text{为图})\text{与}(pr_1G=E)\text{与}(pr_2G\subset F)\Rightarrow G\subset E\times F$,因此$G\in \mathcal{P}(E\times F)$,根据证明规则\ref{C52}得证.

			\begin{de}
				\textbf{映射的图的集合(ensemble des graphe d'applications)}
				\par
				$\{G|(G\text{为图})\text{与}(pr_1G=E)\text{与}(pr_2G\subset F)\}$称为$E$到$F$的映射的图的集合,记作$F^E$.			
			\end{de}
			
			\begin{cor}\label{cor121}
				\hfill\par
				$F^E\subset \mathcal{P}(E\times F)$.
			\end{cor}
			证明:设$G\in F^E$,则$(G\text{为图})\text{与}(pr_1G=E)\text{与}(pr_2G\subset F)\Rightarrow G\subset E\times F$,因此$G\in \mathcal{P}(E\times F)$,得证.
						
			\begin{cor}\label{cor122}
				\textbf{所有从一个集合到另一个集合的映射能够组成集合}
				\par
				$(f\text{为}A\text{到}B\text{的映射})$是$f$上的集合化公式.
			\end{cor}
			证明:$(f\text{的图}\in A\times B$,故$f\in A\times B\times A\times B$,根据证明规则\ref{C52}得证.
			
			\begin{de}
				\textbf{映射的集合(ensemble des applications)}
				\par
				$\{f|f\textbf{为}A\text{到}B\text{的映射}\}$称为$A$到$B$的映射的集合,记作$\mathcal{F}(A; B)$.
			\end{de}
							
			\begin{cor}\label{cor123}
				\hfill\par
				$G\mapsto (G, A, B)$是$A^B$到$\mathcal{F}(A; B)$的双射.
			\end{cor}
			证明:对任意$G\in A^B$,$(G, A, B)\in \mathcal{F}(A; B)$,因此该映射的定义域是$A^B$;$G=G'\Leftrightarrow (G, A, B)=(G', A, B)$,因此该对应是映射并且是单射;对任意$f\in \mathcal{F}(A; B)$,$f$为$A$到$B$的映射,并且$f\text{的图}\in A\times B$,因此该映射为满射.得证.

			\begin{de}
				\textbf{映射的图的集合到映射的集合的规范映射(application canonique de ensemble des graphe d'applications dans ensemble des applications)}
				\par
				$A^B$到$\mathcal{F}(A; B)$的映射$G\mapsto (G, A, B)$,称为$A^B$到$\mathcal{F}(A; B)$的规范映射.
			\end{de}
						
			\begin{theo}\label{theo37}
				\hfill\par
				(1)令$u$为$A'$到$A$的满射,$v$为$B'$到$B$的单射,$f$为$A$到$B$的映射,则$f\mapsto v\circ f\circ u(f\in \mathcal{F}(A; B))$是单射.
				\par
				(2)令$u$为$A'$到$A$的单射,$v$为$B'$到$B$的满射,$f$为$A$到$B$的映射,则$f\mapsto v\circ f\circ u(f\in \mathcal{F}(A; B))$是满射.
			\end{theo}
			证明:
			\par
			(1)令$s$为$u$的右逆,$r$为$v$的左逆,则$r\circ (v\circ f\circ u)\circ s=Id_F\circ f\circ Id_E$,即等于$f$.得证.
			\par
			(2)令$s$为$v$的右逆,$r$为$u$的左逆,则对任意$f$,$v\circ (s\circ f\circ r)\circ u=Id_F\circ f\circ Id_E$,即等于$f$.得证.
						
			\begin{theo}\label{theo38}
				\hfill\par
				令$u$为$A'$到$A$的双射,$v$为$B'$到$B$的双射,$f$为$A$到$B$的映射,则$f\mapsto v\circ f\circ u(f\in \mathcal{F}(A; B))$\\是双射.
			\end{theo}
			证明:根据定理\ref{theo37}可证.
						
			\begin{theo}\label{theo39}
				\hfill\par
				$f$为$B\times C$到$A$的映射,令$g$为映射$y\mapsto f_y(y\in C, f_y\in \mathcal{F}(B; A))$,则$f\mapsto g(f\in \mathcal{F}(B\times C; A), g\in \mathcal{F}(C; \mathcal{F}(B; A)))$为双射.
			\end{theo}
			证明:
			\par
			$f$为$B\times C$到$A$的映射,则$f_y$是$B$到$A$的映射,因此$y\mapsto f_y$是$C$到$\mathcal{F}(B; A)$的映射.
			\par
			反过来,设$g$为$C$到$\mathcal{F}(B; A)$的映射,令二元函数$f=(g(y))(x)$,其定义域为$B\times C$,则当$y\in C$时,$f_y=g(y)$.同时,设定义域为$B\times C$的二元函数$f'$也满足${f'}_y=g(y)$,则当$y\in C$时,$f_y={f'}_y$,即当$x\in B$、$y\in C$时,$f_y(x)={f'}_y(x)$,根据补充定理\ref{cor82}(2),$f=f'$,即$f$是唯一的,得证.

			\begin{de}
				\textbf{映射的集合之间的规范映射(application canonique entre deux ensembles des applications)}
				\par
				$f$为$B\times C$到$A$的映射,令$g$为映射$y\mapsto f_y(y\in C, f_y\in \mathcal{F}(B; A))$,则$f\mapsto g(f\in \mathcal{F}(B\times C; A), g\in \mathcal{F}(C; \mathcal{F}(B; A)))$称为$\mathcal{F}(B\times C; A)$到$\mathcal{F}(C; \mathcal{F}(B; A))$的规范映射.			
			\end{de}
			
			\begin{cor}\label{cor124}
				\textbf{集族的乘积是集合}\par
				$(F\text{为函数图})\text{与}(pr_1F= I)\text{与}(\forall i)(i\in I\Rightarrow F(i)\in X_i)$是$F$上的集合化公式.
			\end{cor}
			证明:$(\forall i)(i\in I\Rightarrow F(i)\in X_i)$,因此$i\in I\Rightarrow F(i)\in \bigcup\limits_{i\in I}X_i$,根据补充定理\ref{cor65},$pr_2F\subset \bigcup\limits_{i\in I}X_i$,又因为$pr_1F=I$,故$F\subset I\times \bigcup\limits_{i\in I}X_i$,因此$F\in \mathcal{P}(I\times \bigcup\limits_{i\in I}X_i)$,根据证明规则\ref{C52}得证.

			\begin{de}
				\textbf{集族的乘积(produit d'une famille d'ensembles),因子(facteur)}
				\par
				令$(X_i)_{i\in I}$为集族,$\{F|(F\text{为函数图})\text{与}(pr_1F= I)\text{与}(\forall i)(i\in I\Rightarrow F(i)\in X_i)\}$称为该集族的乘积,记作$\prod\limits_i\in IX_i$.当$i\in I$时,$X_i$称为乘积$\prod\limits_{i\in I}X_i$的指标$i$的因子.			
			\end{de}
			
			\begin{cor}\label{cor125}
				\hfill\par
				令$(X_i)_{i\in I}$为集族,$i\in I$,则$(\text{对于}F\in \prod\limits_{i\in I}X_i\text{形式为}F(i)\text{的对象集合})\subset X_i$.
			\end{cor}
			证明:$y\in (\text{对于}F\in \prod\limits_{i\in I}X_i\text{形式为}F(i)\text{的对象集合})\Leftrightarrow (\exists F)(y=F(i)\text{与}F\in \prod\limits_{i\in I}X_i)$,因此$y\in (\text{对于}F\in \prod\limits_{i\in I}X_i\text{形式为}F(i)\text{的对象集合})\Rightarrow (\exists F)(y=F(i)\text{与}F(i)\in X_i)$,故$y\in (\text{对于}F\in \prod\limits_{i\in I}X_i\text{形式为}F(i)\text{的对象集合})\Rightarrow (\exists F)(y=F(i)\text{与}y\in X_i)$,故$y\in (\text{对于}F\in \prod\limits_{i\in I}X_i\text{形式为}F(i)\text{的对象集合})\Rightarrow y\in X_i$,得证.

			\begin{de}
				\textbf{坐标函数(fonction coordonnée),射影函数(fonction projection),坐标(coordonnée),射影(projection)}
				\par
				令$(X_i)_{i\in I}$为集族,$i\in I$,映射$F\mapsto F(i)(F\in \prod\limits_{i\in I}X_i, F(i)\in X_i)$称为指标$i$的坐标函数或射影函数,记作$pr_i$,$pr_i(F)$可以简记为$pr_iF$.其中,$F(i)$称为$F$的指标$i$的坐标或射影.
			\end{de}

			\begin{de}
				\textbf{乘积的子集的射影(projection d'une partie de la produit)}
				\par
				令$(X_i)_{i\in I}$为集族,$A\subset \prod\limits_{i\in I}X_i$,则$pr_i\langle A \rangle $称为$A$的指标$i$的射影.
			\end{de}
						
			\begin{cor}\label{cor126}
				\hfill\par
				$(X_i)_{i\in I}$为集族,$A\subset \prod\limits_{i\in I}X_i$,则$A\subset \prod\limits_{i\in I}pr_i\langle A \rangle $.
			\end{cor}
			证明:设$F\in A$,则$F$为函数图、$pr_1F= I$,且对任意$i\in I$,$F(i)\in pr_i\langle A \rangle $,即$(\forall i)(i\in I\Rightarrow F(i)\in X_i)$,因此$F\in \prod\limits_{i\in I}pr_i\langle A \rangle $,得证.

			\begin{cor}\label{cor127}
				\hfill\par
				(1)$(X_i)_{i\in \varnothing}$的乘积为$\{\varnothing\}$.
				\par
				(2)$(X_i)_{i\in I}$为集族,如果存在$i\in I$,使$X_i=\varnothing$,则$(X_i)_{i\in I}$的乘积为$\varnothing$.
			\end{cor}
			证明:
			\par
			(1)一方面,$\varnothing$为函数图且$pr_1\varnothing= \varnothing$.另一方面,设$F\in \prod\limits_{i\in \varnothing}X_i$,则$pr_1F= \varnothing$,因此$F=\varnothing$,得证.
			\par
			(2)根据定义可证.
			
			\begin{cor}\label{cor128}
				\hfill\par
				$(X_i)_{i\in I}$为集族,如果$(\forall i)(i\in I\Rightarrow X_i=E)$,则$\prod\limits_{i\in I}X_i=E^I$.
			\end{cor}
			证明:$F\in \prod\limits_{i\in I}X_i\Leftrightarrow ((F\text{为函数图})\text{与}(pr_1F= I)\text{与}(\forall i)(i\in I\Rightarrow F(i)\in E))$,如果$F\in \prod\limits_{i\in I}X_i$,则$ pr_1F= I$、$pr_2F\subset E$,则$F\in E^I$.
			\par
			反过来,如果$F\in E^I$,则$F$为$I$到$E$的映射的图,根据补充定理64(3),$i\in I\Rightarrow F(i)\in E$,故$F\in \prod\limits_{i\in I}X_i$.得证.
			
			\begin{cor}\label{cor129}
				\hfill\par
				$(X_i)_{i\in I}$为集族,如果$\bigcup\limits_{i\in I}X_i\subset E$,则$\prod\limits_{i\in I}X_i\subset E^I$.
			\end{cor}
			证明:$F\in \prod\limits_{i\in I}X_i\Leftrightarrow ((F\text{为函数图})\text{与}(pr_1F= I)\text{与}(\forall i)(i\in I\Rightarrow F(i)\in X_i))$,由于$\bigcup\limits_{i\in I}X_i\subset E$,故对任意$i\in I$,均有$X_i\subset E$,因此$F\in \prod\limits_{i\in I}X_i\Rightarrow ((F\text{为函数图})\text{与}(pr_1F= I)\text{与}(\forall i)(i\in I\Rightarrow F(i)\in E))$,根据补充定理\ref{cor128},$F\in \prod\limits_{i\in I}X_i\Rightarrow F\in E^I$,得证.
			
			\begin{cor}\label{cor130}
				\hfill\par
				$\prod\limits_{i\in \{a\}}X_i={X_a}^{\{a\}}$,且指标$i$的坐标函数为${X_a}^{\{a\}}$到$X_a$的双射.
			\end{cor}
			证明:如果$i\in \{a\}$,则$i=a$,$X_i=X_a$,根据补充定理\ref{cor128},$\prod\limits_{i\in \{a\}}X_i=X_a^{\{a\}}$.设$F\in {X_a}^{\{a\}}$,$F' \in {X_a}^{\{a\}}$,且$F(a)=F'(a)$,则$i\in \{a\}\Rightarrow F(i)=F'(i)$,故$F=F'$.对任意$b\in X_a$,令$F$为$\{(a, b)\}$,则$F\in {X_a}^{\{a\}}$,且$F(a)=b$.故$F\mapsto F(a)(F\in {X_a}^{\{a\}}, F(a)\in X_a)$为双射.			

			\begin{de}
				\textbf{乘积和集合之间的规范映射(application canonique entre le produit et un ensemble)}
				\par
				$\prod\limits_{i\in \{a\}}X_i$的指标$i$的坐标函数,称为$X_a^{\{a\}}$到$X_a$的规范映射,其逆映射称为$X_a$到${X_a}^{\{a\}}$的规范映射.
			\end{de}
			
			\begin{cor}\label{cor131}
				\hfill\par
				如果$a\neq b$,则$\prod\limits_{i\in \{a, b\}}X_i=\{F|(\exists x)(\exists y)(x\in X_a\text{与}y\in X_b\text{与}F=\{(a, x), (b, y)\})\}$,并且,$(x, y)\mapsto \{(a, x), (b, y)\}$是$X_a\times X_b$到$\prod\limits_{i\in \{a, b\}}X_i$的双射.
			\end{cor}
			证明:
			\par
			$(\exists x)(\exists y)(x\in X_a\text{与}y\in X_b\text{与}F=\{(a, x), (b, y)\})\Rightarrow pr_1F=(a, b)\text{与}pr_2F\subset X_a\cup X_b$,因此,其为$F$上的集合化公式.
			\par
			$\prod\limits_{i\in \{a, b\}}X_i=\{F|(F\text{为函数图})\text{与}(pr_1F=\{a, b\})\text{与}F(a)\in X_a\text{与}F(a)\in X_b\}$.如果\\$(\exists x)(\exists y)(x\in X_a\text{与}y\in X_b\text{与}F=\{(a, x), (b, y)\}$,显然$F\in \prod\limits_{i\in \{a, b\}}X_i$.
			\par
			反过来,如果$F\in \prod\limits_{i\in \{a, b\}}X_i$,则$F$为函数图,设$(u, v)\in F$,则$u=a\text{或}u=b$,如果$u=a$,则$v=f(a)$,如果$u=b$,则$v=f(b)$,因此$F=\{(a, f(a)), (b, f(b))\}$,又因为$x\in X_a\text{与}y\in X_b$,故$(\exists x)(\exists y)(x\in X_a\text{与}y\in X_b\text{与}F=\{(a, x), (b, y)\}$.
			\par
			综上,$\prod\limits_{i\in \{a, b\}}X_i=\{F|(\exists x)(\exists y)(x\in X_a\text{与}y\in X_b\text{与}F=\{(a, x), (b, y)\})\}$.
			\par
			因此,$(x, y)\mapsto \{(a, x), (b, y)\}$是$X_a\times X_b$到$\prod\limits_{i\in \{a, b\}}X_i$的映射,对任意$F\in \prod\limits_{i\in \{a, b\}}X_i$,\\$(\exists x)(\exists y)(x\in X_a\text{与}y\in X_b\text{与}F=\{(a, x), (b, y)\})$,因此该映射为满射;对于$(x, y) \in X_a\times X_b$,$(x', y') \in X_a\times X_b$,由于$a\neq b$,若$\{(a, x), (b, y)\}=\{(a, x'), (b, y')\}$,则$x=x'$,$y=y'$,故该映射为单射.得证.
						
			\begin{cor}\label{cor132}
				\hfill\par
				$\prod\limits_{i\in I}\{a_i\}=\{(a_i)_{i\in I}\}$.
			\end{cor}
			证明:
			$(\exists i)(i\in I\text{与}z=(i, a_i))\Rightarrow z\in \times \bigcup\limits_{i\in I}\{a_i\}$,因此为$z$上的集合化公式.
			$F\in \prod\limits_{i\in I}\{a_i\}\Leftrightarrow ((F\text{为函数图})\text{与}(pr_1F= I)\text{与}(\forall i)(i\in I\Rightarrow F(i)=a_i))$.如果$z\in F$,则$(\exists i)(i\in I且z=(i, a_i))$,则$z\text{为有序对}$,故$F$为图;若$(x, y)\in F$、$(x, y')\in F$,则$(\exists i)(i\in I\text{与}x=i\text{与}y=a_i)$、$(\exists i)(i\in I\text{与}x=i\text{与}y'=a_i)$,故$y=y'$,因此$F$为函数图;同时,$(\exists y)((x, y)\in F)\Leftrightarrow (\exists y)(\exists i)(i\in I\text{与}x=i\text{与}y=a_i)$,等价于$(\exists i)(i\in I\text{与}x=i)$,等价于$x\in I$,故$pr_1F= I$;此外,若$i\in I$,则$(i, a_i)\in F$,同时$F(i)=a_i$.因此,$F\in \prod\limits_{i\in I}\{a_i\}$.
			\par
			反过来,如果$F\in \prod\limits_{i\in I}\{a_i\}$,$F'\in \prod\limits_{i\in I}\{a_i\}$,则$i\in I\Rightarrow F(i)=a_i$、$i\in I\Rightarrow F'(i)=a_i$,则$i\in I\Rightarrow F(i)=F'(i)$,因此$F=F'$.故$x\in \prod\limits_{i\in I}\{a_i\}\Rightarrow x=F$,又$F\in \prod\limits_{i\in I}\{a_i\}$,因此$x\in \prod\limits_{i\in I}\{a_i\}\Leftrightarrow x=F$,得证.
			
			\begin{cor}\label{cor133}
				\hfill\par
				令$(X_i)_{i\in I}$为集族,$F\in \prod\limits_{i\in I}X_i$,则$F=(F(i))_{i\in I}$.
			\end{cor}
			证明:由于$(F\text{为函数图})\text{与}(pr_1F= I)\text{与}(\forall i)(i\in I\Rightarrow F(i)\in X_i)$,因此$(x, y)\in F\Leftrightarrow x\in I\text{与}y=F(x)$,$(x, y)\in (F(i))_{i\in I}\Leftrightarrow (\exists i)(i\in I\text{与}x=i\text{与}y=F(i))$,等价于$x\in I\text{与}y=F(x)$,得证.
						
			\begin{cor}\label{cor134}
				\textbf{对角映射是单射}
				\par
				(1)$pr_1z\in I\text{与}pr_2z=x$是$z$上的集合化公式.
				\par
				(2)令$G_x$表示图$\{z|pr_1z\in I\text{与}pr_2z=x\}$,则$G_x$为函数图,并且$(\exists x)(x\in E\text{与}G=G_x)$是$G$上的集合化公式.同时,$\{G|(\exists x)(x\in E\text{与}G=G_x)\}\subset E^I$,且$x\mapsto G_x$是$E$到$E^I$的单射.
			\end{cor}
			证明:
			\par
			(1)$pr_1z\in I\text{与}pr_2z=x\Rightarrow z\in I\times \{x\}$,因此是集合化公式.
			\par
			(2)设$a\in I$、$b\in I$,且$(a, b)\in G_x$,$(a, b')\in G_x$,则$b=x$,$b'=x$,故$G$为函数图.而$(\exists x)(x\in E\text{与}G=G_x)\Rightarrow G\subset I\times E$,因此$G\in E^I$.因此,$(\exists x)(x\in E\text{与}G=G_x)$是$G$上的集合化公式,且$\{G|(\exists x)(x\in E\text{与}G=G_x)\}\subset E^I$.
			\par
			设$x\in E$、$x'\in E$,若$G_x=G_x'$,则$(\forall z)(pr_1z\in I\text{与}pr_2z=x\Leftrightarrow pr_1z\in I\text{与}pr_2z=x')$,故$x=x'$,因此该映射为单射.

			\begin{de}
				\textbf{到映射的图的集合的对角映射(application diagonale dans ensemble des graphe d'applications)}
				\par
				令$G_x$为函数图$\{z|pr_1z\in I\text{与}pr_2z=x\}$,则映射$x\mapsto G_x(x\in E, G_x\in E^I)$称为对角映射.
			\end{de}
						
			\begin{theo}\label{theo40}
				\hfill\par
				令$(X_i)_{i\in I}$为集族,$u$是$K$到$I$的双射,其图为$U$,则映射$F\mapsto F\circ U$为$\prod\limits_{i\in I}X_i$到$\prod\limits_{k\in K}X_{u(k)}$\\的双射.
			\end{theo}
			证明:根据定理\ref{theo23},$\bigcup\limits_{i\in I}X_i=\bigcup\limits_{k\in K}X_{u(k)}$,设其为$A$,根据定理\ref{theo37},映射$F\mapsto F\circ U(F\in A^I)$为$A^I$到$A^K$的双射,根据补充定理\ref{cor75},$\prod\limits_{i\in I}X_i$到$\prod\limits_{k\in K}X_{u(k)}$的映射$F\mapsto F\circ U$为单射.由于$u$为\\$K$到$I$的双射,因此,若对任意$i\in I$,$F(i)\in X_i$,则对任意$k$,设$k=u^{-1}(i)$,则$F\circ U(k)\in X_{u(k)}$,反之,同理可证$(\forall i)(i\in I\Rightarrow F(i)\in X_i)\Leftrightarrow (\forall k)(k\in K\Rightarrow F\circ U(k)\in X_{u(k)})$.因此,对于$\prod\limits_{i\in I}X_i$到$\prod\limits_{k\in K}X_{u(k)}$的映射$F\mapsto F\circ U$,若$F\circ U\in \prod\limits_{k\in K}X_{u(k)}$,则$F\in \prod\limits_{i\in I}X_i$,故为满射,得证.

			\begin{de}
				\textbf{部分乘积(produit partiel)}
				\par
				令$(X_i)_{i\in I}$为集族,$J\subset I$,则称$\prod\limits_{i\in J}X_i$为其部分乘积.			
			\end{de}
						
			\begin{cor}\label{cor135}
				\hfill\par
				($X_i)_{i\in I}$为集族,$J\subset I$,$f$为函数,其图为$F$,且$F\in \prod\limits_{i\in I}X_i$,则$F\circ \Delta_J$为$f|J$的图.
			\end{cor}
			证明:根据补充定理\ref{cor73}(2)可证.
				
			\begin{cor}\label{cor136}
				\hfill\par
				令($X_i)_{i\in I}$为集族,$J\subset I$,$f$为函数,其图为$F$,且$F\in \prod\limits_{i\in I}X_i$,则$F\circ \Delta_J\in \prod\limits_{i\in J}X_i$.
			\end{cor}
			证明:根据补充定理\ref{cor135},$F\circ \Delta_J$为$f|J$的图.因此$(x, y)\in F\circ \Delta_J\Leftrightarrow x\in J\text{与}(x, y)\in F$.
			\par
			设$i\in J$,由于$(\forall i)(i\in I\Rightarrow F(i)\in X_i)$,故$F(x)\in X_i$,根据补充定理\ref{cor73}(1),$F\circ \Delta_J\in X_i$,即$(\forall i)(i\in J\Rightarrow F\circ \Delta_J (i)\in X_i)$,同时,根据定理\ref{theo17},$F\circ \Delta_J$为函数图且定义域为$J$,根据定义,$F\circ \Delta_J\in \prod\limits_{i\in J}X_i$.

			\begin{de}
				\textbf{指标集的子集的射影(projection de partie de ensemble des indices)}
				\par
				令$(X_i)_{i\in I}$为集族,$J\subset I$,$F$为函数图,则映射$F\mapsto F\circ \Delta_J(F\in \prod\limits_{i\in I}X_i , F\circ \Delta_J\in \prod\limits_{i\in J}X_i)$称为指标$J$的射影,记作$pr_J$.
			\end{de}
			
			\begin{theo}\label{theo41}
				\hfill\par
				令$(X_i)_{i\in I}$为集族,$J\subset I$,如果对任意$i\in I$,均有$X_i\neq \varnothing$,令$A=\bigcup\limits_{i\in I}X_i$,设$g$为$J$到$A$的映射,并且$(\forall i)(i\in J\Rightarrow g(i)\in X_i)$,则存在映射$f$,为$g$在$I$上的延拓,
			\end{theo}
			证明:对于$i\in I-J$,令$T_i=\tau_y(y\in X_i)$,由于$X_i\neq \varnothing$,因此$T_i\in X_i$.设$g$的图为$G$,令$F=G\cup(\bigcup\limits_{i\in I}-J(i, T_i))$,由于$(\exists y)((x, y)\in G)\Leftrightarrow x\in I$,当$x\in J$时,$(\exists y)((x, y)\in G)$,则$(\exists y)((x, y)\in F)$,当$x\in I-J$时,$(x, T_x)\in F$,当$x\notin I$时,$(\exists y)((x, y)\in G)$为假,若存在$y$使$(x, y)\in \bigcup\limits_{i\in I-J}(i, T_i)$,则存在$i\in I-J$,使$(i, T_i)=(x, y)$,则$x\in I$,矛盾.故$pr_1F=I$.类似可证$pr_2F\subset A$.令$f=(F, I, A)$,则$f$和$g$在$J$上重合,故$f$为$g$在$I$上的延拓.
						
			\begin{theo}\label{theo42}
				\hfill\par
				令$(X_i)_{i\in I}$为集族,$J\subset I$,如果对任意$i\in I$,均有$X_i\neq \varnothing$,则$pr_J$为$\prod\limits_{i\in I}X_i$到$\prod\limits_{i\in J}X_i$的满射.
			\end{theo}
			证明:根据定理\ref{theo41}可证.
			
			\begin{theo}\label{theo43}
				\hfill\par
				令$(X_i)_{i\in I}$为集族,$J\subset I$,如果对任意$i\in I$,均有$X_i\neq \varnothing$,则对$a\in I$,$pr_a$为$\prod\limits_{i\in I}X_i$到$X_a$\\的满射.
			\end{theo}
			证明:
			根据补充定理\ref{cor130},$\prod\limits_{i\in J}X\{a\}={X_a}^{\{a\}}$,根据定理\ref{theo42},$pr_{\{a\}}$为$\prod\limits_{i\in I}X_i $到${X_a}^{\{a\}}$的满射,令其图为$G_1$.
			\par
			同时,$pr_{\{a\}}$为$F\mapsto F\circ \{(a, a)\}$,且$F\circ \{(a, a)\}\in \prod\limits_{i\in \{a\}}X_i$.根据补充定理\ref{cor130},$\prod\limits_{i\in \{a\}}X_i$的指标$i$的坐标函数为${X_a}^{\{a\}}$到$X_a$的满射,令该函数为$g$,其图为$G_2$.
			\par
			由于$F\circ \{(a, a)\}(a)=F(a)$,即对任意$F\in \prod\limits_{i\in I}X_i$ ,$(F, F\circ \{(a, a)\})\in G_1$,$(F\circ \{(a, a)\}, F(a))\in G_2$,因此$pr_a=g\circ pr_{\{a\}}$,根据定理\ref{theo21}(2),$pr_a$为$\prod\limits_{i\in I}X_i$ 到$X_a$的满射.
						
			\begin{theo}\label{theo44}
				\hfill\par
				令$(X_i)_{i\in I}$为集族,则$\prod\limits_{i\in I}X_i=\varnothing\Leftrightarrow (\exists i)(i\in I\text{与}X_i=\varnothing)$.
			\end{theo}
			证明:如果$\prod\limits_{i\in I}X_i=\varnothing$,假设$(\forall i)(i\in I\Rightarrow X_i\neq \varnothing)$,设$x\in X_a$,根据定理\ref{theo43},存在$F\in \prod\limits_{i\in I}X_i$,使$F(a)=x$,矛盾,故$(\exists i)(i\in I\text{与}X_i=\varnothing)$.
			\par
			反过来,如果$(\exists i)(i\in I\text{与}X_i=\varnothing)$,假设$\prod\limits_{i\in I}X_i\neq \varnothing$,根据定理\ref{theo43},对任意$a\in I$,$pr_a\langle\prod\limits_{i\in I}X_i\rangle\subset X_a$,则$X_a\neq \varnothing$,矛盾,故$\prod\limits_{i\in I}X_i=\varnothing$.			
			
			\begin{theo}\label{theo45}
				\hfill\par
				令$(X_i)_{i\in I}$、$(Y_i)_{i\in I}$为有相同指标集$I$的集族,如果$(\forall i)(i\in I\Rightarrow X_i\subset Y_i)$,则$\prod\limits_{i\in J}X_i\subset \prod\limits_{i\in J}Y_i$;反过来,如果$\prod\limits_{i\in J}X_i\subset \prod\limits_{i\in J}Y_i$,且对任意$i\in I$,均有$X_i\neq \varnothing$,则$(\forall i)(i\in I\Rightarrow X_i\subset Y_i)$.				
			\end{theo}
			证明:$(\forall i)(i\in I\Rightarrow X_i\subset Y_i)$,则对任意$F\in \prod\limits_{i\in J}X_i$,$F(i)\in X_i\Rightarrow F(i)\in Y_i$,故$F\in \prod\limits_{i\in J}Y_i$.
			\par
			反过来,根据定理\ref{theo43},对任意$a\in I$,$pr_a\langle\prod\limits_{i\in J}X_i\rangle=X_a$,$pr_a\langle\prod\limits_{i\in J}Y_i\rangle=Y_a$,由于$\prod\limits_{i\in J}X_i\subset \prod\limits_{i\in J}Y_i$,因此$pr_a\langle\prod\limits_{i\in J}X_i\rangle\subset pr_a\langle\prod\limits_{i\in J}Y_i\rangle$,得证.
			
			\begin{cor}\label{cor137}
				\hfill\par
				令$(X_i)_{i\in I}$、$\{a_i\}i\in I$为有相同指标集$I$的集族,如果$(\forall i)(i\in I\Rightarrow a_i\in X_i)$,则$(a_i)_{i\in I}\in \prod\limits_{i\in J}X_i$;反过来,如果$(a_i)_{i\in I}\in \prod\limits_{i\in J}X_i$,则$(\forall i)(i\in I\Rightarrow a_i\in X_i)$.
			\end{cor}
			证明:根据定理\ref{theo45}可证.
			
			\begin{theo}\label{theo46}
				\textbf{乘积的结合律}
				\par
				令$(X_i)_{i\in I}$为集族,$(J_l)_{l\in L}$为$I$的划分,则映射$f\mapsto (prJ_lf)_{l\in L}$为$\prod\limits_{i\in I}X_i$到$\prod\limits_{l\in L}(\prod\limits_{i\in J}lX_i)$的双射.
			\end{theo}
			证明:根据补充定理\ref{cor135},$prJ_lf=f|J_l$.设$w\in \prod\limits_{l\in L}(\prod\limits_{i\in J}lX_i )$,则当$l\in L$时,$w(l)\in \prod\limits_{i\in J}lX_i$ ,即$(w(l), J_l, \bigcup\limits_{i\in J_l}X_i)$为映射,又因为$J_l\subset I$,因此$(w(l), J_l, \bigcup\limits_{i\in I}X_i)$为映射,将其记为$v_l$.对于集族$(v_l)_{l\in L}$,根据定理\ref{theo32},存在唯一的$I$到$\bigcup\limits_{i\in I}X_i$的映射$u$,使对任意$l\in L$,$u$在$J_l$上的限制为$v_l$.得证.
			
			\begin{cor}\label{cor138}
				\hfill\par
				令$(X_i)_{i\in I}$为集族,$(J_l)_{l\in \{a, b\}}$为$I$的划分,则映射$f\mapsto (pr_{J_a}f, pr_{J_b}f)$为$\prod\limits_{i\in I}X_i$到$(\prod\limits_{i\in J_a}X_i)\times (\prod\limits_{i\in J_b}X_i)$的双射.如果对任意$i\in J_b$,$X_i$均为单元素集合,则$f\mapsto pr_{J_a}f$为$\prod\limits_{i\in I}X_i$到$\prod\limits_{i\in J_a}X_i$的双射.
			\end{cor}
			证明:根据定理\ref{theo46},$f\mapsto (pr_{J_l}f)(l\in \{a, b\})$为$\prod\limits_{i\in I}X_i$ 到$\prod\limits_{l\in \{a, b\}}(\prod\limits_{i\in J_l}X_i)$的双射,令其为$f$.
			\par
			根据补充定理\ref{cor131},$(x, y)\mapsto \{(a, x), (b, y)\}$是$\prod\limits_{i\in J_a}X_i\times \prod\limits_{i\in J_b}X_i$到$\prod\limits_{l\in \{a, b\}}(\prod\limits_{i\in J_l}X_i)$的双射,令其为$g$.
			\par
			则$g^{-1}\circ f$即为映射$f\mapsto (pr{J_a}f, pr_{J_b}f)$,前半部分得证.
			如果对任意$i\in J_b$,$X_i$均为单元素集合,根据补充定理\ref{cor132},令$\prod\limits_{i\in J_b}X_i=\{y\}$,则对任意$z\in (\prod\limits_{i\in J_a}X_i)\times (\prod\limits_{i\in J_b}X_i)$,$pr_2z=y$,因此,$z\mapsto pr_1f$为$(\prod\limits_{i\in J}aX_i)\times (\prod\limits_{i\in J_b}X_i)$到$\prod\limits_{i\in J_a}X_i$的双射,后半部分得证.
			
			\begin{cor}\label{cor139}
				\hfill\par
				令$(X_i)_{i\in I}$、$(Y_i)_{i\in I}$为集族,则映射$f\mapsto ((pr_1(pr_if))_{i\in I}, (pr_2(pr_if))_{i\in I})$为$\prod\limits_{i\in I}(X_i\times Y_i)$到\\$(\prod\limits_{i\in I}X_i)\times (\prod\limits_{i\in I}Y_i)$的双射.
			\end{cor}
			证明:
			\par
			设$f\in \prod\limits_{i\in I}(X_i\times Y_i)$,则当$i\in I$时,$f(i)\in X_i\times Y_i$,故$pr_1f(i)\in X_i$,因此$(pr_1(pr_if))_{i\in I}\in \prod\limits_{i\in I}X_i$,同理$(pr_2(pr_if))_{i\in I}\in \prod\limits_{i\in I}Y_i$,故$((pr_1(pr_if))_{i\in I}, (pr_2(pr_if))_{i\in I})\in (\prod\limits_{i\in I}X_i)\times (\prod\limits_{i\in I}Y_i)$.
			\par
			对任意$x\in (\prod\limits_{i\in I}X_i)$,$y\in (\prod\limits_{i\in I}Y_i)$,设$f$为$i\mapsto (pr_ix, pr_iy)$,则$(pr_1(pr_if))_{i\in I}=(pr_ix)_{i\in I}$,根据补充定理\ref{cor133},$(pr_1(pr_if))_{i\in I}=x$,同理$(pr_2(pr_if))_{i\in I}=y$.同时,当$i\in I$时,$pr_ix \in X_i$,$pr_iy \in Y_i$,因此$(pr_ix, pr_iy)\in X_i \times Y_i$,所以$f\in \prod\limits_{i\in I}(X_i\times Y_i)$,故该映射为满射.
			\par
			对任意$f\in \prod\limits_{i\in I}(X_i\times Y_i)$,$f'\in \prod\limits_{i\in I}(X_i\times Y_i)$,若$(pr_1(pr_if))_{i\in I}=(pr_1(pr_if'))_{i\in I}$,令其为$A$,则对任意$i'\in I$,$(i', pr_1f(i'))\in A$,因此$(\exists i)(i\in I\text{与}i=i'\text{与}pr_1f(i')=pr_1f'(i))$,因此$(\forall i)(i\in I\Rightarrow pr_1f(i)=pr_1f'(i))$,同理$(\forall i)(i\in I\Rightarrow pr_2f(i)=pr_2f'(i))$,故$(\forall i)(i\in I\Rightarrow f(i)=f'(i))$,根据补充定理\ref{cor133},$f=f'$,故该映射为单射.得证.
			
			\begin{de}
				\textbf{两个集族乘积之间的规范映射(application canonique entre deux produits d'familles d'ensembles)}
				\par
				令$(X_i)_{i\in I}$为集族,$(J_l)_{l\in L}$为$I$的划分,则映射$f\mapsto (prJ_lf)_{l\in L}$称为$\prod\limits_{i\in I}X_i$到$\prod\limits_{l\in L}(\prod\limits_{i\in J_l}X_i)$的规范映射.
				\par
				令$(X_i)_{i\in I}$为集族,$(J_l)_{l\in \{a, b\}}$为$I$的划分,则映射$f\mapsto (pr_{J_a}f, pr_{J_b}f)$称为$\prod\limits_{i\in I}X_i$到$(\prod\limits_{i\in J_a}X_i)\times (\prod\limits_{i\in J_b}X_i)$的规范映射.
				\par
				令$(X_i)_{i\in I}$、$(Y_i)_{i\in I}$为集族,则映射$f\mapsto ((pr_1(pr_if))_{i\in I}, (pr_2(pr_if))_{i\in I})$称为$\prod\limits_{i\in I}(X_i\times Y_i)$到\\$(\prod\limits_{i\in I}X_i)\times (\prod\limits_{i\in I}Y_i)$的规范映射.
			\end{de}
			
			\begin{theo}\label{theo47}
				\textbf{并集和交集的分配律}
				\par
				令$((X_{l, i})_{l\in L})_{i\in I}$为集族,且$L\neq \varnothing$、$(\forall l)(l\in L\Rightarrow J_l\neq \varnothing)$.令$I=\prod\limits_{l\in L}J_l$,则:
				\par
				(1)$\bigcup\limits_{l\in L}(\bigcap\limits_{i\in J_l}X_{l, i})=\bigcap\limits_{f\in I}(\bigcup\limits_{l\in L}X_{l, f(l)})$.
				\par
				(2)$\bigcap\limits_{l\in L}(\bigcup\limits_{i\in J_l}X_{l, i})=\bigcup\limits_{f\in I}(\bigcap\limits_{l\in L}X_{l, f(l)})$.
			\end{theo}
			证明:
			\par
			(1)设$x\in \bigcup\limits_{l\in L}(\bigcap\limits_{i\in J_l}X_{l, i})$,则存在$l\in L$,使$x\in \bigcap\limits_{i\in J_l}X_{l, i}i$,因此对任意$f\in I$,$x\in X_{l, f(l)}$,即对任意$f\in I$,$x\in \bigcup\limits_{l\in L}X_{l, f(l)}$,因此$x\in \bigcap\limits_{f\in I}(\bigcup\limits_{l\in L}X_{l, f(l)})$.
			\par
			反过来,设$x\notin \bigcup\limits_{l\in L}(\bigcap\limits_{i\in J_l}X_{l, i})$,则对任意$l\in L$,$x\notin \bigcap\limits_{i\in J_l}X_{l, i}$,即对任意$l\in L$,存在$J_l$,使$i\in J_l$且$x\notin X_{l, i}$.由于$i\in J_l\text{与}x\notin X_{l, i}$为集合化公式,可令${J'}_l=\{i|i\in J_l\text{与}x\notin X_{l, i}\}$,根据定理\ref{theo43},存在函数图$f$,定义域为$I$,且对任意$l\in L$,$f(l)\in {J'}_l$.因此$f\in I$,且对任意$l\in L$,$x\notin X_{l, f(l)}$.则$x\notin \bigcup\limits_{l\in L}X_{l, f(l)}$,故$x\notin \bigcap\limits_{f\in I}(\bigcup\limits_{l\in L}X_{l, f(l)})$,得证.
			\par
			(2)令$A=\bigcup\limits_{l\in L}(\bigcup\limits_{i\in J_l}X_{l, i})$,令$Y_{l, i}=\complement_AX_{l, i}$,根据定理\ref{theo29}、定理\ref{theo47}(1)可证.
			
			\begin{theo}\label{theo48}
				\hfill\par
				令$(X_i)_{i\in I}$、$(Y_k)k\in K$为集族,则:
				\par
				(1)若$I\neq \varnothing$,$K\neq \varnothing$,则$(\bigcap\limits_{i\in I}X_i)\cup(\bigcap\limits_{k\in K}Y_k)=\bigcap\limits_{(i, k)\in I\times K}(X_i\cup Y_k)$.
				\par
				(2)$(\bigcup\limits_{i\in I}X_i)\cap(\bigcup\limits_{k\in K}Y_k)=\bigcup\limits_{(i, k)\in I\times K}(X_i\cap Y_k)$.
			\end{theo}
			证明:令$L=\{a, b\}$,$J_a=I$,$J_b=K$.根据补充定理\ref{cor131},存在$\prod\limits_{i\in \{a, b\}}J_i$到$A\times B$的双射,令其为$f$,根据定理\ref{theo47}可证.
			
			\begin{theo}\label{theo49}
				\textbf{并集和交集对乘积的分配律}
				\par
				令$((X_{l, i})_{l\in L})_{i\in I}$为集族,$I=\prod\limits_{l\in L}J_l$,则:
				\par
				(1)$\prod\limits_{l\in L}(\bigcup\limits_{i\in J_l}X_{l, i})=\bigcup\limits_{f\in I}(\prod\limits_{l\in L}X_{l, f(l)})$.
				\par
				(2)若$L\neq \varnothing$、$(\forall l)(l\in L\Rightarrow J_l\neq \varnothing)$,则$\prod\limits_{l\in L}(\bigcap\limits_{i\in J_l}X_{l, i})=\bigcap\limits_{f\in I}(\prod\limits_{l\in L}X_{l, f(l)})$.
			\end{theo}
			证明:
			\par
			(1)	若$L=\varnothing$,左边$=\{\varnothing\}$,右边$=\{\varnothing\}$;若存在$l\in L$,使$J_l=\varnothing$,则左边$=\varnothing$,右边$=\varnothing$.
			\par
			在其他情况下,设$g\in \prod\limits_{l\in L}(\bigcup\limits_{i\in J_l}X_{l, i})$,则对任意$l\in L$,存在$i\in J_l$,使$g(l)\in X_{l, i}$.由于$i\in J_l\text{与}g(l)\in X_{l, i}$是$i$上的集合化公式,令$H_l=\{i|i\in J_l\text{与}g(l)\in X_{l, i}\}$,则$H_l\neq \varnothing$.令$f=\bigcup{l\in L}(i, \tau_y(y\in H_l))$,则$f$为函数图且对任意$l\in L$,$f(l) \in H_l$,且$f\in I$.因此$g(l)\in X_{l, f(l)}$,故$g\in \prod\limits_{l\in L}X_{l, f(l)}$,因此$g\in \bigcup\limits_{f\in I}(\prod\limits_{l\in L}X{l, f(l)})$.
			\par
			反过来,如果$g\in \bigcup\limits_{f\in I}(\prod\limits_{l\in L}X_{l, f(l)})$,则存在$f\in I$,使$g\in \prod\limits_{l\in L}X_{l, f(l)}$,因此对任意$l\in L$,均有$g(l)\in X_{l, f(l)}$,由于$f(l)\in J_l$,因此$g(l)\in \bigcup\limits_{i\in J_l}X_{l, i}$,故$g\in \prod\limits_{l\in L}(\bigcup\limits_{i\in J_l}X_{l, i})$.得证.
			\par			
			(2)	类似定理\ref{theo49}(1)可证.
			
			\begin{theo}\label{theo50}
				\hfill\par
				如果对任意$l\in L$,集族$(X_{l, i})_{l\in L}$为$\bigcup\limits_{i\in J_l}X_{l, i}$的划分,则集族$(\prod\limits_{l\in L}X_{l, f(l)})_{f\in I}$是$\prod\limits_{l\in L}(\bigcup\limits_{i\in J_l}X_{l, i})$的划分.
			\end{theo}
			证明:
			令$P_f=\prod\limits_{l\in L}X_{l, f(l)}$.设$f\in I$、$g\in I$,且$f\neq g$,则存在$l\in L$,使$f(l)\neq g(l)$,故$X_{l, f(l)}\cap X_{l, g(l)}=\varnothing$,如果$P_f\cap P_g\neq \varnothing$,则令$G\in P_f\cap P_g$,因此$G(l)\in X_{l, f(l)}$,$G(l)\in X_{l, g(l)}$,矛盾.因此,$P_f\cap P_g=\varnothing$,得证.
						
			\begin{theo}\label{theo51}
				\hfill\par
				令$(X_i)_{i\in I}$、$(Y_k)k\in K$为集族,则:
				\par
				(1)$(\bigcup\limits_{i\in I}X_i)\times (\bigcup\limits_{k\in K}Y_k)=\bigcup\limits_(i, k)\in I\times K(X_i\times Y_k)$.
				\par
				(2)若$I\neq \varnothing$,$K\neq \varnothing$,则$(\bigcap\limits_{i\in I}X_i)\times (\bigcap\limits_{k\in K}Y_k)=\bigcap\limits_(i, k)\in I\times K(X_i\times Y_k)$.
			\end{theo}
			证明:类似定理\ref{theo48}的证明,根据定理\ref{theo49}可证.
			
			\begin{theo}\label{theo52}
				\hfill\par
				令$(X_{i, k})_{(i, k)\in I\times K}$为集族,$k\neq \varnothing$,则$\bigcap\limits_{k\in K}(\prod\limits_{i\in I}X_{i, k})=\prod\limits_{i\in I}(\bigcap\limits_{k\in K}X_{i, k})$.
			\end{theo}
			证明:
			\par
			$f\in \bigcap\limits_{k\in K}(\prod\limits_{i\in I}X_{i, k})\Leftrightarrow (\forall k)(k\in K\Rightarrow f\in \prod\limits_{i\in I}X_{i, k})$.等价于$(f\text{为函数图})\text{与}(f\text{的定义域为}I)\\\text{与}(\forall k)(\forall i)(k\in K\text{与}i\in I\Rightarrow f(i)\in X_{i, k})$.
			\par
			另一方面,$f\in \prod\limits_{i\in I}(\bigcap\limits_{k\in K}X_{i, k})\Leftrightarrow (f\text{为函数图})\text{与}(f\text{的定义域为}I)\text{与}(\forall i)(i\in I\Rightarrow f(i)\in \bigcap\limits_{k\in K}X_{i, k})$,等价于$(f\text{为函数图})\text{与}(f\text{的定义域为}I)\text{与}(\forall k)(\forall i)(k\in K\text{与}i\in I\Rightarrow f(i)\in X_{i, k})$,得证.
			
			\begin{theo}\label{theo53}
				\hfill\par
				令$(X_i)_{i\in I}$、$(Y_i)_{i\in I}$为指标集相同的集族,且$I\neq \varnothing$,则:
				\par
				(1)$(\prod\limits_{i\in I}X_i)\cap(\prod\limits_{i\in I}Y_i)=\prod\limits_{i\in I}(X_i\cap Y_k)$.
				\par
				(2)$(\bigcap\limits_{i\in I}X_i)\times (\bigcap\limits_{i\in I}Y_i)=\bigcap\limits_{i\in I}(X_i\times Y_k)$.
			\end{theo}
			证明:类似定理\ref{theo48}的证明,根据定理\ref{theo52}可证.
			
			\begin{cor}\label{cor140}
				\hfill\par
				令$(X_i)_{i\in I}$、$(Y_i)_{i\in I}$为指标集相同的集族.对任意$i\in I$,$g_i$为$X_i$到$Y_i$的映射.对任意$f\in \prod\limits_{i\in I}X_i$,令$u_f$为映射$i\mapsto g_i(f(i))(i\in I)$,则$u_f\in \prod\limits_{i\in I}Y_i$.
			\end{cor}
			证明:对任意$f\in \prod\limits_{i\in I}X_i$、任意$i\in I$,$g_i(f(i))\in Y_i$,故$u_f\in \prod\limits_{i\in I}Y_i$.
			
			\begin{de}
				\textbf{映射族在乘积上的规范扩展(extension canonique de famille d'applications aux produit),映射族的乘积(produit de famille d'applications)}
				\par
				令$(X_i)_{i\in I}$、$(Y_i)_{i\in I}$为指标集相同的集族,$(g_i)_{i\in I}$为函数族,且对任意$i\in I$,$g_i$为$X_i$到$Y_i$的映射.对任意$f\in \prod\limits_{i\in I}X_i$,令$u_f$为映射$i\mapsto g_i(f(i))(i\in I)$的图,则映射$f\mapsto u_f(f\in \prod\limits_{i\in I}X_i, u_f\\\in \prod\limits_{i\in I}Y_i)$称为映射族$(g_i)_{i\in I}$在乘积上的规范扩展,或称为映射族的乘积.
			\end{de}
			
			\begin{theo}\label{theo54}
				\hfill\par
				令$(X_i)_{i\in I}$、$(Y_i)_{i\in I}$、$(Z_i)_{i\in I}$为指标集相同的集族,$(g_i)_{i\in I}$、$({g'}_i)_{i\in I}$为指标集相同的函数族,且对任意$i\in I$,$g_i$为$X_i$到$Y_i$的映射,${g'}_i$为$Y_i$到$Z_i$的映射.设$g$和$g'$分别为$(g_i)_{i\in I}$和$({g'}_i)_{i\in I}$\\在乘积上的规范扩展,则$g'\circ g$为$({g'}_i\circ g_i)_{i\in I}$在乘积上的规范扩展.
			\end{theo}
			证明:根据定义可证.
			
			\begin{cor}\label{cor141}
				\hfill\par
				令$(X_i)_{i\in I}$为集族,$\prod\limits_{i\in I}X_i=A$,则$(Id_{x_i}){i\in I}$在乘积上的规范扩展为$Id_A$.
			\end{cor}
			证明:令$u_f$为映射$i\mapsto Id_{x_i}(f(i))(i\in I)$的图,即为映射$i\mapsto f(i)$的图,因此,$u_f=f$,得证.

			\begin{theo}\label{theo55}
				\hfill\par
				令$(X_i)_{i\in I}$、$(Y_i)_{i\in I}$为指标集相同的集族,$(g_i)_{i\in I}$为函数族,且对任$i\in I$,$g_i$为$X_i$到$Y_i$的单射(或满射),则$(g_i)_{i\in I}$在乘积上的规范扩展为$\prod\limits_{i\in I}X_i$到$\prod\limits_{i\in I}Y_i$的单射(或满射).
			\end{theo}
			证明:
			\par
			如果存在$i'\in I$,使${X'}_i=\varnothing$,则$\prod\limits_{i\in I}X_i=\varnothing$.如果对任意$i\in I$,$g_i$均为单射,其在乘积上的规范扩展为$(\varnothing, \varnothing, \prod\limits_{i\in I}Y_i)$,是单射.如果对任意$i\in I$,$g_i$均为满射,则${Y'}_i=\varnothing$,故$\prod\limits_{i\in I}Y_i=\varnothing$,因此其在乘积上的规范扩展为$(\varnothing, \varnothing, \varnothing)$,是满射.
			\par
			如果对任意$i\in I$,均有$X_i\neq \varnothing$:
			\par
			对于单射的情况,令$r_i$为$g_i$的左逆,$r$为$(r_i)_{i\in I}$在乘积上的规范扩展,根据定理\ref{theo54},$r\circ g$为$(Id_{x_i})_{i\in I}$在乘积上的规范扩展,根据补充定理\ref{cor140},即$Id_{\prod\limits_{i\in I}X_i}$,其为双射,故$(g_i)_{i\in I}$在乘积上的规范扩展为$\prod\limits_{i\in I}X_i$到$\prod\limits_{i\in I}Y_i$的单射.
			\par
			对于满射的情况,令$s_i$为$g_i$的右逆,同理可证.
						
			\begin{cor}\label{cor142}
				\hfill\par
				令$(X_i)_{i\in I}$为集族,$f$为$E$到$\prod\limits_{i\in I}X_i$的映射,$\tilde{f}$为$(pr_i\circ f)_{i\in I}$在乘积上的规范扩展,$d$为$E$到$E^I$\\的对角映射,则$f=\tilde{f}\circ d$.
			\end{cor}
			证明:设$x\in E$,则$d(x)=\{z|pr_1z\in I\text{与}pr_2z=x\}$,$\tilde{f}\circ d(x)$为映射$i\mapsto pr_i\circ f(x)(i\in I)$,即为映射$i\mapsto (f(x))(i)(i\in I)$,即为映射$f(x)$,得证.
			
			\begin{cor}\label{cor143}
				\hfill\par
				令$(X_i)_{i\in I}$为集族,$(f_i)_{i\in I}$为函数族,且对任意$i\in I$,$f_i$为$E$到$X_i$的映射,$\tilde{f}$为$(f_i)_{i\in I}$在乘积上的规范扩展,$d$为$E$到$E^I$的对角映射,则对任意$i\in I$,$pr_i\circ (\tilde{f}\circ d)=f_i$.
			\end{cor}
			证明:设$x\in E$,则$d(x) =\{z|pr_1z\in I\text{与}pr_2z=x\}$,$\tilde{f}\circ d(x)$为映射$i\mapsto f_i(x)$,故$pr_i\circ (\tilde{f}\circ d)=f_i$.
			
			\begin{cor}\label{cor144}
				\hfill\par
				令$(X_i)_{i\in I}$为集族,$(f_i)_{i\in I}$为函数族,且对任意$i\in I$,$f_i$为$E$到$X_i$的映射,$\tilde{f}$为$(f_i)_{i\in I}$在乘积上的规范扩展,则$f\mapsto \tilde{f}$为$(\prod\limits_{i\in I}X_i)^E$到$\prod\limits_{i\in I}({X_i}^E)$的双射.
			\end{cor}
			证明:根据补充定理\ref{cor142}、补充定理\ref{cor143}可证.

			\begin{de}
				\textbf{积和映射的图的集合之间的规范映射(application canonique entre le produit et la ensemble des graphe d'applications)}
			\end{de}			
			令$(X_i)_{i\in I}$为集族,$f$为$E$到$\prod\limits_{i\in I}X_i$的映射,令$\tilde{f}$为$(pr_i\circ f)_{i\in I}$在乘积上的规范扩展,则称映射$f\mapsto \tilde{f}$为$(\prod\limits_{i\in I}X_i)^E$到$\prod\limits_{i\in I}({X_i}^E)$的规范映射,其逆映射为$\prod\limits_{i\in I}({X_i}^E)$到$(\prod\limits_{i\in I}X_i)^E$的规范映射.			
			
			\begin{exer}\label{exer62}
				\hfill\par
				$(X\subset Y)\Rightarrow( \mathcal{P}(X)\subset \mathcal{P}(Y))$.
			\end{exer}
			证明:即补充定理\ref{cor118}.
			
			\begin{exer}\label{exer63}
				\hfill\par
				$f$为$\mathcal{P}(E)$到$\mathcal{P}(E)$的映射,且$(\forall X)(\forall Y)(X\in \mathcal{P}(E)\text{与}Y\in \mathcal{P}(E)\text{与}X\subset Y\Rightarrow f(X)\subset f(Y))$.令$V=\bigcap\limits_{Z\subset E\text{与}f(Z)\subset Z}Z$,$W=\bigcup\limits_{Z\subset E\text{与}Z\subset f(Z)}Z$,求证:$V=f(V)$,$W=f(W)$,并且,$(\forall Z)(Z\subset E\text{与}f(Z)=Z\Rightarrow V\subset Z\text{与}Z\subset W)$.
			\end{exer}
			证明:
			对任意$Z\in \{Z|Z\subset E\text{与}f(Z)\subset Z\}$,都有$V\subset Z$,因此$f(V)\subset f(Z)$,则$f(V)\subset Z$,故$f(V)\subset V$,因此$f(f(V))\subset f(V)$,故 $f(V)\in \{Z|Z\subset E\text{与}f(Z)\subset Z\}$,因此$V\subset f(V)$,综上,$V=f(V)$.同理可证$W=f(W)$.
			\par
			此外,$f(Z)=Z$,则$Z\in \{Z|Z\subset E\text{与}f(Z)\subset Z\}$、$Z\in \{Z|Z\subset E\text{与}Z\subset f(Z)\}$,根据定义$V\subset Z\text{与}Z\subset W$.
			
			\begin{exer}\label{exer64}
				\hfill\par
				$(X_i)_{i\in I}$为集族,$I\neq \varnothing$,而集族$(Y_i)_{i\in I}$满足$(\forall i)(i\in I\Rightarrow Y_i\subset X_i)$,求证:$\prod\limits_{i\in I}Y_i=\bigcap\limits_{i\in I}{pr_i}^{-1}(Y_i)$.
			\end{exer}
			证明:
			\par
			设$f\in \prod\limits_{i\in I}Y_i$,则对任意$i\in I$,$pr_if\in Y_i$.故$f\in {pr_i}^{-1}(Y_i)$,因此$f\in \bigcap\limits_{i\in I}{pr_i}^{-1}(Y_i)$.
			\par
			反过来,设$f\in \bigcap\limits_{i\in I}{pr_i}^{-1}(Y_i)$,则对任意$i\in I$,$f\in {pr_i}^{-1}(Y_i)$,因此$pr_if\in Y_i$,且$f$是定义域为$I$的函数图,故$f\in \prod\limits_{i\in I}Y_i$.得证.
			
			\begin{exer}\label{exer65}
				\hfill\par
				对任意$G\subset A\times B$,令$\tilde{G}$为$A$到 $\mathcal{P}(B)$的映射$x\mapsto G\langle \{x\} \rangle $,求证:$G\mapsto \tilde{G}$为 $\mathcal{P}(A\times B)$到$( \mathcal{P}(B))^A$的双射.
			\end{exer}
			证明:对任意$A$到$\mathcal{P}(B)$的映射$f$,令$G=\bigcup\limits_{x\in A}(\{x\}\times f(x))$,则对任意$x\in A$,$G\langle \{x\} \rangle =f(x)$;设对任意$x\in A$,$G'\langle \{x\} \rangle =f(x)$,则$G'\langle \{x\} \rangle =G\langle \{x\} \rangle $,故$\{y|(x, y)\in G\}=\{y|(x, y)\\\in G'\}$,则$G=G'$,故$G$具有唯一性.因此$G\mapsto f$为$\mathcal{P}(A\times B)$到$\mathcal{F}(A; \mathcal{P}(B))$的双射.根据补充定理\ref{cor123},$\tilde{G}\mapsto f$为$( \mathcal{P}(B))^A$到$\mathcal{F}(A; \mathcal{P}(B))$的双射.根据定理\ref{theo21}(3)、\ref{theo21}(5)得证.
			
			\begin{exer}\label{exer66}
				\hfill\par
				设$(X_i)_{1\leq i\leq n}$为集族.对任意$[1, n]$的子集$H$,令$P_H=\bigcup\limits_{i\in H}X_i$,$Q_H=\bigcap\limits_{i\in H}X_i$.令$F_k$为\\$[1, n]$中元素数目为$k$的子集集合,求证:
				\par
				(1)	如果$k\leq (n+1)/2$,则$\bigcap\limits_{H\in F_k}P_H\subset \bigcup\limits_{H\in F_k}Q_H$.
				\par
				(2)	如果$k\geq (n+1)/2$,则$\bigcup\limits_{H\in F_k}Q_H\subset \bigcap\limits_{H\in F_k}P_H$.
			\end{exer}
			证明:
			\par
			如果$f\in I$,根据定理\ref{theo47},$\bigcap\limits_{H\in F_k}P_H=\bigcup\limits_{f\in I}(\bigcap\limits_{H\in F_k}X_f(H))$,其中$I=\prod\limits_{H\in F_k}H$.如果$f\in I$,则$f(H)\in H$.
			\par
			(1)如果$k\leq (n+1)/2$,设$x\in \bigcap\limits_{f\in I}(\bigcup\limits_{H\in F_k}X_{f(H)})$,则存在$f\in I$,使$x\in \bigcap\limits_{H\in F_k}X_{f(H)}$.假设$pr_2f$的元素数目小于$k$,则$[1, n]-pr_2f$的元素数目大于等于$k$,因此存在$[1, n]$的$k$个元素的子集$H$,是$[1, n]-pr_2f$的子集,因此$H\cap pr_2f=\varnothing$.而$f(H)\in H\cap pr_2f$,矛盾.因此,存在$k$个元素的子集$H$,即$H\in F_k$,使$x\in \bigcap\limits_{i\in H}X_i$.因此,$x\in \bigcup\limits_{H\in F_k}Q_H$.
			\par
			(2)	如果$k\geq (n+1)/2$,设$x\in \bigcup\limits_{H\in F_k}Q_H$,则存在$H'\in F_k$,使$x\in \bigcap\limits_{i\in H'}X_i$.对任意$H\in F_k$,$H\cap H'\neq \varnothing$,根据定理\ref{theo44},$\prod\limits_{H\in F_k}(H\cap H')\neq \varnothing$,根据定理\ref{theo45},$\prod\limits_{H\in F_k}(H\cap H')\subset I$.因此存在$f\in \prod\limits_{H\in F_k}(H\cap H')$,且$f\in I$.因此对任意$H\in F_k$,$f(H)\in H\cap H'$,即存在$i\in H'$,使$i=f(H)$,因此$X_i=X_{f(H)}$,由于$X_{f(H)}\subset \bigcap\limits_{i\in H'}X_i$,故$x\in X_{f(H)}$.因而$x\in \bigcap\limits_{H\in F_k}X_{f(H)}$,故$x\in \bigcap\limits_{H\in F_k}P_H$.
			注:习题\ref{exer66}涉及尚未介绍的“自然数”知识.
		
		\section{等价关系(Relations d'équivalence)}		
			\begin{metadef}
				\textbf{对称性(symétrique)}
				\par
				令$R$为包含$2$元特别符号$\in$ 、显式公理\ref{ex1}、显式公理\ref{ex2}、显式公理\ref{ex3}和公理模式\ref{Sch8}的等式理论$M$的公式,$x$、$y$、$z$为不同的不是常数的字母,且$R$不包含$z$,如果$R\Rightarrow (x|z)(y|x)(z|y)R$,则称$R$关于$x$、$y$具有对称性,在没有歧义的情况下也可以简称$R$具有对称性.
			\end{metadef}
			注:$(x|z)(y|x)(z|y)R$即为将$R$中的$x$、$y$对换得到的公式.

			\begin{metadef}
				\textbf{传递性(transitive)}
				\par
				令$R$为包含$2$元特别符号$\in$ 、显式公理\ref{ex1}、显式公理\ref{ex2}、显式公理\ref{ex3}和公理模式\ref{Sch8}的等式理论$M$的公式,$x$、$y$、$z$为不同的不是常数的字母,且$R$不包含$z$,如果$R\text{与}(y|x)(z|y)R\Rightarrow (z|y)R$,则称$R$关于$x$、$y$具有传递性,在没有歧义的情况下也可以简称$R$具有传递性.
			\end{metadef}

			\begin{metadef}
				\textbf{等价关系(relation d'équivalence)}
				\par
				令$R$为包含$2$元特别符号$\in$ 、显式公理\ref{ex1}、显式公理\ref{ex2}、显式公理\ref{ex3}和公理模式\ref{Sch8}的等式理论$M$的公式,如果$R$关于$x$、$y$具有对称性和传递性,则称$R$为关于$x$、$y$的等价关系,或称$x$模$R$等价于$y$,记作$x\equiv y(mod R)$,在没有歧义的情况下也可以简称$R$为等价关系.
			\end{metadef}

			\begin{Ccor}\label{Ccor22}
				\hfill\par
				包含$2$元特别符号$\in$ 、显式公理\ref{ex1}、显式公理\ref{ex2}、显式公理\ref{ex3}和公理模式\ref{Sch8}的等式理论$M$中,$R$为关于$x$、$y$的等价关系,则:
				\par
				(1),令$A$、$B$为项,如果$A$不包含$y$、$B$不包含$x$,则$(A|x)(B|y)$为关于$x$、$y$的等价关系;
				\par
				(2)$R$为关于$x$、$y$的等价关系,令$u$、$v$为字母,如果$R$不包含$u$、$v$,则$(u|x)(v|y)R$为关于$u$、$v$的等价关系.
			\end{Ccor}
			证明:根据定义可证.

			\begin{sign}
				\textbf{两个项的等价(équivalence de deux termes)}
				\par
				$R$为关于$x$、$y$的等价关系,如果$A$不包含$y$、$B$不包含$x$,则$(A|x)(B|y)R$记作\\$A\equiv B(mod R)$.
			\end{sign}

			\begin{Ccor}\label{Ccor23}
				\hfill\par
				如果公式$R$为关于$x$、$y$的等价关系,则$R\Rightarrow (y|x)R\text{与}(x|y)R$.
			\end{Ccor}
			证明:令$z$为不同于$x$、$y$的不是常数的字母,且$R$不包含$z$.如果$R$为真,则\\$(x|z)(y|x)(z|y)R$,同时$R\text{与}(y|x)(z|y)R\Rightarrow (z|y)R$.因此$(x|z)(y|x)(z|y)R\Rightarrow (x|z)(z|y)R$,\\故$(x|z)(z|y)R$,即$(y|x)R$.因此,$(x|z)(y|x)(z|y)R$.根据补充替代规则\ref{CScor4},$(x|z)(y|x)(z|y)R$和\\$(y|z)(x|y)(z|x)R$相同,故同理可证$(x|y)R$.

			\begin{Ccor}\label{Ccor24}
				\hfill\par
				包含$2$元特别符号$\in$ 、显式公理\ref{ex1}、显式公理\ref{ex2}、显式公理\ref{ex3}和公理模式\ref{Sch8}的等式理论$M$中,$R$、$S$均为关于$x$、$y$的等价关系,则“$(R\text{与}S)$”为关于$x$、$y$的等价关系.
			\end{Ccor}
			证明:根据定义可证.				
				
			\begin{metadef}
				\textbf{反身性(réflexive)}
				\par
				包含$2$元特别符号$\in$ 、显式公理\ref{ex1}、显式公理\ref{ex2}、显式公理\ref{ex3}和公理模式\ref{Sch8}的等式理论$M$中,令$R$为公式,$x$、$y$为不同的不是常数的字母,如果$(x|y)R\Leftrightarrow x\in E$,则称$R$关于$x$、$y$在$E$上具有反身性,在没有歧义的情况下也可以简称$R$在$E$上具有反身性,或简称$R$具有反身性.
			\end{metadef}

			\begin{metadef}
				\textbf{在集合上的等价关系(relation d'équivalence dans un ensemble)}
				\par
				包含$2$元特别符号$\in$ 、显式公理\ref{ex1}、显式公理\ref{ex2}、显式公理\ref{ex3}和公理模式\ref{Sch8}的等式理论$M$中,令$R$为关于$x$、$y$的等价关系,并且$R$关于$x$、$y$在$E$上具有反身性,则称$R$为关于$x$、$y$在$E$上的等价关系,在没有歧义的情况下也可以简称$R$为在$E$上的等价关系..
			\end{metadef}

			\begin{Ccor}\label{Ccor25}
				\hfill\par
				包含$2$元特别符号$\in$ 、显式公理\ref{ex1}、显式公理\ref{ex2}、显式公理\ref{ex3}和公理模式\ref{Sch8}的等式理论$M$中,公式$R$为关于$x$、$y$在$E$上的等价关系,则$R\Rightarrow x\in E$,$R\Rightarrow y\in E$.				
			\end{Ccor}
			证明:根据补充证明规则\ref{Ccor23}可证.

			\begin{Ccor}\label{Ccor26}
				\hfill\par
				包含$2$元特别符号$\in$ 、显式公理\ref{ex1}、显式公理\ref{ex2}、显式公理\ref{ex3}和公理模式\ref{Sch8}的等式理论$M$中,公式$R$为关于$x$、$y$在$E$上的等价关系,则$R$为生成图的公式.				
			\end{Ccor}
			证明:根据补充证明规则\ref{Ccor25},$R\Rightarrow (x, y)\in E\times E$,根据补充证明规则\ref{Ccor18}可证.

			\begin{de}
				\textbf{等价图(graphe d'équivalence)}
				\par
				$G$为图,如果$(x, y)\in G$为在$E$上的等价关系,则称$G$为在$E$上的等价图.
			\end{de}

			\begin{cor}\label{cor145}
				\hfill\par
				包含$2$元特别符号$\in$ 、显式公理\ref{ex1}、显式公理\ref{ex2}、显式公理\ref{ex3}和公理模式\ref{Sch8}的等式理论$M$中,$(G_i)_{i\in I}$为集族,且$(\forall I)(i\in I\Rightarrow G_i\text{为在}E\text{上的等价图})$,则$\bigcap\limits_{i\in I}G_i$为在$E$上的等价图.
			\end{cor}				
			证明:
			\par
			如果$(x, y)\in \bigcap\limits_{i\in I}G_i$,则对任意$i\in I$,$(x, y)\in G_i$,因此$(y, x)\in G_i$,故$(y, x)\in \bigcap\limits_{i\in I}G_i$,因此对称性成立.
			\par
			同理可证传递性.
			\par
			对任意$i\in I$,$x\in E\Leftrightarrow (x, x)\in G_i$,因此$x\in E\Leftrightarrow (x, x)\in G_i$,故反身性成立.
			\par
			综上得证.

			\begin{Ccor}\label{Ccor27}
				\hfill\par
				包含$2$元特别符号$\in$ 、显式公理\ref{ex1}、显式公理\ref{ex2}、显式公理\ref{ex3}和公理模式\ref{Sch8}的等式理论$M$中,如果公式$R$为关于$x$、$y$的等价关系,则$(\exists y)R\Leftrightarrow (x|y)R$,$(\exists x)R\Leftrightarrow (y|x)R$.
			\end{Ccor}				
			证明:根据公理模式\ref{Sch5},$(x|y)R\Rightarrow (\exists y)R$.另一方面,根据补充证明规则\ref{Ccor23},$(\exists y)R\Rightarrow (\exists y)(x|y)R$,后者即$(x|y)R$.综上,则$(\exists y)R\Leftrightarrow (x|y)R$.
			\par
			同理可证$(\exists x)R\Leftrightarrow (y|x)R$.

			\begin{Ccor}\label{Ccor28}
				\hfill\par
				包含$2$元特别符号$\in$ 、显式公理\ref{ex1}、显式公理\ref{ex2}、显式公理\ref{ex3}和公理模式\ref{Sch8}的等式理论$M$中,如果公式$R$为关于$x$、$y$在$E$上的等价关系,$G$为其生成的图,则:$x\in pr_1G\Leftrightarrow (x|y)R$,$pr_1G=E$,$y\in pr_2G\Leftrightarrow (y|x)R$,$pr_2G=E$.
			\end{Ccor}				
			证明:根据补充证明规则\ref{Ccor27}可证.

			\begin{Ccor}\label{Ccor29}
				\hfill\par
				包含$2$元特别符号$\in$ 、显式公理\ref{ex1}、显式公理\ref{ex2}、显式公理\ref{ex3}和公理模式\ref{Sch8}的等式理论$M$中,公式R为关于x、y在E上的等价关系,G为其生成的图,则:
				\par
				(1)$x\in E\Leftrightarrow (x, x)\in G$;
				\par
				(2)$(x, y)\in G\text{与}(y, z) \in G\Rightarrow (y, z)\in G$;
				\par
				(3)$(x, y)\in G\Rightarrow (x, x)\in G\text{与}(y, y)\in G$.
				\par
				(4)$x\in E\Rightarrow x\in G(x)$.
			\end{Ccor}				
			证明:根据补充证明规则\ref{Ccor23}可证.
			
			\begin{Ccor}\label{Ccor30}
				\hfill\par
				包含$2$元特别符号$\in$ 、显式公理\ref{ex1}、显式公理\ref{ex2}、显式公理\ref{ex3}和公理模式\ref{Sch8}的等式理论$M$\\中:
				\par
				(1)	如果公式$R$、$S$均为关于$x$、$y$在$E$上的等价关系,则“$R\text{与}S$”为关于$x$、$y$在$E$上的等价关系.
				\par
				(2)	如果公式$S$为关于$x$、$y$在$F$上的等价关系,$f$为$E$到$F$的满射,则$(f(y)|y)(f(x)|x)S$\\为关于$x$、$y$在$E$上的等价关系.
				\par
				(3)	如果公式$R$为关于$x$、$y$在$E$上的等价关系,$S$为关于$x$、$y$在$F$上的等价关系,$f$为$E$\\到$F$的映射,则$R\text{与}(f(y)|y)(f(x)|x)S$为关于$x$、$y$在$E$上的等价关系.
			\end{Ccor}				
			证明:
			\par
			(1)根据补充证明规则\ref{Ccor24}可证.
			\par
			(2)对称性和传递性根据补充证明规则\ref{Ccor22}(1)、补充证明规则\ref{Ccor22}(2)可证.根据替代规则\ref{CS2},$(f(x)|y)(f(x)|x)S$和$(f(x)|x)(x|y)S$相同,故$(f(x)|y)(f(x)|x)S\Leftrightarrow f(x)\in F$,等价于$x\in E$,得证.
			\par
			(3)对称性和传递性根据补充证明规则\ref{Ccor22}(1)、补充证明规则\ref{Ccor22}(2)可证.根据替代规则\ref{CS2},$(f(x)|y)(f(x)|x)S$和$(f(x)|x)(x|y)S$相同,故$R\text{与}(f(x)|y)(f(x)|x)S\Leftrightarrow x\in E\text{与}f(x)\in F$,由于$x\in E\Rightarrow f(x)\in F$,故$R\text{与}(f(x)|y)(f(x)|x)S\Leftrightarrow x\in E$,得证.
			
			\begin{cor}\label{cor146}
				\hfill\par
				(1)$x=y$为等价关系,但不是在任何集合上的等价关系.
				\par
				(2)$x=y\text{与}x\in E$为在$E$上的等价关系.
				\par
				(3)$(\exists F)(F\text{为}X\text{到}Y\text{的双射})$为等价关系,但不是在任何集合上的等价关系.
				\par
				(4)$x\in E\text{与}y\in E$为在$E$上的等价关系.
				\par
				(5)如果$A\subset E$,则$(x\in E-A\text{与}y=x)\text{或}(x\in A\text{与}y\in A)$为在$E$上的等价关系.
				\par
				(6)令$f$为函数,其定义域为$E$,则公式“$x\in E\text{与}y\in E\text{与}f(x)=f(y)$”为在$E$上的等价关系.
			\end{cor}
			证明:根据定义可证以上各式的等价关系.
			\par
			对于$x=y$、$(\exists F)(F\text{为}X\text{到}Y\text{的双射})$,根据补充定理\ref{cor10}可以证明其不是在任何集合上的等价关系.
			
			\begin{de}
				\textbf{在集合上的等价对应(correspondance d'équivalence dans un ensemble)}
				\par
				如果$F$为在$E$上的等价图,则$(F, E, E)$称为在$E$上的等价对应.
			\end{de}
			
			\begin{theo}\label{theo56}
				\hfill\par
				当且仅当同时满足下列三个条件时,$X$到$X$的对应$F$为在$X$上的等价对应:
				\par
				第一,$X$为$F$的定义域;
				\par
				第二,$F=F^{-1}$;
				\par
				第三,$F\circ F= F$.
			\end{theo}
			证明:令$F=(G, X, X)$.如果$F$为在$X$上的等价对应,则$x\in X\Leftrightarrow (x, x)\in G$,故$F$的定义域为$X$;由于$(x, y)\in G\Leftrightarrow (y, x)\in G$,故$F=F^{-1}$;由于$(x, y)\in G\text{与}(y, z) \in G\Rightarrow (x, z)\in G$,故$G\circ G\subset G$,同时,$(x, y)\in G\Rightarrow (x, x)\in G$,故$(x, y)\in G\circ G$,故$G\subset G\circ G$,因此$G\circ G=G$.
			\par
			反过来,由于$F=F^{-1}$,因此$(x, y)\in G具有对称性$,由于$F\circ F= F$,因此$(x, y)\in G$具有传递性.由于$F$的定义域为$X$,因此对于$x\in X$,存在$(x, y)\in G$,则$(y, x) \in G$,因此$(x, x)\in G$;同时,如果$(x, x)\in G$,则$x\in X$,故$(x, y)\in G$在$X$上具有反身性.因此,$F$为在$X$上的等价对应.

			\begin{de}
				\textbf{同函数相关的等价关系(relation d'équivalence associée à une fonction)}
				\par
				令$f$为函数,其定义域为$E$,其图为$F$,则公式$(x\in E\text{与}y\in E\text{与}f(x)=f(y))$称为同$f$相关的等价关系.
			\end{de}
			
			\begin{cor}\label{cor147}
				\hfill\par
				如果$f$的图为$F$,则同$f$相关的等价关系生成的图为$F^{-1}\circ F$.
			\end{cor}
			证明:$x\in E\text{与}y\in E\text{与}f(x)=f(y)\Leftrightarrow (\exists z)((x, y)\in F\text{与}(y, z)\in F)$,进而等价于$(\exists z)((x, y)\in F\text{与}(y, z)\in F^{-1})$,等价于$(x, y)\in F^{-1}\circ F$,得证.

			\begin{metadef}
				\textbf{等价类(classe d'équivalence),代表(représentant)}
				\par
				包含$2$元特别符号$\in$ 、显式公理\ref{ex1}、显式公理\ref{ex2}、显式公理\ref{ex3}和公理模式\ref{Sch8}的等式理论$M$中,令$R$为关于$x$、$y$在$E$上的等价关系,$R$生成的图为$G$,且$x\in E$,则称$G(x)$为$x$关于$R$的等价类.如果$z\in (x\text{关于}R\text{的等价类})$,则称$z$为该等价类的代表.
			\end{metadef}

			\begin{Ccor}\label{Ccor31}
				\hfill\par
				包含$2$元特别符号$\in$ 、显式公理\ref{ex1}、显式公理\ref{ex2}、显式公理\ref{ex3}和公理模式\ref{Sch8}的等式理论$M$中,令$R$为关于$x$、$y$在$E$上的等价关系,$R$生成的图为$G$,且$x\in E$,则$(x\text{关于}R\text{的等价类})\subset E$.
			\end{Ccor}
			证明:根据补充证明规则\ref{Ccor25}可证.

			\begin{Ccor}\label{Ccor32}
				\hfill\par
				包含$2$元特别符号$\in$ 、显式公理\ref{ex1}、显式公理\ref{ex2}、显式公理\ref{ex3}和公理模式\ref{Sch8}的等式理论$M$中,令$R$为关于$x$、$y$在$E$上的等价关系,则$x\in(x\text{关于}R\text{的等价类})$.
			\end{Ccor}
			证明:根据补充证明规则\ref{Ccor29}(1)可证.

			\begin{Ccor}\label{Ccor33}
				\hfill\par
				包含$2$元特别符号$\in$ 、显式公理\ref{ex1}、显式公理\ref{ex2}、显式公理\ref{ex3}和公理模式\ref{Sch8}的等式理论$M$中,令$R$为关于$x$、$y$在$E$上的等价关系,$R$生成的图为$G$,则$(\exists x)(x\in E\text{与}X=G(x))$为关于$X$的集合化公式.
			\end{Ccor}
			证明:根据补充证明规则\ref{Ccor31},$(\exists x)(x\in E\text{与}X=G(x))\Rightarrow X\subset E$,即$x\in \mathcal{P}(E)$,根据证明规则\ref{C52}得证.

			\begin{metadef}
				\textbf{商集(ensemble quotient),到商集的规范映射(l'application canonique dans l'ensemble quotient)}
				\par
				包含$2$元特别符号$\in$ 、显式公理\ref{ex1}、显式公理\ref{ex2}、显式公理\ref{ex3}和公理模式\ref{Sch8}的等式理论$M$中,令$R$为关于$x$、$y$在$E$上的等价关系,$R$生成的图为$G$,则称$\{X|(\exists x)(x\in E\text{与}X=G(x))\}$为$E$\\除以$R$的商集,记作$E/R$.$x\mapsto G(x)((x\in E, G(x)\in E/R)$称为$E$到$E/R$的规范映射.
			\end{metadef}

			\begin{Ccor}\label{Ccor34}
				\hfill\par
				包含$2$元特别符号$\in$ 、显式公理\ref{ex1}、显式公理\ref{ex2}、显式公理\ref{ex3}和公理模式\ref{Sch8}的等式理论$M$中,令$R$为关于$x$、$y$在$E$上的等价关系,则:
				\par
				(1)$E=\varnothing\Leftrightarrow E/R=\varnothing$;
				\par
				(2)对任意$X\in E/R$,$X\neq \varnothing$.
			\end{Ccor}
			证明:
			\par
			(1)如果$E=\varnothing$,则$x\in E$为假,根据定义,$E/R=\varnothing$;同时,如果$E\neq \varnothing$,对任意$x\in E$,$G(x)\in E/R$,故$E/R\neq \varnothing$,得证.
			\par
			(2)令$R$生成的图为$G$,若$X=\varnothing$,则$(\exists x)(x\in E\text{与}\varnothing=G(x))$,故$(\exists x)(\varnothing=G(x))$,但$x\in G(x)$,矛盾,得证.

			\begin{Ccor}\label{Ccor35}
				\hfill\par
				包含$2$元特别符号$\in$ 、显式公理\ref{ex1}、显式公理\ref{ex2}、显式公理\ref{ex3}和公理模式\ref{Sch8}的等式理论$M$中,令$R$为关于$x$、$y$在$E$上的等价关系,则$E$到$E/R$的规范映射为满射.
			\end{Ccor}
			证明:根据定义可证.

			\begin{C}\label{C55}
				\hfill\par
				包含$2$元特别符号$\in$ 、显式公理\ref{ex1}、显式公理\ref{ex2}、显式公理\ref{ex3}和公理模式\ref{Sch8}的等式理论$M$中,令$R$为关于$x$、$y$在$E$上的等价关系,$p$为$E$到$E/R$的规范映射,则$R\Leftrightarrow (x\in E\text{与}y\in E\text{与}p(x)\\=p(y))$.
			\end{C}
			证明:令$R$生成的图为$G$,则$R\Leftrightarrow (x, y)\in G$.假设$(x, y)\in G$,则$x\in E\text{与}y\in E$,且$y\in G(x)$,因此$G(y)\subset G\circ G(x)$.根据定理\ref{theo56},$G(y)\subset G(x)$.同时,由于$(x, y)\in G$,故$(y, x)\in G$,同理$G(x)\subset G(y)$,因此$G(x)=G(y)$.
			反过来,假设$G(x)=G(y)$,由于$y\in G(y)$,故$y\in G(x)$,因此$(x, y)\in G$.

			\begin{Ccor}\label{Ccor36}
				\hfill\par
				包含$2$元特别符号$\in$ 、显式公理\ref{ex1}、显式公理\ref{ex2}、显式公理\ref{ex3}和公理模式\ref{Sch8}的等式理论$M$中,令$R$为关于$x$、$y$在$E$上的等价关系,$p$为$E$到$E/R$的规范映射,则:
				\par
				(1)$(\forall x)(\forall y)(x\in E\Rightarrow (p(y)=p(x)\Leftrightarrow y\in p(x)))$;
				\par
				(2)$(\forall x)(\forall X)(x\in E\text{与}X\in E/R\Rightarrow (X=p(x)\Leftrightarrow x\in X))$.
			\end{Ccor}
			证明:
			\par
			(1)令$R$生成的图为$G$,在$x\in E$的情况下,如果$y\in p(x)$,则$(x, y)\in G$,因此$p(x)=p(y)$.
			\par
			反过来,如果$p(x)=p(y)$,根据补充证明规则\ref{Ccor32},$y\in p(y)$,故$y\in p(x)$.
			\par
			(2)假设$x\in E$、$X\in E/R$,则存在$y\in E$,使$p(y)=X$,如果$X=p(x)$,则$p(x)=p(y)$,根据补充证明规则\ref{Ccor36}(1),$x\in X$.
			\par
			反过来,如果$x\in X$,根据补充证明规则\ref{Ccor36}(1),$p(x)=p(y)$,故$p(x)=X$.得证.

			\begin{metadef}
				\textbf{集合对于公式的截面(section d'un ensemble pour une relation)}
				\par
				包含$2$元特别符号$\in$ 、显式公理\ref{ex1}、显式公理\ref{ex2}、显式公理\ref{ex3}和公理模式\ref{Sch8}的等式理论$M$中,令$R$为关于$x$、$y$在$E$上的等价关系,$p$为$E$到$E/R$的规范映射,则$p$的右逆称为$E$对于$R$的截面.
			\end{metadef}

			\begin{Ccor}\label{Ccor37}
				\hfill\par
				包含$2$元特别符号$\in$ 、显式公理\ref{ex1}、显式公理\ref{ex2}、显式公理\ref{ex3}和公理模式\ref{Sch8}的等式理论$M$中,令$R$为关于$x$、$y$在$E$上的等价关系,则$\Delta_{E/R}$为$E$的划分.
			\end{Ccor}
			证明:如果$E=\varnothing$,根据补充证明规则34(1),$E/R=\varnothing$,$\Delta_{\varnothing}$=$\varnothing$,根据划分的定义可证.
			\par
			如果$E\neq \varnothing$,设$R$生成的图为$G$,$X\in E/R$,$X'\in E/R$,设$X=G(x)$,$X'=G(x')$,设$a\in X\cap X'$,则$(x, a)\in G$,$(x', a)\in G$,故$(x, x')\in G$,因此$G(x)= G(x')$,所以$X=X'$.同时,对任意$x\in E$,$x\in G(x)$.综上得证.
			\par
			注:$\Delta_{E/R}$即集族$(X)_{X\in E/R}$.

			\begin{Ccor}\label{Ccor38}
				\hfill\par
				包含$2$元特别符号$\in$ 、显式公理\ref{ex1}、显式公理\ref{ex2}、显式公理\ref{ex3}和公理模式\ref{Sch8}的等式理论$M$中,令$R$为关于$x$、$y$在$E$上的等价关系,则:
				\par
				(1)$R\Leftrightarrow ((\exists X)(x\in X\text{与}y\in X\text{与}X\in E/R))$.
				\par
				(2)$X=G(x)\Leftrightarrow X\in E/R\text{与}x\in X$.
			\end{Ccor}
			证明:
			\par
			(1)设$R$生成的图为$G$,则$R\Leftrightarrow (x, y)\in G$,如果$(x, y)\in G$,则$(\exists X)(x\in X\text{与}X\in E/R)$.若$x\in X$,根据证明规则\ref{C55},则$G(y)=X$,故$y\in X$,因此$(\exists X)(x\in X\text{与}y\in X\text{与}X\in E/R)$.
			\par
			反过来,若$(\exists X)(x\in X\text{与}y\in X\text{与}X\in E/R)$,则$x\in E$、$y\in E$,且$G(x)=G(y)$,根据证明规则\ref{C55},$(x, y)\in G$.
			\par
			(2)如果$X=G(x)$,根据定义,$X\in E/R$、$x\in X$.反过来,如果$X\in E/R$、$x\in X$,由于$G(x)\in E/R$,如果$X\neq G(x)$,则$X\bigcap\limits_G(x)=\varnothing$,矛盾,故$X=G(x)$.

			\begin{cor}\label{cor148}
				\hfill\par
				令$(X_i)_{i\in I}$为$E$的划分,且$i\neq \varnothing$,$(\forall i)(i\in I\Rightarrow X_i\neq \varnothing)$,设$R$为$(\exists i)(i\in I\text{与}x\in X_i\text{与}y\in X_i)$,则$R$为关于$x$、$y$在$E$上的等价关系,且$i\mapsto X_i(i\in I)$为$I$到$E/R$的双射.
			\end{cor}
			证明:$(\exists i)(i\in I\text{与}x\in X_i\text{与}y\in X_i)\Leftrightarrow (\exists i)(i\in I\text{与}y\in X_i\text{与}x\in X_i)$,故该公式具有对称性.若$(\exists i)(i\in I\text{与}x\in X_i\text{与}y\in X_i)$、$(\exists i)(i\in I\text{与}y\in X_i\text{与}z\in X_i)$,设$i\in I$、$i'\in I$,$x\in X_i\text{与}y\in X_i$,使$y\in X_i'$、$z\in X_i'$,故$i=i'$,因此$(\exists i)(i\in I\text{与}x\in X_i\text{与}z\in X_i)$,即该公式具有传递性.若$x\in E$,则存在$i$,使$(i\in I\text{与}x\in X_i)$,故反身性成立.综上,$R$为关于$x$、$y$在$E$上的等价关系.
			\par
			令$R$生成的图为$G$,对任意$i\in I$,设$x\in X_i$,则$y\in G(x)\Leftrightarrow (\exists i)(i\in I\text{与}x\in X_i\text{与}y\in X_i)$,等价于$y\in X_i$,故$G(x)=X_i$.所以$X_i\in E/R$,即该映射到达域为$E/R$.对任意$X\in E/R$,设$x\in X$,故$X=G(x)$,同时存在$i\in I$,使$x\in X_i$.则$X=X_i$,故该映射为满射.又因为$(X_i)_{i\in I}$为$E$的划分,且$i\neq \varnothing$,$(\forall i)(i\in I\Rightarrow X_i\neq \varnothing)$,故$X_i=X_i'\Rightarrow i=i'$,因此该映射为单射.
			
			\begin{de}
				\textbf{代表系统(système de représentants)}
				\par
				令$(X_i)_{i\in I}$为$E$的划分,且$i\neq \varnothing$,$(\forall i)(i\in I\Rightarrow X_i\neq \varnothing)$,设$R$为$(\exists i)(i\in I\text{与}x\in X_i\text{与}y\in X_i)$,$S\subset E$,且$(\forall i)(\exists x)(S\cap X_i=\{x\})$,则称$S$为$R$的等价类的代表系统.
			\end{de}

			\begin{metadef}
				\textbf{同等价关系相容的公式(relation compatible avec une relation \\d'équivalence)}
				\par
				包含$2$元特别符号$\in$ 、显式公理\ref{ex1}、显式公理\ref{ex2}、显式公理\ref{ex3}和公理模式\ref{Sch8}的等式理论$M$中,令$R$为关于$x$、$x'$在$E$上的等价关系,$P$为公式:
				\par
				如果$P$不包含$x'$,且$P\text{与}R\Rightarrow (x'|x)P$,则称$P$关于$x$同等价关系$R$相容,在没有歧义的情况下也可以简称$P$同等价关系$R$相容;
				\par
				如果$P$不包含$x'$、$y'$,R不包含$y$、$y'$,且$P\text{与}R\text{与}(y|x)(y'|x')R\Rightarrow (x'|x)(y'|y)P$,则称$P$关于$x$、$y$同等价关系$R$相容,在没有歧义的情况下也可以简称$P$同等价关系$R$相容.
			\end{metadef}	

			\begin{C}\label{C56}
				\hfill\par
				包含$2$元特别符号$\in$ 、显式公理\ref{ex1}、显式公理\ref{ex2}、显式公理\ref{ex3}和公理模式\ref{Sch8}的等式理论$M$中,令$R$为关于$x$、$x'$在$E$上的等价关系,$P$为公式,$t$为字母,如果$P$关于$x$同等价关系$R$相容,则$t\in E/R\text{与}(\exists x)(x\in t\text{与}P)\Leftrightarrow t\in E/R\text{与}(\forall x)(x\in t\Rightarrow P)$.
			\end{C}
			证明:在$t\in E/R$的情况下,如果$(\exists x)(x\in t\text{与}P)$为真,即存在$a\in t$,使$(a|x)P$、\\$(x|x')(a|x)R$对一切$x\in t$为真,故$P$对一切$x\in t$为真,因此$(\forall a)(a\in t\Rightarrow (a|x)P)$;反过来,若$(\forall x)(x\in t\Rightarrow P)$,由于$t\neq \varnothing$,则$(\exists x)(x\in t\text{与}P)$.得证.
			
			\begin{Ccor}\label{Ccor39}
				\hfill\par
				包含$2$元特别符号$\in$ 、显式公理\ref{ex1}、显式公理\ref{ex2}、显式公理\ref{ex3}和公理模式\ref{Sch8}的等式理论$M$中,令$R$为关于$x$、$x'$在$E$上的等价关系,$P$为公式,$t$为字母,如果$R$不包含$y$,且$P$关于$x$、$y$同等价关系$R$相容,则$t\in E/R\text{与}u\in E/R\text{与}(\exists x)(\exists y)(x\in t\text{与}y\in u\text{与}P)\Leftrightarrow t\in E/R\text{与}u\in E/R\text{与}(\forall x)(\forall y)(x\in t\text{与}y\in u\Rightarrow P)$.
			\end{Ccor}
			证明:类似证明规则\ref{C56}可证.

			\begin{metadef}
				\textbf{通过商导出的公式(relation déduite par passage au quotient)}
				\par
				包含$2$元特别符号$\in$ 、显式公理\ref{ex1}、显式公理\ref{ex2}、显式公理\ref{ex3}和公理模式\ref{Sch8}的等式理论$M$中,令$R$为关于$x$、$x'$在$E$上的等价关系,且$E\neq \varnothing$,$P$为公式:
				\par
				如果$P$关于$x$同等价关系$R$相容,$t$为字母且$P$不包含$t$,则$t\in E/R\text{与}(\exists x)(x\in t\text{与}P)$ 称为$P$关于$x$对于$R$通过商导出的公式,在没有歧义的情况下也可以简称为$P$对于$R$通过商导出的公式.
				\par
				如果$R$不包含$y$,且$P$关于$x$、$y$同等价关系$R$相容,$t$、$u$为字母且$P$不包含$t$、$u$,则$t\in E/R\text{与}u\in E/R\text{与}(\exists x)(\exists y)(x\in t\text{与}y\in u\text{与}P)$称为$P$关于$x$、$y$对于$R$通过商导出的公式,在没有歧义的情况下也可以简称为$P$对于$R$通过商导出的公式.
			\end{metadef}

			\begin{Ccor}\label{Ccor40}
				\hfill\par
				包含$2$元特别符号$\in$ 、显式公理\ref{ex1}、显式公理\ref{ex2}、显式公理\ref{ex3}和公理模式\ref{Sch8}的等式理论$M$中,令$R$为关于$x$、$x'$在$E$上的等价关系,$P$为公式,$t$为字母:
				\par
				(1)	如果$P$关于$x$同等价关系$R$相容,$P'$为$P$为关于$x$对于$R$通过商导出的公式,$f$为$E$到\\$E/R$的规范映射,则$z\in E\text{与}(f(z)|t)P'\Leftrightarrow z\in E\text{与}(z|x)P$.
				\par
				(2)	如果$R$不包含$y$,且$P$关于$x$、$y$同等价关系$R$相容,$P'$为$P$为关于$x$、$y$对于$R$通过商导出的公式,$f$为$E$到$E/R$的规范映射,则$z\in E\text{与}w\in E\text{与}(f(z)|t)(f(w)|u)P'\Leftrightarrow z\in E\text{与}w\in E\text{与}(z|x)(w|y)P$.
			\end{Ccor}
			证明:
			\par
			(1)若$z\in E$,根据证明规则\ref{C56},$(f(z)|t)P'\Leftrightarrow (\forall x)(x\in f(z)\Rightarrow P)$.如果$(\forall x)(x\in f(z)\Rightarrow P)$,则$(z|x)P$.反过来,如果$(z|x)P$,根据证明规则\ref{C56},$(\forall x)(x\in f(z)\Rightarrow P)$.得证.
			\par
			(2)类似补充证明规则\ref{Ccor40}(1)可证.

			\begin{metadef}
				\textbf{浸润子集(partie saturée)}
				\par
				包含$2$元特别符号$\in$ 、显式公理\ref{ex1}、显式公理\ref{ex2}、显式公理\ref{ex3}和公理模式\ref{Sch8}的等式理论$M$中,令$R$为关于$x$、$y$在$E$上的等价关系,$A\subset E$,如果$x\in A$关于$x$同等价关系$R$相容,则称$A$为$E$\\对于$R$的浸润子集.
			\end{metadef}

			\begin{Ccor}\label{Ccor41}
				\hfill\par
				包含$2$元特别符号$\in$ 、显式公理\ref{ex1}、显式公理\ref{ex2}、显式公理\ref{ex3}和公理模式\ref{Sch8}的等式理论$M$中,令$R$为关于$x$、$y$在$E$上的等价关系,$A\subset E$,则:
				\par
				(1)如果$A$为$E$对$R$的浸润子集,则$x\in A\Rightarrow G(x)\subset A$.
				\par
				(2)$(\exists F)(F\subset E/R\text{与}A=\bigcup\limits_X\in FX)\Leftrightarrow (A\text{为}E\text{对于}R\text{的浸润子集})$.
			\end{Ccor}
			证明:令$R$生成的图为$G$.
			\par
			(1)如果$A$为$E$对于$R$的浸润子集,则$x\in A\text{与}(x, y)\in G\Rightarrow y\in A$.对任意$y\in G(x)$,$x\in A \Rightarrow y\in A$,因此$x\in A\Rightarrow G(x)\subset A$.
			\par
			(2)假设存在$F\subset E/R$,使$A=\bigcup\limits_{X\in F}X$,设$x\in A$,则存在$X\in F$,使$x\in X$,则$X\in E/R$,故$X=G(x)$.对于$(x, y)\in G$,可得$y\in X$,故$y\in A$,因此$A$为$E$对于$R$的浸润子集.
			\par
			反过来,如果$A$为$E$对于$R$的浸润子集,则$x\in A\text{与}(x, y)\in G\Rightarrow y\in A$.令$F=\{X|(X\in E/R\text{与}X\bigcap\limits_A\neq \varnothing\}$,故$F\subset E/R$.当$x\in A$时,$G(x)\in F$,故$x\in \bigcup\limits_{X\in F}X$;反过来,若$x\in \bigcup\limits_{X\in F}X$,则存在$X\in F$,使$x\in X$,因此$X=G(x)$,根据补充证明规则\ref{Ccor41}(1),$G(x)\subset A$,故$x\in A$.因此,$(\exists F)(F\subset E/R\text{与}A=\bigcup\limits_{X\in F}X)$.
			注:本补充证明规则表明,当且仅当子集是若干个等价类的并集时,其为浸润子集.
			
			\begin{Ccor}\label{Ccor42}
				\hfill\par
				包含$2$元特别符号$\in$ 、显式公理\ref{ex1}、显式公理\ref{ex2}、显式公理\ref{ex3}和公理模式\ref{Sch8}的等式理论$M$中,令$R$为关于$x$、$y$在$E$上的等价关系,$f$为$E$到$E/R$的规范映射,$A\subset E$,则$f^{-1}\langle f\langle A \rangle \rangle =A\Leftrightarrow (A\text{为}E\text{对于}R\text{的浸润子集})$.
			\end{Ccor}
			证明:设$f$的图为$G$.如果$A$为$E$对于$R$的浸润子集,对任意$x\in A$,$f\langle \{x\} \rangle =f(x)$,即等于$G(x)$,设$y\in E$且$f(y)=G(x)$,根据证明规则\ref{C55},$(x, y)\in G$,故$y\in G(x)$,因此$f^{-1}f\langle \{x\} \rangle \subset G(x)$,根据补充证明规则\ref{Ccor41}(1),$f^{-1}f\langle \{x\} \rangle \subset A$,因此$f^{-1}\langle f\langle A \rangle \rangle\subset A$,根据补充定理\ref{cor50},$f^{-1}\langle f\langle A \rangle \rangle=A$.另一方面,若$f^{-1}\langle f\langle A \rangle \rangle=A$,对任意$x\in A$,则$f(x)\in f\langle A \rangle $.令$K=f(x)$,根据补充证明规则\ref{Ccor36}(1),$y\in K\Leftrightarrow f(y)=K$,故$f^{-1}\langle\{K\}\rangle=K$,因此$K\subset f^{-1}\langle f\langle A \rangle \rangle$,所以$K\subset A$.因此,$A$为$E$对于$R$的浸润子集.
			
			\begin{Ccor}\label{Ccor43}
				\hfill\par
				包含$2$元特别符号$\in$ 、显式公理\ref{ex1}、显式公理\ref{ex2}、显式公理\ref{ex3}和公理模式\ref{Sch8}的等式理论$M$中,令$R$为关于$x$、$y$在$E$上的等价关系,$(X_i)_{i\in I}$为$E$的子集族,$(\forall i)(i\in I\Rightarrow X_i\text{为}E\text{对于}R\text{的浸润}\\\text{子集})$,则$\bigcup\limits_{i\in I}X_i$和$\bigcap\limits_{i\in I}X_i$都是$E$对于$R$的浸润子集.
			\end{Ccor}
			证明:根据定理\ref{theo26}、定理\ref{theo27}、补充证明规则\ref{Ccor42}可证.
			
			\begin{Ccor}\label{Ccor44}
				\hfill\par
				包含$2$元特别符号$\in$ 、显式公理\ref{ex1}、显式公理\ref{ex2}、显式公理\ref{ex3}和公理模式\ref{Sch8}的等式理论$M$中,令$R$为关于$x$、$y$在$E$上的等价关系,$A$为$E$对于$R$的浸润子集,则$\complement_EA$也是$E$对于$R$的浸润子集.
			\end{Ccor}
			证明:令$f$为$E$到$E/R$的规范映射,根据补充证明规则\ref{Ccor42},$A= f^{-1}\langle f\langle A \rangle \rangle$.根据补充定理\ref{cor70}(2),$E=f^{-1}\langle E/R\rangle$.根据定理\ref{theo30},$\complement_EA=f^{-1}\langle E/R-f\langle A \rangle \rangle$,根据补充证明规则\ref{Ccor42}得证.
			
			\begin{Ccor}\label{Ccor45}
				\hfill\par
				包含$2$元特别符号$\in$ 、显式公理\ref{ex1}、显式公理\ref{ex2}、显式公理\ref{ex3}和公理模式\ref{Sch8}的等式理论$M$中,令$R$为关于$x$、$y$在$E$上的等价关系,$f$为$E$到$E/R$的规范映射,$A\subset E$,则$f^{-1}\langle f\langle A \rangle \rangle$为$E$对于$R$的浸润子集.
			\end{Ccor}
			证明:令$R$生成的图为$G$.设$x\in f^{-1}\langle f\langle A \rangle \rangle$,则$f(x)\in f\langle A \rangle $.若$(x, y)\in G$,根据证明规则\ref{C55},$f(y)\in f\langle A \rangle $,故$y\in f^{-1}\langle f\langle A \rangle \rangle$,得证.
			\par

			\begin{Ccor}\label{Ccor46}
				\hfill\par
				包含$2$元特别符号$\in$ 、显式公理\ref{ex1}、显式公理\ref{ex2}、显式公理\ref{ex3}和公理模式\ref{Sch8}的等式理论$M$中,令$R$为关于$x$、$y$在$E$上的等价关系,$f$为$E$到$E/R$的规范映射,$A\subset E$,若$A'$为$E$对于$R$的浸润子集,且$A\subset A'$,则$f^{-1}\langle f\langle A \rangle \rangle\subset A'$.
			\end{Ccor}
			证明:由于$A\subset A'$,因此$f^{-1}\langle f\langle A \rangle \rangle\subset f^{-1}\langle f\langle A'\rangle \rangle$,根据补充证明规则\ref{Ccor42},得证.
			
			\begin{metadef}
				\textbf{子集的浸润子集(partie saturée d'une partie)}
				\par
				包含$2$元特别符号$\in$ 、显式公理\ref{ex1}、显式公理\ref{ex2}、显式公理\ref{ex3}和公理模式\ref{Sch8}的等式理论$M$中,令$R$为关于$x$、$y$在$E$上的等价关系,$f$为$E$到$E/R$的规范映射,$A\subset E$,则称$f^{-1}\langle f\langle A \rangle \rangle$为$A$对于$R$的浸润子集.
			\end{metadef}
			
			\begin{Ccor}\label{Ccor47}
				\hfill\par
				包含$2$元特别符号$\in$ 、显式公理\ref{ex1}、显式公理\ref{ex2}、显式公理\ref{ex3}和公理模式\ref{Sch8}的等式理论$M$中,令$R$为关于$x$、$y$在$E$上的等价关系,$(X_i)_{i\in I}$为$E$的子集族,对任意$i\in I$,令$A_i$为$X_i$对于$R$的浸润子集,则$\bigcup\limits_{i\in I}A_i$为$\bigcup\limits_{i\in I}X_i$对于$R$的浸润子集.
			\end{Ccor}
			证明:根据定理\ref{theo26}、补充证明规则\ref{Ccor42}可证.

			\begin{metadef}
				\textbf{同等价关系相容的映射(application compatible avec une relation d'équivalence)}
				\par
				包含$2$元特别符号$\in$ 、显式公理\ref{ex1}、显式公理\ref{ex2}、显式公理\ref{ex3}和公理模式\ref{Sch8}的等式理论$M$中,令$R$为关于$x$、$y$在$E$上的等价关系,$f$是定义域为$E$的函数,$z$为字母,$R$不包含$z$,如果$z=f(x)$关于$x$同等价关系$R$相容,则称$f$为同等价关系$R$相容的映射.
			\end{metadef}
			注:同等价关系相容的映射,意味着同一个等价类的元素的函数值相等.
			
			\begin{Ccor}\label{Ccor48}
				\hfill\par
				包含$2$元特别符号$\in$ 、显式公理\ref{ex1}、显式公理\ref{ex2}、显式公理\ref{ex3}和公理模式\ref{Sch8}的等式理论$M$中,令$R$为关于$x$、$y$在$E$上的等价关系,$g$为$E$到$E/R$的规范映射,$f$是定义域为$E$的函数,$z$为字母,$R$不包含$z$,则当且仅当$z=f(x)$关于$x$同等价关系$R$相容时,$x\in E\text{与}y\in E\text{与}g(x)=g(y)\Rightarrow f(x)=f(y)$.
			\end{Ccor}
			证明:
			\par
			充分性:$z=f(x)\text{与}R\Rightarrow z=f(y)$,因此$R\Rightarrow f(x)=f(y)$,根据证明规则\ref{C55}得证.
			\par
			必要性:如果$x\in E\text{与}y\in E\text{与}g(x)=g(y)\Rightarrow f(x)=f(y)$,当$x\in E\text{与}y\in E\text{与}x\equiv y(mod R)$时,$g(x)=g(y)$,故$f(x)=f(y)$,因此$z=f(x)$关于$x$同等价关系$R$相容.

			\begin{C}\label{C57}
				\hfill\par
				包含$2$元特别符号$\in$ 、显式公理\ref{ex1}、显式公理\ref{ex2}、显式公理\ref{ex3}和公理模式\ref{Sch8}的等式理论$M$中,令$R$为关于$x$、$y$在$E$上的等价关系,$g$为$E$到$E/R$的规范映射,对任意$E$到$F$的映射$f$,当且仅当存在$E/R$到$F$的映射$h$使$f=h\circ g$时,$f$为同等价关系$R$相容的映射.并且,对任意$f$,$h$是唯一确定的且$h=f\circ s$,其中$s$是$g$的右逆.
			\end{C}
			证明:根据定理\ref{theo22}(1)可证.

			\begin{metadef}
				\textbf{通过商导出的映射(application déduite par passage au quotient)}
				\par
				包含$2$元特别符号$\in$ 、显式公理\ref{ex1}、显式公理\ref{ex2}、显式公理\ref{ex3}和公理模式\ref{Sch8}的等式理论$M$中,令$R$为关于$x$、$y$在$E$上的等价关系,$g$为$E$到$E/R$的规范映射,对$E$到$F$的映射$f$,如果$f$为同等价关系$R$相容的映射,$s$是$g$的右逆,则称$f\circ s$为$f$对于$R$通过商导出的映射.
			\end{metadef}

			\begin{cor}\label{cor149}
				\hfill\par
				令$f$为$E$到$F$的映射,$R$为同$f$相关的等价关系,则$f$为同等价关系$R$相容的映射.
			\end{cor}
			证明:根据定义可证.

			\begin{cor}\label{cor150}
				\hfill\par
				令$f$为$E$到$F$的映射,$R$为同$f$相关的等价关系,则$f$对于$R$通过商导出的映射为$E/R$到$F$\\的单射.
			\end{cor}
			证明:设$f$对于$R$通过商导出的映射为$h$,$t\in E/R$,$t'\in E/R$,若$h(t)=h'(t)$,则对任意$x\in t$、$x\in t'$,$f(x)=h(t)$,$f(x')=h(t')$,故$f(x)=f(x')$,根据$R$的定义,$x=x'$,得证.

			\begin{cor}\label{cor151}
				\hfill\par
				令$f$为$E$到$F$的映射,$R$为同$f$相关的等价关系,$f$对于$R$通过商导出的映射$h$的值域为\\$f\langle E \rangle $,并且,令$k$为$h$通过$F$的子集$f\langle E \rangle $导出的函数,则$k$为$E/R$到$f\langle E \rangle $的双射.
			\end{cor}
			证明:$E$到$E/R$的规范映射为$g$,其右逆为$s$.由于$h=f\circ s$,故$(h\text{的值域})\subset f\langle E \rangle $.而对任意$a\in f\langle E \rangle $.存在$x\in E$,使$a=f(x)$,则$h(g(x))=a$,故$h$的值域为$f\langle E \rangle $,并且,$k$为满射,同时,根据补充定理\ref{cor150},$h$为单射,故$k$为单射,因此$k$为双射.
			
			\begin{cor}\label{cor152}
				\hfill\par
				令$f$为$E$到$F$的映射,$R$为同$f$相关的等价关系,$h$为$f$对于$R$通过商导出的映射,$k$为$h$通过$F$的子集$f\langle E \rangle $导出的函数,$g$为$E$到$E/R$的规范映射,$j$为$f\langle E \rangle $到$F$的规范映射,则$f=j\circ k\circ g$.
			\end{cor}
			证明:根据证明规则\ref{C57}可证.

			\begin{de}
				\textbf{规范分解(décomposition canonique)}
				\par
				令$f$为$E$到$F$的映射,$R$为同$f$相关的等价关系,$h$为$f$对于$R$通过商导出的映射,$k$为$h$通过$F$的子集$f\langle E \rangle $导出的函数,$g$为$E$到$E/R$的规范映射,$j$为$f\langle E \rangle $到$F$的规范映射,则称$j\circ k\circ g$为$f$的规范分解.
			\end{de}

			\begin{metadef}
				\textbf{同两个等价关系相容的映射(application compatible avec deux relations d'équivalence)}
				包含$2$元特别符号$\in$ 、显式公理\ref{ex1}、显式公理\ref{ex2}、显式公理\ref{ex3}和公理模式\ref{Sch8}的等式理论$M$中,令$f$为$E$到$F$的映射,$R$为在$E$上的等价关系,$S$为在$F$上的等价关系,$v$为$S$到$F/S$的规范映射.如果$v\circ f$为同等价关系$R$相容的映射,则称$f$为同等价关系$R$和$S$相容的映射.
			\end{metadef}
			注:同两个等价关系相容的映射,意味着第一个等价关系的同一个等价类的元素的函数值,属于第二个等价关系的同一个等价类.

			\begin{Ccor}\label{Ccor49}
				\hfill\par
				包含$2$元特别符号$\in$ 、显式公理\ref{ex1}、显式公理\ref{ex2}、显式公理\ref{ex3}和公理模式\ref{Sch8}的等式理论$M$中,令$f$为$E$到$F$的映射,$R$为在$E$上的等价关系,$S$为在$F$上的等价关系,$v$为$S$到$F/S$的规范映射,$u$为$E$到$E/R$的规范映射,当且仅当$f$为同等价关系$R$和$S$相容的映射时,存在从$E/R$到\\$F/S$的映射$h$,使$v\circ f=h\circ u$,且$h$是唯一的,同时,令$s$为$u$的右逆,$h=v\circ f\circ s$.
			\end{Ccor}
			证明:
			\par
			必要性:$v\circ f$为同等价关系$R$相容的映射,则$R\Rightarrow v\circ f (x)= v\circ f (y)$,又因为$R\Leftrightarrow x\in E\text{与}y\in E\text{与}u(x)=u(y)$,根据定理\ref{theo22}(1)可证存在性和唯一性,以及,$h=v\circ f\circ s$.
			\par
			充分性:如果存在从$E/R$到$F/S$的映射$h$,使$v\circ f=h\circ u$,则当$x\in E\text{与}y\in E\text{与}x\equiv y(mod R)$,均有$h\circ u(x)=h\circ u(y)$,即$v\circ f(x)=v\circ f(y)$,故$f$为同等价关系$R$和$S$相容的映射.

			\begin{metadef}
				\textbf{通过两个商导出的映射(application déduite par passage au deux quotients)}
				\par
				包含$2$元特别符号$\in$ 、显式公理\ref{ex1}、显式公理\ref{ex2}、显式公理\ref{ex3}和公理模式\ref{Sch8}的等式理论$M$中,令$f$为$E$到$F$的映射,$R$为在$E$上的等价关系,$S$为在$F$上的等价关系,$v$为$S$到$F/S$的规范映射,$u$为$E$到$E/R$的规范映射,如果$E/R$到$F/S$的映射$h$满足$v\circ f=h\circ u$,则称$h$为$f$对于$R$和$S$\\通过商导出的映射.
			\end{metadef}

			\begin{metadef}
				\textbf{等价关系的原像(Image réciproque d'une relation d'équivalence)}
				\par
				包含$2$元特别符号$\in$ 、显式公理\ref{ex1}、显式公理\ref{ex2}、显式公理\ref{ex3}和公理模式\ref{Sch8}的等式理论$M$中,令$f$为$E$到$F$的映射,$S$为在$F$上的等价关系,$v$为$F$到$F/S$的规范映射,则同$v\circ f$相关的等价关系称为$S$在$f$下的原像.
			\end{metadef}

			\begin{Ccor}\label{Ccor50}
				\hfill\par
				包含$2$元特别符号$\in$ 、显式公理\ref{ex1}、显式公理\ref{ex2}、显式公理\ref{ex3}和公理模式\ref{Sch8}的等式理论$M$中,令$f$为$E$到$F$的映射,$S$为在$F$上的等价关系,$v$为$F$到$F/S$的规范映射,令$R$为$S$在$f$下的原像,则$R\Leftrightarrow x\in E\text{与}y\in E\text{与}(f(y)|y)(f(x)|x)S$.
			\end{Ccor}
			证明:$R$即为$(x\in E\text{与}y\in E\text{与}v(f(x))=v(f(y)))$.设$S$的图为$G$,则$R\Leftrightarrow (x\in E\text{与}y\in E\text{与}G\langle f(x)\rangle=G\langle f(y)\rangle)$,等价于$(x\in E\text{与}y\in E\text{与}(f(x), f(y)\in G))$,得证.
			
			\begin{Ccor}\label{Ccor51}
				\hfill\par
				包含$2$元特别符号$\in$ 、显式公理\ref{ex1}、显式公理\ref{ex2}、显式公理\ref{ex3}和公理模式\ref{Sch8}的等式理论$M$中,令$f$为$E$到$F$的映射,$S$为在$F$上的等价关系,$v$为$F$到$F/S$的规范映射,令$R$为$S$在$f$下的原像,则任何一个关于$R$的等价类,都是某个关于$S$的等价类在$f$下的原像,且该等价类与\\$f\langle E \rangle $的交集不为空;反过来,任何关于$S$的等价类,如果与$f\langle E \rangle $的交集不为空,则其在$f$下的原像是$R$的等价类.
			\end{Ccor}
			证明:设$R$生成的的图为$G$,$S$生成的图为$F$,则对于任何$x\in E$,$R$的等价类为$G(x)$,对于$S$的等价类$F(f(x))$,满足$f(x)\in F(f(x))$,故其与$f\langle E \rangle $的交集不为空,同时,当$y\in E$时,$y\in G(x)\Leftrightarrow v(f(x))=v(f(y))$,即$F(f(x))= F(f(y))$,等价于$f(y)\in F(f(x))$,等价于$y\in f^{-1}(F(f(x)))$.故$y$为关于$S$的等价类$F(f(x))$的原像.
			\par
			反过来,任何关于$S$的等价类$X$,如果与$f\langle E \rangle$的交集不为空,设$a\in X\cap f\langle E \rangle $,则存在$x$,使$f(x)=a$,则$X=F(f(x))$,如上所述,其在$f$下的原像是$R$的等价类$G(x)$.

			\begin{metadef}
				\textbf{导出的等价关系(relation d'équivalence induite)}
				包含$2$元特别符号$\in$ 、显式公理\ref{ex1}、显式公理\ref{ex2}、显式公理\ref{ex3}和公理模式\ref{Sch8}的等式理论$M$中,令$R$为在$E$上的等价关系,$A\subset E$,$j$为$A$到$E$的规范映射,则$R$在$j$下的原像,称为在$A$上由$R$导出的等价关系,记作$R_A$.
			\end{metadef}

			\begin{Ccor}\label{Ccor52}
				\hfill\par
				包含$2$元特别符号$\in$ 、显式公理\ref{ex1}、显式公理\ref{ex2}、显式公理\ref{ex3}和公理模式\ref{Sch8}的等式理论$M$中,令$R$为在$E$上的等价关系,$A\subset E$,则$R_A\Leftrightarrow x\in A\text{与}y\in A\text{与}R$.
			\end{Ccor}
			证明:根据补充证明规则\ref{Ccor50}可证.
			
			\begin{Ccor}\label{Ccor53}
				\hfill\par
				包含$2$元特别符号$\in$ 、显式公理\ref{ex1}、显式公理\ref{ex2}、显式公理\ref{ex3}和公理模式\ref{Sch8}的等式理论$M$中,令$R$为在$E$上的等价关系,$A\subset E$,则对任意$x\in A$,$x$关于$R_A$的等价类,是$x$关于$R$的等价类在$A$上的迹.
			\end{Ccor}
			证明:设$R_A$的图为$G$,$R$的图为$F$,$G(x)=\{y|(x, y)\in G\}$,$F(x)=\{y|(x, y)\in F)$.由于$R_A\Leftrightarrow x\in A\text{与}y\in A\text{与}R$,因此$G(x)= \{y|x\in A\text{与}y\in A\text{与}(x, y)\in F\}$,由于$x\in A$,故$G(x)=F(x)\bigcap\limits_A$,得证.
			
			
			\begin{Ccor}\label{Ccor54}
				\hfill\par
				包含$2$元特别符号$\in$ 、显式公理\ref{ex1}、显式公理\ref{ex2}、显式公理\ref{ex3}和公理模式\ref{Sch8}的等式理论$M$中,令$R$为在$E$上的等价关系,$A\subset E$,$j$为$A$到$E$的规范映射,则$j$为同等价关系$R$和$R_A$相容的映射.
			\end{Ccor}
			证明:设$E$到$E/R$的规范映射为$v$,则$R_A$为$x\in A\text{与}y\in A\text{与}v(j(x))=v(j(y))$,因此$z= v(j(x))\text{与}R\Rightarrow z=v(j(y))$,故$v\circ j$为同等价关系$R_A$相容的映射,得证.
			
			\begin{Ccor}\label{Ccor55}
				\hfill\par
				包含$2$元特别符号$\in$ 、显式公理\ref{ex1}、显式公理\ref{ex2}、显式公理\ref{ex3}和公理模式\ref{Sch8}的等式理论$M$中,令$R$为在$E$上的等价关系,$A\subset E$,$j$为$A$到$E$的规范映射,$E$到$E/R$的规范映射为$v$,$h$为$j$对于$R_A$和$R$通过商导出的映射,则:
				\par
				(1)$h$为$A/R_A$到$E/R$的单射.
				\par
				(2)$h\langle A/R_A\rangle =v\langle A \rangle $.
			\end{Ccor}
			证明:设$A$到$A/R_A$的规范映射为$u$,
			\par
			(1)当$x\in A$、$y\in A$时,$h(u(x))=h(u(y))\Leftrightarrow v(j(x))=v(j(y))$,即等价于$v(x)=v(y)$,故$(x, y)\in R_A$生成的图,因此$u(x)=u(y)$,得证.
			\par
			(2)当$x\in A$时,$v(x)=v(j(x))$,等于$h(u(x))$,因此$v\langle A \rangle \subset h\langle A/R_A\rangle$.反过来,对于$y\in A/R_A$,由于$u$为满射,则存在$x\in A$使$y=u(x)$,则$h(y)=v(x)$,因此,$hh\langle A/R_A\rangle\subset v\langle A \rangle $.得证.
			
			\begin{metadef}
				\textbf{商之间的规范映射(application canonique entre deux quotients)}
				\par
				包含$2$元特别符号$\in$ 、显式公理\ref{ex1}、显式公理\ref{ex2}、显式公理\ref{ex3}和公理模式\ref{Sch8}的等式理论$M$中,令$R$为在$E$上的等价关系,$A\subset E$,$j$为$A$到$E$的规范映射,$E$到$E/R$的规范映射为$v$,$h$为$j$对于$R_A$和$R$通过商导出的映射,则$h$通过$E/R$的子集$v\langle A \rangle $导出的映射,称为$A/R_A$到$v\langle A \rangle $的规范映射,其逆映射称为$v\langle A \rangle $到$A/R_A$的规范映射.
			\end{metadef}
									
			\begin{metadef}
				\textbf{更细的等价关系(relations d'équivalence plus fin)}
				\par
				包含$2$元特别符号$\in$ 、显式公理\ref{ex1}、显式公理\ref{ex2}、显式公理\ref{ex3}和公理模式\ref{Sch8}的等式理论$M$中,如果$R$、$S$均为关于$x$、$y$的等价关系,且$S\Rightarrow R$,则称$S$为比$R$更细的等价关系.
			\end{metadef}
			注:在原书中,“更细”这个概念包括与自身相等的情况,即一个等价关系比自身更细.
						
			\begin{Ccor}\label{Ccor56}
				\hfill\par
				包含$2$元特别符号$\in$ 、显式公理\ref{ex1}、显式公理\ref{ex2}、显式公理\ref{ex3}和公理模式\ref{Sch8}的等式理论$M$中,如果$R$、$S$均为关于$x$、$y$的等价关系,且$S$为比$R$更细的等价关系,则任何一个$S$的等价类,都是$R$的一个等价类的子集.
			\end{Ccor}
			证明:设$R$生成的图为$G$,$S$生成的图为$F$,则$F\subset G$.则$F(x)\subset G(x)$,得证.
			
			\begin{Ccor}\label{Ccor57}
				\hfill\par
				包含$2$元特别符号$\in$ 、显式公理\ref{ex1}、显式公理\ref{ex2}、显式公理\ref{ex3}和公理模式\ref{Sch8}的等式理论$M$中,如果$R$、$S$均为关于$x$、$y$的等价关系,且$S$为比$R$更细的等价关系,$f$为$E$到$E/R$的规范映射,则$f$为同等价关系S相容的映射.
			\end{Ccor}
			证明:根据定义可证.
			
			\begin{metadef}
				\textbf{等价关系的商(quotient de relations d'équivalence)}
				\par
				包含$2$元特别符号$\in$ 、显式公理\ref{ex1}、显式公理\ref{ex2}、显式公理\ref{ex3}和公理模式\ref{Sch8}的等式理论$M$中,令$R$、$S$均为关于$x$、$y$在E上的等价关系,且$S$为比$R$更细的等价关系,$f$、$g$分别为$E$到$E/R$\\和$E$到$E/S$的规范映射,$h$为$f$对于$S$通过商集导出的映射,则同$h$相关的等价关系称为$R$除以$S$的商,记作$R/S$.
			\end{metadef}
			
			\begin{Ccor}\label{Ccor58}
				\hfill\par
				包含$2$元特别符号$\in$ 、显式公理\ref{ex1}、显式公理\ref{ex2}、显式公理\ref{ex3}和公理模式\ref{Sch8}的等式理论$M$中,令$R$、$S$均为关于$x$、$y$在$E$上的等价关系,且$S$为比$R$更细的等价关系,$f$、$g$分别为$E$到$E/R$\\和$E$到$E/S$的规范映射,$h$为$f$对于$S$通过商集导出的映射,则:
				\par
				(1)$f=h\circ g$.
				\par
				(2)$x\equiv y(mod R)\Leftrightarrow g(x)\equiv g(y)(mod R/S)$.
				\par
				(3)关于$R/S$的等价类,是关于$R$的等价类在$g$下的像.
			\end{Ccor}
			证明:
			\par
			(1)令$j=Id_E$,则$f\circ j=f$,因此$f=h\circ g$.
			\par
			(2)根据补充证明规则\ref{Ccor58}(1)可证.
			\par
			(3)令$R$的图为$G$,$R/S$的图为$F$,对任意$x\in E$,考虑$R$的等价类$G(x)$和$R/S$的等价类$F(g(x))$.当$y\in E$时,$y\in G(x)\Leftrightarrow f(x)=f(y)$,等价于$h(g(y))=h(g(x))$,等价于$g(y)\in F(g(x))$.同时,当$z\in E/S$时,$z\in g\langle G(x)\rangle\Leftrightarrow (\exists y)(y\in G(x)\text{与}g(y)=z)$,等价于$(\exists y)(g(y)\in F(g(x))\text{与}g(y)=z)$,等价于$z\in F(g(x))\text{与}(\exists y)(g(y)=z)$,根据补充证明规则\ref{Ccor35},$g$为$E$到$E/S$的满射,故$(\exists y)(g(y)=z)$,因此$z\in g\langle G(x)\rangle\Leftrightarrow z\in F(g(x))$,得证.
			
			\begin{Ccor}\label{Ccor59}
				\hfill\par
				包含$2$元特别符号$\in$ 、显式公理\ref{ex1}、显式公理\ref{ex2}、显式公理\ref{ex3}和公理模式\ref{Sch8}的等式理论$M$中,令$S$为关于$x$、$y$在$E$上的等价关系,$T$为关于$x$、$y$在$E/S$上的等价关系,$g$为$E$到$E/S$的规范映射,$R$为$T$在$g$下的原像,则$S$为比$R$更细的等价关系,并且$T\Leftrightarrow R/S$.
			\end{Ccor}
			证明:$R\Leftrightarrow x\in E\text{与}y\in E\text{与}(g(y)|y)(g(x)|x)T$,同时,根据证明规则\ref{C55},$S\Rightarrow (x\in E\text{与}y\in E\text{与}g(x)=g(y))$,当$g(x)=g(y)$时,$(g(y)|y)(g(x)|x)T \Leftrightarrow (g(x)|y)(g(x)|x)T$,后者即$(g(x)|x)(x|y)T$,其为真,因此$S\Rightarrow R$.
			\par
			令$f$为$E$到$E/R$的规范映射,$h$为$f$对于$S$通过商集导出的映射,则$f=h\circ g$,当$x\in E$、$y\in E$时,$(g(y)|y)(g(x)|x)T\Leftrightarrow h(g(x))=h(g(y))$,由于$g$为满射,故对任意$x\in E/S$、$y\in E/S$,均存在$g(a)=x$、$g(b)=y$,故$T\Leftrightarrow h(x)=h(y)$.又因为$T\Rightarrow x\in E/S$、$T\Rightarrow y\in E/S$,因此$T\Leftrightarrow R/S$.
			
			\begin{Ccor}\label{Ccor60}
				\hfill\par
				包含$2$元特别符号$\in$ 、显式公理\ref{ex1}、显式公理\ref{ex2}、显式公理\ref{ex3}和公理模式\ref{Sch8}的等式理论$M$中,$R$为关于$x$、$y$的等价公式,$R'$为关于$x'$、$y'$的等价公式,令$S$为$(\exists x)(\exists y)(\exists x')(\exists y')(u=(x, x')\\\text{与}v=(y, y')\text{与}R\text{与}(y'|y)(x'|x)R)$,则$S$为关于$u$、$v$的等价公式.
			\end{Ccor}
			证明:根据定义可证.
			
			\begin{metadef}
				\textbf{等价关系的积(produit de relations d'équivalence)}
				\par
				包含$2$元特别符号$\in$ 、显式公理\ref{ex1}、显式公理\ref{ex2}、显式公理\ref{ex3}和公理模式\ref{Sch8}的等式理论$M$中,$R$为关于$x$、$y$的等价公式,$R'$为关于$x'$、$y'$的等价公式,则称关于$u$、$v$的等价公式$(\exists x)(\exists y)\\(\exists x')(\exists y')(u=(x, x')\text{与}v=(y, y')\text{与}R\text{与}R')$为$R$和$R'$的积,记作$R\times R'$.
			\end{metadef}
			
			\begin{Ccor}\label{Ccor61}
				\hfill\par
				包含$2$元特别符号$\in$ 、显式公理\ref{ex1}、显式公理\ref{ex2}、显式公理\ref{ex3}和公理模式\ref{Sch8}的等式理论$M$中,$R$为关于$x$、$y$在$E$上的等价公式,$R'$为关于$x'$、$y'$在$E'$上的等价公式,则$R\times R'$为在$E\times E'$上的等价公式.
			\end{Ccor}
			证明:令$S$为$(\exists x)(\exists y)(\exists x')(\exists y')(u=(x, x')\text{与}v=(y, y')\text{与}R\text{与}(y'|y)(x'|x)R)$,则$(u|v)S\\\Leftrightarrow (\exists x)(\exists x')(u=(x, x')\text{与}(x|y)R\text{与}(x'|y')R')$,等价于$(\exists x)(\exists x')(u=(x, x')\text{与}x\in E\text{与}x'\in R')$,等价于$u\in E\times E'$,得证.
			
			\begin{Ccor}\label{Ccor62}
				\hfill\par
				包含$2$元特别符号$\in$ 、显式公理\ref{ex1}、显式公理\ref{ex2}、显式公理\ref{ex3}和公理模式\ref{Sch8}的等式理论$M$中,$R$为关于$x$、$y$在$E$上的等价公式,$R'$为关于$x'$、$y'$在$E'$上的等价公式,$f$和$f'$分别为$E$到$E/R$\\和$E'$到$E'/R'$的规范映射,$g$为$E\times E'$到$(E\times E')/(R\times R')$的映射,则:
				\par
				(1)$g((x, x'))=f(x)\times f'(x')$.
				\par
				(2)关于$R\times R'$的等价类,是关于$R$的等价类和关于$R'$的等价类的积,反过来,关于$R$的等价类和关于$R'$的等价类的积,是关于$R\times R'$的等价类.
			\end{Ccor}
			证明:
			\par
			(1)令$x\in E$,$x'\in E'$,$u=(x, x')$,则$R\times R'\Leftrightarrow (\exists y)(\exists y')(v=(x, x')\text{与}R\text{与}R')$,令$R$和$R'$的图分别为$G$和$G'$,$R\times R'$的图为$F$,则$R\times R'\Leftrightarrow v\in G(x)\times G'(x')$,因此,$(u, v)\in F\Leftrightarrow v\in G(x)\times G'(x')$,即$F(u)= G(x)\times G'(x')$,得证.
			\par
			(2)根据补充证明规则\ref{Ccor62}(1)可证.
			
			\begin{Ccor}\label{Ccor63}
				\hfill\par
				包含$2$元特别符号$\in$ 、显式公理\ref{ex1}、显式公理\ref{ex2}、显式公理\ref{ex3}和公理模式\ref{Sch8}的等式理论$M$中,$R$为关于$x$、$y$在$E$上的等价公式,$R'$为关于$x'$、$y'$在$E'$上的等价公式,$f$和$f'$分别为$E$到$E/R$\\和$E'$到$E'/R'$的规范映射,则同$f\times f'$相关的等价关系等价于$R\times R'$.
			\end{Ccor}
			证明:$f\times f'((x, x'))=(f(x), f'(x'))$,因此,同$f\times f'$相关的等价关系即$x\in E\text{与}x'\in E'\text{与}y\in E\text{与}y'\in E'\text{与}u=(x, x')\text{与}v=(y, y')\text{与}f(x)=f(y)\text{与}f'(x')=f'(y')$,由于$R\Leftrightarrow x\in E\text{与}y\in E\text{与}f(x)=f(y)$,$R'\Leftrightarrow x'\in E'\text{与}y'\in E'\text{与}f'(x')=f'(y')$,得证.
			
			\begin{Ccor}\label{Ccor64}
				\hfill\par
				包含$2$元特别符号$\in$ 、显式公理\ref{ex1}、显式公理\ref{ex2}、显式公理\ref{ex3}和公理模式\ref{Sch8}的等式理论$M$中,$R$为关于$x$、$y$在$E$上的等价公式,$R'$为关于$x'$、$y'$在$E'$上的等价公式,$f$和$f'$分别为$E$到$E/R$\\和$E'$到$E'/R'$的规范映射,$g$为$E\times E'$到$(E\times E')/(R\times R')$的规范映射,则存在唯一的映射$h$,使$f\times f'=h\circ g$,且$h$为$(E\times E')/(R\times R')$到$(E/R)\times (E'/R')$的双射.
			\end{Ccor}
			证明:根据补充证明规则\ref{Ccor62}(1),$g((x, x'))=f(x)\times f'(x')$,$f\times f'((x, x'))=\\(f(x), f'(x'))$,根据定理\ref{theo22}(1),存在唯一的$h$,使$f\times f'=h\circ g$.由于$f\times f'$为满射,故$h$为满射.
			\par
			同时,设$h(a)=h(b)$,由于$g$为满射,故存在$a$、$b$使$a=g(x, x')$、$b=(y, y')$,因此$h(a)= (f(x), f'(x'))$,$h(b)=(f(y), f'(y'))$,故$f(x)=f(y)$, $f'(x')=f'(y')$,故$g((x, x'))=g((y, y'))$,因此$a=b$,所以$h$为单射.
			\par
			综上,$h$为双射.
			
			
			\begin{Ccor}\label{Ccor65}
				\hfill\par
				包含$2$元特别符号$\in$ 、显式公理\ref{ex1}、显式公理\ref{ex2}、显式公理\ref{ex3}和公理模式\ref{Sch8}的等式理论$M$中,令$R$为关于$x$、$y$的等价关系,则:
				\par
				(1)$(x'|y)R\Rightarrow \tau_y(R)=\tau_y((x'|x)R)$.
				\par
				(2)$(x|y)R\Leftrightarrow (\tau_y(R)|y)R$.
				\par
				(3)$(x|y)R\text{与}(x'|x)(x'|y)R\text{与}\tau_y(R)=\tau_y((x'|x)R)\Leftrightarrow (x'|y)R$.
			\end{Ccor}
			证明:
			\par
			(1)如果$(x'|y)R$,则$R\Leftrightarrow (x'|x)R$,根据公理模式\ref{Sch7}可证.
			\par
			(2)$(\tau_y(R)|y)R$即$(\exists y)R$.根据补充证明规则\ref{Ccor27}可证.
			\par
			(3)如果$(x'|y)R$,根据补充证明规则\ref{Ccor23},$(x|y)R\text{与}(x'|x)(x'|y)R$,根据补充证明规则\ref{Ccor65}(1),$(x'|y)R\Rightarrow (x|y)R\text{与}(x'|x)(x'|y)R\text{与}\tau_y(R)=\tau_y((x'|x)R)$.
			\par
			反过来,如果$(x|y)R\text{与}(x'|x)(x'|y)R\text{与}\tau_y(R)=\tau_y((x'|x)R)$,根据公理模式\ref{Sch6},\\$(\tau_y((x'|x)R)|y)(x'|x)R\Leftrightarrow (\tau_y(R)|y)(x'|x)R$,根据补充证明规则\ref{Ccor65}(2),$(x'|x)(x'|y)R \Leftrightarrow\\ (\tau_y((x'|x)R)|y)(x'|x)$,故$(\tau_y(R)|y)(x'|x)R\Leftrightarrow (x'|x)(x'|y)R$,因此$(\tau_y(R)|y)(x'|x)R$,根据补充证明规则\ref{Ccor65}(2),$(x|y)R\Leftrightarrow (\tau_y(R)|y)R$,故$(\tau_y(R)|y)R$,因此$(x'|x)R$.得证.

			\begin{metadef}
				\textbf{等价的对象类(classe d'objets équivalents)}
				\par
				包含$2$元特别符号$\in$ 、显式公理\ref{ex1}、显式公理\ref{ex2}、显式公理\ref{ex3}和公理模式\ref{Sch8}的等式理论$M$中,令$R$为关于$x$、$y$的等价关系,则$\tau_y(R)$称为关于$R$等价于$x$的对象类.
			\end{metadef}
			
			\begin{Ccor}\label{Ccor66}
				\hfill\par
				包含$2$元特别符号$\in$ 、显式公理\ref{ex1}、显式公理\ref{ex2}、显式公理\ref{ex3}和公理模式\ref{Sch8}的等式理论$M$中,令$R$为关于$x$、$y$的等价关系,$T$为项且不包含$x$,如果$(\forall y)((y|x)R\Rightarrow (\exists x)(x\in T\text{与}R))$,则$(\exists x)((x|y)R\text{与}z=\tau_y(R))$为$z$上的集合化公式.
			\end{Ccor}
			证明:根据证明规则\ref{C53},$x\in T\text{与}z=\tau_y(R)$为$z$上的集合化公式,令$X=\{z|x\in T\text{与}z=\tau_y(R)\}$,如果$(x|y)R$,则存在$x\in T$,使$R$为真,则$\tau_y(R)=(y|x)\tau_y(R)$,由于$\tau_y(R)\in X$,故$(y|x)\tau_y(R)\\\in X$,因此$(y|x)R\Rightarrow (y|x)\tau_y(R)\in X$,因此$(x|y)R\Rightarrow \tau_y(R)\in X$,故$(\exists x)((x|y)R\text{与}z=\tau_y(R))\Rightarrow z\in X$,根据证明规则\ref{C52}得证.						
			
			\begin{metadef}
				\textbf{等价的对象类的集合(ensemble de classes d'objets équivalents))}
				\par
				包含$2$元特别符号$\in$ 、显式公理\ref{ex1}、显式公理\ref{ex2}、显式公理\ref{ex3}和公理模式\ref{Sch8}的等式理论$M$中,令$R$为关于$x$、$y$的等价关系,如果存在不包含$x$的项$T$,使$(\forall y)((y|x)R\Rightarrow (\exists x)(x\in T\text{与}R))$,则称$\{z|(\exists x)((x|y)R\text{与}z=\tau_y(R))\}$为等价的对象类的集合.
			\end{metadef}
						
			\begin{Ccor}\label{Ccor67}
				\hfill\par
				包含$2$元特别符号$\in$ 、显式公理\ref{ex1}、显式公理\ref{ex2}、显式公理\ref{ex3}和公理模式\ref{Sch8}的等式理论$M$中,令$R$为关于$x$、$y$的等价关系,存在不包含$x$的项$T$,使$(\forall y)((y|x)R\Rightarrow (\exists x)(x\in T\text{与}R))$,$X$为等价的对象类的集合,$(x|y)$R为真,则存在唯一的$z$,满足$z\in X且(z|y)R$为真.
			\end{Ccor}
			证明:$\tau_y(R)$满足上述条件,存在性得证.
			\par
			另一方面,设$z=\tau_y((u|x)R)$且$(u|y)R$,因此$(u|x)(z|y)R$为真,因此$(u|y)R$为真,根据补充证明规则\ref{Ccor65}(1),$\tau_y((u|x)R)=\tau_y(R)$,故$z=\tau_y(R)$,唯一性得证.
			
			\begin{exer}\label{exer67}
				\hfill\par
				求证:当且仅当满足下列三个条件时,$G$为在$E$上的等价关系生产的图:
				\par
				第一,$pr_1G=E$,$pr_2G=E$,
				\par
				第二,$G\circ G^{-1}\circ G=G$,
				\par
				第三,$\Delta_E\subset G$.
			\end{exer}
			证明:
			\par
			如果$G$为在$E$上的等价关系生产的图:根据补充证明规则\ref{Ccor25},$(x, y)\in G\Rightarrow x\in E$,$(x, y)\in G\Rightarrow y\in E$,同时,$x\in E\Leftrightarrow (x, x)\in G$,故$(\exists y)((x, y)\in G)\Leftrightarrow x\in E$、$(\exists x)((x, y)\in G)\Leftrightarrow y\in E$,即$pr_1G=E$,$pr_2G=E$.
			\par
			$(x, t)\in G\Rightarrow (x, x)\in G\text{与}(x, x)\in G\text{与}(x, t)\in G$,故$(\exists y)(\exists z)((x, y)\in G\text{与}(z, y)\in G\text{与}(z, t)\in G)$,另一方面,如果$(\exists y)(\exists z)((x, y)\in G\text{与}(z, y)\in G\text{与}(z, t)\in G)$,则$(x, t)\in G$,$故G\circ G^{-1}\circ G=G$.
			\par
			$x\in E\Leftrightarrow (x, x)\in G$,根据补充定理\ref{cor59},$(x, y)\in \Delta_E\Leftrightarrow x\in E\text{与}y=x$,由于$x\in E\Leftrightarrow (x, x)\in G$,$(x, y)\in \Delta_E\Rightarrow (x, y)\in G$,因此$\Delta_E\subset G$.
			\par
			反过来,如果三个条件成立:由于$pr_1G=E$,$pr_2G=E$,$\Delta_E\subset G$,反身性成立.
			\par
			如果$(x, y)\in G$,$(y, z) \in G$,由于$(y, y) \in G^{-1}$,故$(x, z)\in G\circ G^{-1}\circ G$,因此$(x, z)\in G$,传递性成立.
			\par
			若$(x, y)\in G$,由于$(y, y)\in G$,因此$(y, x)\in G^{-1}$,$(x, x)\in G$,因此$(y, x)\in G$,对称性成立.
			
			\begin{exer}\label{exer68}
				\hfill\par
				$G$为图,且$G\circ G^{-1}\circ G=G$,求证:$G^{-1}\circ G$和$G\circ G^{-1}$分别为在$pr_1G$上和在$pr_2G$上的等价图.
			\end{exer}
			证明:当$x\in pr_1G$时,$(\exists y)((x, y)\in G)$,则$(y, x)\in G^{-1}$,故$(x, x)\in G^{-1}\circ G$;反过来,若$(x, x)\in G^{-1}\circ G$,$(\exists y)((x, y)\in G)$,故$x\in pr_1G$,反身性成立.
			\par
			若$(x, y)\in G^{-1}\circ G$,$(y, z)\in G^{-1}\circ G$,则$(\exists z)((x, z)\in G)$,$(y, z)\in G$,因此$(y, x)\in G$,对称性成立.
			\par
			若$(x, y)\in G^{-1}\circ G$,则$y\in pr_1G$,因此$(y, y)\in G$,故$(x, y)\in G\circ G^{-1}\circ G$,因此$(x, y)\in G$;由于$(y, z)\in G^{-1}\circ G$,则$(z, y) \in G^{-1}\circ G$,同理$(z, y)\in G$,因此$(x, z) \in G^{-1}\circ G$,传递性成立.
			\par
			故$G^{-1}\circ G$为在$pr_1G$上的等价图.同理可证$G\circ G^{-1}$为在$pr_2G$上的等价图.

			\begin{exer}\label{exer69}
				\hfill\par
				$A\subset E$,$R$是同恒等映射$X\mapsto X\cap A(X\in \mathcal{P}(E), X\bigcap\limits_A\in \mathcal{P}(E))$相关的等价关系.求证:存在$\mathcal{P}(A)$到$\mathcal{P}(E)/R$的双射.
			\end{exer}
			证明:令映射$X\mapsto X\cap A(X\in \mathcal{P}(E), X\bigcap\limits_A\in \mathcal{P}(E))$为$f$,对任意$Y\in \mathcal{P}(A)$,$Y\cap A=Y$,故$f(Y)=Y$,因此$f\langle\mathcal{P}(E)\rangle=\mathcal{P}(A)$,对$f$做规范分解$f=g\circ k\circ j$,其中,$k$为$E/R$到$\mathcal{P}(A)$的映射,根据补充定理\ref{cor150}得证.
			
			\begin{exer}\label{exer70}
				\hfill\par
				$G$为在$E$上的等价图,如果$A\subset G$且$pr_1A=E$(或$pr_2A=E$),$B$为图,则:$G\circ A=G$(或$A\circ G=G$),$(G\cap B)\circ A=G\cap(B\circ A)$(或$A\circ (G\cap B)=G\cap(A\circ B)$).
			\end{exer}
			证明:
			\par
			若$pr_1A=E$,如果$(x, z)\in G$,则$x\in E$,故存在$y$使$(x, y)\in A$,同时$(y, z)\in G$,故$(x, z)\in G\circ A$;反过来,如果$(x, z)\in G\circ A$,则存在$y$使$(x, y)\in A\text{与}(y, z)\in G$,故$(x, z) \in G$,因此$G\circ A=G$.
			\par
			如果$(x, z)\in G\cap(B\circ A)$,则$(x, z)\in G$,且存在$y$,使$(x, y)\in A$,$(y, z)\in B$,因此$(y, z)\in G$,故$(y, z)\in G\cap B$,故$(x, z)\in (G\cap B)\circ A$.反过来,如果$(x, z)\in (G\cap B)\circ A$,则存在$y$,使$(x, y)\in A$,$(y, z)\in G\cap B$,故$(y, z)\in B$,$(y, z)\in G$,因此$(x, z)\in B\circ A$,$(x, z)\in G$,因此$(G\cap B)\circ A=G\cap(B\circ A)$.
			\par
			同理可证$pr_2A=E$的情形.
			
			\begin{exer}\label{exer71}
				\hfill\par
				求证:在$E$上的多个等价图的交集,是在$E$上的等价图.并给出在$E$上的两个等价集,其并集不是在$E$上的等价集.
			\end{exer}
			证明:证明部分即补充定理\ref{cor145}.
			令$a$、$b$、$c$互不相等,$G=\\\{(a, a), (b, b), (c, c)\}\cup\{(a, b), (b, a)\}$,H=$\{(a, a), (b, b), (c, c)\}\cup\{(a, c), (c, a)\}$,则$G$、$H$均为在$\{a, b, c\}$上的等价图,但$G\cup H=\{(a, a), (b, b), (c, c)\}\cup\{(a, b), (b, a)\}\cup\{(a, c), (c, a)\}$,$(b, a)\in G\cup H$、$(a, c)\in G\cup H$,但$(b, c)\notin G\cup H$,故$G\cup H$不是等价图.
			
			\begin{exer}\label{exer72}
				\hfill\par
				$G$、$H$均为在$E$上的等价图,求证:当且仅当$G\circ H=H\circ G$时,$G\circ H$为在$E$上的等价图,并且,这种情况下,令$I={X\in \{X|G\subset X\text{与}H\subset X\text{与}(H\text{为在}E\text{上的等价图})\}}$,则$G\circ H=\bigcap\limits_{X\in I}X$.
			\end{exer}
			证明:
			\par
			对任意$x\in E$,$(x, x)\in G$、$(x, x)\in H$、故$(x, x)\in G\circ H$,反过来,如果$(x, x)\in G\circ H$,则存在$y$使$(x, y)\in H$,故$x\in E$,因此$G\circ H$具有反身性.
			\par
			如果$(x, y)\in G\circ H$,则存在$z$使$(x, z)\in H$,$(z, y)\in G$,因此$(y, x)\in H\circ G$,反之亦然.故当且仅当$G\circ H=H\circ G$时,$G\circ H$具有对称性.
			\par
			如果$G\circ H=H\circ G$,则当$(x, y)\in G\circ H$、$(y, z)\in G\circ H$时,$(x, z)\in G\circ H\circ G\circ H$,因此$(x, z)\in G\circ G\circ H\circ H$.根据定理\ref{theo56},$G\circ G=G$、$H\circ H=H$,因此$(x, z)\in G\circ H$,即$G\circ H$具有传递性.
			\par
			综上,当且仅当$G\circ H=H\circ G$时,$G\circ H$为在$E$上的等价图.
			\par
			设$(x, y)\in G\circ H$,则存在$z$,使$(x, z)\in G$、$(z, y)\in H$,因此对任意$X\in I$,$(x, y)\in X$,故$X\in \bigcap\limits_{X\in I}X$,反过来,若$(x, y)\in \bigcap\limits_{X\in I}X$,则对任意$X\in I$,$(x, y)\in X$,因此$(x, y)\in G$、$(x, y)\in H$,故$(x, y)\in G\circ H$,得证.
			
			\begin{exer}\label{exer73}
				\hfill\par
				令$R$为在$F$上的等价关系,$f$为$E$到$F$的映射,$S$为$R$在$f$下的原像.$A=f\langle E \rangle $,试给出\\$E/S$到$R/R_A$的双射.
			\end{exer}
			答:设$j$为$A$到$F$的规范映射,则$j$为单射;$h$为$f$对于$R$与$S$通过商集导出的映射,根据补充定理\ref{cor150},$h$为$E/S$到$F/R$的单射;$l$为$j$对于$R_A$与$R$通过商集导出的映射,根据补充证明规则\ref{Ccor55}(1),$l$为$A/R_A$到$F/R$的单射.
			\par
			令$p$为$F$到$F/R$的规范映射,根据补充证明规则\ref{Ccor55}(2),$l\langle A/R_A\rangle=p\langle A \rangle $,令$l'$为$l$通过$F/R$的子集$p\langle A \rangle $导出的函数,则$l'$为双射.同时,设$E$到$E/S$的规范映射为$k$,则$p\circ f=h\circ k$.对任意$x\in p\langle A \rangle $,则存在$y\in E$,使$p(f(y))=x$,故$x=h(k(y))$,反过来,对任意$z\in E/S$,设$z=G(u)$,则$h(z)=p(f(u))$,故$h(z)\in p\langle A \rangle $.因此,令$h'$为$h$通过$F/R$的子集$p\langle A \rangle $导出的函数,则$h'$为双射.
			\par
			因此${l'}^{-1}\circ h'$为$E/S$到$R/R_A$的双射.
			
			\begin{exer}\label{exer74}
				\hfill\par
				$R$为在$F$上的等价关系,$p$为$F$到$F/R$的规范映射,$f$为$G$到$F/R$的满射,求证:存在$E$,以及$E$到$F$的满射$g$、$E$到$G$的满射$h$,使$p\circ g=f\circ h$.
			\end{exer}
			证明:$(\exists X)(z\in p^{-1}\langle X \rangle \times f^{-1}\langle X \rangle \text{与}X\in F/R)\Rightarrow z\in F\times G$,故$(\exists X)(z\in p^{-1}\langle X \rangle \times f^{-1}\langle X \rangle \text{与}X\in F/R)$为集合化公式,令$E=\{z|(\exists X)(z\in p^{-1}\langle X \rangle \times f^{-1}\langle X \rangle \text{与}X\in F/R)\}$,映射$g$为$z\mapsto pr_1z(z\in E, pr_1z\in F)$,映射$h$为$z\mapsto pr_2z(z\in E, pr_2z\in G)$.对任意$z\in E$,设$z\in p^{-1}\langle X \rangle \times f^{-1}\langle X \rangle \text{与}X\in F/R$,故$p(g(z))=X$,$f(h(z))=X$,同时,由于$p$、$f$均为满射,故对任意$x\in F$,令$G(x)=X$,故$p^{-1}\langle X \rangle \times f^{-1}\langle X \rangle \neq \varnothing$,因此存在$z\in E$,使$g(z)=x$,故$g$为满射,同理可证$h$为满射.			
			
			\begin{exer}\label{exer75}
				\hfill\par
				(1)令$R$为公式,求证:公式$R\text{与}(x|z)(y|x)(z|y)R$关于$x$、$y$具有对称性,在什么情况下,该公式为关于$x$、$y$在$E$上具有反身性?
				\par
				(2)令$R$为公式,$R$关于$x$、$y$具有对称性并且在$E$上具有反身性,令其生成的图为$G$,且$G\subset E\times E$.$S$为公式:
				\par
				“$\text{存在自然数}n>0\text{以及}(x_i)_{i\in [0, n]}\text{(}(\forall i)(i\in [0, n]\Rightarrow x_i\in G)\text{),其中}x_0=x\text{,}x_n=y\text{,并且,对任意}i\in [0, n-1]\text{,}(x_{i+1}|y)(x_i|x)R\text{为真}$”.
				\par				
				求证:$S$是关于$x$、$y$在$E$上的等价关系,并且,包含$G$的一切等价图,均包含$S$的图.
				\par
				(3)当$x\in E$时,(2)中的$S$的等价类,称为$x$所在的$E$关于$R$的连通分量,在没有歧义的情况下可以简称$E$关于$R$的连通分量.令$F=\{A|A\subset E\text{与}(\forall y)(\forall z)(y\in A\text{与}z\in E-A\Rightarrow \text{非}(y|x)(z|y)R)$,求证:对任意$x\in E$,$\bigcap\limits_{A\in \{A|A\in F\text{与}x\in A\}}A$为$E$关于$R$的连通分量.
			\end{exer}
			证明:
			\par
			(1)对称性根据定义可证.$(x|y)R\text{与}(x|y)(x|z)(y|x)(z|y)R\Leftrightarrow x\in E$,即$(x|y)R\Leftrightarrow x\in E$,故当且仅当$R$在$E$上具有反身性时,公式$R\text{与}(x|z)(y|x)(z|y)R$在$E$上具有反身性.
			\par
			(2)令$R$的图为$G$,$S$的图为$F$.假设$(x, y)\in F$,由于$R$具有对称性,令$y_i=x_{n-i}$,则$(y_i)_{0\leq i\text{与} i\leq n}$满足$S$,故$(y, x) \in F$,因此$S$具有对称性.
			\par
			假设$x\in E$,由于$R$具有反身性,令$y_i=x$($0\leq i\text{与} i\leq n$),则则$(y_i)_{0\leq i\text{与} i\leq n}$满足$S$;反过来若$(x_i)_{0\leq i\text{与} i\leq n}$满足$S$,且其中$x_0=x$,$x_n=x$,则$(x, x_1)\in G$,由于$G\subset E\times E$,故$x\in E$,因此$S$在$E$上具有反身性.
			\par
			假设$(x, y)\in F$、$(y, z)\in F$,将相应的$(x_i)_{i\in [0, n]}$、$(Y_i)_{i\in [0, m]}$合在一起组成$(Zi)_{i \in [0, n+m+1]}$(当$i\in [0, n]$时,$Z_i=X_i$,当$i\in [n+1, n+m+1]$时,$Z_i=Y_{i-n-1}$),因此$(x, z) \in F$,因此$S$具有传递性.令$G^1=G$,$G^n=G^{n-1}\circ G$,则$F=\bigcup\limits_{n\in N-\{0\}}G^n$.对满足$G \subset G'$的任意$G'$,如果$G$为在$E$上的等价关系,根据定理\ref{theo56},对任意自然数$n>0$,$G^n\subset G'$,因此$F\subset G$,得证.
			\par
			(3)令$R$的图为$G$,$S$的图为$F$,则当$A\in F$且$x\in A$时,如果$(x, y)\in F$,则存在满足$S$的$(x_i)_{i\in [0,n]}$,如果存在$i$使$x_{i-1}\in A$但$x_i\notin A$,则$(x_{i-1}, x_i)\notin G$,矛盾,因此$y\in A$,故$y\in \bigcap\limits_{A\in \{A|A\in F\text{与}x\in A\}}A$.
			\par
			反过来,如果$(x, y)\notin F$,则$y\notin F(x)$、$x\in F(x)$、$F(x)\in F$,因此$y\notin \bigcap\limits_{A\in \{A|A\in F\text{与}x\in A\}}A$.得证.
			\par
			注:习题\ref{exer75}(2)、(3)涉及尚未介绍的“自然数”知识.
			
			\begin{exer}\label{exer76}
				\hfill\par
				(1)公式$R$具有对称性并且在$E$上具有反身性.如果对任意互不相等的$x$、$y$、$z$、$t$,$x\in E\text{与}y\in E\text{与}z\in E\text{与}t\in E\text{与}R\text{与}(z|y)R\text{与}(t|y)R\text{与}(y|x)(z|y)R\text{与}(y|x)(t|y)R\Rightarrow (t|y)(z|x)R$,则称$R$具有$1$级非传递性.如果$A\subset E$,并且$x\in A\text{与}y\in A\Rightarrow R$,则称$A$关于$R$稳定.如果$a\in E$、$b\in E$,$a\neq b$,并且$(b|y)(a|x)R$为真,令$C(a, b)=\{x|x\in E\text{与}(x|y)(a|x)R\text{与}\\(x|y)(b|x)R\}$,求证:$C(a, b)$关于$R$稳定;任意$x\in C(a, b)$、$y\in C(a, b)并且x\neq y$,均有$C(x, y)\\=C(a, b)$;此时,集合$C(a, b)$称为$E$关于$R$的组成部分,则$C(a, b)$是$E$关于$R$的连通分量;$E$\\的两个不同组成部分的交集最多有一个元素,并且对于E的三个不同组成部分$A$、$B$、$C$,$A\cap B$、$B\cap C$、$C\cap A$至少有一个为空或者三者相同.
				\par
				(2)反过来,设$(X_l)_{l\in L}$为$E$的覆盖,其中$l\in L\Rightarrow X_l\neq \varnothing$,并且:
				\par
				第一,对任意$l\in L$、$m\in L$且$l\neq m$,$X_l\cap X_m$最多有一个元素;
				\par
				第二,对任意$l\in L$、$m\in L$、$n\in L$且三者互不相等,$X_l\cap X_m$、$X_l\cap X_m$、$X_l\cap X_m$至少有一个为空或者三者相同.
				\par
				令公式$R$为$(\exists l)(l\in L\text{与}x\in X_l\text{与}y\in X_l)$,求证:R在E上具有反身性、具有对称性,并且具有$1$级非传递性.
				\par
				(3)	公式$R$具有对称性并且在$E$上具有反身性.如果对于$E$的$n$个互不相等的元素\\$(x_i)_{1\leq i \text{与}i\leq n}$,只要$(x_i|x)(x_j|y)R$($1\in [1,n]$,$j \in [1, n]$,$i\neq j$,且$(i, j)\neq (n-1, n)$,$(i, j)\neq (n, n-1)$)均为真,就有$(x_{n-1}|x)(x_n|y)R$为真,则称$R$具有$n-3$级非传递性.试类比习题\ref{exer76}(1)、习题\ref{exer76}(2),给出具有任意级非传递性的充分必要条件,并证明:具有p级非传递性的公式,也具有q级非传递性($q>p$).
			\end{exer}
			证明:
			\par
			(1)设$x\in C(a, b)$、$y\in C(a, b)$,则$R(a, x)$、$R(b, x)$、$R(a, y)$、$R(b, y)$、$R(a, b)$,且$a\in E$、$b\in E$、$x\in E$、$y\in E$,根据定义,无论$a$、$b$、$x$、$y$中是否有相等的,$R$均为真.故$C(a, b)$关于$R$稳定.如果$x\in C(a, b)$、$y\in C(a, b)$并且$x\neq y$,则对任意$z\in C(x, y)$,则$z\in E$并且$(z|y)(a|x)R$、$(z|y)(b|x)R$,故$z\in C(a, b)$,反之,同理可证对任意$z\in C(a, b)$,有$z\in C(x, y)$,因此$C(x, y)=C(a, b)$.对任意$x\in C(a, b)$,则$(x|y)(a|x)R$成立,反过来,如果$x$是$a$所在的$E$关于$R$的连通分量的元素,根据数学归纳法可证$x\in C(a, b)$,故$C(a, b)$是$a$所在的$E$关于$R$的连通分量.若$E$的两个不同组成部分的交集有两个元素$x$、$y$,则二者均为\\$C(x, y)$,矛盾,故最多有一个元素.假设$A\cap B$、$B\cap C$、$C\cap A$均不为空,设$A\cap B=\{a\}$,$B\cap C=\{b\}$,$C\cap A=\{c\}$,若其中$a=b$,则$a=c$,三者相同,若三者均不相同,则$A=C(a, c)$,$B=C(a, b)$,$C=C(b, c)$,因此$(b|y)(a|x)R$、$(a|y)(c|x)R$、$(c|y)(b|x)R$,因此$b\in A$、$c\in B$、$a\in C$,故$A\cap B$、$B\cap C$、$C\cap A$均有不少于三个公共元素,矛盾.得证.
			\par
			(2)$(\exists l)(l\in L\text{与}x\in X_l)$为真,故$R$具有反身性;$(\exists l)(l\in L\text{与}x\in X_l\text{与}y\in X_l)\Leftrightarrow (\exists l)(l\in L\text{与}y\in X_l\text{与}x\in X_l)$,故$R$具有对称性.对任意互不相等的$x$、$y$、$z$、$t$如果$x\in E\text{与}y\in E\text{与}z\in E\text{与}t\in E\text{与}R\text{与}(z|y)R\text{与}(t|y)R\text{与}(y|x)(z|y)R\text{与}(y|x)(t|y)R$,则存在$l\in l$,使$\{x, y\}\subset X_i$,存在$m\in L$,使$\{y, z\}\subset X_m$,存在$n\in L$,使$\{z, x\}\subset X_n$,因此$l$、$m$、$n$相等,故$\{x, y, z\}\subset X_l$,同理存在$l' \in l$,使$\{x, y, t\}\subset X_l'$,因此$l=l'$,故$\{x, y, z, t\}\subset X_l$,因此,$(t|y)(z|x)R$,故$R$具有$1$级非传递性.
			\par
			(3)	充分必要条件为:
			\par
			第一,集族任意两个元素的交集最多有$n-3$个元素;
			\par
			第二,集族中任意三个集合$A$、$B$、$C$,如果$A\cap B$、$B\cap C$、$C\cap A$均有$n-3$个元素,且其中$n-4$个元素是三个集合共有的,则全部$n-3$个元素均为三个集合共有.
			\par
			如果上述性质成立,考虑任何元素$x$,集族中包含$x$的集合,去掉$x$后,剩下的集合产生的公式具有$n-4$级非传递性,故充分性成立;
			如果$R$具有$n-3$级非传递性,对任意元素$x\in E$,考虑与$x$满足$R$的元素集合$E'$(不包含元素$x$),令$R'=R\cap (E'\times E')$,则$R'$具有$n-4$级非传递性,因此任意集合$A$、$B$、$C$,上述性质对$n-1$成立,则对于$A\cup\{x\}$、$B\cup\{x\}$、$C\cup\{x\}$,上述性质对$n$成立,必要性成立.
			\par
			如果公式$R$具有$p$级非传递性,则集族任意两个元素的交集最多有$p$个元素,由于$q>p$,故$q$级非传递性的条件一成立;同时,如果集族中任意三个集合$A$、$B$、$C$,如果$A\cap B$、$B\cap C$、$C\cap A$均有$q$个元素,且其中$q-1$个元素是三个集合共有的,则令$M$有$q-p$个元素且$M\subset (A\cap B\cap C)$,令$A'=A-M$、$B'=B-M$、$C'=C-M$,则$A'\cap B'$、$B'\cap C'$、$C'\cap A'$均有$p-3$个元素,且其中$p-4$个元素是三个集合共有的,故全部$p3$个元素均属于$A'\cap B'\cap C'$,也就是说,$q$级非传递性的条件二成立.综上,$R$具有$q$级非传递性.
			\par
			注:习题\ref{exer76}(1)、(3)涉及尚未介绍的“自然数”知识.
		
	\chapter{偏序集,基数,自然数(Ensembles ordonnés, cardinaux, nombres entiers)}
		\section{偏序关系,偏序集(Relations d'ordre, ensembles ordonnés)}		
			\begin{metadef}
				\textbf{偏序关系(relation d'ordre)}
				\par
				包含$2$元特别符号$\in$ 、显式公理\ref{ex1}、显式公理\ref{ex2}、显式公理\ref{ex3}和公理模式\ref{Sch8}的等式理论$M$中,令$R$为公式,$x$、$y$、$z$为不同的不是常数的字母,且$R$不包含$z$,如果下列三个公式为真:
				\par
				第一,$R$关于$x$、$y$具有传递性;
				\par
				第二,$R\text{与}(x|z)(y|x)(z|y)R\Rightarrow x=y$;
				\par
				第三,$R\Rightarrow (x|y)R\text{与}(y|x)R$,
				\par
				则称$R$为关于$x$、$y$的偏序关系,在没有歧义的情况下也可以简称为$R$为偏序关系.
			\end{metadef}
			
			\begin{Ccor}\label{Ccor68}
				\hfill\par
				包含$2$元特别符号$\in$ 、显式公理\ref{ex1}、显式公理\ref{ex2}、显式公理\ref{ex3}和公理模式\ref{Sch8}的等式理论$M$中,$R$为关于$x$、$y$的偏序关系,则$(x|z)(y|x)(z|y)R$也是关于$x$、$y$的偏序关系.
			\end{Ccor}
			证明:根据定义可证.
			
			\begin{metadef}
				\textbf{在集合上的偏序关系(relation d'ordre dans un ensemble)}
				\par
				包含$2$元特别符号$\in$ 、显式公理\ref{ex1}、显式公理\ref{ex2}、显式公理\ref{ex3}和公理模式\ref{Sch8}的等式理论$M$中,令$R$为关于$x$、$y$的偏序关系,并且$R$关于$x$、$y$在E上具有反身性,则称$R$为关于$x$、$y$在$E$上的偏序关系,在没有歧义的情况下也可以简称为$R$为在$E$上的偏序关系.
			\end{metadef}

			\begin{cor}\label{cor153}
				\hfill\par
				(1)$x=y$为关于$x$、$y$的偏序关系.
				\par
				(2)$x\subset y$为关于$x$、$y$的偏序关系.
				\par
				(3)$x=y\text{与}x\in E$为关于$x$、$y$在$E$上的偏序关系.
				\par
				(4)$F\subset \mathcal{P}(E)$,则$x\subset y\text{与}x\in F\text{与}y\in F$为关于$x$、$y$在F上的偏序关系.
				\par
				(5)$E$、$F$为集合,$H$的元素都是$E$的子集到$F$的映射,则$x\in H\text{与}y\in H \text{与}\\(y\text{为}x\text{的延拓})$为关于$x$、$y$在$H$上的偏序关系.
				\par
				(6)$E$为集合,$F=\{A|(\Delta_A\text{为}E\text{的划分})\}$,$X\in F\text{与}Y \in F\text{与}\Delta_X\text{为比}\Delta_Y\text{更细})\}$为关于$X$、$Y$在$F$上的偏序关系.
			\end{cor}
			证明:根据定义可证.
			
			\begin{Ccor}\label{Ccor69}
				\hfill\par
				包含$2$元特别符号$\in$ 、显式公理\ref{ex1}、显式公理\ref{ex2}、显式公理\ref{ex3}和公理模式\ref{Sch8}的等式理论$M$中,公式$R$为关于$x$、$y$在$E$上的偏序关系,则$R\Rightarrow x\in E$,$R\Rightarrow y\in E$.
			\end{Ccor}
			证明:根据定义可证.
			
			\begin{Ccor}\label{Ccor70}
				\hfill\par
				包含$2$元特别符号$\in$ 、显式公理\ref{ex1}、显式公理\ref{ex2}、显式公理\ref{ex3}和公理模式\ref{Sch8}的等式理论$M$中,公式$R$为关于$x$、$y$在$E$上的偏序关系,则$R\text{与}(x|z)(y|x)(z|y)R\Leftrightarrow x\in E\text{与}y\in E\text{与}x=y$.
			\end{Ccor}
			证明:根据定义,$R\text{与}(x|z)(y|x)(z|y)R\Rightarrow x\in E\text{与}y\in E\text{与}x=y$.另一方面,如果$x\in E\text{与}y\in E\text{与}x=y$,则$R\Leftrightarrow (y|x)R$、$R\Leftrightarrow (x|y)R$,进而$x\in E\text{与}y\in E\text{与}x=y\Rightarrow R\text{与}(x|z)(y|x)(z|y)R$.
			
			\begin{Ccor}\label{Ccor71}
				\hfill\par
				包含$2$元特别符号$\in$ 、显式公理\ref{ex1}、显式公理\ref{ex2}、显式公理\ref{ex3}和公理模式\ref{Sch8}的等式理论$M$中,公式$R$为关于$x$、$y$在$E$上的偏序关系,则$R$为生成图的公式.
			\end{Ccor}
			证明:根据补充证明规则\ref{Ccor69},$R\Rightarrow (x, y)\in E\times E$,根据补充证明规则1\ref{Ccor18}可证.

			\begin{de}
				\textbf{偏序图(graphe d'ordre)}
				\par
				$G$为图,如果$(x, y)\in G$为在$E$上的偏序关系,则称$G$为在$E$上的偏序图.
			\end{de}

			\begin{de}
				\textbf{在集合上的偏序对应(correspondance d'ordre dans un ensemble),偏序(ordre)}
				\par
				如果$F$为在$E$上的偏序图,则$(F, E, E)$称为在$E$上的偏序对应,或称为在$E$上的偏序.
			\end{de}

			\begin{theo}
				当且仅当同时满足下列两个条件时,对应$(G, E, E)$为在$E$上的偏序对应:
				\par
				第一,$G\circ G=G$;
				\par
				第二,$G\cap G^{-1}=\Delta_E\times E$.
			\end{theo}
			证明:如果$(G, E, E)$为偏序关系,由于$(x, y)\in G\text{与}(y, z) \in G\Rightarrow (x, z)\in G$,故$G\circ G\subset G$,同时,$(x, y)\in G\Rightarrow (y, y)\in G$,故$(x, y)\in G\circ G$,故$G\subset G\circ G$,因此$G\circ G=G$;根据补充证明规则\ref{Ccor70},$G\cap G^{-1}=\Delta_E\times E$.
			\par
			反过来,如果$G\circ G=G$,则$(x, y)\in G\text{与}(y, z) \in G\Rightarrow (x, z)\in G$;如果$G\cap G^{-1}=\Delta_E\times E$,故$(x, y)\in G\text{与}(y, x)\in G\Rightarrow x=y$;$\Delta_E\times E\subset G$,故$x\in E\Rightarrow (x, x)\in G$,又因为$pr_1G=E$,$pr_2G\subset E$,因此$(x, x)\in G\Rightarrow x\in G$,$(x, y)\in G\Rightarrow x\in E$,$(x, y)\in G\Rightarrow y\in E$,因此$(x, y)\in G\Rightarrow (x, x)\in G\text{与}(y, y)\in G$.故$F$为在$E$上的偏序图.
			
			\begin{metadef}
				\textbf{预序关系(relation de préordre)}
				\par
				包含$2$元特别符号$\in$ 、显式公理\ref{ex1}、显式公理\ref{ex2}、显式公理\ref{ex3}和公理模式\ref{Sch8}的等式理论$M$中,令$R$为公式,$x$、$y$、$z$为不同的不是常数的字母,且$R$不包含$z$,如果以下二个公式为真:
				\par
				(1)$R$关于$x$、$y$具有传递性;
				\par
				(2)$R\Rightarrow (x|y)R\text{与}(y|x)R$,
				\par
				则称$R$为关于$x$、$y$的预序关系,在没有歧义的情况下也可以简称为$R$为预序关系.
			\end{metadef}
						
			\begin{Ccor}\label{Ccor72}
				\hfill\par
				包含$2$元特别符号$\in$ 、显式公理\ref{ex1}、显式公理\ref{ex2}、显式公理\ref{ex3}和公理模式\ref{Sch8}的等式理论$M$中,$R$为关于$x$、$y$的预序关系,则$(x|z)(y|x)(z|y)R$也是关于$x$、$y$的预序关系.
			\end{Ccor}
			证明:根据定义可证.
			
			\begin{Ccor}\label{Ccor73}
				\hfill\par
				包含$2$元特别符号$\in$ 、显式公理\ref{ex1}、显式公理\ref{ex2}、显式公理\ref{ex3}和公理模式\ref{Sch8}的等式理论$M$中,$R$为偏序关系,则$R$为预序关系.
			\end{Ccor}
			证明:根据定义可证.
			
			\begin{metadef}
				\textbf{相反关系(relation opposée)}
				\par
				包含$2$元特别符号$\in$ 、显式公理\ref{ex1}、显式公理\ref{ex2}、显式公理\ref{ex3}和公理模式\ref{Sch8}的等式理论$M$中,令$R$为预序关系,则$(x|z)(y|x)(z|y)R$称为$R$的相反关系.
			\end{metadef}
			
			\begin{Ccor}\label{Ccor74}
				\hfill\par
				包含$2$元特别符号$\in$ 、显式公理\ref{ex1}、显式公理\ref{ex2}、显式公理\ref{ex3}和公理模式\ref{Sch8}的等式理论$M$中,$R$为预序关系(或偏序关系),则$R$的相反关系也是预序关系(或偏序关系).
			\end{Ccor}
			证明:根据补充证明规则\ref{Ccor68}、补充证明规则\ref{Ccor72}可证.
						
			\begin{Ccor}\label{Ccor75}
				\hfill\par
				包含$2$元特别符号$\in$ 、显式公理\ref{ex1}、显式公理\ref{ex2}、显式公理\ref{ex3}和公理模式\ref{Sch8}的等式理论$M$中,令$R$为预序关系,则$R\text{与}(R\text{的相反关系})$是等价关系.
			\end{Ccor}
			证明:根据定义可证.
						
			\begin{metadef}
				\textbf{在集合上的预序关系(relation d'préordre dans un ensemble)}
				\par
				包含$2$元特别符号$\in$ 、显式公理\ref{ex1}、显式公理\ref{ex2}、显式公理\ref{ex3}和公理模式\ref{Sch8}的等式理论$M$中,令$R$为关于$x$、$y$的预序关系,并且$R$关于$x$、$y$在$E$上具有反身性,则称$R$为关于$x$、$y$在$E$上的预序关系,在没有歧义的情况下也可以简称为$R$为在$E$上的预序关系.
			\end{metadef}
			
			\begin{Ccor}\label{Ccor76}
				\hfill\par
				包含$2$元特别符号$\in$ 、显式公理\ref{ex1}、显式公理\ref{ex2}、显式公理\ref{ex3}和公理模式\ref{Sch8}的等式理论$M$中,令$R$为在$E$上的预序关系,$S$为$R\text{与}(R的相反关系)$,$x'$、$y'$为与$x$、$y$不同的字母且$R$不包含$x'$、$y'$,则\\$(x'|y)S\text{与}(y|x)(y'|y)S$也是在$E$上的等价关系,并且$R$关于$x$、$y$同等价关系“$(x'|y)S\text{与}\\(y|x)(y'|y)S$”相容.
			\end{Ccor}
			证明:根据定义,$(x'|y)S\text{与}(y|x)(y'|y)S$是等价关系,同时,根据传递性,$R\text{与}(x'|y)S\text{与}\\(y|x)(y'|y)S\Rightarrow (x'|x)(y'|y)R$,得证.
			
			\begin{Ccor}\label{Ccor77}
				\hfill\par
				包含$2$元特别符号$\in$ 、显式公理\ref{ex1}、显式公理\ref{ex2}、显式公理\ref{ex3}和公理模式\ref{Sch8}的等式理论$M$中,公式$R$为关于$x$、$y$在$E$上的预序关系,则$R\Rightarrow x\in E$,$R\Rightarrow y\in E$.
			\end{Ccor}
			证明:根据定义可证.
			
			\begin{Ccor}\label{Ccor78}
				\hfill\par
				包含$2$元特别符号$\in$ 、显式公理\ref{ex1}、显式公理\ref{ex2}、显式公理\ref{ex3}和公理模式\ref{Sch8}的等式理论$M$中,$R$为关于$x$、$y$在$E$上的预序关系,令$S$为$R\text{与}(R\text{的相反关系})$,$R'$为公式$X\in E/S\text{与}Y\in E/S\\\text{与}(\exists x)(\exists y)(x\in X\text{与}y\in Y\text{与}R)$,则$R'$为关于$X$、$Y$在$E/S$上的偏序关系.
			\end{Ccor}
			证明:$R'\text{与}(Y|X)(Z|Y)R'\Leftrightarrow X\in E/S\text{与}Y\in E/S\text{与}Z\in E/S\text{与}(\forall x)(\forall y)(\forall z)(x\in X\text{与}y\in Y\text{与}z\in Z\Rightarrow R\text{与}(y|x)(z|y)R)$,由于$R$具有传递性,故$R'$具有传递性.
			\par
			$R'\text{与}(X|Z)(Y|X)(Z|Y)R'\Leftrightarrow X\in E/S\text{与}Y\in E/S\text{与}(\forall x)(\forall y)(x\in X\text{与}y\in Y\Rightarrow R\text{与}(x|z)(y|x)(z|y)R)$,等价于$X\in E/S\text{与}Y\in E/S\text{与}(\forall x)(\forall y)(x\in X\text{与}y\in Y\Rightarrow S)$,如果$X\in E/S\text{与}Y\in E/S$,$x\in X\text{与}y\in Y$,根据补充证明规则\ref{Ccor38}(1),$S\Leftrightarrow (\exists X)(x\in X\text{与}y\in X\text{与}X\in E/S)$,则$(\exists X)(x\in X\text{与}y\in X\text{与}X\in E/S)$,因此$X=Y$.故$R'\text{与}\\(X|Z)(Y|X)(Z|Y)R'\Rightarrow X\in E/S\text{与}Y\in E/S\text{与}X=Y$.
			\par
			同时,由于$R\Rightarrow (x|y)R\text{与}(y|x)R$,因此$R'\Rightarrow X\in E/S\text{与}(\forall X)(x\in X\Rightarrow (x|y)R)$,$R'\Rightarrow Y\in E/S\text{与}(\forall y)(y\in Y\Rightarrow (y|x)R)$,故$R'\Rightarrow (X|Y)R'\text{与}(Y|X)R'$.
			\par
			另外,由于$x\in E\Leftrightarrow (y|x)R$,故$X\in E/S\Rightarrow (X|Y)R'$.
			\par
			综上,得证.
			
			\begin{Ccor}\label{Ccor79}
				\hfill\par
				包含$2$元特别符号$\in$ 、显式公理\ref{ex1}、显式公理\ref{ex2}、显式公理\ref{ex3}和公理模式\ref{Sch8}的等式理论$M$中,$R$为偏序关系(或预序关系),$R'$为$R\text{与}x\in E\text{与}y\in E$,如果$x\in E\Rightarrow (y|x)R$,则$R'$为在$E$上的偏序关系(或预序关系).
			\end{Ccor}
			证明:$x\in E\Rightarrow (y|x)R$,故$x\in E\Rightarrow (y|x)R'$,同时,$R'\Rightarrow x\in E$,得证.
			
			\begin{Ccor}\label{Ccor80}
				\hfill\par
				包含$2$元特别符号$\in$ 、显式公理\ref{ex1}、显式公理\ref{ex2}、显式公理\ref{ex3}和公理模式\ref{Sch8}的等式理论$M$中,公式$R$为关于$x$、$y$在$E$上的预序关系,则$R$为生成图的公式.
			\end{Ccor}
			证明:根据补充证明规则\ref{Ccor77},$R\Rightarrow (x, y)\in E\times E$,根据补充证明规则\ref{Ccor18}可证.

			\begin{de}
				\textbf{预序图(graphe d' préordre)}
				\par
				对于图$G$,如果$(x, y)\in G$为在$E$上的预序关系,则称$G$为在$E$上的预序图.
			\end{de}

			\begin{de}
				\textbf{在集合上的预序对应(correspondance d'préordre dans un ensemble),预序(préordre)}
				\par
				如果$F$为在$E$上的预序图,则$(F, E, E)$称为在$E$上的预序对应,或称为在$E$上的预序.
			\end{de}
			
			\begin{cor}\label{cor154}
				\hfill\par
				(1)当且仅当同时满足下列两个条件时,对应$(G, E, E)$为在$E$上的预序对应:
				\par
				第一,$G\circ G\subset G$;
				\par
				第二,$\Delta_{E\times E}\subset G$.
				\par
				(2)$G$为在$E$上的预序图,则$(x, y)\in G\text{与}(y, x)\in G$为生成图的公式,其生成的图为$G\cap G^{-1}$.
			\end{cor}
			证明:
			\par
			(1)$G\circ G\subset G$等价于传递性,$\Delta_E\times E\subset G$与预序的第二个性质等价,得证.
			\par
			(2)$(x, y)\in G\text{与}(y, x)\in G\Rightarrow (x, y)\in G$,故其为生成图的公式.根据定义可证其生成的图为$G\cap G^{-1}$.
			
			\begin{cor}\label{cor155}
				\hfill\par
				(1)在$\varnothing$上的唯一的预序图是$\varnothing$,在$\varnothing$上的唯一的预序是$(\varnothing, \varnothing, \varnothing)$.并且,该预序图(或预序)是偏序图(或偏序).
				\par
				(2)在${x}$上的唯一的预序图是$\{(x, x)\}$,在${x}$上的唯一的预序是$(\{(x, x)\}, {x}, {x})$.并且,该预序图(或预序)是偏序图(或偏序).
			\end{cor}
			证明:
			\par
			(1)根据定义可证.
			\par
			(2)根据补充定理\ref{cor154}(1)可证.
			
			\begin{cor}\label{cor156}
				\hfill\par
				$G$为在$E$上的预序图,令$S$为公式$(x, y)\in G\text{与}(y, x)\in G$,令$R'$为公式$X\in E/S\text{与}Y\in E/S\text{与}(\exists x)(\exists y)(x\in X\text{与}y\in Y\text{与}(x, y)\in G)$,则该公式生成的图$G'$为$(E/S)\times (E/S)$的子集,并且是$G$在$E\times E$到$(E\times E)/(S\times S)$的规范映射下的像.
			\end{cor}
			证明:$X\in E/S\text{与}Y\in E/S\text{与}(\exists x)(\exists y)(x\in X\text{与}y\in Y\text{与}(x, y)\in G)\Rightarrow X\in E/S\text{与}Y\in E/S$,故该公式生成的图$G'$为$(E/S)\times (E/S)$的子集.
			\par
			令$S$的图为$F$,则$F=G\cap G^{-1}$.令$f$为$E$到$E/S$的规范映射,$g$为$E\times E$到$(E\times E)/(S\times S)$的规范映射,根据补充证明规则\ref{Ccor62}(1),$g(x, y)=f(x)\times f(y)$.则$(u, v)\in g\langle G\rangle\Leftrightarrow (\exists x)(\exists y)(u=f(x)\text{与}v=f(y)\text{与}(x, y)\in G)$,等价于$(\exists x)(\exists y)(u=F\langle x \rangle \text{与}v=F\langle y \rangle \text{与}(x, y)\in G)$,根据补充证明规则\ref{Ccor38}(2),等价于$X\in E/S\text{与}Y\in E/S\text{与}(\exists x)(\exists y)( x\in X\text{与}y\in Y\text{与}\\(x, y)\in G)$,得证.
			
			\begin{sign}
				\textbf{不等式(inégalité)}
				\par
				令$R$为关于$x$、$y$的预序关系或偏序关系,在没有歧义的情况下,$R$可以记作$x\leq y$.$x\leq y$也可以记作$y\geq x$,$x\leq y\text{与}x\neq y$记作$x<y$,$x\geq y\text{与}x\neq y$记作$x>y$.
				\par
				令$G$为在$E$上的预序图或偏序图,$S=(G, E, E)$,则$(x, y)\in G$可以记作$x\leq_Sy$.$x\leq_Sy$也可以记作$y\geq_Sx$,$x\leq_Sy\text{与}x\neq_Sy$记作$x<_Sy$, $x\geq_Sy\text{与}x\neq_Sy$记作$x>_Sy$.在没有歧义的情况下,可以分别简记为$x\leq y$、$y\geq x$、$x<y$、$x>y$.
			\end{sign}

			\begin{de}
				\textbf{更细的预序(préordre plus fin),更细的偏序(ordre plus fin)}
				\par
				$(F, E, E)$、$(F', E, E)$均为在$E$上的预序(或偏序),如果$F'\subset F$,则称$(F, E, E)$为比\\$(F', E, E)$更细的预序(或偏序).
			\end{de}
			注:在原书中,“更细”这个概念包括与自身相等的情况,即一个预序(或偏序)比自身更细.
			
			\begin{C}\label{C58}
				\hfill\par
				包含$2$元特别符号$\in$ 、显式公理\ref{ex1}、显式公理\ref{ex2}、显式公理\ref{ex3}和公理模式\ref{Sch8}的等式理论$M$\\中:
				\par
				(1)$x\leq y\Leftrightarrow x<y\text{或}x=y$;
				\par
				(2)$x\leq y\text{与}y<z\Rightarrow x<z$;
				\par
				(3)$x<y\text{与}y\leq z\Rightarrow x<z$;
				\par
				(4)$x\leq y\text{与}y<z\Rightarrow x<z$.
			\end{C}
			证明:根据定义可证.
			
			\begin{de}
				\textbf{偏序集(ensemble ordonné),预序集(ensemble préordonné)}
				\par
				$F$是在$E$上的偏序(或预序),则称$E$为按偏序(或预序)$F$排序的偏序集(或预序集),或称$E$为按偏序关系(或预序关系)$y\in F\langle x \rangle $或与之等价的公式排序的偏序集(或预序集).
			\end{de}

			\begin{de}
				\textbf{集合的同构(isomorphisme de ensembles),集合的逆同构(isomorphisme réciproque de ensembles),同构于一个集合(isomorphe à un ensemble)}
				\par
				如果$E$、$E'$分别为按$F$、$F'$排序的偏序集(或预序集),$f$为$E$到$E'$的双射,且$x\leq y\Leftrightarrow f(x)\leq f(y)$,则称$f$为$E$到$E'$的同构,$f$的逆映射称为$f$的逆同构.如果存在$E$到$E'$的同构,则称$E$同构于$E'$.
			\end{de}

			\begin{cor}\label{cor157}
				\textbf{同构的逆映射为同构}
				\par
				令$f$为$E$到$E'$的同构,则$f$的逆映射为$E'$到$E$的同构.
			\end{cor}

			\begin{de}
				\textbf{集合的逆同构(isomorphisme réciproque de ensembles)}
				\par
				令$f$为$E$到$E'$的同构,则$f$的逆映射称为$f$的逆同构.
			\end{de}

			\begin{de}
				\textbf{按包含关系排序的偏序集(ensemble ordonné par inclusion)}
				\par
				$F\subset \mathcal{P}(E)$,则按偏序关系$x\subset y\text{与}x\in F\text{与}y\in F$排序的偏序集$F$,称为按包含关系排序的偏序集.
			\end{de}

			\begin{de}
				\textbf{可比较的(cornparable),不可比较的(incomparable)}
				\par
				令$E$为预序集,如果$x\leq y\text{或}y\leq x$,则称$x$和$y$为可比较的.否则,称$x$和$y$为不可比较的.
			\end{de}

			\begin{cor}\label{cor158}
				\hfill\par
				令$E$为按$F$排序的偏序集(或预序集),$F$的图为$G$,$A\subset E$,则$(x, y)\in G\cap(A\times A)$为在$A$上关于$x$、$y$的偏序关系(或预序关系),$(G\cap(A\times A), A, A)$是在$A$上的偏序对应(或预序对应).
			\end{cor}
			证明:根据定义可证.
						
			\begin{metadef}
				\textbf{导出的偏序(ordre induits),导出的预序(préordre induits),导出的偏序关系(relation de ordre induits),导出的预序关系(relation de préordre induits),偏序子集(partie ordonné),预序子集(partie préordonné),偏序的延拓(prolongements de l'ordre),预序的延拓(prolongements de l'relation de ordre induits),偏序关系的延拓(prolongements de l'ordre),预序关系的延拓\\(prolongements de relation de l'ordre induits)}
				\par
				包含$2$元特别符号$\in$ 、显式公理\ref{ex1}、显式公理\ref{ex2}、显式公理\ref{ex3}和公理模式\ref{Sch8}的等式理论$M$中,令$E$为按$F$排序的偏序集(或预序集),$F$的图为$G$,$A\subset E$,则$(G\cap (A\times A), A, A)$称为$F$在$A$\\上导出的偏序(或预序);$(x, y)\in G\cap(A\times A)$或与之等价的公式,称为偏序关系$(x, y)\in G$或与之等价的公式在$A$上导出的偏序关系(或预序关系);按该偏序关系(或预序关系)在$A$上导出的偏序排序的偏序集,称为$E$的偏序子集(或预序子集).同时,$F$称为偏序(或预序)$(G\cap(A\times A), A, A)$在$E$上的延拓;$(x, y)\in G$或与之等价的公式,称为偏序关系(或预序关系)$(x, y)\in G\cap(A\times A)$或与之等价的公式在$E$上的延拓.
			\end{metadef}
			
			\begin{cor}\label{cor159}
				\hfill\par
				偏序集(或预序集)的偏序子集(或预序子集)的偏序子集(或预序子集),也是该偏序集(或预序集)的偏序子集(或预序子集).
			\end{cor}
			证明:根据定义可证.
			
			\begin{cor}\label{cor160}
				\hfill\par
				令$(E_i)_{i\in I}$为集族,对任意$i\in I$,$F_i$为在$E_i$上的偏序(或预序),其图为$G_i$,将关于$x_i$、$y_i$的偏序关系(或预序关系)$(x_i, y_i)\in G_i$记作$x_i\leq y_i$,令$F=\prod\limits_{i\in I}E_i$,则公式$(\forall i)(i\in I\Rightarrow pr_ix\leq pr_iy)$是关于$x$、$y$在$F$上的偏序关系(或预序关系).
				\par
			\end{cor}
			证明:根据定义可证.

			\begin{de}
				\textbf{偏序的乘积(ordre produit),预序的乘积(préordre produit),偏序关系的乘积(relation du ordre produit),预序关系的乘积(relation du préordre produit),偏序集的乘积(produit d'ensembles ordonnés),预序集的乘积(produit \\d'ensembles préordonnés)}
				\par
				令$(E_i)_{i\in I}$为集族,对任意$i\in I$,$F_i$为在$E_i$上的偏序(或预序),其图为$G_i$,将关于$x_i$、$y_i$的偏序关系(或预序关系)$(x_i, y_i)\in G_i$记作$x_i\leq y_i$,令$F=\prod\limits_{i\in I}E_i$,$G$为公式$(\forall i)(i\in I\Rightarrow pr_ix\leq pr_iy)$生成的图,则$(G, F, F)$称为偏序(或预序)$(F_i)_{i\in I}$的乘积,$x\leq y$称为上述各偏序(或预序)相应的偏序关系(或预序关系)的乘积,按$(F_i)_{i\in I}$的乘积排序的$F$称为偏序集(或预序集)$(E_i)_{i\in I}$的乘积.
			\end{de}
						
			\begin{cor}\label{cor161}
				\hfill\par
				令$F_1$为在$E_1$上的偏序(或预序),$F_2$为在$E_2$上的偏序(或预序),则公式$pr_1x\leq pr_1y\text{与}\\pr_2x\leq pr_2y$是关于$x$、$y$在$E_1\times E_2$上的偏序关系(或预序关系).
			\end{cor}
			证明:根据定义可证.
			
			\begin{de}
				\textbf{两个偏序的乘积(produit de deux ordres),两个预序的乘积(produit de deux préordres),两个偏序关系的乘积(relation du produit de deux ordres),两个预序关系的乘积(relation du produit de deux préordres),两个偏序集的乘积(produit de deux ensembles ordonnés),两个预序集的乘积(produit de deux ensembles préordonnés)}
				\par
				令$F_1$为在$E_1$上的偏序(或预序),$F_2$为在$E_2$上的偏序(或预序),$G$为公式$pr_1x\leq pr_1y\text{与}\\pr_2x\leq pr_2y$生成的图,则$(G, E_1\times E_2, E_1\times E_2)$称为偏序(或预序)$F_1$和$F_2$的乘积.$x\leq y$称为上述两个偏序(或预序)相应的偏序关系(或预序关系)的乘积,$F_1$和$F_2$的乘积排序的$E_1\times E_2$称为偏序集(或预序集)$E_1$和$E_2$的乘积.
			\end{de}			
			
			\begin{cor}\label{cor162}
				\hfill\par
				令$(E_i)_{i\in I}$为集族,对任意$i\in I$,$F_i$为在$E_i$上的偏序(或预序),其图为$G_i$,将关于$x_i$、$y_i$的偏序关系(或预序关系)$(x_i, y_i)\in G_i$记作$x_i\leq y_i$,令$F=\prod\limits_{i\in I}E_i$,$G$为上述偏序关系(或预序关系)的乘积生成的图,则G为$\prod\limits_{i\in I}G_i$在$\prod\limits_{i\in I}(E_i\times E_i)$到$F\times F$的规范映射下的像.
			\end{cor}
			证明:
			\par
			令该规范映射为$F$,对任意$f\in \prod\limits_{i\in I}G_i$,则$F(f)=((pr_1(pr_if))_{i\in I}, (pr_2(pr_if))_{i\in I})$,等于$((pr_1f(i))_{i\in I}, (pr_2f(i))_{i\in I})$,令$f(i)=(x_i, y_i)$,$x=(x_i)_{i\in I}$,$y=(y_i)_{i\in I}$,故$F(f)=(x, y)$.并且,对任意$i \in I$,$(x_i, y_i)\in G_i$,故$(x, y)\in G$.反过来,如果$(x, y)\in G$,则对任意$i\in I$,$(x_i, y_i)\in G_i$,因此$((x_i, y_i))_{i\in I}\in \prod\limits_{i\in I}G_i$.综上,得证.
						
			\begin{cor}\label{cor163}
				\hfill\par
				令$h$为$B^A$到$\mathcal{F}(A; B)$的规范映射,$F$为偏序集(或预序集),$R$为公式$f\in \mathcal{F}(A; B)\text{与}g\in \mathcal{F}(A; B)\text{与} x\in A\Rightarrow f(x)\leq g(x)$,$R'$为公式$F\in B^A\text{与}g\in B^A\text{与}x\in A\Rightarrow (F, A, B)(x)\leq (G, A, B)(y)$,则$R$为在$\mathcal{F}(A; B)$上的偏序关系(或预序关系),$R'$为在$B^A$上的的偏序关系(或预序关系),并且$h$为$B^A$到$\mathcal{F}(A; B)$的同构.
			\end{cor}
			证明:根据定义可证$R$和$R'$为偏序关系(或预序关系),根据补充定理\ref{cor123},$G\mapsto \\(G, A, B)$为双射,得证.

			\begin{de}
				\textbf{单增映射(application croissante),单减映射(application décroissante),单调映射(application monotone),严格单增映射(application strictement croissante),严格单减映射(application strictement décroissante),严格单调映射\\(application trictement monotone)}
				\par
				令$E$、$F$为预序集,$f$为$E$到$F$的映射,如果$x\leq y\Rightarrow f(x)\leq f(y)$,则称$f$为单增映射,如果$x\leq y\Rightarrow f(x)\geq f(y)$,则称$f$为单减映射,如果$f$是单增映射或单减映射,则称$f$为单调映射.如果$x<y\Rightarrow f(x)<f(y)$,则称$f$为严格单增映射,如果$x<y\Rightarrow f(x)>f(y)$,则称$f$为严格单减映射,如果$f$是严格单增的或严格单减的,则称$f$为严格单调映射.
			\end{de}
						
			\begin{cor}\label{cor164}
				\hfill\par
				令$E$、$F$为偏序集,$f$为$E$到$F$的映射,把$E$、$F$其中之一的偏序关系替代为其相反关系,如果$f$原来是单增映射(或单减映射),则变为单减映射(或单增映射).
			\end{cor}
			证明:根据定义可证.
			
			\begin{cor}\label{cor165}
				\hfill\par
				(1)如果常数函数的定义域和到达域为偏序集,则该常数函数是单增映射,也是单减映射.
				\par
				(2)如果函数是单调映射(或单增映射、单减映射),并且是单射,则该函数是严格单调映射(或单增映射、单减映射)
			\end{cor}
			证明:根据定义可证.
			
			\begin{cor}\label{cor166}
				\hfill\par
				$E$、$F$为偏序集,$f$为$E$到$F$的双射,则$f$为单增映射、$f^{-1}$为单增映射、$f$为$E$到$F$的同构三者等价.
			\end{cor}
			证明:根据定义可证.
		
			\begin{de}
				\textbf{单增子集族(famille de parties croissante),单减子集族(famille de parties décroissante),严格单增子集族(famille de parties strictement croissante),严格单减子集族(famille de parties strictement décroissante)}
				\par
				$(X_i)_{i\in I}$为$E$的子集族,其指标集$I$为偏序集,$\mathcal{P}(E)$为按包含关系排序的偏序集,如果映射$i\mapsto X_i(i\in I, X_i\in \mathcal{P}(E))$为单增映射(或单减映射、严格单增映射、严格单减映射),则称$(X_i)_{i\in I}$为单增子集族(或单减子集族、严格单增子集族、严格单减子集族).
			\end{de}

			\begin{theo}
				\hfill\par
				如果$E$、$E'$为偏序集,$E$到$E'$的映射$u$和$E'$到$E$的映射$v$均为单减映射,并且对任意$x\in E$、$x'\in E'$,$v(u(x))\geq x$,$u(v(x'))\geq x'$,则$u\circ v\circ u=u$、$v\circ u\circ v=v$.
			\end{theo}
			证明:$v(u(x))\geq x$,故$u(v(u(x)))\leq u(x);u(v(x'))\geq x'$,故$u(v(u(x)))\geq u(x)$,因此$u\circ v\circ u=u$,同理可证$v\circ u\circ v=v$.
			
			\begin{Ccor}\label{Ccor81}
				\hfill\par
				包含$2$元特别符号$\in$ 、显式公理\ref{ex1}、显式公理\ref{ex2}、显式公理\ref{ex3}和公理模式\ref{Sch8}的等式理论$M$中:
				\par
				(1)$E$为预序集,$x\leq y$为在$E$上的预序关系,$S$为在$E$上的等价关系,令$R$为公式$X\in E/S\text{与}Y\in E/S\text{与}((\forall x)(x\in X\Rightarrow (\exists y)(y\in Y\text{与}x\leq y)))$,则$R$为关于$X$、$Y$在$E/S$上的预序关系.
				\par
				(2)$E$为偏序集,$x\leq y$为在$E$上的偏序关系,$S$为在$E$上的等价关系,令$R$为公式$X\in E/S\text{与}Y\in E/S\text{与}((\forall x)(x\in X\Rightarrow (\exists y)(y\in Y\text{与}x\leq y)))$,如果$x\leq y\text{与}y\leq z\text{与}x\equiv z(mod S)\Rightarrow x\equiv y(mod S)$,则$R$为关于$X$、$Y$在$E/S$上的偏序关系.
			\end{Ccor}
			证明:
			\par
			(1)如果$(\forall x)(x\in X\Rightarrow (\exists y)(y\in Y\text{与}x\leq y))$、$((\forall y)(y\in Y\Rightarrow (\exists z)(z\in Z\text{与}y\leq z)))$,则$(y\in Y\text{与}x\leq y)\Rightarrow (\exists z)(z\in Z\text{与}y\leq z)\text{与}x\leq y$,因此$(y\in Y\text{与}x\leq y)\Rightarrow (\exists z)(z\in Z\text{与}x\leq z)$,故$((\forall x)(x\in X\Rightarrow (\exists z)(z\in Y\text{与}x\leq z)))$,传递性得证.
			\par
			由于$(x\in X\Rightarrow (x\in X\text{与}x\leq x))$,故$(\forall x)(x\in X\Rightarrow (\exists y)(y\in X\text{与}x\leq y)$.得证.
			\par
			(2)	如果$(\forall x)(x\in X\Rightarrow (\exists y)(y\in Y\text{与}x\leq y))$、$((\forall y)(y\in Y\Rightarrow (\exists z)(z\in X\text{与}y\leq z)))$,则$x\in X\Rightarrow (\exists y)(\exists z)(y\in Y\text{与}x\leq y\text{与}z\in X\text{与}y\leq z)$,故$x\in X\Rightarrow (\exists y)(y\in X\text{与}y\in Y)$,因此$X=Y$,得证.

			\begin{metadef}
				\textbf{预序关系的商(quotient relation de préordre),商预序集(ensemble préordonné quotient),偏序关系的商(quotient relation de ordre),商偏序集\\(ensemble ordonné quotient)}
				\par
				包含$2$元特别符号$\in$ 、显式公理\ref{ex1}、显式公理\ref{ex2}、显式公理\ref{ex3}和公理模式\ref{Sch8}的等式理论$M$\\中:
				\par
				$E$为预序集,$S$为在$E$上的等价关系,公式$X\in E/S\text{与}Y\in E/S\text{与}((\forall x)(x\in X\Rightarrow (\exists y)(y\in Y\text{与}x\leq y)))$称为预序关系$x\leq y$除以$S$的商,按该预序排序的$E/S$,称为预序集$E$除以$S$的商预序集,或称为$E/S$的商预序集.
				\par
				$E$为偏序集,$S$为在$E$上的等价关系,如果$x\leq y\text{与}y\leq z\text{与}x\equiv z(mod S)\Rightarrow x\equiv y\\(mod S)$,则公式$X\in E/S\text{与}Y\in E/S\text{与}((\forall x)(x\in X\Rightarrow (\exists y)(y\in Y\text{与}x\leq y)))$称为偏序关系$x\leq y$除以$S$的商,按该偏序排序的$E/S$,称为偏序集$E$除以$S$的商偏序集,或称为$E/S$的商偏序集.
			\end{metadef}
						
			\begin{Ccor}\label{Ccor82}
				\hfill\par
				包含$2$元特别符号$\in$ 、显式公理\ref{ex1}、显式公理\ref{ex2}、显式公理\ref{ex3}和公理模式\ref{Sch8}的等式理论$M$中,$E$为预序集,$S$为在$E$上的等价关系,$f$为$E$到$E/S$的规范映射:
				\par
				(1)对任何$E/S$的商预序集到预序集$F$的映射$g$,如果$g\circ f$为单增映射,则$g$为单增映射.
				\par
				(2)当且仅当$S$满足下列条件时,$f$为单增映射:	$(x\leq y\text{与}x\equiv x'(mod S))\Rightarrow(\exists y')(y'\in E\text{与}y\equiv y'(mod S)\text{与}x'\leq y')$.
			\end{Ccor}
			证明:
			\par
			(1)如果$X\leq Y$,则对任意$x\in X$,存在$y\in Y$,并且$x\leq y$.根据补充证明规则\ref{Ccor36}(2),$f(x)=X$,$f(y)=Y$,由于$g(f(x))\leq g(f(y))$,故$g(X) \leq g(Y)$,得证.
			\par
			(2)如果$f$是单增映射,对任意$E$的元素$x$、$y$、$x'$,如果$x\leq y$、$x\equiv x'(mod S)$,则$f(x)\\\leq f(y)$,故$f(x')\leq f(y)$,因此,存在$y'\in f(y)$,使$x' \leq y'$.
			\par
			反过来,对任意$E$的元素$x$、$y$,$x\leq y$,则对任意$x'\in f(x)$,都存在$y'\in f(y)$,且$x'\leq y'$,故$f(x)\leq f(y)$,得证.
			
			\begin{metadef}
				\textbf{同预序关系弱相容(faiblement compatible avec une relation de préordre)}
				\par
				包含$2$元特别符号$\in$ 、显式公理\ref{ex1}、显式公理\ref{ex2}、显式公理\ref{ex3}和公理模式\ref{Sch8}的等式理论$M$中,$E$为预序集,$S$为在$E$上的等价关系,$E/S$为商预序集,$f$为$E$到$E/S$的规范映射,如果$f$为单增映射,则称$S$在$x$、$y$上同在$E$上的预序关系$x\leq y$弱相容,在没有歧义的情况下,也可以简称作$S$同在$E$上的预序关系$x\leq y$弱相容.
			\end{metadef}
			
			\begin{Ccor}\label{Ccor83}
				\hfill\par
				包含$2$元特别符号$\in$ 、显式公理\ref{ex1}、显式公理\ref{ex2}、显式公理\ref{ex3}和公理模式\ref{Sch8}的等式理论$M$中,$E$为预序集,$S$为在$E$上的等价关系,$f$为$E$到$E/S$的规范映射,如果$x\leq y$在$x$上同$S$相容,则$S$同$x\leq y$弱相容.
			\end{Ccor}
			证明:根据补充证明规则\ref{Ccor82}(2)可证.
			
			\begin{cor}\label{cor167}
				\hfill\par
				$I$为偏序集,$(E_i)_{i\in I}$为偏序集族,令F为集族$(E_i)_{i\in I}$的和,$G=\{(x, y)|(x, y)\text{为有序对}\text{与}x\\\in F\text{与}y\in F\text{与}(pr_2x<pr_2y\text{或}(pr_2x=pr_2y\text{与}(\text{在}Epr_2x\text{上}pr_1x\leq pr_1y)))\}$,则$G$为在$F$上的偏序.
			\end{cor}
			证明:根据定义可证.
			
			\begin{de}
				\textbf{偏序集族的序数和(somme ordinale de la famille d'ensembles ordonnés)}
				\par
				$I$为偏序集,$(E_i)_{i\in I}$为偏序集族,令F为集族$(E_i)_{i\in I}$的和,$G=\{(x, y)|(x, y)\text{为有序对}\\\text{与}x\in F\text{与}y\in F\text{与}(pr_2x<pr_2y\text{或}(pr_2x=pr_2y\text{与}(\text{在}E_{pr_2x}\text{上}pr_1x\leq pr_1y)))\}$,则称按$G$排序的$F$为偏序集族$(E_i)_{i\in I}$的序数和.
			\end{de}
			注:偏序类族的序数和通常不满足交换律.
			
			\begin{cor}\label{cor168}
				\hfill\par
				$I$为偏序集,$(E_i)_{i\in I}$、$(Fi)_{i\in I}$为偏序集族,对任意$i\in I$,$E_i$同构于$F_i$,则$(E_i)_{i\in I}$的序数和同构于$(F_i)_{i\in I}$的序数和.
			\end{cor}
			证明:令$E_i$到$F_i$的同构为$f_i$,则映射$x\mapsto (f_{pr_2x}(pr_1x), pr_2x)$为$(E_i)_{i\in I}$的序数和到$(F_i)_{i\in I}$\\的序数和的同构.

			\begin{cor}\label{cor169}
				\textbf{偏序集族的序数和的结合律}
				\par
				$L$为偏序集,$I$为偏序集族$(J_l)_{l\in L}$的序数和,令$F_l$为偏序集族$(E_i)_{i\in J_l}$的序数和,则集族\\$(E_i)_{i\in I}$的序数和同构于集族$(F_l)_{l\in L}$的序数和.
			\end{cor}
			证明:令$f$为映射$x\mapsto ((pr_1x, pr_1(pr_2x)), pr_2(pr_2x))$,其为同构.

			\begin{de}
				\textbf{极大元(élément maximal),极小元(éléments minimal)}
				\par
				$E$为预序集,$a\in E$,如果$x\leq a\Rightarrow x=a$(或$x\geq a\Rightarrow x=a$),则称$a$为极小元(或极大元).
			\end{de}
			
			\begin{cor}\label{cor170}
				\hfill\par
				令$E$为预序集,$a$为极小元(或极大元),把$E$的预序关系替代为其相反关系,则$a$变为极大元(或极小元).
			\end{cor}
			证明:根据定义可证.
			
			\begin{de}
				\textbf{最大元(éléments plus petit),最小元(éléments plus grand)}
				\par
				$E$为预序集,$a\in E$,如果$x\in E\Rightarrow x\leq a$(或$x\in E\Rightarrow x\geq a$),则称$a$为$E$的最小元(或最大元).
			\end{de}

			\begin{cor}\label{cor171}
				\hfill\par
				偏序集最多只有一个最小元,一个最大元.
			\end{cor}
			证明:设$a$、$b$均为最小元,则$a\leq b$,$b\leq a$,因此$a=b$.
			
			\begin{cor}\label{cor172}
				\hfill\par
				令$E$为偏序集,$a$为最小元(或最大元),把$E$的偏序关系替代为其相反关系,则$a$变为最大元(或最小元).
			\end{cor}
			证明:根据定义可证.
			
			\begin{cor}\label{cor173}
				\hfill\par
				令$f$为偏序集$E$到偏序集$F$的同构,如果$E$有最小元(或最大元)$a$,则$F$有最小元(或最大元)$f(a)$.
			\end{cor}
			证明:根据定义可证.

			\begin{theo}\label{theo59}
				\hfill\par
				令$E$为偏序集,$a\notin E$,$E'$为$E$和$\{a\}$的和,则存在唯一的在$E'$上的偏序,其为在$E$上的偏序在$E'$上的延拓,且$a$为最大元(或最小元).
			\end{theo}
			证明:对于最大元的情况,令在$E$上的偏序为$G$,令$G'=G\cup \{z|pr_2z=a)\text{与}pr_1z\in E'\}$,则$G'$符合条件.
			\par
			设$G''$也符合条件,则$G\subset G'$,且对任意$x\in E'$,$(x, a)\in G''$,故$G'\subset G''$,同时,设$z\in G''$,如果$pr_1z\in E$且$pr_2z\in E$,由于$G''\cap E\times E=G$,故$z\in G$,因此$z\in G'$,如果$pr_1z\in E$且$pr_2z=a$,则$z\in G'$,如果$pr_1z=a$,由于$x\geq a\Rightarrow x=a$,故$z=(a, a)$,因此$z\in G'$,综上,$G'=G''$.
			\par
			同理可证最小元的情况.

			\begin{de}
				\textbf{向集合添加最大元得到的偏序集(ensemble ordonné obtenu adjoignant à un ensemble un plus grand élément),向集合添加最小元得到的偏序集(ensemble ordonné obtenu adjoignant à un ensemble un plus petit élément)}
				\hfill\par
				令$E$为偏序集,$a\notin E$,$E'$为$E$和$\{a\}$的和,令偏序$F$为在$E$上的偏序在$E'$上的延拓,且$a$\\为最大元(或最小元),则按该偏序排序的$E'$,称为向$E$添加最大元(或最小元)$a$得到的偏序集.
			\end{de}

			\begin{de}
				\textbf{共尾子集(partie cofinale),共首子集(partie coinitiale)}
				令$E$为预序集,$A\subset E$,如果$(\forall x)(x\in E\Rightarrow (\exists y)(y\in A\text{与}x\leq y))$,则称$A$为$E$的共尾子集;如果$(\forall x)(x\in E\Rightarrow (\exists y)(y\in A\text{与}x\geq y))$,则称$A$为$E$的共首子集.
			\end{de}

			\begin{cor}\label{cor174}
				\hfill\par
				任何预序集都是自身的共尾子集和共首子集.
			\end{cor}
			证明:根据定义可证.
			
			\begin{cor}\label{cor175}
				\hfill\par
				当且仅当$\{a\}$是偏序集$E$的共尾子集(或共首子集)时,$a$是$E$的最大元(或最小元).
			\end{cor}
			证明:根据定义可证.

			\begin{de}
				\textbf{下界(minorant),上界(majorant),严格下界(minorant strict),严格上界(majorant strict)}
				\par
				$E$为预序集,$X\subset E$,$x\in E$,如果$(\forall y)(y\in X\Rightarrow x\leq y)$,则称$x$为$X$在$E$上的下界,在没有歧义的情况下也可以简称为$X$的下界;如果$(\forall y)(y\in X\Rightarrow x\geq y)$,则称$x$为$X$在$E$上的上界,在没有歧义的情况下也可以简称为$X$的上界.如果上界不属于$X$,则称其为严格上界;如果下界不属于$X$,则称其为严格下界.			
			\end{de}
			
			\begin{cor}\label{cor176}
				\hfill\par
				(1)令$E$为预序集,$X\subset E$,$x$为$X$在$E$上的下界(或上界),把$E$的预序关系替代为其相反关系,则$x$变为$X$在$E$上的上界(或下界).
				\par
				(2)令$E$为预序集,$X\subset E$,$x$为$X$在$E$上的下界(或上界),$x\leq z$(或$x\geq z$),则$z$为$X$在$E$\\上的下界(或上界).
				\par
			\end{cor}
			证明:根据定义可证.

			\begin{cor}\label{cor177}
				\hfill\par
				令$E$为预序集,$X\subset E$,$Y\subset X$,$x$为$X$在$E$上的下界(或上界),则$x$为$Y$在$E$上的下界(或上界).
			\end{cor}
			证明:根据定义可证.
			
			\begin{cor}\label{cor178}
				\hfill\par
				$E$为偏序集,$X\subset E$,且$X$为$E$的偏序子集,则当且仅当$(\exists x)((x\text{为}X\text{在}E\text{上的下界})\text{与}x\\\in X)$时,$X$有最小元;当且仅当$(\exists x)((x\text{为}X\text{在}E\text{上的上界})\text{与}x\in X)$时,$X$有最大元.
			\end{cor}
			证明:根据定义可证.
			
			\begin{cor}\label{cor179}
				\hfill\par
				偏序集的子集如果有最大元,则最大元是其上界;如果有最小元,则最小元是其下界.
			\end{cor}
			证明:根据定义可证.
			
			\begin{cor}\label{cor180}
				\hfill\par
				$E$为预序集,$X\subset E$,则“$x\text{为}X\text{在}E\text{上的下界(或上界)}$”为关于$x$的集合化公式.
			\end{cor}
			证明:$x\text{为}X\text{的下界}\Rightarrow x\in E$,根据证明规则\ref{C52},下界的情况得证,上界的情况同理可证.

			\begin{de}
				\textbf{下界集(ensemble des minorant),上界集(majorant)}
				\par
				$E$为预序集,$X\subset E$,则$\{x|x\text{为}X\text{在}E\text{上的下界}\}$($\{x|x\text{为}X\text{在}E\text{上的上界}\}$)称为$X$在$E$\\上的下界集(或上界集),在没有歧义的情况下也可以简称为$X$的下界集(或上界集).
			\end{de}
			
			\begin{de}
				\textbf{有下界(minorée),有上界(majorée),有界(bornée)}
				\par
				$E$为预序集,$X\subset E$,如果“$X\text{在}E$\text{上的下界集(或上界集)}$\neq \varnothing$”,则称$X$在$E$上有下界(或有上界).如果$X$在$E$上有下界或有上界,则称$X$在$E$上有界.在没有歧义的情况下也可以简称为$X$有上界(或有下界、有界).
			\end{de}
						
			\begin{cor}\label{cor181}
				\hfill\par
				$E$为预序集,$X\subset E$,$Y\subset X$,$X$在$E$上有上界(或有下界、有界),则$Y$在$E$上有上界(或有下界、有界).
			\end{cor}
			证明:根据补充定理\ref{cor177}可证.

			\begin{de}
				\textbf{有下界的映射(application minorée),有上界的映射(application majorée),有界的映射(application bornée)}
				\par
				$E$为预序集,$f$为$A$到$E$的映射,如果$f\langle A \rangle $在$E$上有下界(或有上界、有界),则称$f$为有下界的映射(或有上界的映射、有界的映射).
			\end{de}

			\begin{de}
				\textbf{集合的最大下界(borne inférieure d'un ensemble),集合的最小上界\\(borne supérieure d'un ensemble),族的最大下界(borne inférieure d'une \\famille),族的最小上界(borne supérieure d'une famille)}
				\par
				令$E$为偏序集,$X\subset E$,如果$X$在$E$上的下界集(或上界集)有最大元(或最小元),则称其为$X$在$E$上的最大下界(或最小上界),记作$inf_EX$(或$sup_EX$),在没有歧义的情况下也可以简记为$inf\ X$(或$sup\ X$).
				\par
				如果$X$为二元集合$\{x, y\}$、三元集合$\{x, y, z\}$、四元集合$\{x, y, z, t\}$,为$X$在$E$上的最大下界(或最小上界)也可以记作$inf_E(x, y)$、$inf_E(x, y, z)$、$inf_E(x, y, z, t)$(或$sup_E(x, y)$、\\$sup_E(x, y, z)$、$sup_E(x, y, z, t)$),在没有歧义的情况下也可以简记为$inf(x, y)$、$inf(x, y, z)$、$inf(x, y, z, t)$(或$sup(x, y)$、$sup(x, y, z)$、$sup(x, y, z, t)$).
				\par
				对于族$(a_i)_{i\in I}$,如果$\bigcup\limits_{i\in I}\{a_i\}\subset E$,则称$\bigcup\limits_{i\in I}\{a_i\}$的最大下界(或最小上界)为该族的最大下界(或最小上界),记作$inf_E(a_i)_{i\in I}$(或$sup_E(a_i)_{i\in I}$),在没有歧义的情况下也可以简记为$inf(a_i)_{i\in I}$(或$sup(a_i)_{i\in I}$),或者简记作$\mathop{inf}\limits_{i\in I}a_i$(或$\mathop{sup}\limits_{i\in I}a_i$).
			\end{de}
			
			\begin{cor}\label{cor182}
				\hfill\par
				令$E$为偏序集,$X\subset E$,把$E$的偏序关系替代为其相反关系,则$X$在$E$上的最小上界变为最大下界,在$E$上的最大下界变为最小上界.
			\end{cor}
			证明:根据定义可证. 

			\begin{cor}\label{cor183}
				\hfill\par
				令$E$为偏序集,$X$是$E$的偏序子集:
				\par
				(1)如果$X$有最大元(或最小元),则$X$在$E$上的最小上界(或最大下界)是其最大元(或最小元).
				\par
				(2)如果$X$在$E$上有最小上界(或最大下界)$x$,且$x\in X$,则$x$是$X$的最大元(或最小元).
			\end{cor}
			证明:
			\par
			(1)	根据补充定理\ref{cor179}可证.
			\par
			(2)	根据定义可证.

			\begin{cor}\label{cor184}
				\hfill\par
				令$E$为偏序集,$X$是$E$的偏序子集,且$X \neq \varnothing$,如果$x$是$X$的最小上界,也是$X$的最大下界,则$X=\{x\}$.
			\end{cor}
			证明:对任意$y\in X$,$x\leq y$、$y\leq x$,故$x=y$,因此$X\subset \{x\}$,又因为)$X \neq \varnothing$,故$X=\{x\}$.

			\begin{de}
				\textbf{映射的最大下界(borne inférieure d'une application),映射的最小上界(borne supérieure d'une application)}
				\par
				$E$为预序集,$f$为$A$到$E$的映射,$f\langle A \rangle $的最大下界称为映射$f$的最大下界,令$x$为不出现在$f$的图、A、E的任何一个字母,则记作$\mathop{inf}\limits_{x\in A}f(x)$,$f\langle A \rangle $的最小上界称为映射$f$的最小上界,记作$\mathop{sup}\limits_{x\in A}f(x)$.
			\end{de}
			
			\begin{theo}\label{theo60}
				\hfill\par
				令$E$为偏序集,$A\subset E$,并且有在$E$上的最大下界和最小上界,当$A\neq \varnothing$时,$inf\ A\leq sup\ A$;当$A=\varnothing$时,$inf\ A$为$E$的最大元,$sup\ A$为$E$的最小元.
			\end{theo}
			证明:根据定义可证.
						
			\begin{theo}\label{theo61}
				\hfill\par
				令$E$为偏序集,$A$、$B$均为$E$的子集,并且均有在$E$上的最大下界(或最小上界),如果$A\subset B$,则$inf\ A\geq inf\ B$($sup\ A\leq sup\ B$).
			\end{theo}
			证明:根据定义可证.
						
			\begin{theo}\label{theo62}
				\hfill\par
				令$E$为偏序集,E的元素族$(x_i)_{i\in I}$有在$E$上的最小上界(或最大下界),则对任意$J\subset I$,$(x_i)i\in J$也有在$E$上的最小上界(或最大下界),并且$\mathop{sup}\limits_{i\in J}x_i\leq \mathop{sup}\limits_{i\in I}x_i$($\mathop{inf}\limits_{i\in J}x_i\geq \mathop{inf}\limits_{i\in I}x_i$).
			\end{theo}
			证明:根据定义可证.
			
			\begin{theo}\label{theo63}
				\hfill\par
				令$E$为偏序集,$E$的元素族$(x_i)_{i\in I}$、$(y_i)_{i\in I}$是两个$E$的子集族,其指标集相同且均有最小上界(或最大下界),如果$(\forall i)(i\in I\Rightarrow x_i\leq y_i)$,则$\mathop{sup}\limits_{i\in I}x_i\leq \mathop{sup}\limits_{i\in I}y_i$($\mathop{inf}\limits_{i\in I}x_i\leq \mathop{inf}\limits_{i\in I}y_i$).
			\end{theo}
			证明:设$a=\mathop{sup}\limits_{i\in I}y_i$,则$(\forall i)(i\in I\Rightarrow yi\leq a)$,故$(\forall i)(i\in I\Rightarrow x_i\leq a)$,因此$a是(x_i)_{i\in I}$的上界,上界的情况得证.下界的情况同理可证.
			
			\begin{theo}\label{theo64}
				\hfill\par
				令$E$为偏序集,$(x_i)_{i\in I}$是$E$的元素族,$(J_l)l\in L$为指标集$I$的覆盖,对任意$l\in L$,$(x_i)_{i\in J_l}$\\有在$E$上的最小上界(或最大下界),则当且仅当$(\mathop{sup}\limits_{i\in J_l}x_i)_{l\in L}$有在$E$上的最小上界(或最大下界)时,$(x_i)_{i\in I}$有在$E$上的最小上界(或最大下界),且$\mathop{sup}\limits_{i\in I}x_i=\mathop{sup}\limits_{l\in L}(\mathop{sup}\limits_{i\in J_l}x_i)$($\mathop{inf}\limits_{i\in I}x_i=\mathop{inf}\limits_{l\in L}(\mathop{inf}\limits_{i\in J_l}x_i)$).
			\end{theo}
			证明:令$b_l=\mathop{sup}\limits_{i\in J_l}x_i$,设$(x_i)_{i\in I}$有最小上界$a$,根据定理\ref{theo61},对任意$l\in L$,$a\geq b_l$.另一方面,由于$(J_l)_{l\in L}$为指标集$I$的覆盖,故若$c$满足对任意$l\in L$,$c\geq b_l$,则对任意$i\in I$,$c\geq x_i$,故$c为(x_i)_{i\in I}$的上界,因此$c\geq a$,故$a= \mathop{sup}\limits_{l\in L}(\mathop{sup}\limits_{i\in J_l}x_i)$.反过来,如果$(\mathop{sup}\limits_{i\in J_l}x_i)_{l\in L}$有最小上界$a'$,则对任意$l\in L$,$a\geq x_i$.另一方面,如果$c'$满足对任意$l\in L$,$c'\geq x_i$,则对任意$l\in L$,$c'\geq b_l$,因此$c' \geq a'$,故$a=\mathop{sup}\limits_{i\in I}x_i$.
			\par
			最大下界的情况同理可证.
			
			\begin{theo}\label{theo65}
				\hfill\par
				令$E$为偏序集,E的元素族$(x_{l,m})_{(l, m)\in L\times M}$为双族,如果对任意$m\in M$,$(x_{l,m})_{l\in L}$在$E$上有最小上界(或最大下界),则当且仅当$(\mathop{sup}\limits_{l\in L}x_{l,m})_{m\in M}$在$E$上有最小上界(或最大下界)时,$(x_{l,m})_{(l, m)\in L\times M}$在$E$上有最小上界(或最大下界),且$\mathop{sup}\limits_{(l, m)\in L\times M}x_{l,m}=\\\mathop{sup}\limits_{m\in M}(\mathop{sup}\limits_{l\in L}x_{l,m})$($\mathop{inf}\limits_{(l, m)\in L\times M}x_{l,m}=\mathop{inf}\limits_{m\in M}(\mathop{inf}\limits_{l\in L}x_{l,m})$).
			\end{theo}
			证明:根据定理\ref{theo64}可证.
			
			\begin{theo}\label{theo66}
				\hfill\par
				令$(E_i)_{i\in I}$为偏序集族,$E=\prod\limits_{i\in I}E_i$,$E$为按各偏序的乘积排序的偏序族,$A\subset E$,对任意$i\in I$,令$A_i=pr_iA$,则当且仅当对任意$i\in I$,$A_i$在$E_i$上有最小上界(或最大下界)时,$A$在$E$上有最小上界(或最大下界),且$sup\ A=\mathop{sup}\limits_{i\in I}pr_iA$.
			\end{theo}
			证明:如果存在$E_i=\varnothing$,则$E=\varnothing$,$A_i$和$A$均无上界和下界,无需考虑.
			\par
			如果对任意$i\in I$,$E_i=\varnothing$,令$A_i$的最小上界为$b_i$,设$(c_i)_{i\in I}$为$A$的上界,则对任意$i\in I$,$c_i\geq b$,故$(b_i)_{i\in I}$为$A$的最小上界.反过来,设$A$的最小上界为$(a_i)_{i\in I}$,对任意$i\in I$、$x_i\in A_i$,根据定理\ref{theo41},存在$x\in A$,使$pr_ix=x_i$;由于$x\leq (a_i)_{i\in I}$,故$x_i\leq a_i$,因此,对任意$i\in I$,$a_i$为$A_i$的上界;另一方面,假设${a'}_i$为$A_i$的上界,令$c=(a-(i, a_i))\cup (i, {a'}_i)$,由于$c\geq a$,故$a_i'\geq a_i$,因此,对任意$i\in I$,$a_i$为$A_i$的最小上界.最大下界的情况同理可证.

			\begin{cor}\label{cor185}
				\hfill\par
				$E$为偏序集,$f$为$E$到$E$的单增映射,令$A=\{z|z\in E\text{与}f(z)\leq z\}$,$B=\{z|z\in E\text{与}z\leq f(z)\}$,则:
				\par
				(1)	如果$A$有最小上界$v$,则$v=f(v)$;
				\par
				(2)	如果$B$有最大下界$w$,则$w=f(w)$.
			\end{cor}
			证明:
			\par
			(1)对任意$z\in A$,$v\leq z$,因此$f(v)\leq f(z)$,则$f(v)\leq z$,故$f(v)$是$A$的下界,因此$f(v)\leq v$,故$f(f(v))\leq f(v)$,因此$f(v)\in A$,$v\leq f(v)$,综上,$v=f(v)$.
			\par
			(2)类似补充定理\ref{cor185}(1)可证.
			\par
			注:本补充定理是习题\ref{exer63}的推广.
			
			\begin{theo}\label{theo67}
				\hfill\par
				令$E$为偏序集,$F\subset E$,$A\subset F$,如果$A$在$E$上和在$F$上均有最小上界(或最大下界),则$sup_EA\leq sup_FA$($inf_EA\geq inf_FA$);如果$A$在$E$上有最小上界,且$sup_EA\in F$,则$sup_EA\\=sup_FA$.
			\end{theo}
			证明:根据定义可证.

			\begin{de}
				\textbf{右方有向集(ensemble filtrant à droite/ensemble filtrant croissant),左方有向集(ensemble filtrant à gauche/ensemble filtrant décroissant)}
				\par
				$E$是预序集,如果$E$的任意二元子集在$E$上都有上界(有下界),此时称$E$为右方有向集(左方有向集).
			\end{de}
			
			\begin{theo}\label{theo68}
				\hfill\par
				$E$是偏序集,如果$E$是右方有向集(或左方有向集),则$E$的极大元是最大元(或极小元是最小元).
			\end{theo}
			证明:设$E$是右方有向集,$a$为极大元.对任意$x\in E$,设$\{x, a\}$的上界为$y$,则$a\leq y$,$x\leq y$,故$a=y$,因此$x\leq a$,因此$a$为最大元.左方有向集的情形同理可证.
						
			\begin{cor}\label{cor186}
				\hfill\par
				令$E$为偏序集,如果$E$为右方有向集(或左方有向集),把$E$的偏序关系替代为其相反关系,则$E$变为为左方有向集(或右方有向集).
			\end{cor}
			证明:根据定义可证.
			
			\begin{cor}\label{cor187}
				\hfill\par
				$I$、$L$均为右方有向集,$I\times L$的预序关系为$(x\in I\times L\text{与}y\in I\times L\text{与}pr_1x\leq pr_1y\text{与}pr_2x\leq pr_2y)$,则$I\times L$为右方有向集.
			\end{cor}
			证明:设$(a, b)\in I\times L$,$(c, d)\in I\times L$,$\{a, c\}$在$I$上的上界为$x$,$\{b, d\}$在$L$上的上界为$y$,则$(x, y) \in I\times L$,且$(x, y)$为$\{(a, b), (c, d)\}$在$I\times L$上的上界,得证.			
			
			\begin{de}
				\textbf{格(ensemble réticulé)}
				\par
				如果偏序集$E$的任何二元子集都有在$E$上的最大下界和最小上界,则称E为格.
			\end{de}

			\begin{cor}\label{cor188}
				\hfill\par
				(1)	格的乘积是格.
				\par
				(2)	格的偏序子集为格.
				\par
				(3)	如果格有极小元,则其为格的最小元;如果格有极大元,则其为格的最大元.
			\end{cor}
			证明:
			\par
			(1)	根据定理\ref{theo66}可证.
			\par
			(2)	根据定义可证.
			\par
			(3)	根据定义可证.

			\begin{de}
				\textbf{不可约元素(élément irréductible)}
				\par
				$E$为格,$a\in E$,如果$(\forall x)(\forall y)(x\in E\text{与}y\in E\text{与} sup(x, y)=a\Rightarrow x=a\text{或}y=a)$,则称$a$为$E$的不可约元素.
			\end{de}
			
			\begin{cor}\label{cor189}
				\hfill\par
				$E$为格,$a$为$E$的最小元,则$a$为$E$的不可约元素.
			\end{cor}
			证明:根据定义可证.

			\begin{de}
				\textbf{内部格(ensemble coréticulée)}
				\par
				$E$为格,$A\subset E$,如果对任意$x\in A$、$y\in A$,均有$sup_E(x, y)\in A$、$inf_E(x, y)\in A$,则称$A$为$E$的内部格.
			\end{de}

			\begin{de}
				\textbf{全序集(ensemble totalement ordonné),全序(ordre total),全序关系(relation d'ordre total),全序图(graphe d'ordre total),全序子集(partie totalement ordonné),链(chaîne d'ensemble)}
				\par
				令$E$为偏序集,如果$E$的任何两个元素都是可比较的,则称$E$为全序集.$E$的偏序称为全序,$E$的偏序关系称为全序关系,其偏序图称为全序图.
				\par
				令$E$为偏序集,如果$E$的偏序子集是全序集,则称其为$E$的全序子集,或称其为$E$的链.
			\end{de}
			
			\begin{cor}\label{cor190}
				\hfill\par
				(1)$E$为全序集,$x$、$y$为$E$的元素,则$x=y\text{或}x<y\text{或}x>y$.
				\par
				(2)令$E$为全序集,把$E$的全序关系替代为其相反关系,则E仍为全序集.
			\end{cor}
			证明:根据定义可证.
			
			\begin{cor}\label{cor191}
				\hfill\par
				全序集是左方有向集,是右方有向集,也是格.
			\end{cor}
			证明:根据定义可证.
			
			\begin{cor}\label{cor192}
				\textbf{偏序集族的序数和为右方有向集、全序集、格的条件}
				\par
				令$F$为偏序集族$(E_i)_{i\in I}$的序数和,且对任意$i\in I$,$E_i\neq \varnothing$:
				\par
				(1)	当且仅当$I$为右方有向集且对$I$的任意极大元$i$,并且$E_i$均为右方有向集时,$F$为右方有向集.
				\par
				(2)	当且仅当$I$为全序集,且对任意$i\in I$,$E_i$均为全序集时,$F$为全序集.
				\par
				(3)	当且仅当满足下列条件时,F为格:
				\par
				第一,$I$为格,并且,对任意$i\in I$、$j\in I$,如果$i$和$j$是不可比较的,则$E_{sup(i, j)}$有最小元,$E_{inf(i, j)}$有最大元;
				\par
				第二,对任意$i\in I$,如果$x\in E_i$,$y\in E_i$,且$\{x, y\}$在$E_i$上有上界(或下界),则$\{x, y\}$在\\$E_i$上有最小上界(或最大下界);
				\par
				第三,对任意$i\in I$,如果$x\in E_i$,$y\in E_i$,且$\{x, y\}$在$E_i$上没有上界(或下界),则$\{k|k\\\in I\text{与}k>i\}$(或$\{k|k\in I\text{与}k<i\}$)有最小元(或最大元)$j$,且$E_j$有最大元(或最小元).
			\end{cor}
			证明:
			\par
			(1)	充分性:
			\par
			对任意$x\in F$、$y\in F$,如果$pr_2x=pr_2y$,那么:若存在$j \in I$使$j>pr_2x$,设$z\in Ej$,则$z$为$\{x, y\}$的上界;若$pr_2x$为$I$的极大元,则存在$z$使$z\geq pr_1x$且$z\geq y$,则$z$为$\{x, y\}$的上界;
			\par
			如果$pr_2x\neq pr_2y$,则存在$j\geq pr_2x$、$j\geq pr_2y$:
			若$j>pr_2x$、$j>pr_2y$,设$z\in E_j$,则$z$为$\{x, y\}$的上界;若$j>pr_2x$、$j=pr_2y$,则$y$为$\{x, y\}$的上界,若$j=pr_2x$、$j>pr_2y$,则$x$为$\{x, y\}$的上界.
			\par
			必要性:
			\par
			对任意$i\in I$、$j\in I$,$i\neq j$,设$x\in E_i$,$y\in E_j$,则存在$z\geq (x, i)$、$z\geq (y, j)$,设$z\in E_k$,则$k$为$\{i, j\}$的上界;
			\par
			对$I$的任意极大元$i$,设$x\in E_i$,$y\in E_i$,则存在$z\geq (x, i)$、$z\geq (y, i)$,故$pr_2z=i$,因此$pr_1z\geq x$、$pr_1z\geq y$,所以$E_i$为右方有向集.
			\par
			(2)	充分性:
			\par
			对任意$x\in F$、$y\in F$:
			\par
			如果$pr_2x=pr_2y$,由于$pr_1x$和$pr_1y$是可比较的,因此$x$和$y$是可比较的;
			\par
			如果$pr_2x\neq pr_2y$,由于$pr_2x$和$pr_2y$是可比较的,因此$x$和$y$是可比较的;
			\par
			必要性:对任意$i\in I$、$j\in I$,$i\neq j$,设$x\in E_i$,$y\in E_j$,由于$(x, i)$和$(y, j)$是可比较的,因此$i$和$j$是可比较的,所以$I$为全序集;
			\par
			对任意$i\in I$,设$x\in E_i$,$y\in E_i$,由于$(x, i)$和$(y, i)$是可比较的,因此$x$和$y$是可比较的,所以$E_i$为全序集.
			\par
			(3)	充分性:
			\par
			对任意$x\in F$、$y\in F$:
			\par
			如果$pr_2x=pr_2y$,根据第二个条件和第三个条件,其有最小上界和最大下界;
			\par
			如果$pr_2x\neq pr_2y$,根据第一个条件,其有最小上界和最大下界.
			\par
			必要性:
			\par
			对任意$i\in I$、$j\in I$,$i\neq j$,设$x\in E_i$,$y\in E_j$,由于$(x, i)$和$(y, j)$有最大上界和最小下界,因此第一个条件成立;
			\par
			对任意$i\in I$,设$x\in E_i$,$y\in E_i$,由于$(x, i)$和$(y, i)$有最大上界和最小下界,因此第二个条件、第三个条件成立.
			
			\begin{theo}\label{theo69}
				\hfill\par
				全序集$E$到全序集$F$的严格单调映射,为单射.如果$f$为严格单增映射,则$f$为$E$到$f(E)$\\的同构.
			\end{theo}
			证明:如果$x\neq y$,则$x<y\text{或}x>y$,故$f(x)<f(y)\text{或}f(y)<f(x)$,因此$f(x)\neq f(y)$,故$f$为单射.如果$f$为严格单增映射,则$x\leq y\Rightarrow f(x)\leq f(y)$,同时$x>y\Rightarrow f(x)>f(y)$,故$f(x)\leq f(y)\Rightarrow x<y$,因此$f$为$E$到$f(E)$的同构.
						
			\begin{theo}\label{theo70}
				\hfill\par
				令$E$为全序集,$X\subset E$,当且仅当$b$为$X$在$E$上的上界(或下界),并且$(\forall c)(c\in E\text{与}c<b\Rightarrow (\exists x)(x\in X\text{与}c<x\text{与}x\leq b))$(或$(\forall c)(c\in E\text{与}c>b\Rightarrow (\exists x)(x\in X\text{与}c>x\text{与}x\geq b))$)时,$b$为$X$在$E$上的最小上界(或最大下界).
			\end{theo}
			证明:如果$(\forall c)(c\in E\text{与}c<b\Rightarrow (\exists x)(x\in X\text{与}c<x\text{与}x\leq b))$,同时$b$为上界,则对任意$c\in E$且$c<b$,$c$都不是$X$的上界,故$b$是最小上界;
			\par
			反过来,如果$b$是最小上界,则$b$是上界,同时,对任意$c$ $c\in E$且$c<b$,由于$c$不是$X$的上界,故存在$x\in X$,使$c<x$,同时,由于$b$是上界,故$x\leq b$,得证.
			\par
			下界的情况同理可证.
			
			\begin{de}
				\textbf{自由子集(partie libre),反链(antichaîne d'ensemble)}
				\par
				令$E$为偏序集,$X\subset E$,如果$X$的任何两个不同元素都是不可比较的,则称$X$为$E$的自由子集,或称$X$为$E$的反链.
			\end{de}

			\begin{cor}\label{cor193}
				\hfill\par
				$F=\{X|X\text{为}E\text{的自由子集}\}$,则:
				\par
				(1)$X\in F\text{与}Y\in F\text{与}(\forall x)(x\in X\Rightarrow (\exists y)(y\in Y\text{与}x\leq y))$为关于$X$、$Y$在$F$上的偏序关系;
				\par
				(2)$F$按(1)的偏序关系排序,则$x\mapsto \{x\}$为$E$到$F$的子集的同构;
				\par
				(3)$F$按(1)的偏序关系排序,则如果$X\in F$、$Y\in F$、$X\subset Y$,则$X\leq Y$;
				\par
				(4)$F$按(1)的偏序关系排序,则当且仅当$E$为全序集、$E$到$F$存在同构时,F为全序集.
			\end{cor}
			证明:
			\par
			(1)如果$x\in X\Rightarrow (\exists y)(y\in Y\text{与}x\leq y)$,$y\in Y\Rightarrow (\exists x)(x\in X\text{与}y\leq x))$,那么对任意$x\in X$,存在$y\in Y$且$x\leq y$,故存在$x' \in X$且$y\leq x'$,因此$x\leq x'$,故$x=x'$,故$x=y$,故$X\subset Y$,同理$Y\subset X$,因此$X=Y$;另外两个条件类似补充定理\ref{Ccor81}(1)可证.
			\par
			(2)	根据定义,$x\leq y\Leftrightarrow \{x\}\leq \{y\}$,得证.
			\par
			(3)	根据定义可证.
			\par
			(4)	充分性根据定义可证.如果$F$为全序集,则$E$为全序集,故$F=\{X|(\exists x)(X=\{x\})\}$,因此$E$到$F$存在同构,必要性得证.

			\begin{de}
				\textbf{完备格(ensemble réticulé achevé)}
				\par
				如果偏序集$E$的任何子集在$E$上都有最大下界和最小上界,则称$E$为完备格.
			\end{de}
			
			\begin{cor}\label{cor194}
				\hfill\par
				如果偏序集$E$的任何子集在$E$上都有最小上界,则$E$为完备格.
			\end{cor}
			证明:对任意$F\subset E$,令$G=\{x|x\text{为}F\text{在}E\text{上的下界}\}$,$G$在$E$上有最小上界$x$,如果存在$y\in F$且$y<x$,则$y$也是$G$的上界,矛盾,故$x$是$F$的下界,所以$x\in G$,故$x$是$G$的最大元,因此$F$有最大下界,得证.
			
			\begin{cor}\label{cor195}
				\hfill\par
				当且仅当各偏序集都是完备格时,偏序集的积是完备格.
			\end{cor}
			证明:根据定义可证.
			
			\begin{cor}\label{cor196}
				\textbf{偏序集的序数和为完备格的条件}
				\par
				当且仅当集族$(E_i)_{i\in I}$满足下列条件时,其序数和为完备格:
				\par
				第一,$I$为完备格;
				\par
				第二,对任意$J\subset I$,如果$J$没有最大元,令$d=sup\ J$,则$E_d$有最小元;
				\par
				第三,对任意$i\in I$和任意$E_i$的子集,如果在$E_i$上有上界,则有最小上界;
				\par
				第四,对任意$i\in I$,如果$E_i$没有最大元,则$\{x|x>i\text{与}x\in I\}$有最小元$a$,并且$E_a$有最小元,
			\end{cor}
			证明: 
			\par
			必要性:
			\par
			如果$(E_i)_{i\in I}$的序数和为完备格,对$I$的任意子集$K$,$(E_i)_{i\in K}$的序数和有最大下界$x$和最小上界$y$,故$K$有最大下界$pr_2x$和最小上界$pr_2y$;如果$J\subset I$,且$J$没有最大元,令$(E_i)_{i\in J}$的最小上界$x$,则$pr_2x=d$,$pr_1x$为$d$的最小元;令$E_i\times \{i\}$的最小上界为$x$,则$pr_2x=a$,其最小元为$pr_1x$.
			\par
			充分性:
			\par
			对$(E_i)_{i\in I}$的序数和的任意子集$K$,令$J=pr_2k$,则$J$有最小上界$d$.如果$d\notin J$,则$E_d$有最小元$a$,$(a, d)$即为$K$的最小上界,如果$d\in J$,那么,若$K\cap E_d$有上界,则有最小上界$y$,$(y, d)$即为$K$的最小上界,若$K\cap E_d$没有上界,则$E_d$没有最大元,故$\{x|x>d\text{与}x\in I\}$有最小元$b$,并且$E_b$有最小元$z$,$(z, b)$即为$K$的最小上界.
			
			\begin{cor}\label{cor197}
				\hfill\par
				$E$、$F$为偏序集,令$A(E, F)=\{X|X\in FE\text{与}((X, E, F)\text{为单增函数})\}$,并为按$f\in A(E, F)\text{与}g\in A(E, F)\text{与}(\forall x)(x\in E\Rightarrow f(x)\leq g(x))$排序的关于$f$、$g$的偏序集,则当且仅当$F$为完备格时,$A(E, F)$为完备格.
			\end{cor}
			证明:
			\par
			充分性:
			\par
			对$A(E, F)$的任何子集$G$,映射$x\mapsto (\{y|(y, E, F)(x)\}\text{的最小上界})$的图,是$G$的最小上界,同理可证$G$有最大下界.
			\par
			必要性:
			\par
			对任意$X\subset F$,令$G=\{A|(\exists x)(A=E\times \{x\}\text{与}x\in X)\}$,设$G$的最小上界为$H$,对任意$z\in E$,令$u=(H, E, F)(z)$,则$u$是$X$的上界,如果$u'<u$且为$X$的上界,则令$H'=(H-\{(z, u)\})\cup\{(z, u')\}$,$H'<H$,且$H'$也是$G$的上界,矛盾,故$u$是$X$的最小上界,同理可证$X$的最大下界存在.

			\begin{de}
				\textbf{闭包(fermeture)}
				\par
				$E$为偏序集,$E$到$E$的映射$f$如果满足下列条件,则称$f$为在$E$上的闭包,在没有歧义的情况下也可以简称$f$为闭包:
				\par
				第一,f是单增映射;
				\par
				第二,对任意$x\in E$,$f(x)\geq x$;
				\par
				第三,对任意$x\in E$,$f(f(x))=f(x)$.
			\end{de}
			
			\begin{cor}\label{cor198}
				\textbf{分配格的判定和性质}
				\par
				$E$为格,则以下五个公式等价:
				\par
				第一,$(\forall x)(\forall y)(\forall z)(x\in E\text{与}y\in E\text{与}z\in E\Rightarrow sup(x, inf(y, z))=inf(sup(x, y), \\sup(x, z)))$;
				\par
				第二,$(\forall x)(\forall y)(\forall z)(x\in E\text{与}y\in E\text{与}z\in E\Rightarrow inf(x, sup(y, z))=sup(inf(x, y), \\inf(x, z)))$;
				\par
				第三,$(\forall x)(\forall y)(\forall z)(x\in E\text{与}y\in E\text{与}z\in E\Rightarrow sup(inf(x, y), inf(y, z), inf(z, x))=inf(sup(x, y), sup(y, z), sup(z, x)))$;
				\par
				第四,$(\forall x)(\forall y)(\forall z)(x\in E\text{与}y\in E\text{与}z\in E\Rightarrow inf(z, sup(x, y))\leq sup(x, inf(y, z)))$;
				\par
				第五,$(\forall x)(\forall y)(\forall z)(x\in E\text{与}y\in E\text{与}z\in E\Rightarrow inf(sup(x, y), sup(z, inf(x, y)))=sup(inf(x, y), inf(y, z), inf(z, x)))$.
			\end{cor}
			证明:
			\par
			以上五个公式分别记作$R_1$、$R_2$、$R_3$、$R_4$、$R_5$:
			\par
			如果$sup(x, inf(y, z))=inf(sup(x, y), sup(x, z))$,则$sup(inf(x, y), inf(y, z), inf(z, x))\\=sup(inf(sup(x, inf(y, z)), sup(y, inf(y, z)))), inf(z, x))$,等于$sup(inf(sup(x, y), sup(x, z), \\y, sup(y, z)), inf(z, x))$,等于$inf(sup(x, y, z), sup(x, z), sup(y, z), sup(x, y))$,等于\\$inf(sup(x, y), sup(y, z), sup(z, x))$,即$R_1\Rightarrow R3$.
			\par
			同理可证$R_2\Rightarrow R_3$.
			\par
			根据定义可证$sup(z, inf(x, y))=inf(x, sup(y, z))$.并且,$sup(sup(inf(x, y), inf(y, z), \\inf(z, x)), x)=sup(inf(sup(x, y), sup(y, z), sup(z, x)), x)$,$\text{左边}=sup(x, inf(y, z))$,$\text{右边}=inf(sup(x, z), sup(x, y), sup(x, y, z))$,等于$inf(sup(x, y), sup(x, z))$,即$R_3\Rightarrow R_1$.
			\par
			同理可证$R_3\Rightarrow R_2$.
			\par
			如果$inf(x, sup(y, z))=sup(inf(x, y), inf(x, z))$,则$inf(z, sup(x, y))\leq \\sup(x, inf(y, z)$),即$R_2\Rightarrow R_4$.
			\par
			如果$inf(z, sup(x, y))\leq sup(x, inf(y, z))$,则$sup(inf(x, y), inf(x, z))\geq \\inf(x, sup(inf(x, y), z))$,故$sup(inf(x, y), inf(x, z))\geq inf(x, sup(y, z))$,同时,由于\\$inf(x, y)\leq x$、$inf(x, y)\leq sup(y, z)$,故$inf(x, y)\leq inf(x, sup(y, z))$,同理$inf(x, z) \leq inf(x, sup(y, z))$,因此$sup(inf(x, y), inf(x, z))=inf(x, sup(y, z))$.因此$R_4\Rightarrow R_2$.
			\par
			如果$inf(x, sup(y, z))=sup(inf(x, y), inf(x, z))$,则$inf(sup(x, y), sup(z, inf(x, y)))=sup(inf(x, y), inf(y, z), inf(z, x))$,即$R_2\Rightarrow R_5$.
			\par
			如果$inf(sup(x, y), sup(z, inf(x, y)))=sup(inf(x, y), inf(y, z), inf(z, x))$,而\\$inf(z, sup(x, y)) \leq inf(sup(x, y), sup(z, inf(x, y)))$,$sup(inf(x, y), inf(y, z), inf(z, x))\leq \\sup(x, inf(y, z))$,故$R_5\Rightarrow R_4$.
			
			\begin{de}
				\textbf{分配格(ensemble réticulé distributif)}
				\par
				$E$为格,如果$(\forall x)(\forall y)(\forall z)(x\in E\text{与}y\in E\text{与}z\in E\Rightarrow sup(x, inf(y, z))=\\inf(sup(x, y), sup(x, z)))$,则称$E$为分配格.
			\end{de}		
			
			\begin{cor}\label{cor199}
				\hfill\par
				(1)全序集是分配格.
				\par
				(2)按偏序的乘积排列的分配格的乘积,是分配格.
				\par
				(3)分配格的内部格是分配格.
				\par
				(4)$E$为分配格,$a$是$E$的不可约元素,则$a\leq sup(x, y)\Rightarrow a\leq x\text{或}a\leq y$.
			\end{cor}
			证明:
			\par
			(1)	根据定义可证.
			\par
			(2)	根据定理\ref{theo66}可证.
			\par
			(3)	根据定义可证.
			\par
			(4)	由于$inf(z, sup(x, y))=z$,故$sup(inf(x, z), inf(y, z))=z$,因此$inf(x, z)=z\text{或}\\inf(y, z)=z$,得证.

			\begin{de}
				\textbf{互补格(ensemble réticulé relativement complémenté),补(complément relatif)}
				\par
				令$E$为格,并且有最小元$a$,如果对任意$x\in E$、$y\in E$、$x\leq y$,均存在$x'\in E$,使$sup(x, x')=y$,$inf(x, x')=a$,则称$E$为互补格,$x'$称为$x$对$y$的补.
			\end{de}
			
			\begin{cor}\label{cor200}
				\hfill\par
				$E$为分配格和互补格,则对任意$x\in E$、$y\in E$、$x\neq y$,x对y的补唯一.
			\end{cor}
			证明:设$t$、$t'$均为$x$对$y$的补,根据补充定理\ref{cor198},$inf(t, t')=sup(t, t')$,根据补充定理\ref{cor184}(3),$t=t'$.

			\begin{de}
				\textbf{布尔网络(réseau booléien)}
				\par
				如果$E$是分配格,也是互补格,并且有最大元,则称$E$为布尔网络.				
			\end{de}

			\begin{cor}\label{cor201}
				\hfill\par
				$E$为布尔网络,最大元为$z$,对任意$x\in E$,令$x^*$为$x$对$z$的补,则$x\mapsto x^*$为$E$到按在$E$上的偏序关系的相反关系排序的偏序集的同构,并且$(x^*)^*=x$.
			\end{cor}
			证明:根据定义可证$(x^*)^*=x$.对任意$x\leq y$,$sup(y, inf(x^*, y^*))=sup(y, x^*)$.根据定理\ref{theo63},$sup(y, x^*)\geq sup(x, x^*)$,故$sup(y, x^*)=z$,因此$sup(y, inf(x^*, y^*))=z$,又因为$inf(y, inf(x^*, y^*))=a$,因此$inf(x^*, y^*)=y^*$,所以$x^*\geq y^*$,得证.
									
			\begin{cor}\label{cor202}
				\hfill\par
				对任意集合$A$,按包含关系排序的$\mathcal{P}(A)$,是布尔网络.
			\end{cor}
			证明:根据定义可证.
			
			\begin{cor}\label{cor203}
				\hfill\par
				$E$为布尔网络,且为完备格,$(x_i)_{i\in I}$为$E$的元素族,求证:$inf(y, sup(x_i)_{i\in I})=\\sup(inf(y, x_i))_{i\in I}$.
			\end{cor}
			证明:
			\par
			令$E$的最大元为$z$,对任意$x\in E$,令$x^*$为$x$对$z$的补.则:
			\par
			$inf(y, sup(y^*, sup(inf(y, x_i))_{i\in I}))= sup(inf(y, x_i))_{i\in I})$,
			\par
			$inf(y, sup(y^*, sup(x_i)_{i\in I}))=inf(y, sup(x_i)_{i\in I})$.
			\par
			根据定理\ref{theo63},$inf(y, sup(y^*, sup(inf(y, x_i))_{i\in I}))\leq inf(y, sup(y^*, sup(x_i)_{i\in I}))$,
			\par
			因此$sup(inf(y, x_i))_{i\in I})\leq inf(y, sup(x_i)_{i\in I})$.
			\par
			同时,$inf(y, sup(x_i)_{i\in I})\leq y$,
			\par
			$sup(inf(y, x_i))_{i\in I}\leq y$.
			\par
			对任意$i\in I$,$inf(y, x_i)\leq sup(inf(y, x_i))_{i\in I})$,
			\par
			根据定理\ref{theo63},$sup(y, x_i)\leq sup(y^*, sup(inf(y, x_i))_{i\in I})$,
			\par
			因此$x_i\leq sup(y^*, sup(inf(y, x_i))_{i\in I})$,
			\par
			因此,$sup(x_i)_{i\in I}\leq sup(y^*, sup(inf(y, x_i))_{i\in I})$,
			\par
			所以$inf(y, sup(x_i)_{i\in I})\leq inf(y, sup(y^*, sup(inf(y, x_i))_{i\in I}))$,
			\par
			故$inf(y, sup(x_i)_{i\in I})\leq sup(inf(y, x_i))_{i\in I})$.
			
			\begin{cor}\label{cor204}
				\hfill\par
				令$E$为偏序集,$a$、$b$是$E$的元素,则$x\in E\text{与}a\leq x\text{与}x\leq b$、$x\in E\text{与}a<x\text{与}x\leq b$、$x\in E\text{与}a\leq x\text{与}x<b$、$x\in E\text{与}a<x\text{与}x<b$均为关于$x$的集合化公式.
			\end{cor}
			证明:根据证明规则\ref{C52}可证.

			\begin{de}
				\textbf{闭区间(interval fermé),左半开区间(intervalle semi-ouvert à gauche),右半开区间(intervalle semi-ouvert à droite),开区间(intervalle ouvert)}
				\par
				令$E$为偏序集,$a$、$b$是$E$的元素,$a\leq b$,则称$\{x|x\in E\text{与}a\leq x\text{与}x\leq b\}$为在$E$上以$a$为起点、以$b$为终点的闭区间,记作$[a, b]$;$\{x|x\in E\text{与}a<x\text{与}x\leq b\}$为在$E$上以$a$为起点、以$b$为终点的左半开区间,记作$]a, b]$;$\{x|x\in E\text{与}a\leq x\text{与}x<b\}$为在$E$上以$a$为起点、以$b$为终点的右半开区间,记作$[a, b[$;$\{x|x\in E\text{与}a<x\text{与}x<b\}$为在$E$上以$a$为起点、以$b$为终点的开区间,记作$]a, b[$.在没有歧义的情况下,也可以省略“在$E$上”字样.
			\end{de}
			
			\begin{de}
				\textbf{左无穷闭区间(interval fermé illimité à gauche),左无穷开区间\\(intervalle semi-ouvert à gauche),右无穷闭区间(interval fermé illimité à \\droite),右无穷开区间(intervalle semi-ouvert à droite)}
				\par
				令$E$为偏序集,$a$是$E$的元素,则称$\{x|x\in E\text{与}x\leq a\}$为在$E$上以$a$为终点的左无穷闭区间,记作$]\gets, a]$;$\{x|x\in E\text{与}x<a\}$为在$E$上以$a$为终点的左无穷开区间,记作$]\gets, a[$;$\{x|x\in E\text{与}a\leq x\}$为在$E$上以$a$为起点的右无穷闭区间,记作$[a, \to [$;$\{x|x\in E\text{与}a<x\}$为在$E$上以$a$为起点的右无穷开区间,记作$]a, \to [$.在没有歧义的情况下,也可以省略“在E上”字样.
			\end{de}
			
			\begin{de}
				\textbf{区间(interval)}
				\par
				左半开区间、右半开区间、闭区间、开区间、左无穷闭区间、左无穷开区间、右无穷闭区间、右无穷开区间统称为区间.
			\end{de}
						
			\begin{theo}\label{theo71}
				\hfill\par
				格的任何两个区间的交集和并集,都是区间.
			\end{theo}
			证明:根据定义可证.

			\begin{de}
				\textbf{无间隙的偏序集(ensemble ordonné sans trou)}
				\par
				$E$为偏序集,且至少有一对可比较的不同元素,并且,对$E$的任何一对可比较的元素$x$、$y$,如果$x<y$,则$]x, y[\neq \varnothing$,则称$E$为无间隙的.				
			\end{de}

			\begin{de}
				\textbf{离散的偏序集(ensemble ordonné dispersé)}
				\par
				$E$为偏序集,如果$E$的任何偏序子集,都不是无间隙的,则称$E$为离散的.
			\end{de}
			
			\begin{cor}\label{cor205}
				\hfill\par
				离散集合的任何偏序子集都是离散的.
			\end{cor}
			证明:根据定义可证.

			\begin{de}
				\textbf{开集(ensemble ouvert),正则开集(ensemble ouvert régulier)}
				\par
				$E$为偏序集,如果$E$的子集$U$满足对任意$x\in U$,$[x, \to [\subset U$,则称$U$为开集;如果$U$为开集,且不存在开集$V$,满足$U\neq V$且$U$是$V$的共尾子集,则称$U$为正则开集.
			\end{de}
			
			\begin{cor}\label{cor206}
				\hfill\par
				(1)开集族的并集仍是开集.
				\par
				(2)$E$为偏序集,对任意$x\in E$,$[x, \to [$为开集.
			\end{cor}
			证明:根据定义可证.
			
			\begin{cor}\label{cor207}
				\hfill\par
				$E$为偏序集,则:
				\par
				(1)对任意开集$U\subset E$,存在唯一的正则开集$\tilde{U}$,使$U$为$\tilde{U}$的共尾子集.
				\par
				(2)$U$为开集,$x\in E$,如果对任意$y\geq x$,均存在$z\in U$使$y\leq z$,则$x\in \tilde{U}$.
				\par
				(3)$\mathcal{P}(E)$按包含关系排序,则映射$U\mapsto \tilde{U}$为单增映射.
				\par
				(4)$\mathcal{P}(E)$按包含关系排序,则映射$U\mapsto \tilde{U}$为闭包.
				\par
				(5)对开集$U$、$V$,如果$U\cap V=\varnothing$,则$\tilde{U}\cap \tilde{V}=\varnothing$.
			\end{cor}
			证明:
			\par
			(1)令$K=\{V|(V\text{为开集})\text{与}(U\neq V)\text{与}(U\text{是}V\text{的共尾子集})\}$,$\tilde{U}=\bigcup\limits_{X\in K}X$,根据补充定理\ref{cor206}(1),$\tilde{U}$为开集;根据定义可证,$\tilde{U}$为正则开集.
			\par
			同时,如果存在正则开集$U'$也满足条件,则$U'\subset \tilde{U}$,根据正则开集的定义,$U'=\tilde{U}$,$\tilde{U}$的唯一性得证.
			\par
			(2)由于$U\cup [x, \to [$也是开集并且$U$是其共尾子集,根据补充定理\ref{cor207}(1)的证明过程,$U\cup [x, \to [\subset \tilde{U}$,得证.
			\par
			(3)设$U\subset V$,$x\in \tilde{U}$,对任意$y\geq x$,由于$y\in \tilde{U}$,故存在$z\geq y$且$z\in U$,因此$z\in V$,根据补充定理\ref{cor207}(2),$x\in \tilde{V}$,因此映射$U\mapsto \tilde{U}$为单增映射.
			\par
			(4)根据补充定理\ref{cor207}(1)、补充定理\ref{cor207}(3)可证.
			\par
			(5)$U\mapsto \tilde{U}$为单增映射.若存在$a\in \tilde{U}$且$a\in \tilde{V}$,则存在$x\geq a$且$x\in U$,因此$x\in \tilde{V}$,故存在$y\geq x$且$y\in V$,因此$y\in U$,矛盾,得证.
			
			\begin{cor}\label{cor208}
				\hfill\par
				$E$为偏序集,$U$、$V$为正则开集,则$U\cap V$为正则开集.
			\end{cor}
			证明:令$\tilde{X}$为正则开集且使$U$为$\tilde{X}$的共尾子集,$f$为映射$X\mapsto \tilde{X}$.
			\par
			根据补充定理\ref{cor207}(3),$f(U\cap V)\subset f(U)$,$f(U\cap V)\subset f(V)$,故$f(U\cap V)\subset f(U)$.根据补充定理\ref{cor207}(1),$f(U)=U$、$f(V)=V$,因此$f(U\cap V)\subset U\cap V$.又因为$U\cap V\subset f(U\cap V)$,得证.
			
			\begin{cor}\label{cor209}
				\hfill\par
				$E$为偏序集,令$R(E)$为$E$的正则开子集集合,并按包含关系排序:
				\par
				(1)$\varnothing$为$R(E)$的最小元,$E$为$R(E)$的最大元.
				\par
				(2)$R(E)$为完备格.
				\par
				(3)$R(E)$为互补格.
				\par
				(4)对任意$U\in R(E)$、$V\in R(E)$,$U\cap V\in R(E)$,且$inf(U, V)= U\cap V$.
				\par
				(5)$R(E)$为分配格,
				\par
				(6)$R(E)$布尔网络.
				\par
				(7)当且仅当$E$非空且为右方有向集时,$R(E)$为仅有两个元素的结合.
			\end{cor}
			证明:对于$E$的子集$X$,令$\tilde{X}$表示正则开集,并且$X$为其共尾子集:
			\par
			(1)根据定义可证.
			\par
			(2)对$E$的任意子集族,令其并集为$F$,则$\tilde{F}$为其上界,假设存在$H \in R(E)$、$F\subset H$且$H\subset \tilde{F}$,则$H=\tilde{H}$,并且,根据补充定理\ref{cor207}(3),$\tilde(F)\subset \tilde{H}$,故$H=\tilde{F}$,即$\tilde{F}$是其最小上界,故$E$为完备格.
			\par
			(3)设$X\in R(E)$,$Y\in R(E)$,令$Z=\{x|x\in Y\text{与}(\forall y)(y\in X\Rightarrow (\text{非}x\leq y))\}$,则$Z$为开集,根据补充定理\ref{cor207}(2),$Y\subset X\cup \tilde{Z}$,同时,$X\cap Z=\varnothing$,根据补充定理\ref{cor207}(5),$X\cap \tilde{Z}=\varnothing$.又因为$Z\subset Y$,故$\tilde{Z}\subset Y$,又$X\subset Y$,因此$Y=X\cup \tilde{Z}$,即$\tilde{Z}$是$X$对$Y$的补,因此$E$为互补格.
			\par
			(4)根据补充定理\ref{cor208}可证.
			\par
			(5)设$X\in R(E)$,$Y\in R(E)$,$Z\in R(E)$,令$x\in Z\cap sup(X, Y)$,故存在$y\geq x$使$y\in X\cap Y$,同时$y\in Z$,又因为$Z\cap (X\cup Y)\subset X\cup(Y\cap Z)$,故$y\in sup(X, Y\cap Z)$,根据补充定理\ref{cor198},$E$为分配格.
			\par
			(6)根据定义可证.
			\par
			(7)如果$E$不是右方有向集,则存在$x$、$y$,使$[x, \to [\cap [y, \to [=\varnothing$,根据补充定理\ref{cor207}(5),$R(E)$至少有两个非空集元素;反过来,如果$E$是右方有向集,设$U\in R(E)$,且$U\neq \varnothing$.令$x\in U$,对任意$y\in E$,对任意$z\geq y$,存在$t$为$\{x, z\}$的上界,则$t\in U$,根据补充定理\ref{cor207}(2),$y\in U$,故$U=E$,因此$R(E)=\{\varnothing, E\}$.

			\begin{de}
				\textbf{区间到正则开集的规范映射(application canonique de interval dans ensemble ouvert régulier)}
				\par
				$E$为偏序集,令$R(E)$为$E$的正则开子集集合,$R_0(E)=R(E)-\{\varnothing\}$.对任意$x\in E$,令$r(x)$为正则开集,且$[x, \to [$是其共尾子集,则$r$称为$E$到$R_0(E)$的规范映射.
			\end{de}
			
			\begin{cor}\label{cor210}
				\hfill\par
				$E$为偏序集,令$R(E)$为$E$的正则开子集集合,$R_0(E)=R(E)-\{\varnothing\}$,并按包含关系的相反关系排序.$r$为$E$到$R_0(E)$的规范映射,则$r$为单增映射,并且$r(E)$的像是$R_0(E)$的共尾子集.
			\end{cor}
			证明:根据补充定理\ref{cor207}(3),$r$为单增映射.对任意$X\in R_0(E)$,$x\in X$,$r(x)\subset X$,故$r(E)$的像是$R_0(E)$的共尾子集.			
			
			\begin{cor}\label{cor211}
				\hfill\par
				$E$为偏序集,令$R(E)$为$E$的正则开子集集合,$R_0(E)=R(E)-\{\varnothing\}$,并按包含关系的相反关系排序.$r$为$E$到$R_0(E)$的规范映射,则:
				\par
				(1)$x\in E$、$y\in E$,则当且仅当对任意$z\geq y$且$z\in E$,$\{x, z\}$在$E$上均有上界时,$y\in r(x)$.
				\par
				(2)$x\in E$、$y\in E$,则当且仅当存在$z\geq y$且$z\in E$,使$[x, \to [\cap[z, \to [=\varnothing$时,$y\notin r(x)$.
				\par
				(3)$x\in E$、$y\in E$,如果$x\in r(y)$、$y\in r(x)$,则$r(x)=r(y)$.
			\end{cor}
			证明:
			\par
			(1)	如果$y\in r(x)$,则$[y, \to [\subset r(x)$,由于$[x, \to [$是$r(x)$的共尾子集,故对任意$z\geq y$,$\{x, z\}$在$E$上均有上界;反过来,如果对任意$z\geq y$,$\{x, z\}$在$E$上均有上界,根据补充定理\ref{cor207}(2),$y\in r(x)$.
			\par
			(2)	根据补充定理\ref{cor211}(1)可证.
			\par
			(3)	如果$x\in r(y)$、$y\in r(x)$,则$[x, \to [\subset r(y)$,$[y, \to [\subset r(x)$,根据补充定理\ref{cor207}(3),$r(y)\subset r(x)$,$r(x)\subset r(y)$,得证.

			\begin{de}
				\textbf{右方无向集(ensemble afiltrant à droite)}
				\par
				$E$为偏序集,令$R(E)$为$E$的正则开子集集合,$R_0(E)=R(E)-\{\varnothing\}$.对任意$x\in E$,令$r(x)$为区间$[x, \to [$相应的正则开集.如果$r$为单射,则称$E$为右方无向集.
			\end{de}
			
			\begin{cor}\label{cor212}
				\textbf{偏序集为右方无向集的条件}
				\par
				$E$为偏序集,当且仅当同时满足下列两个条件时,E为右方无向集:
				\par
				第一,对任意$x\in E$、$y\in E$、$x<y$,均存在$z\in E$、$x<z$,使$[y, \to [\cap[z, \to [=\varnothing$;
				\par
				第二,令$x$、$y$是$E$的一对不可比较的元素,那么,或者存在$x'\geq x$,使$[x', \to [\cap [y, \to [=\varnothing$,或者存在$y'\geq y$,使$[x, \to [\cap [y', \to [=\varnothing$.
			\end{cor}
			证明:
			\par
			令$R(E)$为$E$的正则开子集集合,$R_0(E)=R(E)-\{\varnothing\}$.$r$为$E$到$R_0(E)$的规范映射,
			\par
			根据补充定理\ref{cor211}(2),两个条件等价于$x\notin r(y)\text{或}y\notin r(x)$.
			如果$r$为单射,即当$x\neq y$时,$r(x)\neq r(y)$,根据补充定理\ref{cor211}(3),$x\notin r(y)\text{或}y\notin r(x)$.
			反过来,当$x\neq y$时,如果$x\notin r(y)\text{或}y\notin r(x)$,则$r(x)\neq r(y)$,得证.

			\begin{de}
				\textbf{右方分叉集(ensemble fourchu à droite)}
				\par
				$E$为偏序集,如果对任意$x\in E$,均存在$y\in E$、$z\in E$并且$x\leq y$、$x\leq z$,使$[y \to [\cap [z, \to [=\varnothing$,在则称$E$为右方分叉集.
			\end{de}
			
			\begin{cor}\label{cor213}
				\hfill\par
				(1)右方分叉集没有极大元.
				\par
				(2)E为右方分叉集,$x\in E$,则存在$y\in E$、$z\in E$,使$x<y$、$x<z$且$y$和$z$不可比较.
				\par
				(2)没有极大元的右方无向集,是右方分叉集.
			\end{cor}
			证明:
			\par
			(1)	根据定义可证.
			\par
			(2)	根据定义可证.
			\par
			(3)	根据补充定理\ref{cor212}可证.
			
			\begin{de}
				\textbf{右方分支集(ensemble ramifié à droite),完全右方分支集(ensemble complément ramifié à droite)}
				\par
				$E$为偏序集,如果对任意$x\in E$、$y\in E$、$x<y$,均存在$z\in E$、$x<z$,使$y$和$z$是不可比较的,则称$E$为右方分支集.没有极大元的右方分支集,称为完全右方分支集.
			\end{de}
			
			\begin{cor}\label{cor214}
				\hfill\par
				右方无向集都是右方分支集.
			\end{cor}
			证明:根据补充定理\ref{cor212}可证.
			
			\begin{exer}\label{exer77}
				\hfill\par
				$E$为偏序集,且至少有一对可比较的不同元素,令$R$为$x\in E\text{与}y\in E\text{与}x<y$,求证:R满足偏序关系前两个条件,不满足第三个条件.
			\end{exer}
			证明:由于$E$的偏序具有传递性,因此$x\leq y\text{与}y\leq z\Rightarrow x\leq z$.如果$x<y$、$y<z$且$x=z$,则$x=y$,矛盾,故$R$具有传递性;由于$x<y\text{与}y<x$为假,故第二式为真.但$E$的两个不同的可比较的元素$x$、$y$若满足$x<y$,而$x<x$、$y<y$为假,故第三式为假.
			
			\begin{exer}\label{exer78}
				\hfill\par
				(1)$E$为预序集,$x\leq y$为在$E$上的预序关系,令$R$为公式$X\in E/S\text{与}Y\in E/S\text{与}\\((\forall x)(x\in X\Rightarrow (\exists y)(y\in Y\text{与}x\leq y)))$.求证:$R$为关于$X$、$Y$在$E/S$上的预序关系.
				\par
				(2)	$f$为$E$到$E/S$的规范映射,求证:
				\par
				对任意$E/S$的商预序集到预序集F的映射$g$,如果$g\circ f$为单增映射,则$g$为单增映射.
				\par
				当且仅当$S$满足下列条件时,$f$为单增映射:
				\par
				$(x\leq y\text{与}x\equiv x'(mod S))\Rightarrow(\exists y')(y'\in E\text{与}y\equiv y'(mod S)\text{与}x'\leq y')$.
				\par
				如果$x\leq y$关于$x$同$S$相容,则$S$同$x\leq y$弱相容.
				\par
				(3)	$E_1$、$E_2$为预序集,令$S_1$为公式$pr_1z=pr_1z'$,求证:$S_1$在$z$、$t$上同在$E_1\times E_2$上的预序关系的乘积$z\leq t$弱相容;并且,令$h_1$为$E_1\times E_2$到$(E_1\times E_2)/S_1$的规范映射,$pr_1=f_1\circ h_1$,则$f_1$是$(E_1\times E_2)/S_1$到$E_1$的同构.
				\par
				(4)$E$为偏序集,$x\leq y$ 为在$E$上的偏序关系,$S$为在$E$上的等价关系,令$R$为公式$X\in E/S\text{与}Y\in E/S\text{与}((\forall x)(x\in X\Rightarrow (\exists y)(y\in Y\text{与}x\leq y)))$,且$x\leq y\text{与}y\leq z\text{与}x\equiv z(mod S)\Rightarrow x\equiv y(mod S)$.求证:$R$为关于$X$、$Y$在$E/S$上的偏序关系.
				\par
				(5)给出元素数目为$4$的全序集$E$以及在$E$上的等价关系$S$,使$E/S$为一个偏序集,但“$(x\leq y\text{与}x\equiv x'(mod S))\Rightarrow(\exists y')(y'\in E\text{与}y\equiv y'(mod S)\text{与}x'\leq y')$”、 “$x\leq y\text{与}y\leq z\text{与}x\equiv z(mod S)\Rightarrow x\equiv y(mod S)$”都不成立.
				\par
				(6)$E$、$F$为偏序集,$f$为$E$到$F$的单增函数,$S$为公式$f(x)=f(y)\text{与}x\in E\text{与}y\in E$,则$x\leq y\text{与}y\leq z\text{与}x\equiv z(mod S)\Rightarrow x\equiv y(mod S)$.
				\par
				令$R$为公式$X\in E/S\text{与}Y\in E/S\text{与}((\forall x)(x\in X\Rightarrow (\exists y)(y\in Y\text{与}x\leq y)))$,$E/S$为按$R$排序的偏序集.当且仅当$x'\in E\text{与}x\leq y\text{与}f(x)=f(x')\Rightarrow (\exists y')(y'\in E\text{与}x'\leq y'\text{与}f(y)=f(y'))$时,$(x\leq y\text{与}x\equiv x'(mod S))\Rightarrow(\exists y')(y'\in E\text{与}y\equiv y'(mod S)\text{与}x'\leq y')$.
				\par
				令$f=g\circ h$,$h$为$E$到$E/S$的规范映射,则当且仅当下列两个条件同时成立时,$g$为$E/S$\\到$f\langle E \rangle $的同构:
				\par
				第一,$x'\in E\text{与}x\leq y\text{与}f(x)=f(x')\Rightarrow (\exists y')(y'\in E\text{与}x'\leq y'\text{与}f(y)=f(y'))$;
				\par
				第二,$x\in E\text{与}y\in E\text{与}f(x)\leq f(y)\Rightarrow(\exists x')(\exists y')(f(x)=f(x')\text{与}f(y)=f(y')\text{与}x' \leq y'$.
			\end{exer}
			证明:
			\par
			(1)	即补充证明规则\ref{Ccor81}(1).
			\par
			(2)	即补充证明规则\ref{Ccor82}、补充证明规则\ref{Ccor83}.
			\par
			(3)	根据定义可证$S_1$在$z$、$t$上同在$E_1\times E_2$上的预序关系的乘积$z\leq t$弱相容,根据补充证明规则\ref{Ccor82}(1)可证$f_1$是单增映射,即$X\leq Y\Rightarrow f_1(X)\leq f_1(Y)$.
			\par
			反过来,如果$f_1(X)\leq f_1(Y)$,对任意$x\in X$,设$y\in Y$,令$y'=(pr_1y, pr_2x)$,则$y'\in Y$,且$pr_1x\leq pr_1y'$,故$x\leq y'$,因此$X\leq Y$.
			\par
			(4)	即补充证明规则\ref{Ccor81}(2).
			\par
			(5)	设$a$、$b$、$c$、$d$互不相等,令$E=\{a, b, c, d\}$,$S=\{(a, b), (b, a), (a, d), (d, a), (b, d), \\(d, b), (a, a), (b, b), (c, c), (d, d)\}$,在$E$上的偏序的图为$\{(a, b), (a, c), (a, d), (b, c), (b, d), (a, a), \\(b, b), (c, c), (d, d)\}$,则符合要求.
			\par
			(6)	第一部分、第二部分根据定义可证.
			\par
			对于第三部分:
			\par
			如果$g$为$E/S$到$f\langle E\rangle$的同构,则当$X\in E/s$、$Y\in E/s$时,$X\leq Y\Leftrightarrow g(X)\leq g(Y)$.对任意E的元素$x$、$y$、$x'$,如果$x\leq y$以及$x\equiv x'(mod S)$,则$g(h(x))\leq g(h(y))$,故$h(x)\leq h(y)$,因此$h(x')\leq h(y)$,因此存在$y'\in h(y)$且$x'\leq y'$;同时,如果$f(x)\leq f(y)$,则$g(h(x))\leq g(h(y))$,故对任意$x'\in h(x)$,存在$y'\in h(y)$且$x'\leq y'$.
			\par
			反过来,如果两个条件成立,若$X\leq Y$,则对任意$x\in X$,存在$y\in Y$,使$x\leq y$,故$g(h(x))\leq g(h(y))$,因此$g(X)\leq g(Y)$;如果$g(X)\leq g(Y)$,则对任意$x\in X$、$y\in Y$,有$X=h(x)$、$Y=h(y)$,故$f(x)\leq f(y)$,因此,存在$x'$、$y'$,使$f(x)=f(x')$、$f(y)=f(y')$且$x'\leq y'$,故存在$y''\in Y$使$x\leq y''$,因此$X\leq Y$,得证.
			
			\begin{exer}\label{exer79}
				\hfill\par
				$I$为偏序集,$(E_i)_{i\in I}$为偏序集非空族:
				\par
				(1)令$F$为集族$(E_i)_{i\in I}$的序数和,$G=\{(x, y)|(x, y)\text{为有序对}\text{与}x\in F\text{与}y\in F\text{与}(pr_2x<pr_2y\text{或}(pr_2x=pr_2y\text{与}\text{在}E_{pr_2x}\text{上}pr_1x\leq pr_1y))\}$,求证:
				\par
				G为在F上的偏序.
				\par
				令S为公式$x\in F\text{与}y\in F\text{与}pr_2x=pr_2y$,则:
				\par
				第一,$(x\leq y\text{与}x\equiv x'(mod S))\Rightarrow(\exists y')(y'\in E\text{与}y\equiv y'(mod S)\text{与}x'\leq y')$;
				\par
				第二,$x\leq y\text{与}y\leq z\text{与}x\equiv z(mod S)\Rightarrow x\equiv y(mod S)$.
				\par
				$E/S$的商偏序集,同构于$I$.
				\par
				(2)$L$为偏序集,$I$为偏序集族$(J_l)_{l\in L}$的序数和,令$F_l$为偏序集族$(E_i){i\in J_l}$的序数和,求证:集族$(E_i)_{i\in I}$的序数和同构于集族$(F_l)l\in L$的序数和.但令$I$为全序集$\{1, 2\}$,$E_1$、$E_2$为偏序集,$F_2=E_1$,$F_1=E_2$,偏序集族$(_Ei)_{i\in I}$的序数和不同构于偏序集族$(F_i)_{i\in I}$的序数和.
				(1)求证:当且仅当$I$为右方有向集且对$I$的任意极大元$i$,并且$E_i$均为右方有向集时,$F$为右方有向集.
				\par
				(2)求证:当且仅当$I$为全序集,且对任意$i\in I$,$E_i$均为全序集时,$F$为全序集.
				\par
				(3)求证:当且仅当满足下列条件时,F为格:
				\par
				第一,$I$为格,并且,对任意$i\in I$、$j\in I$,如果$i$和$j$是不可比较的,则$E_{sup(i, j)}$有最小元,$E_{inf(i, j)}$有最大元;
				\par
				第二,对任意$i\in I$,如果$x\in E_i$,$y\in E_i$,且$\{x, y\}$在$E_i$上有上界(或下界),则$\{x, y\}$在\\$E_i$上有最小上界(或最大下界);
				\par
				第三,对任意$i\in I$,如果$x\in E_i$,$y\in E_i$,且$\{x, y\}$在$E_i$上没有上界(或下界),则$\{k|k\\\in I\text{与}k>i\}$(或$\{k|k\in I\text{与}k<i\}$)有最小元(或最大元)$j$,且$E_j$有最大元(或最小元).
			\end{exer}
			证明:
			\par
			(1)第一部分即补充定理\ref{cor167}.
			\par
			第二部分根据定义可证.
			\par
			第三部分:令$g$为映射$x\mapsto E_x\times{x}(x\in I)$,则$g$为双射,且$g^{-1}$为$E/S$到$I$的同构,
			\par
			(2)前一部分即补充定理\ref{cor169}.设$a$、$b$、$c$互不相等,$E_1=\{a\}$,$E_2=\{b, c\}$,均按偏序关系$x=y$排序,则偏序集族$(E_i)_{i\in I}$的序数和不同构于偏序集族$(F_i)_{i\in I}$的序数和.
			\par
			(3)	即补充定理\ref{cor192}(1).
			\par
			(4)	即补充定理\ref{cor192}(2).
			\par
			(5)	即补充定理\ref{cor192}(3).
			
			\begin{exer}\label{exer80}
				\hfill\par
				$E$为偏序集,$(E_i)_{i\in I}$为$E$的划分,并且均为$E$关于公式$(x=y\text{或}x\text{和}y\text{是不可比较的})$的连通分量:
				\par
				(1)求证:如果$i\leq k$,$x\in E_i$,$y\in E_k$,则$x$和$y$是可比较的;并且,如果$x\leq y$,$y'\in E_k$,$y\neq y'$,则$x\leq y'$.
				\par
				(2)令$S$为公式$x\in F\text{与}y\in F\text{与}(\exists i)(i\in I\text{与}x\in Ei\text{与}y\in E_i)$,求证:$x\leq y$关于$x$、$y$同$S$相容,且$E/S$的商预序集为全序集.
				\par
				(3)$F$、$G$为全序集,$E$为$F$和$G$两个全序集的乘积,那么,$E$的连通分量是什么?
			\end{exer}
			证明:
			\par
			(1)	根据定义可证$x$和$y$是可比较的;如果$y'<x$,则$y'\leq y$,又因为$y\in E_k$、$y'\in E_k$,故$y=y'$,矛盾,因此,$x\leq y'$.
			\par
			(2)	根据习题\ref{exer80}(1)可证$x\leq y$关于$y$同$S$相容,同理可证$x\leq y$关于$x$同$S$相容,根据习题\ref{exer80}(1)和定义,可证$E/S$的商预序集为全序集.
			\par
			(3)	如果$F$、$G$分别有最小元$a$、$b$和最大元$c$、$d$,则$E$的连通分量为$\{(a,b)\}$、$\{(c, d)\}$、$E-\{(a, b), (c, d)\}$;如果$E$、$F$至少有一个没有最小元,并且至少有一个没有最大元,则$E$的连通分量为$E$;如果$E$、$F$至少有一个没有最小元,分别有最大元$a$、$b$,或者至少有一个没有最大元,分别有最小元$a$、$b$,则E的连通分量为$\{(a,b)\}$、$E-\{(a, b)\}$.
			\par
			注:习题\ref{exer80}中的概念“联通分量”,涉及尚未介绍的“自然数”知识.
			
			\begin{exer}\label{exer81}
				\hfill\par
				$F=\{X|X\text{为}E\text{的自由子集}\}$,求证:
				\par
				(1)$X\in F\text{与}Y\in F\text{与}(\forall x)(x\in X\Rightarrow (\exists y)(y\in Y\text{与}x\leq y))$为关于$X$、$Y$在$F$上的偏序关系;
				\par
				(2)$F$按(1)的偏序关系排序,则$x\mapsto \{x\}$为$E$到$F$的子集的同构;
				\par
				(3)$F$按(1)的偏序关系排序,则如果$X\in F$、$Y\in F$、$X\subset Y$,则$X\leq Y$;
				\par
				(4)$F$按(1)的偏序关系排序,则当且仅当$E$为全序集、$E$到$F$存在同构时,F为全序集.
			\end{exer}
			证明:即补充定理\ref{cor193}.
			
			\begin{exer}\label{exer82}
				\hfill\par
				$E$、$F$为偏序集,令$\mathcal{A}(E, F)=\{X|X\in FE\text{与}((X, E, F)\text{为单增函数})\}$,并为按$f\in \mathcal{A}(E, F)\text{与}g\in \mathcal{A}(E, F)\text{与}(\forall x)(x\in E\Rightarrow f(x)\leq g(x))$排序的关于$f$、$g$的偏序集:
				\par
				(1)$E$、$F$、$G$均为偏序集,$F\times G$为按$x\in F\times G\text{与}y\in F\times G\text{与}pr_1x\leq pr_1y\text{与}pr_2x\leq pr_2y$排序的关于$x$、$y$的偏序集,求证:$\mathcal{A}(E, F\times G)$同构于偏序集$\mathcal{A}(E, F)$和$\mathcal{A}(E, G)$的积.
				\par
				(2)$E$、$F$、$G$均为偏序集,$E\times F$为按$x\in E\times F\text{与}y\in E\times F\text{与}pr_1x\leq pr_1y\text{与}pr_2x\leq pr_2y$排序的关于$x$、$y$的偏序集,求证:$\mathcal{A}(E\times F, G)$同构于$\mathcal{A}(E, \mathcal{A}(F, G))$.
				\par
				(3)$E\neq \varnothing$,求证:当且仅当$F$为格时,$\mathcal{A}(E, F)$为格.
				\par
				(4)$E\neq \varnothing$,$F\neq \varnothing$,求证:当且仅当满足下列条件之一时,$\mathcal{A}(E, F)$为全序集:第一,$F$的元素数目为$1$;第二,$E$的元素数目为$1$且$F$为全序集;第三,$E$为全序集,$F$的元素数目为$2$并且为全序集.
			\end{exer}
			证明:
			\par
			(1)映射$X\mapsto (x\mapsto pr_1(X, E, F\times G)(x)\text{的图}, x\mapsto pr_2(X, E, F\times G)(x)\text{的图})$,为其同构,得证.
			\par
			(2)映射$X\mapsto (x\mapsto (y\mapsto (X, E\times F, G)(x, y)\text{的图})\text{的图})$ ,为其同构,得证.
			\par
			(3)类似补充定理\ref{cor197}可证.
			\par
			(4)充分性根据定义可证.
			\par
			必要性:$\mathcal{A}(E, F)$为全序集,则$F$为全序集.如果$F$的元素数目大于$2$,$E$的元素数目大于$1$,设$x\in E$,$y\in E$,且$x\neq y$,$\{a, b, c\}\in F$,且$a<b$,$b<c$,则对任意$y\in E$,令$f(x)=b$,对任意$z\leq x$,令$g(z)=a$,对任意$u\geq y$,令$g(u)=c$,对其他$v\in E$,令$g(v)=b$,则$f$和$g$是不可比较的,矛盾,故$E$的元素数目为$1$;
			\par
			如果F的元素数目为$2$,设$\{a, b\}\in F$,且$a<b$,如果$x\in E$,$y\in E$,且$x$和$y$不可比较,则对任意$z\leq x$,令$f(z)=a$,对其他$v\in E$,令$f(v)=b$,对任意$z\leq y$,令$g(z)=a$,对其他$v\in E$,令$g(v)=b$,则$f$和$g$是不可比较的,矛盾,故$E$为全序集.必要性得证.
			\par
			注:习题\ref{exer82}中的概念“元素数目”,涉及尚未介绍的知识.	
			
			\begin{exer}\label{exer83}
				\hfill\par
				$E$为偏序集,$F$为元素数目不少于$2$的偏序集,求证:当且仅当$E$为关于公式\\“$(x=y\text{或}x\text{和}y\text{是不可比较的})$”的连通分量时,下列公式为真:
				\par
				$(f\text{为}E\text{到}F\text{的映射})\text{与}(f\text{是单增映射})\text{与}(f\text{是单减映射})\Rightarrow(f\text{是常数映射})$.
				\par
				特别是,当$E$为右方有向集或左方有向集时,满足上述条件.
			\end{exer}
			证明:根据定义可证.
			\par
			注:习题\ref{exer83}中的概念“联通分量”,涉及尚未介绍的“自然数”知识.
			
			\begin{exer}\label{exer84}
				\hfill\par
				$E$、$F$为偏序集,$f$为$E$到$F$的单增映射,$g$为$F$到$E$的单增映射,$A=\{x|x\in E\text{与}g(f(x))\\=x\}$,$B=\{y|y\in F\text{与}f(g(y))=y\}$,求证:$A$同构于$B$.
			\end{exer}
			证明:当$x\in A$时,$f(x)=f(g(f(x)))$,故$f(x)\in B$,因此,令$f'$为$f$通过$A$和$B$导出的函数,类似的,可以令$g'$为$g$通过$B$和$A$导出的函数.则$f'$和$g'$为单增映射,$f'\circ g'=Id_B$,$g'\circ f'=Id_A$,根据定理\ref{theo20},$f'$和$g'$均为双射,得证.
			
			\begin{exer}\label{exer85}
				\hfill\par
				$E$为格,$I$、$J$均为有限集合,对任意$i\in I$、$j\in J$,$(x_{i,j})_{(i, j)\in I\times J}$为$E$的元素族,求证:$sup(inf(x_{i,j})_{i\in I})_{j\in J}\leq inf(sup(x_{i,j})_{j\in J})_{i\in I}$.
			\end{exer}
			证明:
			\par
			$E$为格,$I$、$J$均为有限集合,故$inf(x_{i,j})_{i\in I}$、$sup(x_{i,j})_{j\in J}$存在,$sup(inf(x_{i,j})_{i\in I})_{j\in J}$、\\$inf(sup(x_{i,j})_{j\in J})_{i\in I}$也存在.
			\par
			对任意$i\in I$、$j\in J$,$(inf(x_{i,j})_{i\in I}\leq x_{i,j}$,根据定理\ref{theo63},对任意$i\in I$,\\$sup(inf(x_{i,j})_{i\in I})_{j\in J}\leq sup(x_{i,j})_{j\in J}$,得证.
			\par
			注:习题\ref{exer85}涉及尚未介绍的“有限集合”知识.
						
			\begin{exer}\label{exer86}
				\hfill\par
				$E$、$F$为格,$f$为$E$到$F$的映射,求证:当且仅当对任意$x\in E$、$y\in E$均有$f(inf(x, y))\leq inf(f(x), f(y))$时,$f$为单增函数.$N$为自然数集,$N\times N$为偏序集$N$和$N$的乘积,并试给出$N\times N$到$N$的单增映射$f$,且存在$(x, y)$,使$f(inf(x, y))=inf(f(x), f(y))$不成立.
			\end{exer}
			证明;
			\par
			必要性:如果$x\leq y$,则$f(x)\leq f(y)$,故$f(inf(x, y))\leq inf(f(x), f(y))$.
			\par
			充分性:如果$x\leq y$,则$f(x)\leq inf(f(x), f(y))$,故$f(x)\leq f(y)$.
			\par
			不成立的例子:令$f$为$(x, y)\mapsto x+y$,$x=(0, 1)$,$y=(1, 0)$,则$f(inf(x, y))=0$,$inf(f(x), f(y))=1$.
			\par
			注:习题\ref{exer86}涉及尚未介绍的“自然数”知识.
			
			\begin{exer}\label{exer87}
				\hfill\par
				(1)求证:如果偏序集$E$的任何子集在$E$上都有最小上界,则$E$为完备格.
				\par
				(2)求证:当且仅当各偏序集都是完备格时,偏序集的积是完备格.
				\par
				(3)求证:当且仅当集族$(E_i)_{i\in I}$满足下列条件时,其序数和为完备格:
				\par
				第一,$I$为完备格;
				\par
				第二,对任意$J\subset I$,如果$J$没有最大元,令$d=sup\ J$,则$E_d$有最小元;
				\par
				第三,对任意$i\in I$和任意$E_i$的子集,如果在$E_i$上有上界,则有最小上界;
				\par
				第四,对任意$i\in I$,如果$E_i$没有最大元,则$\{x|x>i\text{与}x\in I\}$有最小元$a$,并且$E_a$有最小元,
				\par
				(4)$E$、$F$为偏序集,令$A(E, F)=\{X|X\in FE\text{与}((X, E, F)\text{为单增函数})\}$,并为按$f\in A(E, F)\text{与}g\in A(E, F)\text{与}(\forall x)(x\in E\Rightarrow f(x)\leq g(x))$排序的关于$f$、$g$的偏序集,求证:当且仅当$F$为完备格时,$A(E, F)$为完备格.
			\end{exer}
			证明:
			\par
			(1)	即补充定理\ref{cor194}.
			\par
			(2)	即补充定理\ref{cor195}.
			\par
			(3)	即补充定理\ref{cor196}.
			\par
			(4)	即补充定理\ref{cor197}.
			
			\begin{exer}\label{exer88}
				\hfill\par
				$E$为$A$到$A$的映射集合,$F=\{X|X\subset A\text{与}f(X)\subset X\text{与}f\in E\}$,且为按包含关系排序的偏序集,则$F$为完备格.
			\end{exer}
			证明:$F$的任何子集的最小上界,是其并集,得证.
			
			\begin{exer}\label{exer89}
				\hfill\par
				$E$为偏序集,$f$为闭包;令$F$为$f$的不动点集合:
				\par
				(1)求证:
				\par
				对任意$x\in E$,令$F_x=\{y|y\in F\text{与}x\leq y\}$,则其有最小元$f(x$).
				\par
				反过来,如果$G\subset E$,对任意$x\in E$,$\{y|y\in G\text{与}x\leq y\}$均有最小元$g(x)$,则$g$为闭包,且$G$为$g$的不动点集合.
				\par
				(2)$E$为完备格,求证:$F$的任何非空子集在$E$上的最大下界属于$F$.
				\par
				(3)如果$E$为格,求证:对任意$x\in E$、$y\in E$,$f(sup(x, y))=sup(f(x), f(y))$.
			\end{exer}
			证明:
			\par
			(1)$f(x)\geq x$,因此$f(x)\in F_x$;设$y\in F_x$,则$x\leq y$,故$f(x)\leq f(y)$,$y\geq f(x)$,故$f(x)$为最小元.反过来,根据定义可证对任意$x\in E$ ,$g(x)\geq x$,$g(x)\in G$,故$g(g(x))=g(x)$,同时,当$x\leq y$时,$\{z|z\in G\text{与}y\leq z\}\subset \{z|z\in G\text{与}x\leq z\}$,故$g(x)\leq g(y)$,因此,$G$为闭包.另外,对任意$x\in G$,$x=g(x)$,同时,对任意$x=g(x)$,$x\in G$,得证.
			\par
			(2)如果该子集的最大下界$x$不属于$F$,则$f(x)>x$,且$f(x)$也是该子集的下界,矛盾.
			\par
			(3)设$z=sup(x, y)$,对任意$u$,如果使$f(x)\leq u$,$f(y)\leq u$,则$x\leq f(u)$,$y\leq f(u)$,故$z\leq f(u)$,因此$f(z)\leq u$,所以$f(z)=sup(f(x), f(y))$,得证.
			
			\begin{exer}\label{exer90}
				\hfill\par
				$R\subset A\times B$,对$X\subset A$、$Y\subset B$,令映射$r(X)=\{y|y\in B\text{与}(\forall x)(x\in X\Rightarrow (x, y)\in R)\}$,$s(Y)=\{x|x\in A\text{与}(\forall y)(y\in Y\Rightarrow (x, y)\in R)\}$,$\mathcal{P}(A)$、$\mathcal{P}(B)$为按包含关系排序的偏序集,求证:$r$、$s$为单减映射,并且映射$s\circ r$、$r\circ s$均为闭包.
			\end{exer}
			证明:根据定义可证,$r$、$s$为单减映射,因此,$s\circ r$、$r\circ s$为单增映射.对任意$x\in X$,令$Y=r(X)$,则对任意$y\in Y$,$(x, y)\in R$,故$x\in s(r(X))$,因此$s(r(X))\geq X$,同理$r(s(Y))\geq Y$.因此$r(s(r(X)))\geq r(X)$,同时,由于$r$为单减函数,故$r(s(r(X)))\leq r(X)$,因此$r(s(r(X)))=r(X)$,故$s(r(s(r(X))))=s(r(X))$,同理$r(s(r(s(Y))))=r(s(Y))$,根据定义,$s\circ r$、$r\circ s$均为闭包.
			
			\begin{exer}\label{exer91}
				\hfill\par
				(1)$E$为偏序集,对任意$X\subset E$,令$r(X)=\{A|A\text{为}X\text{在}E\text{上的上界}\}$,$s(X)=\\\{A|A\text{为}X\text{在}E\text{上的下界}\}$,映射$i$为$x\mapsto s(\{x\})$.
				\par
				求证:
				\par
				令$E'=\{X|X\subset E\text{与}X=s(r(X))\}$,并按包含关系排序,则$E'$是完备格.
				\par
				并且,映射$i$是$E$到$E'$的一个子集的同构.
				\par
				同时,对任意$x_i\in E$,如果$\{x_i\}$有最小上界$a$,则在$E'$上,$i(a)$是$\{i(\{x_i\})\}$的最小上界.
				\par
				(2)求证:对任意$X\subset E$,$s(r(X))$是$i(X)$在$E'$上的最小上界.同时,确定对$E$到完备格$F$的任意单增映射$f$,是否存在唯一的$E'$到$F$的单增映射$f'$,使$f=f'\circ i$,并且,对$E'$的任意子集$Z$,均有$f'(sup\ Z) =sup(f'(Z))$.
				\par
				(3)	如果$E$为全序集,求证:$E'$为全序集.
			\end{exer}
			证明:
			\par
			(1)令$f=s\circ r$,根据定义,$f$是闭包.因此,对$E'$的子集$I$,令$J=\bigcup\limits_{X\in I}X$,$f(J)$为$I$的最小上界.根据补充定理\ref{cor194},$E'$是完备格.
			\par
			根据定义,$s\circ r$、$r\circ s$为增函数,$r$为减函数,故$s(r(s(\{x\})))\geq s(\{x\})$,$s(r(s(\{x\})))\leq s(\{x\})$,所以$s(r(s(\{x\})))=s(\{x\})$,因此映射$i$是$E$到$E'$的子集$i(E)$的双射.同时,如果$x\leq y$,则$s(\{x\})\leq s(\{y\})$,故$i$是$E$到$i(E)$的同构.根据定义可证,$i(a)是\{i(\{xi\})\}$的最小上界.
			\par
			(2)根据定义,$s(r(\{x\}))=s(\{x\})$,由于$s\circ r$为增函数,因此,对任意$x\in X$,$s(r(X))\\\geq s(\{x\})$.如果对任意$x\in X$,$Y\geq s(\{x\})$且$s(r(Y)=Y$,则$Y\geq \bigcup\limits_{x\in X}s(\{x\})$,因此$Y\geq s(r(\bigcup\limits_{x\in X}s(\{x\})))$,又因为$r(\bigcup\limits_{x\in X}s(\{x\}))=r(X)$,故$s(r(X))=(r(\bigcup\limits_{x\in X}s(\{x\})))$,因此$Y\geq s(r(X))$,故$s(r(X))$是$i(X)$在$E'$上的最小上界.
			\par
			设a、b、c互不相等,令E=\{a, b, c\},按(\{a, a\}, \{b, b\}, \{c, c\}, \{a, c\}, \{b, c\})排序,F=\{a, b, c\},按(\{a, a\}, \{b, b\}, \{c, c\}, \{a, c\}, \{b, c\}, \{a, b\})排序,映射$f=Id_E$,$Z=\{\{a\}, \{b\}\}$,则$f'(sup\ Z)=c$,$sup(f'(Z))=b$,故为反例.
			\par
			(3)	根据定义可证.
			\par
			注:原书习题\ref{exer91}(2)后半部分是假命题.
			
			\begin{exer}\label{exer92}
				\hfill\par
				$E$为格:
				\par
				(1)	求证:如果以下两个条件之中任何一个成立,则对任意$x\in E$、$y\in E$、$z\in E$,均有$sup(inf(x, y), inf(y, z), inf(z, x))=inf(sup(x, y), sup(y, z), sup(z, x))$:
				\par
				第一,对任意$x\in E$、$y\in E$、$z\in E$,$sup(x, inf(y, z))=inf(sup(x, y), sup(x, z))$;
				\par
				第二,对任意$x\in E$、$y\in E$、$z\in E$,$inf(x, sup(y, z))=sup(inf(x, y), inf(x, z))$.
				\par
				(2)	求证:如果对任意$x\in E$、$y\in E$、$z\in E$,均有$sup(inf(x, y), inf(y, z), inf(z, x))\\=inf(sup(x, y), sup(y, z), sup(z, x))$,则当$x\geq z$时,$sup(z, inf(x, y))=inf(x, sup(y, z))$,并且,(1)当中的两个条件都成立.即对任意$x\in E$、$y\in E$、$z\in E$,$sup(x, inf(y, z))=inf(sup(x, y), sup(x, z))$、$inf(x, sup(y, z))=sup(inf(x, y), inf(x, z))$、\\$sup(inf(x, y), inf(y, z), inf(z, x))=inf(sup(x, y), sup(y, z), sup(z, x))$都等价.
				\par
				(3)	求证:下列两个条件中的任何一个,均为$E$为分配格的充分必要条件:
				\par
				第一,对任意$x\in E$、$y\in E$、$z\in E$,$inf(z, sup(x, y))\leq sup(x, inf(y, z))$;
				\par
				第二,对任意$x\in E$、$y\in E$、$z\in E$,$inf(sup(x, y), sup(z, inf(x, y)))=\\sup(inf(x, y), inf(y, z), inf(z, x))$.
			\end{exer}
			证明:即补充定理\ref{exer198}.
			
			\begin{exer}\label{exer93}
				\hfill\par
				(1)求证:按包含关系排序的维数不小于$2$的向量空间的子空间集合$E$,是互补格.但对$x\in E$、$y\in E$、$x\neq y$,通常$x$对$y$的补不是唯一的.
				\par
				(2)$E$为分配格和互补格,求证:对任意$x\in E$、$y\in E$、$x\neq y$,$x$对$y$的补唯一.如果$E$为分配格和互补格,并且有最大元$z$,则称$E$为布尔网络.对任意$x\in E$,令$x^*$为$x$对$z$的补,则$x\mapsto x^*$为$E$到按在$E$上的偏序关系的相反关系排序的同构,并且$(x^*)^*=x$.对任意集合$A$,按包含关系排序的$\mathcal{P}(A)$,是布尔网络.
				\par
				(3)$E$为布尔网络,且为完备格,$(x_i)_{i\in I}$为$E$的元素族,求证:$inf(y, sup(x_i)_{i\in I})=sup(inf(y, x_i))_{i\in I}$.
			\end{exer}
			证明:
			\par
			(1)对$E$的任何两个元素,其交集为最大下界,其和为最小上界,故$E$为格.对任意$x\leq y$,将$x$的基扩展为$y$的基,则扩展的基,为$x$对$y$的补,故$E$为互补格.同时,当$y$不是$E$的最大元时,扩展的基和$x$的基线性组合可以得到不同的补,故$x'$存在且通常不是唯一的.
			\par
			(2)即补充定理\ref{cor200}、补充定理\ref{cor201}、补充定理\ref{cor202}.
			\par
			(3)即补充定理\ref{cor203}.
			\par
			注:习题\ref{exer93}(1)涉及尚未介绍的“向量空间”知识.
			
			\begin{exer}\label{exer94}
				\hfill\par
				$A$的元素数目不小于$3$$F=\{A|(\Delta_A\text{为}E\text{的划分})\}$,并按偏序关系“$X\in F\text{与}Y \in F\text{与}\\\Delta_X\text{为比}\Delta_Y\text{更细}$”排序.求证:$F$是完备格,不是分配格,但是互补格.
			\end{exer}
			证明:
			对$F$的任意子集$X$,$\{B|B\neq \varnothing\text{与}(\forall x)(\forall z)(x\in B\text{与}z\in X\Rightarrow (\exists t)(t\in z\text{与}x\in t\text{与}t\subset B))\}$为其最小上界,$\{C|C\neq \varnothing\text{与}(\forall x)(\forall y)(\forall z)(x\in C\text{与}y\in C\text{与}z\in X\Rightarrow (\exists t)(t\in z\text{与}x\in t\text{与}y\in t))\}$为其最大下界,故$F$为完备格.
			\par
			对F的任意子集$X$、$Y$,且$X\leq Y$,对任意$z\in Y$,令$H_z=\{x|x\in X\text{与}x\subset z\}$,$T_z=\bigcup\limits_{x\in H_z}\{\tau_u(u\in x)\}$.$D=\{u|(\exists z)(u=T_z\text{与}z\in Y)\}$,$E=A-\bigcup\limits_{x\in D}x$,$K=D\cup\{X|(\exists x)(X=\{x\}\text{与}x\in E)\}$,则$K$为$X$的补.
			\par
			$F$不是分配格的反例:
			\par
			令$x$、$y$、$z$互不相等,$A=\{x, y, z\}$,$X=\{\{x\}, \{y, z\}\}$,$Y=\{\{y\}, \{x, z\}\}$,$Z=\{\{z\}, \{x, y\}\}$,则$sup(X, inf(Y, Z))\neq inf(sup(X, Y), sup(X, Z))$.
			\par
			注:证明互补格时,$K$的含义是:从$Y$的每个集合对应的$X$的各集合中,各取一个元素,组成若干个新集合($D$),然后将$A$剩下的每个元素均作为一个集合.
						
			\begin{exer}\label{exer95}
				\hfill\par
				求证:当且仅当集族$(E_i)_{i\in I}$满足下列条件时,其序数和为无间隙的:
				\par
				第一,$I$至少有一对可比较的不同元素,或者存在$i\in I$使$E_i$至少有一对可比较的不同元素;
				\par
				第二,对任意$i\in I$,如果$E_i$至少有一对可比较的不同元素,则$E_i$为无间隙的;
				\par
				第三,$a\in I$、$b\in I$,$a<b$且$]a, b[$为空,则$E_a$没有极大元或者$E_b$没有极小元.
				\par
				特别是:
				\par
				对任意$i\in I$,$E_i$都是无间隙的,且没有极大元(或极小元),则其序数和是无间隙的.
				\par
				如果$I$是无间隙的,并且对任意$i\in I$,$E_i$都是无间隙的或者没有可比较的元素,则其序数和是无间隙的.
			\end{exer}
			证明:根据定义可证.
			
			\begin{exer}\label{exer96}
				\hfill\par
				(1)$E$是离散的,求证:对任意$x\in E$、$y\in E$、$x<y$,存在$x'\in E$、$y'\in E$,使$x\leq x'$、$x'<y'$、$y'\leq y$,且$]x', y'[=\varnothing$.并给出一个满足该条件,但不是离散的全序集.
				\par
				(2)$(E_i)_{i\in I}$为集族,其中$I\neq \varnothing$,对任意$i\in I$,$E_i\neq \varnothing$,求证:当且仅当$I$为离散的,并且对任意$i\in I$,$E_i$为离散的,该集族的序数和是离散的,
			\end{exer}
			证明:
			\par
			(1)根据定义可证.康托尔集是一个例子.
			\par
			(2)令$E$为其序数和.
			\par
			必要性:如果$E$是离散的,根据习题\ref{exer96}(1),对任意$i\in I$,$E_i\times \{i\}$是离散的,因此$E_i$是离散的;同时,令$F=\{x|(\exists i)(i\in I\text{与}x=\tau_y(y\in Ei))\}$,则$F$是离散的,因此$I$是离散的,必要性得证.
			\par
			充分性:对$E$的任何偏序子集$F$,令$J=pr_2F$,则$F$为集族$(E_i\cap F)_{i\in J}$的序数和,如果$F$是无间隙的,根据习题\ref{exer95}, $J$至少有一对可比较的不同元素,令其为$x$、$y$且$x<y$,则存在$x'\in E$、$y'\in E$,使$x\leq x'$、$x'<y'$、$y'\leq y$,且$]x', y'[=\varnothing$.同时,根据习题\ref{exer95},对任意$i\in J$,$E_i\cap F$没有可比较的不同元素,因此,对任意$i\in J$,$E_i\cap F$有极大元也有极小元,矛盾.
			\par
			注:习题\ref{exer96}(1)反例部分部分涉及尚未介绍的“康托尔集”知识.

			\begin{exer}\label{exer97}
				\hfill\par
				$E$为\text{非}空全序集,公式$S$为$([x, y]\text{是离散的})$,求证:$S$和在$E$上的偏序$x\leq y$弱相容;关于$S$的等价类是离散的;$E/S$的商偏序集存在,其或者是仅有一个元素的集合,或者是无间隙的.并且,$E$同构于某个离散集合族的序数和,其指标集或者是仅有一个元素的集合,或者是无间隙的.
			\end{exer}
			证明:根据补充定理\ref{cor169},$S$和在$E$上的偏序$x\leq y$弱相容.
			\par
			考虑关于$S$的某个等价类的任何偏序子集$F$,设$x\in F$,$y\in F$,则$[x, y]$是离散的,令$G=F\cap [x, y]$,根据习题\ref{exer96}(3),存在$x'\in G$、$y'\in G$,使$x\leq x'$、$x'<y'$、$y'\leq y$,且$]x', y'[=\varnothing$,因此,关于$S$的等价类是离散的.
			\par
			同时,$E$同构于集族$(X)_{X\in E/S}$的序数和.如果$E$是离散的,根据习题\ref{exer96}(1),对任意$x\in E$、$y\in E$,$[x, y]$都是离散的,故$E/S=\{E\}$,即是仅有一个元素的集合;如果$E$不是离散的,则集族$(X)_{X\in E/S}$的序数和也不是离散的,根据习题\ref{exer96}(4),$E/S$不是离散的,得证.
			
			\begin{exer}\label{exer98}
				\hfill\par
				(1)$E$为偏序集,求证:对任意开集$U\subset E$,存在唯一的正则开集$\tilde{U}$,使$U$为$\tilde{U}$的共尾子集,并且,映射$U\mapsto \tilde{U}$为单增映射.同时,对开集$U$、$V$,如果$U\cap V=\varnothing$,则$\tilde{U}\cap \tilde{V}=\varnothing$.
				\par
				(2)$E$为偏序集,令$R(E)$为$E$的正则开子集集合,并按包含关系排序,求证:$R(E)$为布尔网络,且为完备格.并且,当且仅当$E$非空且为右方有向集时,$R(E)$为仅有两个元素的结合.
				\par
				(3)令$F$为$E$的共尾子集,$R(E)$、$R(F)$均按包含关系排序,求证:映射$U\mapsto U\cap F$为$R(E)$到$R(F)$的同构.
				\par
				(4)$E_1$、$E_2$为偏序集,$E1\times E2$按偏序关系$(pr_1x\leq pr_1y)\text{与}(pr_2x\leq pr_2y)$排序,求证:$E_1\times E_2$的任意开集,均可表示为$U_1\times U_2$的形式,其中$U_1$、$U_2$分别为在$E_1$上的开集、在$E_2$上的开集.并确定$R(E1\times E2)$是否同构于$R(E1)\times R(E2)$.
			\end{exer}
			证明:
			\par
			(1)即补充定理\ref{cor207}(1)、补充定理\ref{cor207}(3)、补充定理\ref{cor207}(5).
			\par
			(2)即补充定理\ref{cor209}(2)、补充定理\ref{cor209}(6)、补充定理\ref{cor209}(7).
			\par
			(3)$U\in R(E)$,因此在$F$上$U\cap F$是开集,设$U\cap F$在$F$上相应的正则开集是$V$,对任意$x\in V$,显然$x\in F$,同时,对任意$y\geq $x,存在$z\in F$且$y\leq z$,因此,$z\in V$,因此存在$z\leq t$且$t\in U\cap F$,根据补充定理\ref{cor207}(2),$x\in \tilde{U}$,故$x\in U$,因此$x\in U\cap F$,所以$U\cap F$在$F$上是正则开集.
			\par
			对任意$V\in R(F)$,设$V$在$E$上相应的正则开集为$U$,则$U$是唯一的;同时,对任意$x\in U\cap F$,对任意$y\geq x$且$y\in F$,存在$z\in F$且$y\leq z$,故$z\in U\cap F$,故存在$t\in V$使$t\geq z$,根据补充定理\ref{cor207}(2),$x\in V$,因此,$U\cap F=V$.因此,$U\mapsto U\cap F$为$R(E)$到$R(F)$的双射.根据定义$U\subset V\Rightarrow U\cap F\subset V\cap F$;反过来,如果$U\cap F\subset V\cap F$,如果$x\in U$,对任意$y\geq x$,存在$z\in F$且$y\leq z$,因此$z\in V$,根据补充定理\ref{cor207}(2),$x\in V$,故$U\subset V$,综上,该映射为同构.
			\par
			(4)	对于$F\subset E_1\times E_2$,且$F$为开集,令$U_1=pr_1F$,$U_2=pr_2F$,根据定义可证$F=U_1\times U_2$且$U_1$、$U_2$均为开集.
			\par
			令$E_1=\{a\}$、$E_2=\{b\}$,则$R(E_1\times E_2)$不同构于$R(E_1)\times R(E_2)$.
			\par
			注:原书习题\ref{exer98}(4)后半部分是假命题.
			
			\begin{exer}\label{exer99}
				\hfill\par
				$E$为偏序集,令$R_(E)$为$E$的正则开子集集合,$R_0(E)=R(E)-\{\varnothing\}$,并按包含关系的相反关系排序.令为$E$到$R_0(E)$的规范映射:
				\par
				(1)	求证:$r$为单增映射,并且$r(E)$的像是$R_0(E)$的共尾子集.
				\par
				(2)	求证:$E$为偏序集,当且仅当同时满足下列两个条件时,E为右方无向集:
				\par
				第一,对任意$x\in E$、$y\in E$、$x<y$,均存在$z\in E$、$x<z$,使$[y, \to [\cap[z, \to [=\varnothing$;
				\par
				第二,令$x$、$y$是$E$的一对不可比较的元素,那么,或者存在$x'\geq x$,使$[x', \to [\cap [y, \to [=\varnothing$,或者存在$y'\geq y$,使$[x, \to [\cap [y', \to [=\varnothing$.
				\par
				(3)	求证:$R_0(E)$为右方无向集,并且$R_0(E)$到$R_0(R_0(E))$的规范映射为双射.
			\end{exer}
			证明:
			\par
			(1)	即补充定理\ref{cor210}.
			\par
			(2)	即补充定理\ref{cor212}.
			\par
			(3)	对任意$X\in R_0(E)$、$Y\in R_0(E)$,如果$Y\subset X$或者$Y$、$X$不可比较,根据补充定理\ref{cor209}(3),存在$Z\in R_0(E)$且$Z\subset X$、$Z\cap Y=\varnothing$;根据补充定理\ref{cor212},$R_0(E)$为右方无向集,并且$R_0(E)$到$R_0(R_0(E))$的规范映射为单射.对任意$U\in R_0(R_0(E))$,设其最小元为$X$,则对任意$Y\subset X$,存在$X\cap Y\subset U$,根据补充定理\ref{cor207}(2),$X\in U$,故$[X, \to [\subset U$,因此$r(X)=U$,得证.
			
			\begin{exer}\label{exer100}
				\hfill\par
				(1)求证:没有极大元的右方无向集,是右方分叉集.
				\par
				(2)	E为实数区间$[k2-n, (k+1)2-n]$($n$为自然数,$k$为整数且$k\in [0, 2^n-1]$)的集合,按包含关系的相反关系排列,求证:$E$为右方分叉集,并且没有极大元.
				\par
				(3)	给出一个右方分叉集,其不存在右方无向的共尾子集.
				\par
				(4)	给出偏序集,它不是右方分叉集,但存在右方分叉的共尾子集.
			\end{exer}
			证明:
			\par
			(1)	即补充定理\ref{cor213}(3).
			\par
			(2)	根据补充定理\ref{cor212}、补充定理\ref{cor213}(3)可证.
			\par
			(3)	设$E$为习题\ref{exer100}(2)所称的集合,$F$为不包含可数共尾子集的全序集,根据定义,$E\times F$为右方分叉集.同时,$E\times F$的任何共尾子集,都有元素$(x, a)$,$(x, b)$,故该共尾子集不是右方无向集.
			\par
			(4)	设$E$为习题\ref{exer100}(2)所称的集合,F为非空全序集,则偏序集族$(F)_{i\in E}$的序数和,不是右方分叉集,但令其序数和为$A$,$y\in F$,则$\{x|x\in A\text{与}pr_1x=y\}$为右方分叉的共尾子集.
			\par
			注:习题\ref{exer100}(2)、(3)、(4)涉及尚未介绍的“有理数”知识.

		\section{良序集(Ensembles bien ordonnés)}		
			\begin{metadef}
				\textbf{良序关系(relation de bon ordre),良序图(graphe d'un bon ordre)}
				\par
				包含$2$元特别符号$\in$ 、显式公理\ref{ex1}、显式公理\ref{ex2}、显式公理\ref{ex3}和公理模式\ref{Sch8}的等式理论$M$中,令$R$为关于$x$、$y$的偏序关系,并且对任意满足$E\neq \varnothing$并且$x\in E\Rightarrow (x|y)R$的集合$E$,$E$均为按偏序关系$R\text{与}x\in E\text{与}y\in E$排序的偏序集,并且有最小元,则称$R$为$x$、$y$之间的良序关系,在没有歧义的情况下简称为$R$为良序关系.良序关系生成的图称为良序图.
			\end{metadef}
			
			\begin{de}
				\textbf{良序(bien ordonné)}
				\par
				$F$为在$E$上的偏序,如果$y\in F\langle x \rangle $为$x$、$y$之间的良序关系,则称$F$为在$E$上的良序.
			\end{de}
			
			\begin{de}
				\textbf{良序集(ensemble bien ordonné),良序子集(partie bien ordonné)}
				\par
				任何非空子集都有最小元的偏序集,称为良序集.偏序集的偏序子集如果是良序集,则称为良序子集.
			\end{de}
						
			\begin{Ccor}\label{Ccor84}
				\hfill\par
				包含$2$元特别符号$\in$ 、显式公理\ref{ex1}、显式公理\ref{ex2}、显式公理\ref{ex3}和公理模式\ref{Sch8}的等式理论$M$中,如果$E$是按照偏序关系$R$排序的良序集,则$R$是良序关系;如果$E$是根据偏序$F$排序的良序集,则$F$是良序.
			\end{Ccor}
			证明:由于$E$的任何非空子集$E'$,都满足$x\in E'\Rightarrow (x|y)R$,且$E'$为按偏序关系$R\text{与}x\in E\text{与}y\in E$排序的偏序集,故$R$为偏序关系.进而,如果$E$是良序集,则$y\in F\langle x \rangle $为良序关系,$F$为良序.
						
			\begin{Ccor}\label{Ccor85}
				\hfill\par
				包含$2$元特别符号$\in$ 、显式公理\ref{ex1}、显式公理\ref{ex2}、显式公理\ref{ex3}和公理模式\ref{Sch8}的等式理论$M$中,如果$R$是关于$x$、$y$的良序关系,$E\neq \varnothing$并且$x\in E\Rightarrow (x|y)R$,则按$R\text{与}x\in E\text{与}y\in E$排序的$E$,为良序集.
			\end{Ccor}
			证明:根据定义可证.
			
			\begin{cor}\label{cor215}
				\hfill\par
				良序集是全序集.
			\end{cor}
			证明:根据定义可证.
			
			\begin{cor}\label{cor216}
				\hfill\par
				良序集的子集如果有上界,则有最小上界.
			\end{cor}
			证明:令良序集为$E$,其子集为$A$,$A$的上界集为$B$,则$B$的最小元为其最小上界.
			
			\begin{cor}\label{cor217}
				\hfill\par
				(1)良序集的偏序子集也是良序集.	
				\par
				(2)良序集是离散的.
				\par
				(3)当且仅当指标集和各偏序集均为良序集时,偏序集族的序数和为良序集.
			\end{cor}
			证明:
			\par
			(1)根据定义可证.
			\par
			(2)设$E$为良序集,$F$为其偏序子集,且有一对可比较的不同元素,则$F$有最小元$a$,$F-\{a\}$有最小元b,则区间$]a, b[=\varnothing$,得证.
			\par
			(3)根据定义可证.
			
			\begin{cor}\label{cor218}
				\hfill\par
				$\varnothing$、${x}$按其唯一的偏序排序得到的偏序集,都是良序集.
			\end{cor}
			证明:根据定义可证.

			\begin{de}
				\textbf{片段(segement)}
				\par
				$E$为偏序集,$S\subset E$,如果$x\in S\text{与}y\in E\text{与}y\leq x\Rightarrow y\in S$,则称$S$为$E$的片段.
			\end{de}
						
			\begin{cor}\label{cor219}
				\hfill\par
				(1)	偏序集是其自身的片段.
				\par
				(2)	偏序集的任何两个片段的交集和并集,都是其片段.
			\end{cor}
			证明:根据定义可证.
			
			\begin{cor}\label{cor220}
				\hfill\par
				令$E$为偏序集,$S$为$E$的片段,则$E$的偏序子集$S$的片段,也是$E$的片段.
			\end{cor}
			证明:根据定义可证.
			
			\begin{theo}\label{theo72}
				\hfill\par
				$E$为良序集,$S$为$E$的片段,$S\neq E$,则$(\exists a)(a\in E\text{与}S=]\gets, a[)$.
			\end{theo}
			证明:由于$E-S$非空,故令$a$为$E-S$的最小元,由于$a\notin S$,又因为$x\in S\text{与}a\leq x\Rightarrow a\in S$,故$x\geq a\Rightarrow x\notin S$,又因为$x\in E-S\Rightarrow x\geq a$,因此$E-S=[a, \to [$,得证.
			
			\begin{cor}\label{cor221}
				\textbf{偏序集的每个元素均确定一个以该元素为终点的片段}
				\par
				$E$为偏序集,$a\in E$,则$]\gets, a[$为$E$的片段.
			\end{cor}
			证明:假设$x\in ]\gets, a[$、$y\in E$、$y\leq x$,则$x\leq a$,故$y\leq a$,得证.
			
			\begin{de}
				\textbf{以元素为终点的片段(segment d'extrémité un élément)}
				\par
				$E$为良序集,$a\in E$,则$]\gets, a[$称为以$a$为终点的片段,记作$S_a$.
			\end{de}
			
			\begin{cor}\label{cor222}
				\hfill\par
				$E$为全序集,$A=\bigcup\limits_{x\in E}S_x$,如果$E$没有最大元,则$A=E$,如果$E$有最大元$b$,则$A=E-\{b\}$.
			\end{cor}
			证明:对任意$c\in E$,如果$c$不是$E$的最大元,则存在$d$,使$c<d$,因此$c\in S_d$,故$c\in A$;反过来,对任意$c\in A$,存在$S_d$,使$c<d$,故$c\in E$,并且$c$不是$E$的最大元,得证.
						
			\begin{cor}\label{cor223}
				\hfill\par
				$E$为良序集,$a\in E$,$b\in E$,$S_a=S_b$,则$a=b$.
			\end{cor}
			证明:如果$a<b$,则$a\notin S_a$,但$a\in S_b$,矛盾.同理$a>b$也矛盾,得证.
			
			\begin{cor}\label{cor224}
				\hfill\par
				映射$f$为偏序集$E$到偏序集$F$的同构,则:
				\par
				(1)如果$E$为良序集,则$F$为良序集.
				\par
				(2)如果$S$为$E$的片段,则$f\langle S\rangle$为$F$的片段.
			\end{cor}
			证明:
			(1)对于$F$的任何非空子集$B$,令$A=f^{-1}\langle B\rangle$,设$A$的最小元为$a$,由于对任意$x\in A$,$a\leq x$,故对任意$y\in B$,令$z=f^{-1}(y)$,则$f(a)\leq f(z)$,即$f(a)\leq y$,因此$f(a)$为$B$的最小元.得证.
			\par
			(2)如果$S=E$,则$f\langle S \rangle =F$,得证;
			\par
			如果$S\neq E$,根据定理\ref{theo72},令$S$为$] \gets, a[$,$a\in E$,则$f(x)<f(a)\Leftrightarrow x<a$,故$f\langle S \rangle $为$] \gets, f(a)[$,得证.
			
			\begin{cor}\label{cor225}
				\hfill\par
				(1)$E$为良序集,如果其片段$S_a$同构于$S_b$,则$a=b$.
				\par
				(2)$E$为良序集,如果$E$同构于其片段$S$,则$E=S$.
			\end{cor}
			证明:
			\par
			(1)	令$f$为$S_a$到$S_b$的同构,令$X=\{x|x\in S_a\text{与}x\neq f(x)\}$,如果$X\neq \varnothing$,则$X$有最小元$y$,因此$f(y)\neq y$.令$z=f^{-1}(y)$,则$z\neq y$,故$f(z)\neq z$,因此$z>y$,故$f(y)<y$,故$f(f(y))<f(y)$,因此$f(y)\in X$,矛盾,因此$X=\varnothing$,故对任意$x\in S_a$,$x=f(x)$,所以$S_a=S_b$,根据补充定理\ref{cor223},$a=b$.
			\par
			(2)	类似补充定理\ref{cor225}(1)可证.
			
			\begin{theo}\label{theo73}
				\hfill\par
				E为良序集:
				\par
				(1)偏序集$F=\{X|(X\text{为}E\text{的片段})\text{与}X\neq E\}$,并按包含关系排序,则映射$x\mapsto S_x$为$E$到$F$的同构.
				\par
				(2)$E^*=\{S|S\text{为}E\text{的片段}\}$,则$E^*$为按包含关系排序的良序集.
			\end{theo}
			证明:
			\par
			(1)$x<y\Rightarrow S_x\subset S_y$,因此$S_x\neq S_y$,根据定理\ref{theo69},映射$x\mapsto S_x$为$E$到$F$的同构.
			\par
			(2)$E^*=F\cup\{E\}$,根据补充定理\ref{cor224}(1),$F$的任意非空子集$G$均有最小元,进而,$G\cup\{E\}$也有同样的最小元,同时,$\{E\}$也有最小元$E$,得证.
						
			\begin{theo}\label{theo74}
				\hfill\par
				令$(X_a)_{a\in A}$为偏序集族,集合$\bigcup\limits_{a\in A}\{X_a\}\}$按包含关系排序,且为右方有向集.对任意$a\in A$、$b\in A$,如果$X_a\subset X_b$,则在$X_a$上的偏序,都是在$X_b$上的偏序在$X_a$上导出的偏序.令$E=\\\bigcup\limits_{a\in A}X_a$,则存在唯一的在$E$上的偏序,令$E$为按该偏序排序的偏序集,对任意$a\in A$, $X_a$都是偏序集$E$的偏序子集.
			\end{theo}
			证明:令$G_a$为在$X_a$上的偏序的图,如果$G$为在$E$上的偏序,且对任意$a\in A$时,该偏序在$X_a$上导出的偏序,等于$X_a$上的偏序,则对任意$a\in A$,$G_a\subset G$,因此$\bigcup\limits_{a\in A}G_a\subset G$;同时,由于该集族为右方有向集,故对任意$x\in E$、$y\in E$,存在$x\in X_a$,$y\in X_a$,因此$(x, y)\in G\Rightarrow (x, y)\in G_a$,故$G=\bigcup\limits_{a\in A}G_a$,唯一性成立.
			\par
			设$G=\bigcup\limits_{a\in A}G_a$,由于对任意$a\in A$、$b\in A$,$X_a\subset X_b\Rightarrow G_b\cap (X_a\times X_a)=G_a$,因此当$x\in G\cap(X_a\times X_a)$时,存在$c\in A$,使$x\in G_c$且$x\in X_a\times X_a$,因此$x\in G_c\cap(X_a\times X_a)$,故$x\in G_a$.反过来,当$x\in G_a$时,$x\in G$、$x\in X_a\times X_a$.故$a\in A$时,$G\cap(X_a\times X_a)=G_a$.
			\par
			同时,对$E$的任意三个元素$x$、$y$、$z$,均存在$a\in A$,使$\{x, y, z\}\subset X_a$,故G为偏序图.存在性成立.
						
			\begin{theo}\label{theo75}
				\hfill\par
				令$(X_i)_{i\in I}$为良序集族,对任意$i\in I$、$k\in I$,$X_i$和$X_k$其中都有一个是另一个的片段,令$E=\bigcup\limits_{i\in I}X_i$,则存在唯一的在$E$上的偏序满足下列条件:
				\par
				第一,$E$为良序集;
				第二,对任意$i\in I$, $X_i$都是$E$的偏序子集.
				\par
				同时,对任意$i\in I$,$X_i$的片段都是$E$的片段,并且,对任意$x\in X_i$,$X_i$的以$x$为终点的片段,是$E$的以$x$为终点的片段.反过来,$E$的片段或者是E,或者存在$i\in I$,使之为$X_i$的片段.
			\end{theo}
			证明:
			根据定理\ref{theo74},存在唯一$E$上的偏序.
			\par
			对$E$的任意非空子集$H$,存在$i\in I$,使$H\cap X_i\neq \varnothing$,设$a$是$H\cap X_i$的最小元,对任意$x\in H$,设$x\in X_k$,若$X_k\subset X_i$,则$a\leq x$,若$X_i\subset X_k$,且$x<a$,则$x\in ]\gets, a[$,根据补充定理\ref{cor220}、定理\ref{theo72},该区间是$X_i$的片段,故$x\in H\cap X_i$,矛盾,因此$a$是$H$的最小元,故$E$为按该偏序排序的良序集.
			\par
			如果$x\in X_i$,$y\in E$,则存在$k\in I$,使$x\in X_k$,$y\in X_k$.同时,根据补充定理\ref{cor220}、定理\ref{theo72},对任意$x\in X_i$,$X_i$的以$x$为终点的片段,是$E$的区间$]\gets, x[$,因此是$E$的以$x$为终点的片段.
			\par
			反过来,$E$的片段如果不是$E$,则存在$x\in E$,该片段为$]\gets, x[$,由于存在$X_i$使$x\in X_i$,故该区间也是$X_i$的片段.
			
			\begin{theo}\label{theo76}
				\hfill\par
				$E$是良序集,$F$的元素均为$E$的片段,并且,任何$F$的元素族的并集也是$F$的元素,同时,如果$S_x\in F$,则$S_x\cup\{x\}\in F$,则$F$是$E$的片段集合.
			\end{theo}
			证明:设有$E$的片段不属于$F$,根据定理\ref{theo73}(2),不属于$F$的$E$的片段的集合,有最小元$S$.如果$S$没有最大元,根据补充定理\ref{cor222},$S$是所有片段的并集,因此$S\in F$,矛盾;如果$S$有最大元$a$,则$S-\{a\}\in F$,则$S\in F$,同样矛盾.

			\begin{C}\label{C59}
				\textbf{超限归纳法}
				\par
				包含$2$元特别符号$\in$ 、显式公理\ref{ex1}、显式公理\ref{ex2}、显式公理\ref{ex3}和公理模式\ref{Sch8}的等式理论$M$中,$R$为公式,$x$不是常数,$E$为按良序关系$x\leq y$排序的良序集,如果$(x\in E\text{与}(\forall y)(y\in E\text{与}y<x\Rightarrow (y|x)R))\Rightarrow R$是$M$的定理,则$x\in E\Rightarrow R$是$M$的定理.
			\end{C}					
			证明:令$F$为$E$的满足$x\in S\Rightarrow R$的片段$S$的集合,因此$F$任何元素的并集也是$F$的元素,同时,如果$S_x\in F$,则$S_x\cup\{x\}\in F$,根据定理\ref{theo76}可证.

			\begin{C}\label{C60}
				\textbf{超限归纳法定义的映射的存在性和唯一性}
				\par
				包含$2$元特别符号$\in$ 、显式公理\ref{ex1}、显式公理\ref{ex2}、显式公理\ref{ex3}和公理模式\ref{Sch8}的等式理论$M$中,$E$为良序集,$u$为字母,$T$为项,则存在唯一的项$U$和$E$到$U$的满射$f$,使$(f^{(x)}|u)T=f(x)$,其中$f^{(x)}$为任意一个$S_x$到$f\langle S_x\rangle$的满射,并且在$S_x$上和$f$重合.
			\end{C}					
			证明:设$f$和$U$、$f'$和$U'$均满足条件,令$F$为使$f$和$f'$在$S$上重合的$E$的片段$S$的集合,因此,$F$任意若干个元素的并集都是$F$的元素,并且对于$F$的元素$S_x$,$f^{(x)}= {f'}^{(x)}$,$f(x)=(f^{(x)}|u)T$,$f'(x)= ({f'}^{(x)}|u)T$,因此,$S_x\cup\{x\}\in F$,根据定理\ref{theo76},$F$是$E$的片段集合,故$E\in F$.由于$U=f(E)$,$U'=f'(E)$,因此$f$和$f'$在$E$上重合,且$U=U'$.故唯一性成立.
			\par
			令$G$为存在满足条件的项和满射的$E$的片段的集合,对任意$S\in G$,根据唯一性,存在唯一的满足条件的项和满射,分别记作$f_S$和$U_S$.则对任意$S'\in G$、$S''\in G$,设$S'\subset S''$,$f_{S'}$和$f_{S''}$在$S'$上重合.根据定理\ref{theo32}(2),$G$的元素的并集也属于$G$.同时,令$S_x\in G$,由于$f_{S_x}(x)= ({f_{S_x}}^{(x)}|u)T$,根据定理\ref{theo33},$S_x\cup\{x\}\in F$,根据定理\ref{theo76}可证存在性.

			\begin{theo}\label{theo77}
				\hfill\par
				$G$为$\mathcal{P}(E)$的子集,$p$为$G$到$E$的映射,对任意$X\in G$,$p(X)\notin X$,则存在$E$的子集$M$和在$M$上的良序$F$,满足下列条件:
				\par
				第一,令$x\leq y$表示$y\in F\langle x \rangle $,$S_x$表示在$M$上的区间$]\gets, x[$,$(\forall x)(x\in M\Rightarrow S_x\in G\text{与}p(S_x)=x)$;
				\par
				第二,$M\notin G$.
			\end{theo}
			证明:
			\par
			令$K$为满足下列条件的图$H$的集合:
			\par
			第一,$H\subset E\times E$;
			\par
			第二,令$U=pr_1H$,$H$为在$U$上的良序图;
			\par
			第三,偏序关系$(x, y)\in H$记作$x\leq y$,令$S_x$为$U$的片段,则$(\forall x)(x\in U\Rightarrow S_x\in G\text{与}p(S_x)=x)$.
			\par
			设$H$、$H'$都是$K$的元素,相应的第一射影为$U$、$U'$,设$V=\{x|x\in U\cap U'\text{与}\\(U\text{的以}x\text{为终点的片段})=(U'\text{的以}x\text{为终点的片段})\}$,对任意$x\in V$、$y\in V$,若在$U$上$x\leq y$,则在$U'$上$x\leq y$,故$(H, U, U)$和$(H', U', U')$在$V$上导出的偏序相同.如果$x\in V$,$y\leq x$,$y\in U$,则$y\in U'$,同时,在$U$上$z\leq y$等价于在$U'$上$z\leq y$,故$y\in V$,因此$V$是$U$的片段,同理,$V$也是$U'$的片段.
			\par
			如果$V\neq U$且$V\neq U'$,设$x$为$U-V$的最小元,$x'$为$U'-V$的最小元,则在$U$上$V=S_x$,在$U'$上$V=S_{x'}$,且$p(S_x)=x$,$p(S_{x'})=x'$,故$x=x'$,根据$V$的定义,$x\in V$,矛盾.
			\par
			因此,$V=U$或$V=U'$.不妨设$V=U$,故$U\subset U'$,且$U$是$U'$的片段.
			\par
			令$M=\bigcup\limits_{H\subset K}H$,根据定理\ref{theo75},存在唯一的偏序,使$M$为按该偏序排序的良序集,且任意$K$的元素$H$,都是$M$的偏序子集.并且,对任意$x\in M$,存在$H\in K$,使$x\in H$,则$(\forall x)(x\in U\Rightarrow S_x\in H\text{与}p(S_x)=x)$;由于在$H$上的$S_x$和在$M$上的$S_x$相等,故$(\forall x)(x\in M\Rightarrow S_x\in G\text{与}p(S_x)=x)$.
			\par
			如果$M\in G$,令$a=p(M)$,则$a\notin M$,把$a$作为最大元加入$M$,则$M'$也是良序集,由于在$M'$上,$M=S_a$,因此对于上述在偏序集$M'$上延拓的偏序,相应的偏序图也是$K$的元素,矛盾.
			\par
			注:本定理的证明利用了并集,即:对于所有符合条件一的子集,证明这些子集彼此具有包含关系;然后,取所有子集的并集,证明其仍然符合条件一,并符合条件二.
			
			\begin{theo}\label{theo78}
				\textbf{策梅洛定理}
				\par
				在任何集合上均存在良序.
			\end{theo}
			证明:令$G=\mathcal{P}(E)-\{E\}$,对于$X\in G$,令$p(X)=\tau_x(x\in E-X)$,根据定理\ref{theo77},存在子集$M$以及在$M$上的良序使$M\notin G$,故$M=E$,得证.

			\begin{de}
				\textbf{归纳集(ensemble inductif)}
				\par
				如果偏序集的任何全序子集在该偏序集上都有上界,则称其为归纳集.
			\end{de}
			
			\begin{cor}\label{cor226}
				\hfill\par
				(1)$F\subset \mathcal{P}(E)$,并且$(\forall G)(G\subset F\text{与}G\text{为按包含关系排序的全序集}\Rightarrow \bigcup\limits_{X\in G}X\in F)$,则F是归纳集.
				\par
				(2)“$x\text{为}A\text{的子集到}B\text{的映射}$”,为$x$上的集合化公式.并且,令$F=\{x|x\text{为}A\text{的子集到}\\B\text{的映射}\}$,$R$为公式$(v\text{为}u\text{在}pr_1v\text{上的延拓})$,则$R$为关于$u$、$v$的偏序关系,并且按$R\text{与}u\in F\text{与}v\in F$排序的偏序集$F$,为归纳集.
			\end{cor}
			证明:
			\par
			(1)	根据定义可证.
			\par
			(2)	由于“$x\text{为}A\text{的子集到}B\text{的映射}\Rightarrow x\in A\times B\times A\times B$”,故“$x\text{为}A\text{的子集到}B\text{的映}\\\text{射}$”为$x$上的集合化公式.根据定义可证$R$为偏序关系.根据定理\ref{theo32}(2),$F$的任何全序子集,存在唯一映射,是各元素在所有元素的定义域的并集上的延拓,该映射即为上界,故$F$为归纳集.

			\begin{theo}\label{theo79}
				\hfill\par
				如果偏序集$E$的任何良序子集在$E$上都有上界,则$E$有极大元.
			\end{theo}
			证明:令$G$为$E$的有严格上界的子集的集合,$p$为$G$到$E$的映射,其中$p(x)=\tau_v(v\text{为}x\\\text{的严格上界})$,根据定理\ref{theo77},存在$E$的子集$M$和在$M$上的良序$F$,使$(\forall x)(x\in M\Rightarrow S_x\in G\text{与}p(S_x)=x)$,且$M$没有严格上界.
			\par
			在$M$上,$y\in F\langle X \rangle \text{与}x\neq y\Leftrightarrow x\in S_y$,同时,$p(S_y)=y$,因此,在$E$上,$y$为$S_y$的严格上界,			进而,在$E$上,$x<y$.
			\par
			又因为$M$是全序集,故良序$F$是在$E$上的偏序在$M$上的导出的偏序.
			\par
			由于$M$有上界,且$M$没有严格上界,因此该上界是$E$的极大元.

			\begin{theo}\label{theo80}
				\textbf{佐恩引理}
				\par
				归纳集有极大元.
			\end{theo}
			证明:根据定理\ref{theo79}可证.

			\begin{theo}\label{theo81}
				\hfill\par
				$E$为归纳集,$a\in E$,则存在$E$的极大元$m$,使$m\geq a$.
			\end{theo}
			证明:令$F=\{x|x\in E\text{与}x\geq a\}$,则$F$为归纳集.设$m$为$F$的极大元,则$m$同时是$E$的极大元,得证.

			\begin{theo}\label{theo82}
				\hfill\par
				$F$的元素都是$E$的子集,并且对$F$的任意子集$G$,只要$G$是按包含关系排序的全序集,$G$的所有元素的并集(或交集)都是$F$的元素,则$F$有极大元(或极小元).
			\end{theo}
			证明:对于并集的情况,根据补充定理\ref{cor226}(1),$F$为归纳集,根据定理\ref{theo80}可证.将$F$的所有全序子集变为按相反关系排序的全序集,则可证得交集的结论.
			
			\begin{cor}\label{cor227}
				\hfill\par
				$K=\{F|F\text{为在}E\text{上的偏序}\}$,并且按$(F\in K\text{与}F'\in K\text{与}F\text{是比}F'\text{更细的偏序})$排序,则$K$的极小元是在$E$上的全序.
			\end{cor}
			证明:令$H$为$K$的极小元,其图为$G$,设$a$、$b$是不可比较的,令$A=\{z|z\in E\text{与}z\leq a\}$,$B=\{z|z\in E\text{与}b\leq z\}$,因此$A\cap B=\varnothing$.令$G'=G\cup(A\times B)$,$H'$为偏序$(G', E, E)$.对任意$(x, y)\in G'$,$(y, z)\in G'$:如果$(x, y)\in G$,$(y, z)\in G$,则$(x, z)\in G$;如果$(x, y)\in A\times B$,$(y, z)\in G$,由于$(b, y)\in G$,故$(b, z)\in G$,因此$(x, z)\in A\times B$,故$(x, z)\in G'$; 如果$(x, y)\in G$,$(y, z)\in A\times B$,同理可证$(x, z)\in G'$.故对$G'$,传递性成立,根据定义可以证明其他两个条件也成立,因此$H'$是比$H$更细的偏序,矛盾.
			
			\begin{cor}\label{cor228}
				\hfill\par
				$F$为在$E$上的偏序,则$F$的图是所有不等于$F$且比$F$更细的全序的图的交集.
			\end{cor}
			证明:令$F$的图为$G$,对任意不等于$F$且比$F$更细的全序的图的交集的元素$(a, b)$,如果$(a, b)\notin G$,则$a\neq b$.按照补充定理\ref{cor227}的证明过程构造$F'$,使$(b, a)\in F'$,考虑$M=\{H|H\in K\text{与}F'\subset H\}$,对于按相反关系排序的偏序集,根据定理\ref{theo80},其有极大元$Q$,其同时为$K$的极小元.根据补充定理\ref{cor227},$Q$为全序.$(b, a)\in Q$,因此$(a, b)\in Q$,矛盾.
			
			\begin{cor}\label{cor229}
				\hfill\par
				任何偏序集同构于全序集族的乘积的一个子集.
			\end{cor}
			证明:对于按偏序$F$排序的偏序集$E$,令$F$的图为$G$,$M=\\\{H|H\neq F\text{与}(H\text{为比}F\text{更细的全序})\}$,根据补充定理\ref{cor228},$G=\bigcap\limits_{H\in M}(H\text{的图})$.根据定义,映射$x\mapsto (x)_{i\in M}(x\in E)$为$E$到$\prod\limits_{H\in M}E$的同构.
						
			\begin{cor}\label{cor230}
				\hfill\par
				对于任何偏序集,存在在该集合上的全序图,其偏序图是该全序图的子集.
			\end{cor}
			证明:设$E$为偏序集,其偏序图为$G$,$H=\{K|G\subset K\text{与}K\text{为在}E\text{上的偏序图}\}$,根据定理\ref{theo82},$H$有极大元$M$,如存在$x\in E$、$y\in E$且$(x, y)\notin M$,则令$M'=M\cup{(x, y)}\cup\{(a, b)|b=y\text{与}a\leq x\}\cup\{(a, b)| a=x\text{与}y\leq b\}$,则$M'\subset H$,矛盾.故$M$为全序图.

			\begin{theo}\label{theo83}
				\hfill\par
				$E$、$F$为良序集,$f$、$g$为$E$到$F$的两个单增映射,并且,$f(E)$是$F$的片段,$g$为严格单增映射,则对任意$x\in E$,$f(x)\leq g(x)$.
			\end{theo}
			证明:如果$\{x|x\in E\text{与}f(x)>g(x)\}\neq \varnothing$,设其最小元为$a$,令$x<a$,则$f(x) \leq g(x)$,$g(a)<f(a)$,$g(x)<g(a)$.由于$f(E)$是$F$的片段,故存在$z$,使$g(a)=f(z)$,则$f(z)<f(a)$,因此$z<a$,则$f(z)<f(a)$,$g(z)<g(a)$,故$f(z)<g(a)$,因此$f(z)<f(z)$,矛盾.
			
			\begin{theo}\label{theo84}
				\hfill\par
				$E$、$F$为良序集,则下列两个公式至少有一个为真:
				\par
				第一,存在唯一的$E$到$F$的片段的同构;
				\par
				第二,存在唯一的$F$到$E$的片段的同构.
			\end{theo}
			证明:令$G$为$E$的片段到$F$的片段的同构的集合,并为按“$v$是$u$的延拓”排序的偏序集.设$G$的全序子集为$H$,由于$H$各元素的定义域都是$E$的片段,故其并集$S$也是$E$的片段,根据定理\ref{theo32},存在以$S$为定义域的映射,是$H$各元素的延拓,故为$H$的最小上界,设其为$v$,则$v(S)$为$H$各元素值域的并集,因此也是$F$的片段.由于$H$为全序子集,因此对于$S$\\的任意两个元素$x<y$,存在$u\in H$,使$x$、$y$均为$u$的定义域的元素,故$u(x)<u(y)$,因此$v(x)<v(y)$,故$v$是$S$到$v(S)$的同构,因此$v\in G$,故$G$为归纳集,因此$G$有极大元.
			\par
			设$G$的极大元为$u$,定义域为$S$,如果$S\neq E$且$u(S)\neq F$,则存在$a\in E$、$b\in F$,使$S=]\gets, a[$,$u(S)= ]\gets, b[$,$u$为$S$到$u(S)$的延拓,将$u$延拓为$w$,其定义域为$] \gets, a]$,值域为$] \gets, b]$,其中$w(a)=b$,则$w\in G$,与$u$是$G$的极大元矛盾.
			\par
			故$S$=$E$或$u(S)$=$F$,存在性得证.
			\par
			根据定理\ref{theo83}可证唯一性.
			
			\begin{theo}\label{theo85}
				\hfill\par
				$E$为良序集,$E$到$E$的片段的唯一同构,是$E$到$E$的恒等映射.
			\end{theo}
			证明:设存在$E$到其片段$S$的同构,根据补充定理\ref{cor225}(2),$S=E$,根据定理\ref{theo84}可证.
						
			\begin{theo}\label{theo86}
				\hfill\par
				$E$、$F$为良序集,如果存在E到F的片段S的同构f,以及F到E的片段T的同构g,则S=E,T=F并且f和g互为反函数.
			\end{theo}
			证明:根据补充定理\ref{cor224}(2),$g(T)$为$E$的片段,故$g\circ f$为$E$到$E$的片段的同构,根据定理\ref{theo85},$S=E$,故$g\circ f=Id_E$,同理$f\circ g=Id_F$,$T=F$,得证.
			
			\begin{theo}\label{theo87}
				\hfill\par
				良序集的任何子集同构于该良序集的一个片段.
			\end{theo}
			证明:设$A$为良序集的子集,对于该良序集的片段$S_a$,如果$A$不同构于$E$的任何一个片段,则存在映射$g$,为$E$到$A$的某个片段$S_a$的同构,故$g(a)\in S_a$,且$g$为严格单增映射,令$f=Id_E$,根据定理\ref{theo83},$a\leq g(a)$,矛盾.
			
			\begin{de}
				\textbf{部分良序集(ensemble partiellement bien ordonné)}
				\par
				如果$E$的任意全序子集都是良序集,则称$E$为部分良序集.
			\end{de}
						
			\begin{cor}\label{cor231}
				\hfill\par
				(1)对任意偏序集$E$,存在$F\subset E$,$F$为部分良序集,并且是$E$的共尾子集.
				\par
				(2)对任意全序集$E$,存在$F\subset E$,$F$为良序集,并且是$E$的共尾子集.
			\end{cor}
			证明:
			\par
			(1)令$H=\{X|X\subset E\text{与}X\text{为部分良序集}\}$,$R$为公式$X\in H\text{与}Y\in H\text{与}X\subset Y\text{与}\\(\forall x)(\forall y)(x\in X\text{与}y\in Y-X\Rightarrow \text{非}(y\leq x))$,则$R$为在$\mathcal{P}(E)$上的偏序关系.
			\par
			对$H$的任意全序子集$K$,令$Z=\bigcup\limits_{X\in K}X$,对任意$X\in K$,$x\in X$,$y\in Z-X$,则存在$Y\in K$使$y\in Y$,因此$X\leq Y$,故$\text{非}(y\leq x)$,$X\leq Z$,因此$Z$为$K$的上界,故$H$为归纳集,根据定理\ref{theo80},$H$有极大元$M$.如果存在$y\in E$,对任意$x\in M$,均有$\text{非}(y\leq x)$,令$M'=M+\{x\}$,则$M'$为部分良序集,且$M<M'$,矛盾,故$M$为$E$的共尾子集.
			\par
			(2)	根据补充定理\ref{cor231}(1)可证.

			\begin{de}
				\textbf{关于函数的链(chaîne pour une fonction)}
				\par
				$E$为偏序集,$f$为$E$到$E$的映射,并且对任意$x\in E$,$f(x)\geq x$:令$H$为满足下列条件的$E$的子集$M$的集合:
				\par
				第一,$x\in M\Rightarrow f(x)\in M$;
				\par
				第二,如果$M$的非空子集在$E$上有最小上界$x$,则$x$是$M$的元素.
				\par
				令$K=\{X|X\in H\text{与}a\in X\}$,则称$\bigcap\limits_{X\in K}X$为$a$关于函数$f$的链,记作$C_a$,在没有歧义的情况下可以简称为$a$的链.
			\end{de}
			
			\begin{cor}\label{cor232}
				\hfill\par
				$E$为偏序集,$f$为$E$到$E$的映射,并且对任意$x\in E$,$f(x)\geq x$.令$H$为满足下列条件的$E$的子集$M$的集合:
				\par
				第一,$x\in M \Rightarrow f(x)\in M$;
				\par
				第二,如果$M$的非空子集在$E$上有最小上界$x$,则$x$是$M$的元素.
				\par
				则:
				\par
				(1)对任意$a\in E$,$C_a\in H$;
				\par
				(2)$E$的偏序子集$C_a$,为良序集;
				\par
				(3)如果$C_a$在$E$上有最小上界$b$,则$b\in C_a$并且$f(b)=b$.
			\end{cor}
			证明:
			\par
			(1)由于$E\in H$,故$H\neq \varnothing$,根据定义可证$C_a\in H$.
			\par
			(2)令$K=\{X|X\subset E\text{与}a\in X\text{与}(X\text{在}E\text{上有最小上界}m)\text{与}(m\notin X\text{或}f(m)>m)\}\cup\{\varnothing\}$,并定义$K$到$E$的映射$p$,其中:
			\par
			$p(\varnothing)=a$;
			\par
			如果$sup_EX\notin X$,则$p(X)= sup_EX$;
			\par
			如果$sup_EX\in X$,则$p(X)=f(sup_EX)$.
			\par
			根据定理\ref{theo77},存在$E$的子集$U$及在$U$上的良序$F$,使$U$满足:
			\par
			第一,在$U$上,$(\forall x)(x\in U\Rightarrow S_x\in K\text{与}p(S_x)=x)$;
			\par
			第二,$U\notin K$.
			\par
			在按$F$排序的$U$上,如果$y<x$,则$y\in S_x$,令$c=sup_ES_x$,则在$E$上$y\leq c$,又因为$c\leq p(S_x)$,故$c\leq x$,因此在$E$上$y\leq x$,由于$x\neq y$,因此$y<x$,故良序$F$是在$E$上的偏序在$U$上的导出的偏序.
			\par
			由于$U\not in K$,故$U\neq \varnothing$;同时,令$x\in U$,如果在$U$上,$S_x=\varnothing$,则$x=a$;如果$S_x\neq\varnothing$,则$a\in S_x$.综上有,$a\in U$.
			\par
			在$U$上,对任意$y\in U$:
			\par
			如果$y$为$U$的最大元,由于$a\in S_y$,故$a\in U$,而$U \notin K$,因此$f(y)=y$,故$f(y)\in U$;
			\par
			如果$y$不是$U$的最大元,令$z$为$\{x|x>y\text{与}x\in U\}$的最小元,则$y=sup_E(S_z)$,故$p(S_z)=f(y)$,因此$z=f(y)$,即$f(y)\in U$.
			\par
			综上,$U$满足第一个条件.
			\par
			对任意$U$的子集$V$,设$V$在$E$上有最小上界$m$:
			\par
			如果$\{y|y\geq m\text{与}y\in U\}=\varnothing$,则$m=sup\ U$,故$m\in U$;
			\par
			如果$\{y|y\geq m\text{与}y\in U\}\neq\varnothing$,设其最小元为$z$,如果$z\in V$,则$z=m$,且$z\in U$;如果$z\notin V$,则$V\subset S_z$,令$t$为$S_z$在$E$上的最小上界,则$m\leq t$,$t\leq z$,如果$t\in S_z$,则$t\in \{y|y\geq m\text{与}y\in U\}$,则$z=t$,与$t\in S_z$矛盾,故$t\notin S_z$,$p(S_z)=t$,因此$z=t$,如果存在$q\in S_z$,使$m<q$,由于$q<z$,与$z$为$\{y|y \geq m\text{与}y\in U\}$的最小元矛盾,因此$m$为$S_z$在$E$上的上界,故$t\leq m$,因此$t=m$,所以$m=z$,故$U$满足第二个条件.
			\par
			综上,$U\in H$,故$C_a\subset U$,因此$C_a$为良序集.
			\par
			(3)	根据定义可证.
			
			\begin{cor}\label{cor233}
				\textbf{布尔巴基-维特定理}
				\par
				$E$为归纳集,$f$为$E$到$E$的映射,并且对任意$x\in E$,$f(x)\geq x$,则存在$b\in E$,使$f(b)=b$.
			\end{cor}
			证明:根据补充定理\ref{cor232}可证.
						
			\begin{cor}\label{cor234}
				\textbf{字典式排序为偏序关系}
				\par
				令$(E_i)_{i\in I}$为良序集族,$I$为良序集,$E=\prod\limits_{i\in I}E_i$,对于$x\in E$、$y\in E$,且$x\neq y$,令$j$为集合$\{i|i\in I\text{与}pr_ix\neq pr_iy\}$的最小元,则公式$(x\in E\text{与}y\in E\text{与}(x\neq y\Rightarrow pr_jx<pr_jy))$为在$E$上的偏序关系.
			\end{cor}
			证明:
			\par
			令$R$为$x\in E\text{与}y\in E\text{与}pr_jx<pr_jy$,则$(y|x)R\Leftrightarrow x\in E$,$R\Rightarrow (y|x)R\text{与}(x|y)R$,$R\text{与}(y|z)(x|y)(z|x)R\Rightarrow x=y$.对于$R\text{与}(y|x)(z|y)R$,如果$x=y\text{或}y=z$,显然$(z|y)R$为真,如果$x\neq y$、$y\neq z$,设满足$pr_ix\neq pr_iy$的最小$i$为$j$,满足$pr_iy\neq pr_iz$的最小$i$为$k$,若$j<k$,则$pr_jx\neq pr_jz$,且$pr_jx<pr_jy$,$pr_iy=pr_iz$,因此$pr_jx<pr_jz$;若$j>k$,同理可得$pr_kx<pr_kz$;若$j=k$,则$pr_jx<pr_jy$,$pr_jy<pr_jz$,因此$pr_jx<pr_jz$.综上,得证.
			
			\begin{de}
				\textbf{字典式偏序关系(relation d'ordre lexicographique),字典式偏序(ordre \\lexicographique),字典式乘积(produit lexicographique)}
				\par
				令$(E_i)_{i\in I}$为良序集族,$I$为良序集,$E=\prod\limits_{i\in I}E_i$,对于$x\in E$、$y\in E$,且$x\neq y$,令$j$为集合$\{i|i\in I\text{与}pr_ix\neq pr_iy\}$的最小元,则公式$(x\in E\text{与}y\in E\text{与}(x\neq y\Rightarrow pr_jx<pr_jy))$称为在$E$上的字典式偏序关系,令该公式生成的图为$G$,则$(G, E, E)$称为在$E$上的字典式偏序;按该偏序排序的偏序集$E$,称为偏序集族$(E_i)_{i\in I}$的字典式乘积.
			\end{de}
						
			\begin{cor}\label{cor235}
				\hfill\par
				$I$为良序集,如果对任意$i\in I$,$E_i$为全序集,则$(E_i)_{i\in I}$的字典式乘积是全序集.
			\end{cor}
			证明:根据定义可证.
			
			\begin{cor}\label{cor236}
				\textbf{偏序的同构为等价关系}
				\par
				令$R$为公式$(F\text{为在}E\text{上的偏序})\text{与}(F'\text{为在}E'\text{上的偏序})\text{与}\\(\text{按}F\text{排序的}E\text{同构于在}F'\text{上排序的}E')$,则$R$为关于$F$、$F'$的等价关系.
			\end{cor}
			证明:根据定义可证.
			
			\begin{de}
				\textbf{偏序类(type d'ordre)}
				\par
				令$R$为公式$(F\text{为在}E\text{上的偏序})\text{与}(F'\text{为在}E'\text{上的偏序})\text{与}\\(\text{按}F\text{排序的}E\text{同构于在}F'\text{上排序的}E')$,则称$\tau_{F'}(R)$为$F$的偏序类,记作$Ord(F)$,在没有歧义的情况下也可以记作$Ord(E)$.
			\end{de}
			
			\begin{cor}\label{cor237}
				\hfill\par
				当且仅当两个偏序的偏序类相等时,这两个偏序集同构.
			\end{cor}
			证明:根据定义可证.
			
			\begin{cor}\label{cor238}
				\textbf{偏序类之间的预序关系}
				\par
				令$S$为公式$(X\text{为偏序类})\text{与}(Y\text{为偏序类})\text{与}(\exists A)(\exists B)\\((A\text{为按偏序类为}X\text{的偏序排序的偏序集})\text{与}(B\text{为按偏序类为}Y\text{的偏序排序的偏序集})\text{与}(\exists Z)\\(Z\subset Y\text{与}X\text{同构于}Z))$.则$S$为关于$X$、$Y$的预序关系.
			\end{cor}
			证明:根据定义可证.
			
			\begin{sign}
				\textbf{偏序类之间的不等式(inégalité entre types d'ordre)}
				\par
				$X$、$Y$为偏序类,则预序关系$(X\text{为偏序类})\text{与}(Y\text{为偏序类})\text{与}(\exists A)(\exists B)\\((A\text{为按偏序类为}X\text{的偏序排序的偏序集})\text{与}(B\text{为按偏序类为}Y\text{的偏序排序的偏序集})\text{与}(\exists Z)\\(Z\subset Y\text{与}X\text{同构于}Z))$记作$X\prec Y$或$Y\succ X$.
			\end{sign}

			\begin{de}
				\textbf{偏序类族的序数和(somme ordinale de la famille des types d'ordre)}
				\par
				令$I$为偏序集,$(l_i)_{i\in I}$为偏序类族,对任意$i\in I$,令$E_i=l_i\text{的定义域}$.令$E$为偏序集族$(E_i)_{i\in I}$的序数和,则称$Ord(E)$为偏序类族$(l_i)_{i\in I}$的序数和,记作$\sum\limits_{i\in I}l_i$.
			\end{de}
			
			\begin{de}
				\textbf{两个偏序类的序数和(somme ordinale de deux types d'ordre)}
				\par
				令$a\neq b$,$I=\{a, b\}$,按$\{(a, a), (a, b), (b, b)\}$排序,$l$、$m$为偏序类,$E_a=l\text{的定义域}$,$E_b=m\text{的定义域}$.令$E$为偏序集族$(E_i)_{i\in I}$的序数和,则称$Ord(E)$为偏序类$l$和$m$的序数和,记作$l+m$.
			\end{de}
			
			\begin{cor}\label{cor239}
				\hfill\par
				$I$为偏序集,$(E_i)_{i\in I}$为偏序集族,$F$为其序数和,则$\sum\limits_{i\in I}Ord(E_i)=Ord(F)$.
			\end{cor}
			证明:根据补充定理\ref{cor168}可证.
			
			\begin{cor}\label{cor240}
				\textbf{序数和的结合律}
				\par
				$(J_k)_{k\in K}$为偏序集族,其和为$I$,$(l_i)_{i\in I}$为偏序类族,则$\sum\limits_{k\in K}(\sum\limits_{i\in J_k\times \{k\}}l_i)=\sum\limits_{i\in I}l_i$.
			\end{cor}
			证明:根据补充定理\ref{cor169}可证.
			
			\begin{de}
				\textbf{偏序类族的序数乘积(produit ordinale de la famille des types d'ordre)}
				\par
				令$I$为良序集,$(l_i)_{i\in I}$为偏序类族,对任意$i\in I$,令$E_i=l_i\text{的定义域}$.令$E$为偏序集族$(E_i)_{i\in I}$的字典式乘积,则称$Ord(E)$为偏序类族$(l_i)_{i\in I}$的序数乘积,记作$\mathop{\mathsf{P}}\limits_{i\in I}l_i$.
			\end{de}
						
			\begin{de}
				\textbf{两个偏序类的序数乘积(produit ordinale de deux types d'ordre)}
				\par
				令$a\neq b$,$I=\{a, b\}$,按$\{(a, a), (a, b), (b, b)\}$排序,$l$、$m$为偏序类,$E_a=l\text{的定义域}$,$E_b=m\text{的定义域}$.令$E$为偏序集族$(E_i)_{i\in I}$的序数乘积,则称$Ord(E)$为偏序类$l$和$m$的序数乘积,记作$ml$.
			\end{de}
			
			\begin{cor}\label{cor241}
				\hfill\par
				令$I$为良序集,$(E_i)_{i\in I}$、$(F_i)_{i\in I}$为偏序集族,且对任意$i\in I$,$E_i$同构于$F_i$,则$(E_i)_{i\in I}$的字典式乘积同构于$(F_i)_{i\in I}$的字典式乘积.
			\end{cor}
			证明:令$f_i$为$E_i$到$F_i$的同构,则$x\mapsto\bigcap\limits_{i\in I}\{(i, f_i(pr_ix))\}$是$(E_i)_{i\in I}$的字典式乘积到$(F_i)_{i\in I}$的字典式乘积的同构.
			
			\begin{cor}\label{cor242}
				\hfill\par
				$I$为良序集,$(E_i)_{i\in I}$为偏序集族,$F$为其字典式乘积,则$\mathop{\mathsf{P}}\limits_{i\in I}Ord(E_i)=Ord(F)$.
			\end{cor}
			证明:根据补充定理\ref{cor241}可证.
			
			\begin{cor}\label{cor243}
				\textbf{序数乘积的结合律}
				\par
				$(J_k)_{k\in K}$为良序集族,其序数和为良序集$I$,$(l_i)_{i\in I}$为偏序类族,则$\mathop{\mathsf{P}}\limits_{k\in K}(\mathop{\mathsf{P}}\limits_{i\in J_k\{k\}}l_i)=\mathop{\mathsf{P}}\limits_{i\in I}l_i$.
			\end{cor}
			证明:根据定理\ref{theo64}可证.
			
			\begin{cor}\label{cor244}
				\hfill\par
				$I$为良序集,$(m)_{i\in I}$为偏序类族,$Ord(I)=l$,则$\sum\limits_{i\in I}m=ml$.
			\end{cor}
			证明:令$a\neq b$,$J={a, b}$,$E_a=I$,$E_b=m$的定义域,偏序集族$(E_i)_{i\in J}$的序数乘积为$E$.$(E_b)i\in I$的序数为$F$,则$x\mapsto\{(a, pr_2x), (b, pr_1x)\}$是E到F的同构,得证.
			
			\begin{cor}\label{cor245}
				\textbf{两个序数的和与乘积的结合律、分配律}
				\par
				$l$、$m$、$n$为偏序类,则:
				\par
				(1)$(l+m)+n=l+(m+n)$;
				\par
				(2)$(lm)n=l(mn)$;
				\par
				(3)$l(m+n)=lm+ln$.
			\end{cor}
			证明:根据定义可证.
			
			\begin{cor}\label{cor246}
				\hfill\par
				(1)令$I$为偏序集,$(l_i)_{i\in I}$、$(m_i)_{i\in I}$为两个偏序类族,对任意$i\in I$,$l_i\prec m_i$,则$\sum\limits_{i\in I}l_i\prec \sum\limits_{i\in I}m_i$,并且,如果$I$是良序集,则$\mathop{\mathsf{P}}\limits_{i\in I}l_i\prec \mathop{\mathsf{P}}\limits_{i\in I}m_i$.
				\par
				(2)令$I$为偏序集,$(l_i)_{i\in I}$为偏序类族,$J\subset I$,则$\sum\limits_{i\in J}l_i\prec \sum\limits_{i\in I}l_i$,并且,如果$I$是良序集,并且对任意$i\in I$,$l_i\neq \varnothing$,则$\mathop{\mathsf{P}}\limits_{i\in J}l_i\prec \mathop{\mathsf{P}}\limits_{i\in I}l_i$.
			\end{cor}
			证明:
			\par
			(1)根据补充定理\ref{cor97}(1)、定理\ref{theo45}可证.
			\par
			(2)根据补充定理\ref{cor246}(1)可证.
									
			\begin{de}
				\textbf{序数(ordinal)}
				\par
				良序集的偏序类,称为序数.
			\end{de}
			
			\begin{cor}\label{cor247}
				\hfill\par
				(1)$I$为良序集,$(l_i)_{i\in I}$为序数族,则$\sum\limits_{i\in I}l_i$为序数.
				\par
				(2)$m$、$l$为序数,则$m+l$、$ml$为序数.
			\end{cor}
			证明:
			\par
			(1)令$E_i$为$l_i$的定义域,$(E_i)_{i\in I}$的和为$S$,对$S$的任意子集$F$,令$pr_2f$的最小元为$i$,$S\cap(——Ei\times \{i\})$的最小元为$a$,则$(a, i)$为$S$的最小元,得证.
			\par
			(2)$m+l$部分根据补充定理\ref{cor247}(1)可证;$ml$部分根据定义可证.
	
			\begin{de}
				\textbf{序数0(ordinal zéro),序数1(ordinal un),序数2(ordinal deux),序数3(ordinal trois)}
				\par
				$Ord(\varnothing)$称为序数$0$;$Ord (\{\varnothing\})$称为序数$1$.在没有歧义的情况下也可以分别简称为$0$、$1$.
				\par
				$1$和$1$的序数和,称为序数$2$,在没有歧义的情况下也可以简称为$2$.
				\par
				$2$和$1$的序数和,称为序数$3$,在没有歧义的情况下也可以简称为$3$.
			\end{de}
			
			\begin{cor}\label{cor248}
				\hfill\par
				(1)$0=\varnothing$;
				\par
				(2)$Ord(A)=0\Leftrightarrow A=\varnothing$;
				\par
				(3)$Ord(A)=1\Leftrightarrow (\exists x)(A=\{x\})$.
				\par
				(4)序数族$(l_i)_{i\in \varnothing}$的序数和为$0$,序数乘积为$1$.
				\par
				(5)序数族$(\varnothing)_{i\in I}$的序数和为$0$.
				\par
				(6)序数族$(l_i)_{i\in I}$中,如果存在$i \in I$,使$l_i=0$,则其序数乘积为$0$.
			\end{cor}
			证明:
			\par
			(1)根据定义可证.
			\par
			(2)根据定义可证.
			\par
			(3)根据定义可证.
			\par
			(4)根据补充定理\ref{cor116}(4)、补充定理\ref{cor127}(1)可证.
			\par
			(5)根据补充定理\ref{cor116}(5)可证.
			\par
			(6)根据补充定理\ref{cor127}(2)可证.
			
			\begin{cor}\label{cor249}
				\hfill\par
				$a$为序数,则:
				\par
				(1)$a+0=a$;$0+a=a$;$a1=a$;$1a=a$.
				\par
				(2)$a_0=0$;$0a=0$.
				\par
				(3)$a+a=a2$.
			\end{cor}
			证明:
			\par
			(1)根据定义可证.
			\par
			(2)根据补充定理\ref{cor248}(4)、补充定理\ref{cor248}(5)可证.
			\par
			(3)根据定义可证.
			
			\begin{cor}\label{cor250}
				\hfill\par
				对任意两个序数$l$、$m$,均有$l\prec m$或$m\prec l$.
			\end{cor}
			证明:根据定理\ref{theo84}可证.
			
			\begin{cor}\label{cor251}
				\hfill\par
				公式$(l\text{为序数})\text{与}(m\text{为序数})\text{与}(l\prec m)$,为良序.
			\end{cor}
			证明:
			\par
			对任意元素均为序数的非空集合$X$,令$x\in X$,$x$的定义域为$E$,$Y=\{y|y\prec x\text{与}y\neq x\}$,如果$Y=\varnothing$,则$x$为$X$的最小元;如果$y\neq \varnothing$,根据定理\ref{theo87}、定理\ref{theo72},对任意$y\in Y$且$y\neq x$,存在$a\in E$使$y=Ord(]\gets, a[)$,令$f(y)=\tau_a(y=Ord(]\gets, a[))$,则$\{z|z=f(y)\text{与}y\in Y\text{与}y\neq x\}$有最小元,令其最小元为$b$,设$f(u)=b$,因此$u$是$Y$的最小元,进而,$u$也是$X$的最小元,得证.
	
			\begin{sign}
				\textbf{序数之间的不等式(inégalité entre ordinaux)}
				\par
				$l$、$m$为序数,如果$l\prec m$,则记作$l\leq m$.
			\end{sign}
			
			\begin{cor}\label{cor252}
				\hfill\par
				$a$为序数:
				\par
				(1)公式$x\text{为序数}\text{与}x<a$是$x$上的集合化公式.
				\par
				(2)公式$x\text{为序数}\text{与}x\leq a$是x上的集合化公式.
			\end{cor}
			证明:
			\par
			(1)令$E=pr_1a$.则$x<a\Leftrightarrow (\exists y)(y\in E\text{与}x=Ord(Sy))$.令$F_y=\{S_y\}$,$A=\bigcup\limits_{y\in E}F_y$,则$x<a\Rightarrow x\in A$,得证.
			\par
			(2)根据补充定理\ref{cor252}(1)可证.
			
			\begin{cor}\label{cor253}
				\hfill\par
				(1)令$O_a=\{x|x\text{为序数}\text{与}x<a\}$,则$O_a$为良序集且$Ord(O_a)=a$. 
				\par
				(2)$\{x|x\text{为序数}\text{与}x\leq a\}$为良序集,且$Ord(\{x|x\text{为序数}\text{与}x\leq a\})=a+1$.
			\end{cor}
			证明:
			\par
			(1)	令$E=a\text{的定义域}$.根据定理\ref{theo87},$x<a\Leftrightarrow (\exists y)(y\in E\text{与}x=Ord(S_y))$,根据补充定理\ref{cor225}、定理\ref{theo73}(1),$O_a$同构于E,根据补充定理\ref{cor224}(1)得证.
			\par
			(2)根据补充定理\ref{cor253}(1)可证.
						
			\begin{Ccor}\label{Ccor86}
				\hfill\par
				包含$2$元特别符号$\in$ 、显式公理\ref{ex1}、显式公理\ref{ex2}、显式公理\ref{ex3}和公理模式\ref{Sch8}的等式理论$M$中,$R$为公式,$x$不是常数,如果$(x\text{为序数}\text{与}(\forall y)(y\text{为序数}\text{与}y<x\Rightarrow (y|x)R))\Rightarrow R$是$M$的定理,则$x\text{为序数}\Rightarrow R$是$M$的定理.
			\end{Ccor}
			证明:根据证明规则\ref{C59}、补充定理\ref{cor253}(2)可证.
			
			\begin{cor}\label{cor254}
				\hfill\par
				(1)令$(x_i)_{i\in I}$为序数族,则存在唯一的序数$a$,使$(l\text{为序数}\text{与}(\forall i)(i\in I\Rightarrow xi\leq l))\Leftrightarrow (a\leq l)$.
				\par
				(2)$H$的 元素都是序数,则存在唯一的序数$a$,使$(l\text{为序数}\text{与}(\forall x)(x\in H\Rightarrow x\leq l))\Leftrightarrow (a\leq l)$.
			\end{cor}
			证明:
			\par
			(1)令$x=\sum\limits_{i\in I}x_i$,$A=\{l|l\leq a\text{与}l\text{为序数}\text{与}(\forall i)(i\in I\Rightarrow xi\leq l)\}$,根据补充定理\ref{cor246}(2),对任意$i\in I$,$x_i\leq x$,故$A\neq \varnothing$,因此令$A$的最小元为$a$,存在性得证.根据定义可证唯一性.
			\par
			(2)类似补充定理\ref{cor254}(1)可证.
									
			\begin{de}
				\textbf{序数族的最小上界(borne supérieure de la famille d'ordinaux),序数集合的最小上界(borne supérieure du ensemble d'ordinaux)}
				\par
				令$(x_i)_{i\in I}$为序数族,使$(l\text{为序数}\text{与}(\forall i)(i\in I\Rightarrow x_i\leq l))\Leftrightarrow (a\leq l)$成立的$a$,称为$(x_i)_{i\in I}$的最小上界,记作$\mathop{sup}\limits_{i\in I}(x_i)$.令$H$的所有元素均为序数,使$(l\text{为序数}\text{与}(\forall x)(x\in H\Rightarrow x\leq l))\Leftrightarrow (a\leq l)$成立的$a$,称为$H$的最小上界,记作$sup\ H$.
			\end{de}
									
			\begin{de}
				\textbf{前导(prédécesseur)}
				\par
				$a$、$b$为序数,如果$a=b+1$,则称$b$为$a$的前导.
			\end{de}
			
			\begin{cor}\label{cor255}
				\hfill\par
				a为序数,则序数族$(x)_{x<a}$的最小上界为a或a的前导.
			\end{cor}
			证明:
			\par
			设最小上界为$b$,假设$b\neq a$且$b+1\neq a$.
			\par
			由于$b<a$,设$a$的定义域为$E$,根据定理\ref{theo87},存在$y\in E$,使$b=Ord(S_y)$.
			\par
			由于$b+1=Ord(S_y\cup\{y\})$,故$b+1\neq a$,又因为$b+1\neq a$,因此$b+1<a$.但根据补充定理\ref{cor246}(2),$b<b+1$,矛盾.
						
			\begin{cor}\label{cor256}
				\hfill\par
				$a$为序数,且$a\neq 0$,则$a>0$并且$a\geq 1$.
			\end{cor}
			证明:根据定义可证.
						
			\begin{cor}\label{cor257}
				\hfill\par
				$a$、$b$、$x$为序数,则:
				\par
				(1)$a<b\Leftrightarrow a+1\leq b$.
				\par
				(2)如果$a<b$,则$x+a<x+b$,$a+x\leq b+x$,$ax\leq bx$;
				\par
				(3)如果$a<b$,$x>0$,则$xa<xb$.
				\par
				(4)$a<a+1$.
				\par
				(5)$a>1\Leftrightarrow a\geq 2$.
				\par
				(6)$a+1=b+1\Leftrightarrow a=b$.
				\par
				(7)$a+1<b+1\Leftrightarrow a<b$.
			\end{cor}
			证明:
			\par
			(1)	令$b$的定义域为$E$,如果$a<b$,则$a$同构于$E$的片段$S_x$,$a+1$同构于$E$的片段$S_x\cup\{x\}$.
			\par
			反过来,如果$a+1\leq b$,根据补充定理\ref{cor246}(1)、补充定理\ref{cor249}(1),$a<a+1$,因此$a<b$.
			\par
			(2)	根据补充定理\ref{cor257}(1)、补充定理\ref{cor246}(1)、补充定理\ref{cor245}(1)可证.
			\par
			(3)	根据补充定理\ref{cor257}(1)、补充定理\ref{cor246}(1)、补充定理\ref{cor245}(3)可证.
			\par
			(4)	根据补充定理\ref{cor257}(1)可证.
			\par
			(5)	根据补充定理\ref{cor257}(1)可证.
			\par
			(6)	根据公理模式\ref{Sch6},$a=b\Rightarrow a+1=b+1$.
			反过来,如果$a+1=b+1$,令$a$的定义域为$E$,$b$的定义域为$E'$,设$x \not in E$,$y \not in y$,把$x$作为最大元加入$E$,把$y$作为最大元加入$E'$,则$a+1=Ord(E\cup\{x\})$,$b+1=Ord(E'\cup\{y\})$,故$E\cup\{x\}$同构于$E'\cup \{y\}$.因此$E$同构于$E'$,故$a=b$.
			\par
			(7)	根据补充定理\ref{cor257}(2)、补充定理\ref{cor257}(6)可证.
			
			\begin{cor}\label{cor258}
				\textbf{前导的唯一性}
				\par
				一个序数的前导如果存在,则是唯一的.
			\end{cor}
			证明:根据补充定理\ref{cor257}(6)可证.
			
			\begin{cor}\label{cor259}
				\textbf{所有序数不能组成集合}
				\par
				$\text{非}Coll_x(x\text{为序数})$.	
			\end{cor}
			证明:设序数的集合的最小上界为$a$,但$a<a+1$,矛盾.
						
			\begin{cor}\label{cor260}
				\hfill\par
				$a$、$b$、$c$为序数,则:
				\par
				(1)$c+a<c+b\Rightarrow a<b$,$a+c<b+c\Rightarrow a<b$;
				\par
				(2)如果$c>0$,则$ca<cb\Rightarrow a<b$,$ac<bc\Rightarrow a<b$;
				\par
				(3)$c+a=c+b\Rightarrow a=b$;
				\par
				(4)$ca=cb\Rightarrow a=b$.
			\end{cor}
			证明:
			\par
			(1)	根据补充定理\ref{cor246}(1)可证.
			\par
			(2)	根据补充定理\ref{cor246}(1)可证.
			\par
			(3)	根据补充定理\ref{cor257}(2)可证.
			\par
			(4)	根据补充定理\ref{cor257}(3)可证.
						
			\begin{cor}\label{cor261}
				\textbf{序数的差的存在性和唯一性}
				\par
				$a$、$b$为序数,$a\leq b$,则存在唯一的序数$c$,使$a+c=b$.
			\end{cor}
			证明:令$b$的定义域为$E$,则$a$同构于$E$的区间$S_x$,令$c=Ord(E-S_x)$,则$a+c=b$;根据补充定理\ref{cor260}(3),$c$具有唯一性.
									
			\begin{de}
				\textbf{序数的差(différence ordinale)}
				\par
				$a$、$b$为序数,$a\leq b$,如果序数$c$满足$a+c=b$,则称$c$为$a$和$b$的差,记作$(-a)+b$.
			\end{de}
					
			\begin{cor}\label{cor262}
				\textbf{序数的商和余数的存在性和唯一性}
				\par
				$a$、$b$、$c$为序数,$c<ab$,则存在唯一的一组序数$d$、$e$,使$c=ae+d$,且$d<a$,$e<b$.
			\end{cor}
			证明:
			\par
			由于$c<ab$,故$a\neq 0$.
			\par
			令$a$的定义域为$E$,$b$的定义域为$F$,$c$的定义域为$G$,则$G$同构于$ab$的区间$S_x$,令$x$的射影分别为$m$、$n$,其中$m\in F$、$n\in E$.令$Ord(\{t|t\text{为序数}\text{与}t<m\})=e$,$Ord(\{t|t\text{为序数}\text{与}t<n\})=d$,则$d<a$,$e<b$.令$E$和序数集合$\{t|t\text{为序数}\text{与}t<m\}$的字典式乘积为$X$,则\\$Ord(X)=ae$.令$Y=\{t|t\text{为序数}\text{与}t<n\}$,根据定义,$Ord(X)+Ord(Y)=Ord(S_x)$,因此$c=ae+d$.存在性得证.
			\par
			设$c=ae+d$,$c=ae'+d'$.设$e<e'$,则$e+1\leq e'$,因此$ae+a\leq ae'$,故$ae+a+d'\leq ae+d$,故$a+d'\leq d$,和$d<a$矛盾.唯一性得证.
												
			\begin{de}
				\textbf{序数的商(quotient ordinal),序数的余数(reste ordinal)}
				\par
				$a$、$b$、$c$为序数,$c<ab$,如果序数$d$、$e$,使$c=ae+d$,且$d<a$,$e<b$,则称$e$为$c$除以$a$的商,$d$为$c$除以$a$的余数.
			\end{de}
									
			\begin{de}
				\textbf{可约的序数(ordinal décompasable),不可约的序数(ordinal \\indécompasable)}
				\par
				序数$r>0$,如果存在序数$s<r$、$t<r$使$s+t=r$,则称$r$可约,否则,称$r$不可约.
			\end{de}
			
			\begin{cor}\label{cor263}
				\hfill\par
				序数$r>0$,当且仅当对任意序数$s<r$均有$s+r=r$时,$r$不可约.	
			\end{cor}
			证明:根据补充定理\ref{cor261}可证.
			
			\begin{cor}\label{cor264}
				\hfill\par
				序数$r>0$,当且仅当对任意序数$a<r$、$b<r$,均有$a+b<r$时,$r$不可约.
			\end{cor}
			证明:
			\par
			充分性:根据补充定理\ref{cor263},$a+r=r$,由于$b<r$,根据补充定理\ref{cor257}(2),$a+b<r$.
			\par
			必要性:根据定义可证.
			
			\begin{cor}\label{cor265}
				\hfill\par
				$a$、$r$为序数,$r>1$,$a>0$,则当且仅当$r$不可约时,$ar$不可约.
			\end{cor}
			证明:
			\par
			如果$r$可约,令$r=x+y$,其中$x<r$,$y<r$,根据补充定理\ref{cor245}(3),$ar=ax+ay$,根据补充定理\ref{cor257}(3),$ax<ar$,$ay<ar$,故$ar$可约.
			\par
			反过来,如果$ar$可约,令$ar=p+q$,$p<ar$、$q<ar$,则存在$e$、$d$使$p=ae+d$,故$ar=ae+d+q$,因此$e<r$.如果$r$不可约,根据补充定理\ref{cor263},$r=e+r$,根据补充定理\ref{cor260}(3),$ar=d+q$,由于$d+r=r$,故$d+ar=ar$,因此$q=ar$,矛盾.
			
			\begin{cor}\label{cor266}
				\hfill\par
				$a$、$r$为序数,$a>0$,$r>0$,$r$不可约,则存在不可约的序数$x$,使$r=ax$.
			\end{cor}
			证明:如果$r=ar$,得证;如果$r<ar$,则存在$d$、$e$使$r=ae+d$,且$d<r$,故$r=ae$,根据补充定理\ref{cor265},$e$不可约,得证.
			
			\begin{cor}\label{cor267}
				\hfill\par
				不可约的序数族的最小上界是不可约的序数.
			\end{cor}
			证明:令$c=\mathop{sup}\limits_{i\in I}(x_i)$,设$a<c$,$b<c$,则存在$i\in I$,使$a<x_i$、$b<x_i$,故$a+b<x_i$,因此$a+b<c$,得证.
			
			\begin{cor}\label{cor268}
				\hfill\par
				序数$a>0$,则$\{x|x\text{为序数}\text{与}x\leq a\text{与}x\text{不可约}\}$有最大元.
			\end{cor}
			证明:根据补充定理\ref{cor267}可证. 
									
			\begin{metadef}
				\textbf{序数函数符号(symbole fonctionnel ordinal)}
				\par
				包含$2$元特别符号$\in$ 、显式公理\ref{ex1}、显式公理\ref{ex2}、显式公理\ref{ex3}和公理模式\ref{Sch8}的等式理论$M$中,令$T$为项:
				\par
				令$a_0$为序数,如果$(x\text{为序数}\text{与}x\geq a_0)\Rightarrow T\text{为序数}$,则称$T$为关于$x$定义在$x\geq a_0$上的序数函数符号,或简称为定义在$x\geq a_0$上的序数函数符号.
				\par
				令$a_0$、$b_0$为序数,如果$(x\text{为序数}\text{与}x\geq a_0\text{与}y\text{为序数}\text{与}y\geq b_0)\Rightarrow T\text{为序数}$,则称$T$为关于$x$、$y$定义在$x\geq a_0$、$y\geq b_0$上的序数函数符号,或简称为定义在$x\geq a_0$、$y\geq b_0$上的序数函数符号.
			\end{metadef}
			注:由于所有序数不能组成集合,故“序数函数符号”只是类似函数的一种表达式,并非真正的函数.
			\par
			使用“序数函数符号”的主要意义是用于定义序数幂.
									
			\begin{metadef}
				\textbf{标准序数函数符号(symbole fonctionnel ordinal)}
				\par
				包含$2$元特别符号$\in$ 、显式公理\ref{ex1}、显式公理\ref{ex2}、显式公理\ref{ex3}和公理模式\ref{Sch8}的等式理论$M$中,令$T$为项:
				\par
				令$a_0$为序数,$T$为关于$x$定义在$x\geq a_0$上的序数函数符号,如果对任意序数$x$、$y$,$x<y\text{与}x\geq a_0\Rightarrow T<(y|x)T$,并且,对任意序数族$(x_i)_{i\in I}$且$i\neq \varnothing$,如果对任意$i\in I$,均有$x_i\geq a_0$,则$(\mathop{sup}\limits_{i\in I}(x_i)|x)T=\mathop{sup}\limits_{i\in I}((x_i|x)T)$,则称$T$为关于$x$的标准序数函数符号,在没有歧义的情况下也可以简称$T$为标准序数函数符号.
			\end{metadef}
			
			\begin{cor}\label{cor269}
				\hfill\par
				序数$a>0$,则$a+x$、$a$x均为定义在$x\geq 0$上的序数函数符号.
			\end{cor}
			证明:根据补充定理\ref{cor257}(2)、补充定理\ref{cor257}(3)可证.
						
			\begin{cor}\label{cor270}
				\hfill\par
				$w(x)$为定义在$x\geq a_0$上的序数函数符号,对任意序数$x\geq a_0$,$w(x)\geq x$,并且,\\$(x\text{为序数})\text{与}(y\text{为序数})\text{与}x<y\text{与}x\geq a_0\Rightarrow w(x)<w(y)$.
				\par
				令$g(x, y)$为定义在$x\geq a_0$、$y\geq a_0$上的序数函数符号,并满足$((x\text{为序数})\text{与}(y\text{为序数})\text{与}\\x\geq a_0\text{与}y\geq a_0)\Rightarrow g(x, y)>x$.
				\par
				$f(x, y)$为定义在$x\geq a_0$、$y\geq 1$上的序数函数符号,其按下列方式定义:
				\par
				第一,对任意序数$x\geq a_0$,$f(x, 1)=w(x)$;
				\par
				第二,对任意序数$x\geq a_0$, $y>1$,$f(x, y)=\mathop{sup}\limits_{z\in ]0, y[}g(f(x, z), x)$.
				\par
				那么,
				\par
				(1)如果序数函数符号$f_1(x, y)$也满足上述条件,则对任意序数$x\geq a_0$, $y\geq 1$,$f_1(x, y)\\=f(x, y)$.
				\par
				(2)对任意序数$x\geq a_0$,$f(x, y)$是关于$y$定义在$y\geq 1$上的标准序数函数符号.
				\par
				(3)对任意序数$x\geq a_0$, $y\geq 1$,均有$f(x, y)\geq x$,对任意序数$x\geq sup(a_0, 1)$,$y\geq 1$,均有$f(x, y)\geq y$.
				\par
				(4)如果序数$a$、$b$满足$a>0$,$a\geq a_0$,$b\geq w(a)$,则存在唯一的序数$x$,使f$(a, x)\leq b$,$b<f(a, x+1)$,并且$x\leq b$.
				\par
				(5)令$a_0=0$,$w(x)=x+1$,$g(x, y)=x+1$,则$f(x, y)=x+y$.
				\par
				令$a_0=1$,$w(x)=x$,$g(x, y)=x+y$,则$f(x, y)=xy$.
				\par
				(6)如果$a_0\leq x\text{与}x\leq x'\text{与}a_0\leq y\text{与}y\leq y'\Rightarrow g(x, y)\leq g(x', y')$,则$a_0\leq x\text{与}x\leq x'\text{与}1\leq y\text{与}y\leq y'\Rightarrow f(x, y)\leq f(x', y')$.
				\par
				如果$a_0\leq x\text{与}x\leq x'\text{与}a_0\leq y\text{与}y<y'\Rightarrow g(x, y)<g(x, y')\text{与}g(x, y)\leq g(x', y)$,则$a_0\leq x\text{与}x<x'\text{与}0\leq y\Rightarrow f(x, y+1)<f(x', y+1)$.
				\par
				(7)如果$w(x)=x$,并且,$a_0\leq x\text{与}x\leq x'\text{与}a_0\leq y\text{与}y<y'\Rightarrow g(x, y)<g(x, y')\text{与}\\g(x, y)\leq g(x', y)$,同时,对任意序数$x\geq a_0$,$g(x, y)$为关于$y$定义在$y\geq a_0$上的标准序数函数符号,此外,对任意序数$x\geq a_0$,$y\geq a_0$,$z\geq a_0$,均有$g(g(x, y), z))=g(x, g(y, z))$,则对任意$x\geq a_0$,$y\geq 1$,$z\geq 1$,$g(f(x, y), f(x, z))=f(x, y+z)$,$f(f(x, y), z)=f(x, yz)$.
				\par
				(8)如果$a_0\leq x\text{与}x\leq x'\text{与}a_0\leq y\text{与}y\leq y'\Rightarrow g(x, y)\leq g(x', y')$,则对任意序数$x\geq a_0$,$y>0$,均有$f(x, y+1)\geq w(x)+y$.
			\end{cor}
			证明:
			\par
			(1)	根据证明规则\ref{C60}可证.
			\par
			(2)	根据定义,对任意$a>b$、$b\geq 1$,$f(x, a)>f(x, b)$,同时,对于集族$(x_i)_{i\in I}$且$i\neq \varnothing$,设$a=\mathop{sup}\limits_{i\in I}(x_i)$,故$f(x, a)=\mathop{sup}\limits_{z\in ]0, a[}(g(f(x, z), x))$.同时,$\mathop{sup}\limits_{i\in I}(f(x_i))=\\\mathop{sup}\limits_{i\in I}(\mathop{sup}\limits_{z\in ]0, x_i[}(g(f(x, z), x)))$,对任意$i\in I$,$\mathop{sup}\limits_{z\in ]0, x_i[}(g(f(x, z), x))\leq f(a)$,故\\$\mathop{sup}\limits_{i\in I}(f(x, x_i))\leq f(x, a)$,同时,对任意z$\in ]0, a[$,存在$x_i\geq z$,故$g(f(x, z), x)\leq \\\mathop{sup}\limits_{z\in ]0, x_i[}(g(f(x, z), x))$,故$g(f(x, z), x)\leq \mathop{sup}\limits_{i\in I}(f(x_i))$,因此$f(x, a) \leq \mathop{sup}\limits_{i\in I}(f(x_i))$,故\\$f(x, a)= \mathop{sup}\limits_{i\in I}(f(x_i))$,综上,对任意$x\geq a_0$,$f(x, y)$是关于$y$定义在$y\geq 1$上的标准序数函数符号.
			\par
			(3)根据定义可证$f(x, y)\geq x$,同时,$f(x, y)=\mathop{sup}\limits_{z\in ]0, y[}(g(f(x, z), x))$,$y=1$时,显然$f(x, 1)\geq 1$,$y>1$时,设命题对$]0, y[$成立,则$f(x, y)\geq =\mathop{sup}\limits_{z\in ]0, y[}z$,根据补充定理\ref{cor255},$f(x, y)\geq y\text{或}f(x, y)\geq b$,其中$b$为$y$的前导,如果$f(x, y)\geq b$,由于$b\in ]0, y[$,因此$g(f(x, b), x)\\>b$,矛盾,因此$f(x, y)\geq y$,根据补充证明规则86,对任意$x\geq sup(a_0, 1)$,$y\geq 1$,$f(x, y)\geq y$.
			\par
			(4)令$A=\{y|f(a, y)\leq b\}$,则$A\neq \varnothing$,令$z=\mathop{sup}\limits_{i\in A}f(a, i)$,则$z=f(a, \mathop{sup}\limits_{i\in A}i)$,故$\mathop{sup}\limits_{i\in A}i$满足条件,存在性得证;设$x$、$x'$都满足条件,如果$x<x'$,则$x+1\leq x'$,$f(a, x+1)\leq f(a, x')$,矛盾,唯一性得证.
			\par
			(5)前一部分:$y=1$显然成立,$y>1$时,设命题对$]0, y[$成立,则$f(x, y)=\\\mathop{sup}\limits_{z\in ]0, y[}(x+z+1)$,根据补充定理\ref{cor257}(1)、补充定理\ref{cor259}(1)可证.
			\par
			后一部分:$y=1$显然成立,$y>1时$,设命题对$]0, y[$成立,则$f(x, y)=\mathop{sup}\limits_{z\in ]0, y[}(xz+x)$,根据补充定理\ref{cor262}可证.
			\par
			(6)根据补充证明规则\ref{Ccor86}可证.
			\par
			(7)前半部分:根据定义可证$g(f(x, y), f(x, 1))=f(x, y+1)$.设命题对$]1, z[$成立,则$g(f(x, y), f(x, z))=g(f(x, y), \mathop{sup}\limits_{i\in ]0, z[}g(f(x, i), x))$,等于$\mathop{sup}\limits_{i\in ]0, z}[g(f(x, y), g(f(x, i), x))$,等于\\$\mathop{sup}\limits_{i\in ]0, z}[g(f(x, y+i), x)$,等于$\mathop{sup}\limits_{j\in ]0, y+z[}g(f(x, j), x)$,等于$f(x, y+z)$.
			\par
			后半部分:$f(x, yz)=\mathop{sup}\limits_{j\in ]0, yz[}f(x, j+1)$,$f(f(x, y), z)= \mathop{sup}\limits_{j\in ]0, z}[f(x, y(i+1))$,如果存在$i+1=z$,则$f(x, yz)=f(f(x, y), z)$,如果对任意$i<z$均有$i+1<z$,对任意$j<yz$,根据补充定理\ref{cor262},存在$h$、$k$使$j=yh+k$,且$h<z$,$k<y$,故$j+1\leq y(h+1)$,因此$\mathop{sup}\limits_{j\in ]0, yz[}f(x, j+1)\leq \mathop{sup}\limits_{j\in ]0, z[}f(x, y(i+1))$,反过来,对任意$i<z$,$y(i+1)<yz$,故$\mathop{sup}\limits_{j\in ]0, yz[}f(x, j+1)\geq \mathop{sup}\limits_{j\in ]0, z[}f(x, y(i+1))$,得证.
			\par
			(8)$y=0$显然成立;设命题对$]0, y[$成立,$z\in ]0, y[$时,$g(f(x, z+1), x)\geq x+z+1$,则$f(x, y+1)\geq z\in ]0, y[(w(x)+z+1)$,故$f(x, y+1)\geq w(x)+y$.
									
			\begin{de}
				\textbf{序数幂(exponentiation ordinale)}
				\par
				按下列方式定义序数函数符号$f(x, y)$:
				\par
				第一,对任意$x\geq 2$,$f(x, 1)=x$;
				\par
				第二,对任意$x\geq 2$,$y>1$,$f(x, y)=\mathop{sup}\limits_{z\in ]0, y[}f(x, z)x$;
				\par
				第三,对任意序数$a$,$f(a, 0)=1$;
				\par
				第四,对任意序数$b\geq 1$,$f(0, b)=0$,$f(1, b)=1$.
				则称$f(a, b)$为$a$的$b$次序数幂,记作$a^b$.
			\end{de}
						
			\begin{cor}\label{cor271}
				\hfill\par
				$a>1\Rightarrow a^b>1$.
			\end{cor}
			证明:根据补充证明规则\ref{Ccor86}可证.
						
			\begin{cor}\label{cor272}
				\hfill\par
				$a$、$b$、$b'$为序数,如果$a>1$,$b>b'$,则$a^{b'}<a^b$,并且,$a^b$为关于$b$的标准序数函数符号.
			\end{cor}
			证明:根据补充定理\ref{cor270}(2)可证.
						
			\begin{cor}\label{cor273}
				\hfill\par
				$a$、$a'$、$b$为序数,如果$a'>0$,$a\geq a'$,则${a'}^b\leq a^b$.
			\end{cor}
			证明:根据补充证明规则\ref{Ccor86}可证.
			
			\begin{cor}\label{cor274}
				\hfill\par
				$a$、$x$、$y$为序数,则$a^xa^y=a^{x+y}$,$(a^x)^y=a^{xy}$.
			\end{cor}
			证明:根据补充定理\ref{cor270}(7)可证.
			
			\begin{cor}\label{cor275}
				\hfill\par
				$a$、$b$为序数,$a\geq 2$,$b\geq 1$,则$a^b\geq ab$.
			\end{cor}
			证明:$a^b=\mathop{sup}\limits_{i\in ]0, b[}a^ia$,$ab=\mathop{sup}\limits_{i\in ]0, b[}(ai+a)$,根据补充证明规则\ref{Ccor86}可证.
			
			\begin{cor}\label{cor276}
				\hfill\par
				$a$、$b$为序数,$a\geq 2$,$b\geq 1$,则存在唯一的序数$c$、$d$、$e$,使$b=a^cd+e$,且$d>0$、$d<a$、$e<a^c$.
			\end{cor}
			证明:根据补充定理\ref{cor270}(4)、补充定理\ref{cor262}可证.
												
			\begin{de}
				\textbf{传递集合(ensemble transitif)}
				\par
				如果$(\forall x)(x\in X\Rightarrow x\subset X)$,则称$X$为传递集合.
			\end{de}
			
			\begin{cor}\label{cor277}
				\hfill\par
				(1)$\varnothing$、$\{\varnothing\}$、$\{\varnothing, \{\varnothing\}\}$、$\{\varnothing, \{\varnothing\}, \{\varnothing, \{\varnothing\}\}\}$为传递集合.
				\par
				(2)如果$Y$为传递集合,则$Y\cup\{Y\}$为传递集合.
				\par
				(3)如果$(Y_i)_{i\in I}$为传递集族,则$\bigcup\limits_{i\in I}Y_i$为传递集合,如果$i\neq \varnothing$,则$\bigcap\limits_{i\in I}Y_i$为传递集合.
			\end{cor}
			证明:根据定义可证.
												
			\begin{de}
				\textbf{伪序数(pseudo-ordinal)}
				\par
				如果$(\forall Y)(Y\subset X\text{与}(Y\text{为传递集合})\text{与}Y\neq X\Rightarrow Y\in X)$,则称$X$为伪序数.
			\end{de}
									
			\begin{de}
				\textbf{正式集合(ensemble decent)}
				\par
				如果$(\forall x)(x\in X\Rightarrow x\notin x)$,则称$X$为正式集合.
			\end{de}
			
			\begin{cor}\label{cor278}
				\hfill\par
				伪序数是传递集合也是正式集合.
			\end{cor}
			证明:令$A$为伪序数,$B=\{Y|Y\subset A\text{与}(Y\text{为传递集合})\text{与}(Y\text{为正式集合})\}$,$C=\bigcup\limits_{Y\in B}Y$,因此$C\subset A$.根据补充定理\ref{cor277}(3),$C$为传递集合.同时,对任意$x \in C$,故存在$Y\in B$使$x \in Y$,故$x \notin x$,因此$C$为正式集合.
			\par
			如果$C\in C$,则存在$Y\in B$使$C\in Y$,故$C\notin C$,矛盾;因此,$C\notin C$.
			\par
			如果$C\neq A$,则$C\in A-C$,根据补充定理\ref{cor277}(2),$C\cup \{C\}$为传递集合,
			\par
			同时,对任意$x \in C\cup \{C\}$,如果$x=C$,则$x \notin x$,如果$x \neq C$,则$x \in C$,故$x \notin x$,因此,$C\cup \{C\}$为正式集合,故$C\cup \{C\}\subset C$,矛盾.
			\par
			因此$C=A$,得证.
			
			\begin{cor}\label{cor279}
				\hfill\par
				如果$X$是伪序数,则$X\cup \{X\}$也是伪序数.
			\end{cor}
			证明:根据定义可证.
			
			\begin{cor}\label{cor280}
				\hfill\par
				$X$、$Y$均为伪序数,则$X\subset Y\text{或}Y\subset X$.
			\end{cor}
			证明:根据补充定理\ref{cor279}、补充定理\ref{cor277}(3),$X\cap Y$为传递集合,且$X\cap Y\notin X\cap Y$,因此,$X\cap Y=X\text{或}X\cap Y=Y$,得证.
			
			\begin{cor}\label{cor281}
				\hfill\par
				$X$为传递集合,如果对任意$x\in X$,$x$均为伪序数,则$X$为伪序数.
			\end{cor}
			证明:如果$Y\subset X$,$Y\neq X$,$Y$为传递集合,对任意$x\in X-Y$,$y\in Y$,根据补充定理\ref{cor280},$x\subset y\text{或}y\subset x$.如果$x\subset y$,由于$x\neq y$,故$x\in y$,又因为$y\subset Y$,故$x\in Y$,矛盾;因此,$y\subset x$,由于$x\neq y$,故$y\in x$,因此$Y\subset x$.如果$Y=x$,则$Y\in X$,如果$Y\neq x$,则$Y\in x$,由于$x\subset X$,故$y\in X$,得证.
			
			\begin{cor}\label{cor282}
				\hfill\par
				$\varnothing$为伪序数.
			\end{cor}
			证明:根据定义可证.
			
			\begin{cor}\label{cor283}
				\hfill\par
				伪序数的每一个元素都是伪序数.
			\end{cor}
			证明:设$X$为伪序数,$A=\{Y|Y\subset X\text{与}(Y\text{为传递集合})\text{与}(\forall x)(x\in Y\Rightarrow x\text{为伪序数})\}$,$B=\bigcup\limits_{Y\subset A}Y$,则$B$为传递集合,故$B$为伪序数,因此$B\notin B$.根据补充定理\ref{cor279},$B\cup\{B\}$也是伪序数,且$B\cup\{B\}\neq B$.如果$B\neq X$,则$B\in X$,故$B\cup\{B\}\subset X$,因此,$B\cup\{B\}\in A$,所以$B\cup\{B\}\subset B$,矛盾.因此$B=X$,得证.
			
			\begin{cor}\label{cor284}
				\hfill\par
				(1)令$(X_i)_{i\in I}$是伪序数族,且$i\neq \varnothing$,则$\bigcap\limits_{i\in I}X_i$是$\{Y|(\exists i)(i\in I\text{与}Y=X_i)\}$按包含关系排序的偏序集的最小元.
				\par
				(2)$E$为伪序数,则$x\in E\text{与}y\in E\text{与}x\subset y$为良序.
			\end{cor}
			证明:
			\par
			(1)令$X=\bigcap\limits_{i\in I}X_i$,根据补充定理\ref{cor281}、补充定理\ref{cor277}(3),$X$为伪序数,故$X\notin X$.根据补充定理\ref{cor279},$X\cup\{X\}$为伪序数,如果对任意$i\in I$,均有$X_i\neq X$,则$X\cup\{X\}\subset X$,矛盾.因此,存在$i\in I$使$X_i=X$,其即为最小元.
			\par
			(2)根据补充定理\ref{cor284}(1)可证.
			
			\begin{cor}\label{cor285}
				\hfill\par
				对任意序数$a$,存在唯一的伪序数$E_a$,使$Ord(Ea)=a$.
			\end{cor}
			证明:
			由于$Ord(\varnothing)=0$,故命题对$[0, 1[$成立,设命题对$[0, a[$成立,令$X=\bigcup\limits_{i\in [0, a[}\{E_i\}$,并且是按包含关系排序的偏序集.根据补充定理\ref{cor281}、补充定理\ref{cor277}(3),$E_a$为伪序数,根据补充定理\ref{cor253}(1),$Ord(E_a)=a$,同时,根据补充定理\ref{cor280},这样的伪序数是唯一的.根据补充证明规则\ref{Ccor86}得证.
			\par
			注:本补充定理表明,序数和伪序数一一对应.
			
			\begin{exer}\label{exer101}
				\hfill\par
				$K=\{F|F\text{为在E上的偏序}\}$,并且是按$(F\in K\text{与}F'\in K\text{与}F\text{是比}F'\text{更细的偏序})$排序的偏序集,求证:$K$的极小元是在$E$上的全序;对在$E$上的任意偏序$F$,$F$的图是所有不等于$E$且比$F$更细的全序的图的交集;进而,任何偏序集同构于全序集族的乘积的一个子集.
			\end{exer}
			证明:即补充定理\ref{cor227}、补充定理\ref{cor228}、补充定理\ref{cor229}.
			
			\begin{exer}\label{exer102}
				\hfill\par
				$E$为集合,$P=\{F|F\subset E\text{与}(F\text{是按}E\text{的偏序在}F\text{上导出的偏序排序的良序集})\}$,求证:
				\par
				(1)“$(X\text{是}Y\text{的片段}$”是关于$X$、$Y$在$P$上的偏序关系;
				\par
				(2)$P$是归纳集;
				\par
				(3)存在$E$的良序子集,在$E$上没有严格上界.
			\end{exer}
			证明:
			\par
			(1)根据定义可证.
			\par
			(2)根据定义可证.
			\par
			(3)	根据定理\ref{theo80},$P$有极大元$F$.假设$F$有严格上界$x$,则$F\cup\{x\}\in P$,且$F<F\cup\{x\}$,矛盾,因此$F$没有严格上界.
			
			\begin{exer}\label{exer103}
				\hfill\par
				$E$为偏序集,求证:存在$A$、$B$,使$A\cup B=E$,$A\cap B=\varnothing$,且$A$为良序集,$B$没有最小元.并举出这样的例子.
			\end{exer}
			证明:
			\par
			令$B$为$E$的所有没有最小元的子集的并集,如果$B$有最小元x,设$x\in X$,$X$为没有最小元的$E$的子集,则$x$为$X$的最小元,矛盾.令$A=E-B$,如果$A$的子集Y没有最小元,则$Y\subset B$,矛盾.
			\par
			例:$E$为整数集,$A$为$E$的任意有限子集,$B=E-A$.
			\par
			注:习题\ref{exer103}例子部分涉及未介绍的“整数”知识.
						
			\begin{exer}\label{exer104}
				\hfill\par
				对任意偏序集$E$,存在$F\subset E$,$F$为部分良序集,并且是$E$的共尾子集.
			\end{exer}
			证明:即补充定理\ref{cor231}(1).
						
			\begin{exer}\label{exer105}
				\hfill\par
				$E$为偏序集,$F=\{X|X\text{为}E\text{的自由子集}\}$,并且为按$X\in F\text{与}Y\in F\text{与}(\forall x)(x\in X\Rightarrow (\exists y)(y\in Y\text{与}x\leq y))$排序的偏序集,求证:如果$E$为归纳集,则$F$有极大元.
			\end{exer}
			证明:根据定理\ref{theo80},$E$有极大元,令$X=\{x|x\text{为}E\text{的极大元}\}$,则$X\in F$.如果存在$E$的自由子集$Y$使$X\leq Y$,则$X\subset Y$,若$Y-X\neq \varnothing$,则令$u\in Y-X$,根据定理\ref{theo81},存在$E$的极大元$v$,使$u\leq v$,又因为$v\in X$,故$v\in Y$,矛盾.故$X$是$F$的极大元.
						
			\begin{exer}\label{exer106}
				\hfill\par
				$E$为偏序集,$f$为$E$到$E$的映射,并且对任意$x\in E$,$f(x)\geq x$:
				\par
				(1)$C_a$为$a$关于函数$f$的链,求证:
				\par
				对任意$a\in E$,$C_a\in H$;
				\par
				$E$的偏序子集$C_a$,为良序集;
				\par
				如果$C_a$在$E$上有最小上界$b$,则$b\in C_a$并且$f(b)=b$.
				\par
				(2)	如果$E$为归纳集,求证:存在$b\in E$,使$f(b)=b$.
			\end{exer}
			证明:
			\par
			(1)	即补充定理\ref{cor232}.
			\par
			(2)	即补充定理\ref{cor233}.
						
			\begin{exer}\label{exer107}
				\hfill\par
				$E$为良序集,$F$为在$E$上的闭包的集合.$F$按关于$u$、$v$的偏序关系“$u\in F\text{与}v \in F\text{与}(\forall x)\\(x\in E\Rightarrow u(x)\leq v(x))$”排序.对任$u\in F$,令$I(u)$为$u$的不动点集合:
				\par
				(1)	求证:当且仅当$I(v)\subset I(u)$时,$u\leq v$.
				\par
				(2)	求证:如果$E$的任何两个元素在$E$上有最大下界,则$F$的任何两个元素在$F$上有最大下界;如果$E$是完备格,则$F$是完备格.
				\par
				(3)	求证:如果$E$为归纳集,则$F$的任何两个元素在$F$上有最小上界.
			\end{exer}
			证明:
			\par
			(1)当$u\leq v$时,根据定义可证$I(v)\subset I(u)$;反过来,当$I(v)\subset I(u)$时,对任意$x\in E$,$u(v(x))=v(x)$,由于$u(v(x))\geq u(x)$,得证.
			\par
			(2)对任意$u\in F$、$v\in F$,令$t$为映射$x\mapsto inf(u(x), v(x))$,根据定义,$t\in F$,且$t$为$u$、$v$的最大下界.同理可证映射$x\mapsto sup(u(x), v(x))$ 为$u$、$v$的最小上界,因此,如果$E$是完备格,则$F$是完备格.
			\par
			(3)对于$u\in F$、$v\in F$,令$f=u\circ v$,根据定义,$I(f)=I(u)\bigcap\limits_I(v)$.对任意$x\in E$,令$w(x)$为$x$的链$C_x$的最大元.则对任意$x\in E$、$y\in E$、$x\leq y$,$x\leq w(y)$,同时,根据习题\ref{exer106}(1),$C_x$为良序集.令$X=\{a|a\in C_x\text{与}a\leq w(y)\}$,则$X$非空,令$Y=C_x-X$,如果$Y$非空,则$Y$有最小元$b$,因此对任意$c\in X$,$c\leq w(y)$,故$f(c)\leq w(y)$,因此$f(c)\in X$,故$f(c)<b$.所以$w(y)$、$b$都是X的上界.根据定义,存在$X$的子集$Z$,$Z$在$X$上有最小上界,且该最小上界是$Y$的元素,故$sup\ Z=b$.又因为$w(y)$也是$X$的上界,故也是$Z$的上界,因此$b\leq w(y)$,矛盾.故$Y$为空,即$w(x)\leq w(y)$.$w$满足闭包的其他条件,故$w$是闭包,并且,$I(w)=I(u)\cap I(v)$;根据习题\ref{exer107}(1),$w$为最小上界.
			
			\begin{exer}\label{exer108}
				\hfill\par
				$E$为偏序集,$a\in E$,$R_a$为$E$的以$a$为最小元的右方分支子集的集合,并按包含关系排序:
				\par
				(1)求证:$R_a$有极大元.
				\par
				(2)$E$为右方分叉集,求证:$R_a$的极大元均为完全右方分支集.
				\par
				(3)给出一个是右方分叉集但不是右方分支集的例子.并且,设E为习题\ref{exer100}(2)所称的集合,$F$为不包含可数共尾子集的全序集,$E\times F$为完全右方分支集.
				\par
				(4)$E$为集合,对任意$x\in E$,区间$]\gets, x]$均为全序集,是否一定存在$E$的共尾右方无向子集.
			\end{exer}
			证明:
			\par
			(1)根据定理\ref{theo82}可证.
			\par
			(2)如果$R_a$的极大元$U$有极大元x,根据补充定理\ref{cor213}(2),存在$y\in E$、$z\in E$使$y>x$、$z>x$,且$y$和$z$是不可比较的.因此$U\cup \{y, z\}$也是右方分支集,矛盾,得证.
			\par
			(3)设$E$为习题\ref{exer100}(2)所称的集合,$\{R\}\cup E$按包含关系的相反关系排序,则其是右方分叉集,但不是右方分支集.同时,根据定义可证$E\times F$为完全右方分支集.
			\par
			(4)如果$E$为无最大元的全序集,显然不存在.
			\par
			注:
			\par
			习题\ref{exer108}(3)涉及尚未介绍的“实数”知识.
			\par
			原书习题\ref{exer108}(4)有误.
			
			\begin{exer}\label{exer109}
				\hfill\par
				求证:当且仅当指标集和各偏序集均为良序集时,偏序集族的序数和为良序集.
			\end{exer}
			证明:即补充定理\ref{cor217}(3).
						
			\begin{exer}\label{exer110}
				\hfill\par
				$E$、$I$为偏序集,$\{a, b\}$为良序集,其中$a<b$;$F_a=I$,$F_b=E$.求证:$(E)_{i\in I}$的偏序集族的序数和,同构于集族$(F_i)_{i\in \{a, b\}}$的字典式乘积.
			\end{exer}
			证明:根据定义可证.
						
			\begin{exer}\label{exer111}
				\hfill\par
				$I$为良序集,$(E_i)_{i\in I}$为偏序集族,并且对任意$i\in I$,$E_i$均中少有一对可比较的不同元素.则当且仅当$I$为有限集合,并且对任意$i\in I$,$E_i$均为良序集时,$(E_i)_{i\in I}$的字典式乘积为良序集.
			\end{exer}
			证明:
			\par
			必要性:如果$(E_i)_{i\in I}$的字典式乘积为良序集,根据定义,对任意$i\in I$,$E_i$均为良序集.
			同时,如果$I$为无穷集合,令$k$的自然数集$N$到$I$的子集的同构,$G(n)=\{(x, y)|x\in E_k(n)\text{与}y\in E_k(n) \text{与}x<y\}$,则对任意$n \in N$,$G(n)\neq \varnothing$.进而,令$f(n)=pr_1(\tau_z(z\in G(n)))$,$g(n)=pr_2(\tau_z(z\in G(n)))$.
			\par
			令$F(n)=\{z|(\exists i)(i\in N-{n}\text{与}z=(k(i), g(i)))\}\cup\{z|(\exists i)(i\in I-k\langle N\rangle\text{与}z=(i, tau_a(a\in E_i)))\}$,$A=\{X|(\exists n)(n\in N\text{与}X=F(n))\}$,那么对任意$A$的元素$F(n)$,令自然数$m>n$,则$F(n)>F(m)$,故$A$没有最小元,矛盾.必要性得证.
			\par
			充分性:对$I$的元素数目用数学归纳法可证.
			\par
			注:习题\ref{exer111}涉及尚未介绍的“有限集合”知识.
			
			\begin{exer}\label{exer112}
				\hfill\par
				$I$为全序集,$(E_i)_{i\in I}$为偏序集族,$E=\prod\limits_{i\in I}E_i$,$R$为公式$(\{i|i\in I\text{与}pr_ix\neq pr_iy\}\text{是良序集})\\\text{与}((\exists j)(j\text{是}\{i|i\in I\text{与}pr_ix\neq pr_iy\}\text{的最小元})\text{与}(pr_jx\leq pr_jy))$.
				\par
				求证:$R$为关于$x$、$y$在$E$上的偏序关系.如果对任意$i\in I$,$E_i$均为全序集,则$E$关于“$x\text{和}y\text{是可比较的}$”的连通分量是全序集.如果对任意$i\in I$,$E_i$都有不少于两个元素,则当且仅当$I$为良序集,且对任意$i\in I$,$E_i$均为全序集时,$E$为全序集,并且此时,$E$为$(E_i)_{i\in I}$的字典式乘积.
			\end{exer}
			证明:
			\par
			根据定义可证$R$为关于$x$、$y$在$E$上的偏序关系.
			\par
			设$x$、$y$是可比较的,$y$、$z$是可比较的,令$A=\{i|i\in I\text{与}pr_ix\neq pr_iy\}$,$B=\{i|i\in I\text{与}pr_iy\neq pr_iz\}$,则$A\cup B$是良序集,$\{i|i\in I\text{与}pr_ix\subset pr_iy\}A\cup B$,也是良序集.故$x$、$z$是可比较的.使用数学归纳法可证$E$关于“$x\text{和}y\text{是可比较的}$”的连通分量是全序集.
			\par
			充分性根据定义可证.
			\par
			必要性:若$E$为全序集,令$f(i)=\tau_x(x\in E_i)$,$g(i)=\tau_x(x\in E_i-\{f(i)\})$.则对$I$的任意子集$J$,令$x=\bigcup\limits_{i\in I}\{(i, f(i))\}$,$y=(\bigcup\limits_{i\in I-J}\{(i, f(i))\})\cup(\bigcup\limits_{i\in J}\{(i, g(i))\})$,由于$x$、$y$是可比较的,故$J$为良序集,根据定义,$I$为良序集.
			\par
			同时,对任意$i\in I$、$x\in E_i$、$y\in E_i$,令$A=(\bigcup\limits_{j\in I-i}\{(j, f(j))\})\cup\{i, x\}$,$B=\\(\bigcup\limits_{j\in I-i}\{(j, f(j))\})\cup\{i, y\}$,由于$A$、$B$是可比较的,故$x$、$y$是可比较的.
			\par
			综上,必要性成立.
			\par
			根据定义可证,此时,$E$为$(E_i)_{i\in I}$的字典式乘积.
			\par
			注:习题\ref{exer112}涉及尚未介绍的知识.
						
			\begin{exer}\label{exer113}
				\hfill\par
				(1)令$R$为公式$(F\text{为在}E\text{上的偏序})\text{与}(F'\text{为在}E'\text{上的偏序})\text{与}\\(\text{按}F\text{排序的}E\text{同构于在}F'\text{上排序的}E')$,求证:$R$为关于$F$、$F'$的等价关系.同时,当且仅当两个偏序的偏序类相等时,这两个偏序集同构.
				\par
				(2)令$S$为公式$(X\text{为偏序类})\text{与}(Y\text{为偏序类})\text{与}(\exists A)(\exists B)\\((A\text{为按偏序类为}X\text{的偏序排序的偏序集})\text{与}(B\text{为按偏序类为}Y\text{的偏序排序的偏序集})\text{与}(\exists Z)\\(Z\subset Y\text{与}X\text{同构于}Z))$.求证:$S$为关于$X$、$Y$的预序关系.
				\par
				(3)$I$为偏序集,$(E_i)_{i\in I}$为偏序集族,$F$为其序数和,求证:$\sum\limits_{i\in I}Ord(E_i)=Ord(F)$.
				\par
				$(J_k)_{k\in K}$为偏序集族,其和为$I$,$(l_i)_{i\in I}$为偏序类族,求证:$\sum\limits_{k\in K}(\sum\limits_{i\in J_k\times \{k\}}l_i)=\sum\limits_{i\in I}l_i$.
				\par
				(4)$I$为良序集,$(E_i)_{i\in I}$为偏序集族,$F$为其字典式乘积,求证:$\mathop{\mathsf{P}}\limits_{i\in I}Ord(E_i)=\\Ord(F)$.
				\par
				$(J_k)_{k\in K}$为良序集族,其序数和为良序集$I$,$(l_i)_{i\in I}$为偏序类族,求证:$\mathop{\mathsf{P}}\limits_{k\in K}(\mathop{\mathsf{P}}\limits_{i\in J_k\{k\}}l_i)=\mathop{\mathsf{P}}\limits_{i\in I}l_i$.
				\par
				(5)$I$为良序集,$(m)_{i\in I}$为偏序类族,$Ord(I)=l$,求证:$\sum\limits_{i\in I}m=ml$.
				\par
				$l$、$m$、$n$为偏序类,求证:$(l+m)+n=l+(m+n)$,$(lm)n=l(mn)$,$l(m+n)=lm+ln$.
				\par
				(6)令$I$为偏序集,$(l_i)_{i\in I}$、$(m_i)_{i\in I}$为两个偏序类族,对任意$i\in I$,$l_i\prec m_i$,求证:$\sum\limits_{i\in I}l_i\prec \sum\limits_{i\in I}m_i$,并且,如果$I$是良序集,则$\mathop{\mathsf{P}}\limits_{i\in I}l_i\prec \mathop{\mathsf{P}}\limits_{i\in I}m_i$.
				\par
				令$I$为偏序集,$(l_i)_{i\in I}$为偏序类族,$J\subset I$,求证:$\sum\limits_{i\in J}l_i\prec \sum\limits_{i\in I}l_i$,并且,如果$I$是良序集,并且对任意$i\in I$,$l_i\neq \varnothing$,则$\mathop{\mathsf{P}}\limits_{i\in J}l_i\prec \mathop{\mathsf{P}}\limits_{i\in I}l_i$.
				\par
				(7)$l$为偏序类,$l^*$表示按$l$相反关系排序的$l$的定义域,求证:$(l^*)^*=l$,$(\sum\limits_{i\in I}l_i)^*=(\sum\limits_{i\in I^*}l_i^*)$,其中$I^*$表示按相反关系排序的$I$.
			\end{exer}
			证明:
			\par
			(1)	即补充定理\ref{cor236}、补充定理\ref{cor237}.
			\par
			(2)	即补充定理\ref{cor238}.
			\par
			(3)	即补充定理\ref{cor239}、补充定理\ref{cor240}.
			\par
			(4)	即补充定理\ref{cor242}、补充定理\ref{cor243}.
			\par
			(5)	即补充定理\ref{cor244}、补充定理\ref{cor245}.
			\par
			(6)	即补充定理\ref{cor246}.
			\par
			(7)	根据定义可证.
						
			\begin{exer}\label{exer114}
				\hfill\par
				(1)$I$为良序集,$(l_i)_{i\in I}$为序数族,求证:$\sum\limits_{i\in I}l_i$为序数;如果$I$为有限集合,则$\mathop{\mathsf{P}}\limits_{i\in I}l_i$为序数;$a$为序数,则$a+0=a$;$0+a=a$;$a1=a$;$1a=a$.
				\par
				(2)公式$l\text{为序数}\text{与}m\text{为序数}\text{与}l\prec m$,为良序.
				\par
				(3)$a$为序数,求证:公式$x\text{为序数}\text{与}x\leq a$是$x$上的集合化公式.并且,令$O_a=\\\{x|x\text{为序数}\text{与}x<a\}$,则$O_a$为良序集且$Ord(O_a)=a$.
				\par
				(4)$a$为序数,求证:序数族$(x)_{x<a}$的最小上界为$a$或$a$的前导.
			\end{exer}
			证明:
			\par
			(1)前半部分序数和即补充定理\ref{cor247}(1);序数乘积对$I$的元素数目用数学归纳法可证.后半部分即补充定理\ref{cor249}(1).
			\par
			(2)即补充定理\ref{cor251}.
			\par
			(3)前半部分即补充定理\ref{cor252}(2),后半部分即补充定理\ref{cor253}(1).
			\par
			(4)即补充定理\ref{cor255}.
			\par
			注:习题\ref{exer114}(1)涉及尚未介绍的“有限集合”知识.
						
			\begin{exer}\label{exer115}
				\hfill\par
				(1)$a$、$b$为序数:
				\par
				求证:$a<b\Leftrightarrow a+1\leq b$;
				\par
				$x$为序数,$a<b$,求证:$x+a<x+b$,$a+x\leq b+x$,$ax\leq bx$;
				\par
				$x$为序数,$a<b$,$x>0$,求证:$xa<xb$.
				\par
				(2)	不存在一个集合,使所有序数都是其元素.
				\par
				(3)	$a$、$b$、$c$为序数:
				\par
				求证:$c+a<c+b\Rightarrow a<b$,$a+c<b+c\Rightarrow a<b$.
				\par
				如果$c>0$,求证:$ca<cb\Rightarrow a<b$,$ac<bc\Rightarrow a<b$.
				\par
				(4)$a$、$b$、$c$为序数,求证:$c+a<c+b\Rightarrow a<b$;如果$c>0$,则$ca<cb\Rightarrow a<b$;
				\par
				(5)$a$、$b$为序数,$a\leq b$,求证:存在唯一的序数$c$,使$a+c=b$;
				\par
				(6)$a$、$b$、$c$为序数,$c<ab$,求证:存在唯一的一组序数$d$、$e$,使$c=ae+d$,且$d<a$,$e<b$.
			\end{exer}
			证明:
			\par
			(1)即补充定理\ref{cor257}(1)、补充定理\ref{cor257}(2)、补充定理\ref{cor257}(3);
			\par
			(2)即补充定理\ref{cor259};
			\par
			(3)即补充定理\ref{cor260}(1)、补充定理\ref{cor260}(2);
			\par
			(4)即补充定理\ref{cor260}(3)、补充定理\ref{cor260}(4);
			\par
			(5)即补充定理\ref{cor261};
			\par
			(6)即补充定理\ref{cor262}.
			
			\begin{exer}\label{exer116}
				\hfill\par
				(1)$r>0$,求证:当且仅当对任意序数$s<r$均有$s+r=r$时,$r$为不可约的序数.
				\par
				(2)$a$、$r$为序数,$r>1$,$a>0$,求证:当且仅当$r$不可约时,$ar$不可约.
				\par
				(3)$a$、$r$为序数,$a>0$,$r>0$,$r$不可约,求证:存在不可约的序数$x$,使$r=ax$.
				\par
				(4)序数$a>0$,求证:$\{x|x\text{为序数}\text{与}x\leq a\text{与}x\text{不可约}\}$有最大元.
				\par
				(5)求证:不可约的序数族的最小上界是不可约的序数.
			\end{exer}
			证明:
			\par
			(1)	即补充定理\ref{cor263}.
			\par
			(2)	即补充定理\ref{cor265}.
			\par
			(3)	即补充定理\ref{cor266}.
			\par
			(4)	即补充定理\ref{cor268}.
			\par
			(5)	即补充定理\ref{cor267}.

			\begin{exer}\label{exer117}
				\hfill\par
				(1)序数$a>0$,求证:$a+x$、$a$x均为定义在$x\geq 0$上的序数函数符号.
				\par
				(2)$w(x)$为定义在$x\geq a_0$上的序数函数符号,对任意序数$x\geq a_0$,$w(x)\geq x$,并且,\\$(x\text{为序数})\text{与}(y\text{为序数})\text{与}x<y\text{与}x\geq a_0\Rightarrow w(x)<w(y)$.
				\par
				令$g(x, y)$为定义在$x\geq a_0$、$y\geq a_0$上的序数函数符号,并满足$((x\text{为序数})\text{与}(y\text{为序数})\text{与}\\x\geq a_0\text{与}y\geq a_0)\Rightarrow g(x, y)>x$.
				\par
				$f(x, y)$为定义在$x\geq a_0$、$y\geq 1$上的序数函数符号,其按下列方式定义:
				\par
				第一,对任意序数$x\geq a_0$,$f(x, 1)=w(x)$;
				\par
				第二,对任意序数$x\geq a_0$, $y>1$,$f(x, y)=\mathop{sup}\limits_{z\in ]0, y[}g(f(x, z), x)$.
				\par
				求证:
				\par
				如果序数函数符号$f_1(x, y)$也满足上述条件,则对任意序数$x\geq a_0$, $y\geq 1$,$f_1(x, y)= f(x, y)$.
				\par
				对任意序数$x\geq a_0$,$f(x, y)$是关于$y$定义在$y\geq 1$上的标准序数函数符号.
				\par
				对任意序数$x\geq a_0$, $y\geq 1$,均有$f(x, y)\geq x$,对任意序数$x\geq sup(a_0, 1)$,$y\geq 1$,均有$f(x, y)\geq y$.
				\par
				如果序数$a$、$b$满足$a>0$,$a\geq a_0$,$b\geq w(a)$,则存在唯一的序数$x$,使f$(a, x)\leq b$,$b<f(a, x+1)$,并且$x\leq b$.
				\par
				(3)令$a_0=0$,$w(x)=x+1$,$g(x, y)=x+1$,求证:$f(x, y)=x+y$.
				\par
				令$a_0=1$,$w(x)=x$,$g(x, y)=x+y$,求证:$f(x, y)=xy$.
				\par
				(4)如果$a_0\leq x\text{与}x\leq x'\text{与}a_0\leq y\text{与}y\leq y'\Rightarrow g(x, y)\leq g(x', y')$,求证:$a_0\leq x\text{与}x\leq x'\text{与}1\leq y\text{与}y\leq y'\Rightarrow f(x, y)\leq f(x', y')$.
				\par
				如果$a_0\leq x\text{与}x\leq x'\text{与}a_0\leq y\text{与}y<y'\Rightarrow g(x, y)<g(x, y')\text{与}g(x, y)\leq g(x', y)$,求证:$a_0\leq x\text{与}x<x'\text{与}0\leq y\Rightarrow f(x, y+1)<f(x', y+1)$.
				\par
				(5)如果$w(x)=x$,并且,$a_0\leq x\text{与}x\leq x'\text{与}a_0\leq y\text{与}y<y'\Rightarrow g(x, y)<g(x, y')\text{与}\\g(x, y)\leq g(x', y)$,同时,对任意序数$x\geq a_0$,$g(x, y)$为关于$y$定义在$y\geq a_0$上的标准序数函数符号,此外,对任意序数$x\geq a_0$,$y\geq a_0$,$z\geq a_0$,均有$g(g(x, y), z))=g(x, g(y, z))$,求证:对任意$x\geq a_0$,$y\geq 1$,$z\geq 1$,$g(f(x, y), f(x, z))=f(x, y+z)$,$f(f(x, y), z)=f(x, yz)$.
			\end{exer}
			证明:
			\par
			(1)即补充定理\ref{cor269}.
			\par
			(2)即补充定理\ref{cor270}(1)、补充定理\ref{cor270}(2)、补充定理\ref{cor270}(3)、补充定理\ref{cor270}(4).
			\par
			(3)即补充定理\ref{cor270}(5).
			\par
			(4)即补充定理\ref{cor270}(6).
			\par
			(5)即补充定理\ref{cor270}(7).
			
			\begin{exer}\label{exer118}
				\hfill\par
				(1)$a$、$a'$、$b$、$b'$为序数,求证:如果$a>1$,$b>b'$,则$a^{b'}<a^b$,并且,$a^b$为关于$b$的标准序数函数符号.此外,如果$a'>0$,$a\geq a'$,则${a'}^b\leq a^b$.
				\par
				(2)$a$、$x$、$y$为序数,求证:$a^xa^y=a^{x+y}$,$(a^x)^y=a^{xy}$.
				\par
				(3)$a$、$b$为序数,$a\geq 2$,$b\geq 1$,求证:$a^b\geq ab$.
				\par
				(4)$a$、$b$为序数,$a\geq 2$,$b\geq 1$,求证:存在唯一的序数$c$、$d$、$e$,使$b=a^cd+e$.
			\end{exer}
			证明:
			\par
			(1)	即补充定理\ref{cor272}、补充定理\ref{cor273}.
			\par
			(2)	即补充定理\ref{cor274}.
			\par
			(3)	即补充定理\ref{cor275}.
			\par
			(4)	即补充定理\ref{cor276}.
			
			\begin{exer}\label{exer119}
				\hfill\par
				$a$、$b$为序数,$E$、$F$为良序集,且$a=Ord(E)$,$b=Ord(F)$,令$G=\{g|g\in E^F\text{与}F-\{y|y\in F\text{与}(y, E\text{的最小元})\in g\}\text{为有限集合}\}$,$F^*$为$F$按相反关系排序的偏序集.$E^{F^*}$按公式$(\{i|i\in I\text{与}(x, F, E)(i)\neq (y, F, E)(i)\}\text{是良序集})\text{与}((\exists j)(j\text{是}\{i|i\in I\text{与}(x, F, E)(i)\neq \\(y, F, E)(i)\}\text{的最小元})\text{与}(x, F, E)(i)\leq (y, F, E)(i))$排序,求证:$G$是$E^{F^*}$关于\\“$x\text{和}y\text{是可比较的}$”的连通分量,并且,$G$是良序集,$Ord(G)=a^b$.
			\end{exer}
			证明:
			\par
			令$T_{x,y}$表示集合$\{i|i\in I\text{与}(x, F, E)(i)\neq (y, F, E)(i)\}$,如果$x$和$y$是可比较的,则$T_{x,y}$的任何子集均有最大元和最小元,故$T_{x,y}$是有限集合,反过来,如果$T_{x,y}$是有限集合,则$x$和$y$是可比较的.
			\par
			令$x\in G$,如果$y$和$x$属于同一个连通分量,运用数学归纳法可证$y\in G$.反过来,如果$x\in G$、$y\in G$,则$x$、$y$是可比较的.故$G$是$E^{F^*}$关于“$x\text{和}y\text{是可比较的}$”的联通分量,并且,$G$是全序集.
			\par
			令$F$的最小元为$m$,$I_g=F-\{y|y\in F\text{与}(y, E\text{的最小元})\in g\}$,对$G$的任意子集$Y$,令$A_Y=\bigcup\limits_{g\in Y}I_g$.对于$G$的子集$X$,考虑所有满足$(\forall x)(\forall y)(x\in Y\text{与}y\in X-Y\Rightarrow x<y)$的集合$Y$,则所有$A_Y$均为有限集合,
			\par
			设其中元素数目最小的是$A_Z$,如果$A_Z=\varnothing$,即$x\mapsto m$的图是$G$的最小元;如果$A_Z\neq \varnothing$,令其最小元为$v$,设$\{u|(\exists g)(g\in Z\text{与}(g, F, E)(v)=u)\}$的最小元为$w$,且$x\in Z$、\\$(g, F, E)(v)=u$,若有$y\in Z$、$(g, F, E)(v)=u$、$y\neq x$,则$A\{g|(g, F, E)(v)=u)\}$的元素数目小于$A_Z$的元素数目,矛盾,故$x$为$G$的最小元.
			\par
			如果良序集$E$和$E'$同构、$F$和$F'$同构,则相应的$G$、$G'$同构.令$P_{a,b}$为和$a$、$b$相应的\\$Ord(G)$,则$P_{a,0}=1$,当$b=c+d$时,$P_{a,b}=P_{a,c}P_{a,d}$,同时,当$b<b'$时,$P_{a,b}\leq P_{a,b'}$,因此,$P_{a,b}\geq \mathop{sup}\limits_{i\in ]0, b[}P_{a^i,a}$.如果$P_{a,b}>\mathop{sup}\limits_{i\in ]0, b[}P_{a^i,a}$,设在$G$上,$\mathop{sup}\limits_{i\in ]0, b[}P_{a^i,a}=Ord(S_x)$,$Q=\{i|(i, m)\notin x\}$,设$Q$的最大元是$t$,则$x\in P_{a,Ord[0, t]}$.一方面,$Ord(]0, x])> \mathop{sup}\limits_{i\in ]0, b[}P_{a^i,a}$,另一方面,$P_{a,Ord([0, t])}=P_{a,Ord([0, t[)}a$,$Ord(]0, x])\leq P_{a,t}a$,同时,$Ord([0, t[)<b$,矛盾.故$P_{a,b}=\mathop{sup}\limits_{i\in ]0, b[}P_{a^i,a}$.因此$P_{a,b}=a^b$.
			\par
			注:习题\ref{exer119}涉及尚未介绍的“自然数”知识.
			
			\begin{exer}\label{exer120}
				\hfill\par
				(1)如果$Y$为传递集合,求证:$Y\cup\{Y\}$为传递集合.如果$(Y_i)_{i\in I}$为传递集族,求证:$\bigcup\limits_{i\in I}Y_i$为传递集合,如果$i\neq \varnothing$,求证:$\bigcap\limits_{i\in I}Y_i$为传递集合.
				\par
				(2)求证:伪序数是传递集合也是正式集合.并且,如果$X$是伪序数,$X\cup\{X\}$也是伪序数.
				\par
				(3)$X$、$Y$均为伪序数,求证:$X\subset Y\text{或}Y\subset X$.
				\par
				(4)$X$为传递集合,如果对任意$x\in X$,$x$均为伪序数,求证:$X$为伪序数.
				\par
				(5)求证:$\varnothing$为伪序数.伪序数的每一个元素都是伪序数.
				\par
				(6)令$(X_i)_{i\in I}$是伪序数族,且$i\neq \varnothing$,求证:$\bigcap\limits_{i\in I}X_i是\{Y|(\exists i)(i\in I\text{与}Y=X_i)\}$按包含关系排序的偏序集的最小元.并且,如果$E$为伪序数,则$x\in E\text{与}y\in E\text{与}x\subset y$为良序.
				\par
				(7)求证:对任意序数$a$,存在唯一的伪序数$E_a$,使$Ord(E_a)=a$.并且,序数$0$、$1$、$2$、$3$相应的伪序数分别是$\varnothing$、$\{\varnothing\}$、$\{\varnothing, \{\varnothing\}\}$、$\{\varnothing, \{\varnothing\}, \{\varnothing, \{\varnothing\}\}\}$.
			\end{exer}
			证明:
			\par
			(1)	即补充定理\ref{cor277}.
			\par
			(2)	即补充定理\ref{cor278}、补充定理\ref{cor279}.
			\par
			(3)	即补充定理\ref{cor280}.
			\par
			(4)	即补充定理\ref{cor281}.
			\par
			(5)	即补充定理\ref{cor282}、补充定理\ref{cor283}.
			\par
			(6)	即补充定理\ref{cor284}.
			\par
			(7)	前半部分即补充定理\ref{cor285},后半部分根据定义可证.

		\section{集合等势,基数(Ensembles équipotents, cardinaux)}		
			\begin{de}
				\textbf{等势(équipotent)}
				\par
				如果存在$X$到$Y$的双射,则称$X$和$Y$等势,记作$Eq(X, Y)$.
			\end{de}
			
			\begin{cor}\label{cor286}
				\hfill\par
				(1)$Eq(X, Y)$为等价关系.
				\par
				(2)如果$Eq(X, Y)$,则$\tau_Z(Eq(X, Z))=\tau_Z(Eq(Y, Z))$.
			\end{cor}
			证明:
			\par
			(1)	根据定义可证.
			\par
			(2)	根据补充定理\ref{cor286}(1)、公理模式\ref{Sch7}可证.
			
			\begin{de}
				\textbf{基数(cardinal)}
				\par
				$\tau_Z(Eq(X, Z))$称为$X$的基数,记作$Card(X)$.
			\end{de}
			
			\begin{cor}\label{cor287}
				\hfill\par
				$Eq(Card(X), X)$.
			\end{cor}
			证明:根据公理模式\ref{Sch5}可证.
			
			\begin{theo}\label{theo88}
				\hfill\par
				$Eq(X, Y)\Leftrightarrow Card(X)= Card(Y)$.
			\end{theo}
			证明:根据补充定理\ref{cor287}可证.
			
			\begin{cor}\label{cor288}
				\hfill\par
				$(\exists Z)(Z\subset Y\text{与}Eq(X, Z))\Leftrightarrow (\exists f)(f\text{为}X\text{到}Y\text{的单射})$.
			\end{cor}
			证明:根据定义可证.

			\begin{cor}\label{cor289}
				\hfill\par
				$a$为基数,则$Card(a)=a$.
			\end{cor}
			证明:$a$为基数,故存在$X$,使$Card(X)=a$,根据补充定理\ref{cor287},$Eq(a, X)$,根据定理\ref{theo88},$Card(a)=a$.

			\begin{de}
				\textbf{基数0(cardinal zéro),基数1(cardinal un),基数2(cardinal deux)}
				\par
				$Card(\varnothing)$称为基数$0$;$Card(\{\varnothing\})$称为基数$1$;$Card(\{\varnothing, \{\varnothing\}\})$称为基数$2$.在没有歧义的情况下也可以分别简称为$0$、$1$、$2$.				
			\end{de}
			
			\begin{cor}\label{cor290}
				\hfill\par
				(1)$0=\varnothing$;
				\par
				(2)$Card(A)=0\Leftrightarrow A=\varnothing$;
				\par
				(3)$Card(A)=1\Leftrightarrow (\exists x)(A=\{x\})$;
				\par
				(4)$Card(A)=2\Leftrightarrow (\exists x)(\exists y)(x\neq y\text{与}A=\{x, y\})$.
			\end{cor}
			证明:
			\par
			(1)$(\exists Z)(Z=\varnothing)$,即$\tau_Z(Z=\varnothing))=\varnothing$,又因为$Z=\varnothing\Leftrightarrow Eq(\varnothing, Z)$,因此$0=\varnothing$.
			\par
			(2)	根据定义可证.
			\par
			(3)	根据定义可证.
			\par
			(4)	根据定义可证.

			\begin{theo}\label{theo89}
				\textbf{基数之间的良序关系}
				\par
				$(X\text{为基数})\text{与}(Y\text{为基数})\text{与}(\exists Z)(Z\subset Y\text{与}Eq(X, Z))$为$X$、$Y$之间的良序关系.
			\end{theo}
			证明:
			\par
			根据定义可知其为偏序关系.
			\par
			令$R$为$(\exists Z)(Z\subset Y\text{与}Eq(X, Z))$,对任意元素均为基数的集合$E$,令$A=\bigcup\limits_{X\in E}X$,根据定理\ref{theo78},在$A$上存在良序.
			\par
			根据定理\ref{theo87},任何$A$的子集同构于$A$的一个片段,对任意$X\in E$,考虑和$X$等势的$A$的片段的集合,根据定理\ref{theo73}(2),该集合有最小元,用$f(X)$表示该最小元.
			\par
			当$f(X)\subset f(Y)$时,存在$f(X)$到$f(Y)$的单射,因此存在$X$到$Y$的单射,故$(\exists Z)\\(Z\subset Y\text{与}Eq(X, Z))$;反过来,若$(\exists Z)(Z\subset Y\text{与}Eq(X, Z))$,则存在$X$和$f(Y)$的某个子集等势,此时,若$f(Y)\subset f(X)$且$f(X)\neq f(Y)$,则根据定理\ref{theo87}、定理\ref{theo75},$f(X)$的某个片段和$X$等势,与$f(X)$的定义矛盾,故公式$R\text{与}X\in E\text{与}Y\in E\Leftrightarrow f(X)\subset f(Y)$.根据定理\ref{theo73}(2),$A$的片段集合为良序集,得证.
			
			\begin{sign}
				\textbf{基数之间的不等式(inégalité entre cardinaux)}
				\par
				$(X\text{为基数})\text{与}(Y\text{为基数})\text{与}(\exists Z)(Z\subset Y\text{与}Eq(X, Z))$记作$X\leq Y$或$Y\geq X$.
			\end{sign}

			\begin{cor}\label{cor291}
				\hfill\par
				(1)$X\text{和}Y\text{的子集等势}\Leftrightarrow Card(X)\leq Card(Y)$.
				\par
				(2)$X\subset Y\Rightarrow Card(X)\leq Card(Y)$.
				\par
				(3)$a$为基数,对任意$Y$,如果$a\leq Card(Y)$,则存在$Z\subset Y$,使$Card(Z)=a$.
				\par
				(4)$a$为基数,则$a\geq 0$.
				\par
				(5)$a$为基数,则$a\neq 0\Leftrightarrow a\geq 1$.
			\end{cor}
			证明:
			\par
			(1)根据补充定理\ref{cor288}可证.
			\par
			(2)由于$X$和$X$等势,根据补充定理\ref{cor291}(1),得证.
			\par
			(3)根据补充定理\ref{cor289},$Card(a)=a$,根据补充定理\ref{cor291}(1),$a$和$Y$的子集等势,设该子集为$Z$,得证.
			\par
			(4)令$Card(A)=a$,$\varnothing\subset A$,得证.
			\par
			(5)令$Card(A)=a$,如果$a\neq 0$,则$a\neq \varnothing$,故存在$x\in A$,因此$\{x\}\subset A$;反过来,如果$a\geq 1$,则$a\neq \varnothing$,故$a\neq 0$,得证.
			
			\begin{theo}\label{theo90}
				\hfill\par
				任意两个集合中,必有一个集合和另一个集合的子集等势.
			\end{theo}
			证明:根据定理\ref{theo84}、定理\ref{theo78}可证.

			\begin{theo}\label{theo91}
				\hfill\par
				两个集合之中的任何一个都和另一个集合的子集等势,则两个集合等势.
			\end{theo}
			证明:根据定理\ref{theo86}、定理\ref{theo78}可证.
			
			\begin{cor}\label{cor292}
				\hfill\par
				$(\exists X)(x=Card(X)\text{与}X\subset A)$是$x$上的集合化公式.
			\end{cor}
			证明:由于$X\in \mathcal{P}(A)$,根据证明规则\ref{C53}可证.
			
			\begin{cor}\label{cor293}
				\hfill\par
				$a$为基数,则$(x\text{为基数})\text{与}x\leq a$是$x$上的集合化公式.	
			\end{cor}
			证明:根据定义,$(x\text{为基数})\text{与}x\leq a\Leftrightarrow (\exists X)(x=Card(X)\text{与}X\subset a)$,根据补充定理\ref{cor292}可证.
			
			\begin{theo}\label{theo92}
				\hfill\par
				令$(a_i)_{i\in I}$为基数族,则存在唯一的基数$b$,使对任意$i\in I$,$a_i\leq b$,并且,如果$c$满足对任意$i\in I$,$a_i\leq c$,则$b\leq c$.
			\end{theo}
			证明:令$E$为$(a_i)_{i\in I}$的和,则对任意$i\in I$,$a_i\leq Card(E)$,令$F=\{x|(x\text{为基数})\text{与}x\leq Card(E)\}$,根据定理\ref{theo89},$F$为良序集,且对任意$i\in I$,$a_i\in F$,根据补充定理\ref{cor216},集族$(a_i)_{i\in I}$有最小上界$b$.另一方面,如果$c$满足对任意$i\in I$,$a_i\leq c$,且$c<b$,则与$b$为最小上界矛盾,得证.
			
			\begin{theo}\label{theo93}
				\hfill\par
				如果存在$X$到$Y$的满射,则$Card(Y)\leq Card(X)$.
			\end{theo}
			证明:设满射为$f$,则其右逆$s$为$Y$到$X$的单射,得证.
			
			\begin{cor}\label{cor294}
				\hfill\par
				$f$为$X$到$Y$的映射,则$Card(f\langle X\rangle)\leq Card(X)$.
			\end{cor}
			证明:$f$为$X$到$f\langle X\rangle$的满射,根据定理\ref{theo93}得证.
						
			\begin{de}
				\textbf{基数乘积(produit cardinal),基数和(somme cardinal)}
				\par
				令$(a_i)_{i\in I}$为基数族,则$(a_i)_{i\in I}$的乘积及和的基数,分别称为该基数族的基数乘积及基数和,分别记作$\mathop{\mathsf{P}}\limits_{i\in I}a_i$、$\sum\limits_{i\in I}a_i$.
			\end{de}
			
			\begin{cor}\label{cor295}
				\hfill\par
				令$(E_i)_{i\in I}$为集族,其乘积为$P$,其和为$S$,对任意$i\in I$,$Card(E_i)=a_i$,则$Card(P)=\mathop{\mathsf{P}}\limits_{i\in I}a_i$,$Card(S)=\sum\limits_{i\in I}a_i$.
			\end{cor}
			证明:对任意$i\in I$,存在$E_i$到$a_i$的双射,根据定理\ref{theo35}、定理\ref{theo55},存在$P$到$\mathop{\mathsf{P}}\limits_{i\in I}a_i$、$S$到\\$\sum\limits_{i\in I}a_i$的双射,得证.
						
			\begin{cor}\label{cor296}
				\hfill\par
				$(a_i)_{i \in \varnothing}$为基数族,则$\mathop{\mathsf{P}}\limits_{i\in \varnothing}a_i=1$;$\sum\limits_{i\in \varnothing}a_i=0$.
			\end{cor}
			证明:根据补充定理\ref{cor116}(4)、补充定理\ref{cor127}(1)可证.
									
			\begin{theo}\label{theo94}
				\hfill\par
				(1)令$(E_i)_{i\in I}$为集族,则$\bigcup\limits_{i\in I}E_i\leq \sum\limits_{i\in I}Card(E_i)$.
				\par
				(2)令$(E_i)_{i\in I}$为两两不相交的集族,则$\bigcup\limits_{i\in I}E_i=\sum\limits_{i\in I}Card(E_i)$.
			\end{theo}
			证明:
			\par
			(1)根据补充定理\ref{cor116}(1)可证.
			\par
			(2)根据定理\ref{theo35}可证.

			\begin{theo}\label{theo95}
				\textbf{基数的和与乘积的交换律、结合律、分配律}
				\par
				(1)令$(a_i)_{i\in I}$为基数族,$f$为$K$到$I$的双射,则$\sum\limits_{k\in K}a_{f(k)}=\sum\limits_{i\in I}a_i$,$\mathop{\mathsf{P}}\limits_{k\in K}a_{f(k)}=\mathop{\mathsf{P}}\limits_{i\in I}a_i$.
				\par
				(2)令$(a_i)_{i\in I}$为基数族,$(J_l)_{l\in L}$为$I$的划分,则$\sum\limits_{i\in I}a_i=\sum\limits_{l\in L}(\sum\limits_{i\in J_l}a_i)$,$\mathop{\mathsf{P}}\limits_{i\in I}a_i=\\\mathop{\mathsf{P}}\limits_{l\in L}(\mathop{\mathsf{P}}\limits_{i\in J_l}a_i)$.
				\par
				(3)令$((a_{l,i})_{i\in J_l})_{l\in L}$为基数族,令$I=\mathop{\mathsf{P}}\limits_{l\in L}J_l$,则$\mathop{\mathsf{P}}\limits_{l\in L}(\sum\limits_i\in J_lali)=\sum\limits_{f\in I}(\mathop{\mathsf{P}}\limits_{l\in L}a_{l,f(l)})$.
			\end{theo}
			证明:
			\par
			(1)根据定理\ref{theo23}、定理\ref{theo40}可证.
			\par
			(2)根据定理\ref{theo25}、定理\ref{theo46}、定理\ref{theo47}可证.
			\par
			(3)根据定理\ref{theo49}、定理\ref{theo50}可证.
			
			\begin{de}
				\textbf{两个基数的乘积(produit de deux cardinaux),两个基数的和(somme de deux cardinaux)}
				\par
				$x\neq y$,$a$,$b$为基数,基数族$(a_i)_{i\in \{x, y\}}$中,$a_x=a$,$a_y=b$,则$\mathop{\mathsf{P}}\limits_{i\in \{x, y\}}a_i$称为$a$和$b$的乘积,记作$ab$或者$a\cdot b$,$\sum\limits_{i\in \{x, y\}}a_i$称为$a$和$b$的和,记作$a+b$.
			\end{de}
			
			\begin{theo}\label{theo96}
				\textbf{两个基数的和与乘积的交换律、结合律、分配律}
				\par
				$a$、$b$、$c$为基数,则:
				\par
				(1)$a+b=b+a$;
				\par
				(2)$(a+b)+c=a+(b+c)$;
				\par
				(3)$ab=ba$;
				\par
				(4)$(ab)c=a(bc)$;
				\par
				(5)$(a+b)c=ac+bc$.
			\end{theo}
			证明:根据定理\ref{theo95}可证.
			
			\begin{theo}\label{theo97}
				\hfill\par
				令$(a_i)_{i\in I}$为基数族,$J\subset I$,对于$i\in I$但$i\notin J$,$a_i=0$(或$a_i=1$),则$\sum\limits_{i\in J}a_i=\sum\limits_{i\in I}a_i$(或$\mathop{\mathsf{P}}\limits_{i\in J}a_i=\mathop{\mathsf{P}}\limits_{i\in I}a_i$).
			\end{theo}
			证明:
			\par
			对于加法,$i\in I-J$时,$a_i=\varnothing$,根据定理\ref{theo95}(2)可证.
			\par
			对于乘法,$i\in I-J$时,$a_i$为单元素集合,根据补充定理\ref{cor138},$\prod\limits_{i\in J}a_i$到$\prod\limits_{i\in I}a_i$存在一一对应,得证.
						
			\begin{theo}\label{theo98}
				\hfill\par
				$a$为基数,则$a+0=a$,$0+a=a$,$a\cdot 1=a$,$1\cdot a=a$.
			\end{theo}
			证明:根据定理\ref{theo97}可证.
						
			\begin{theo}\label{theo99}
				\hfill\par
				令$(a_i)_{i\in I}$、$(c_i)_{i\in I}$为基数族,$a$、$b$为基数,$b=Card(I)$,并且对任意$i\in I$,$a_i=a$,$c_i=1$,则$ab=\sum\limits_{i\in I}a_i$,$b=\sum\limits_{i\in I}c_i$.
			\end{theo}
			证明:$c_i=1$,故$c_i$为单元素集合,故存在$I$到$\bigcup\limits_{i\in I}c_i$的双射,因此第二式成立.第一式根据定理\ref{theo95}(3)、定理\ref{theo98}可证.
						
			\begin{theo}\label{theo100}
				\hfill\par
				令$(a_i)_{i\in I}$为基数族,$\mathop{\mathsf{P}}\limits_{i\in I}a_i\neq 0$,则对任意$i\in I$,$a_i\neq 0$.
			\end{theo}
			证明:根据定理\ref{theo44}可证.
						
			\begin{theo}\label{theo101}
				\hfill\par
				$a$、$b$为基数,如果$a+1=b+1$,则$a=b$.
			\end{theo}
			证明:令$X=a+1$,则$X=b+1$,则存在$A\subset X$、$B\subset X$,使$Card(A)=a$,$Card(B)=b$,故$Card(X-A)=1$,$Card(X-B)=1$,因此$X-A=\{x\}$、$X-B=\{y\}$.如果$x=y$,则$A=B$,故$a=b$;如果$x\neq y$,令$C=A\cap B$,则$A=C\cup \{y\}$,$B=C\cup\{x\}$,因此$a=Card(C)+1$,$b=Card(C)+1$,得证.
						
			\begin{cor}\label{cor297}
				\hfill\par
				$a$、$b$为基数,则$a+1=b+1\Leftrightarrow a=b$.
			\end{cor}
			证明:根据公理模式\ref{Sch7}、定理\ref{theo101}可证.
						
			\begin{de}
				\textbf{基数幂(exponentiation des cardinaux)}
				\par
				$a$、$b$为基数,则$a$到$b$的映射集合的基数,称为$b$的$a$次基数幂,记作$b^a$.
			\end{de}
			
			\begin{theo}\label{theo102}
				\hfill\par
				$Card(X)=a$,$Card(Y)=b$,则$Card(X^Y)=a^b$.
			\end{theo}
			证明:根据定理\ref{theo38}可证.
			
			\begin{theo}\label{theo103}
				\hfill\par
				令$(a_i)_{i\in I}$为基数族,$a$、$b$为基数,$b=Card(I)$,并且对任意$i\in I$,$a_i=a$,则$a^b=\mathop{\mathsf{P}}\limits_{i\in I}a_i$.
			\end{theo}
			证明:根据集合的乘积的定义可证.
						
			\begin{theo}\label{theo104}
				\hfill\par
				令$a$为基数,$(b_i)_{i\in I}$为基数族,则$a^{(\sum\limits_ {i\in I} b_i)}=\mathop{\mathsf{P}}\limits_{i\in I}(a^{b_i})$.
			\end{theo}
			证明:令$S$为$(b_i)_{i\in I}$的和,当$s\in S$时,令$a_s=a$,根据定理\ref{theo103},左边等于$\mathop{\mathsf{P}}\limits_{s\in S}a_s$,根据定理\ref{theo95}(2),右边等于$\mathop{\mathsf{P}}\limits_{s\in S}a_s$,得证.
						
			\begin{theo}\label{theo105}
				\hfill\par
				令$(a_i)_{i\in I}$为基数族,$b$为基数,则$\mathop{\mathsf{P}}\limits_{i\in I}{a_i}^b=(\mathop{\mathsf{P}}\limits_{i\in I}a_i)^b$.
			\end{theo}
			证明:对于$(x, y)\in I\times b$,令$a_{x,y}=a_x$,根据定理\ref{theo95}(2),$\mathop{\mathsf{P}}\limits_{i\in I}{a_i}^b=\mathop{\mathsf{P}}\limits_{y\in b}(\mathop{\mathsf{P}}\limits_{x\in I}a_{x,y})$,等于$\mathop{\mathsf{P}}\limits_{(x, y)\in I\times b}a_{x,y}$,等于$\mathop{\mathsf{P}}\limits_{x\in I}(\mathop{\mathsf{P}}\limits_{y\in b}a_{x,y})$,等于$(\mathop{\mathsf{P}}\limits_{i\in I}a_i)^b$.
						
			\begin{theo}\label{theo106}
				\hfill\par
				$a$、$b$、$c$为基数,则$a^{bc}=(a^b)^c$.
			\end{theo}
			证明:对任意$i\in c$,令$b_i=b$,则$a^{bc}=a^{\sum\limits_{i\in c} b_i}$,根据定理\ref{theo104},等于$\mathop{\mathsf{P}}\limits_{i\in c}a^{b_i}$,等于$\mathop{\mathsf{P}}\limits_{i\in c}a^b$,根据定理\ref{theo103},等于$(a^b)c$.
						
			\begin{theo}\label{theo107}
				\hfill\par
				令$a$为基数,则$a^0=1$,$a^1=a$,$1^a=1$,如果$a\neq 0$,则$0^a=1$.
			\end{theo}
			证明:根据补充定理\ref{cor127}(1)、补充定理\ref{cor132}及定义可证.
			
			\begin{theo}\label{theo108}
				\hfill\par
				$Card(X)=a$,则$Card(\mathcal{P}(X))=2^a$.
			\end{theo}
			证明:令$A=\{a, b\}$.对任意$Y\subset X$,令$f_Y$为$X$到$A$的映射,对于$x\in X$,如果$x\in Y$,则$f_Y(x)=a$,如果$x\in X-Y$,则$f_Y(x)=b$,令$u$为映射$Y\mapsto f_Y(Y\in \mathcal{P}(X), f_Y\in A^X)$.反过来,对任意$X$到$A$的映射$g$,$g^{-1}(a)\subset X$,令$v$为映射$g\mapsto g^{-1}(a)(g\in A^X, g^{-1}(a)\in \mathcal{P}(X))$,则$u\circ v$和$v\circ u$均为恒等映射,根据定理\ref{theo20},$u$、$v$均为双射,故$Card(\mathcal{P}(X))= Card(A^X)$,得证.
						
			\begin{theo}\label{theo109}
				\hfill\par
				$a$、$b$均为基数,则当且仅当存在基数$c$使$a=b+c$时,$a\geq b$.
			\end{theo}
			证明:$a\geq b\Leftrightarrow (\exists B)(B\subset a\text{与}Card(B)=b)$,令$c=Card(a-B)$,因此等价于$a=b+c$.
						
			\begin{theo}\label{theo110}
				\hfill\par
				令$(a_i)_{i\in I}$、$(_bi)_{i\in I}$为基数族,$(\forall I)(i\in I\Rightarrow a_i\geq b_i)$,则$\sum\limits_{i\in I}a_i\geq \sum\limits_{i\in I}b_i$,$\mathop{\mathsf{P}}\limits_{i\in I}a_i\geq \sum\limits_{i\in I}b_i$.
			\end{theo}
			证明:根据定理\ref{theo45}、补充定理\ref{cor97}(1)可证.
			
			\begin{cor}\label{cor298}
				\hfill\par
				令$a$、$b$、$c$为基数,$a\leq b\Rightarrow a+c\leq b+c$,$a\leq b\Rightarrow ac\leq bc$.
			\end{cor}
			证明:根据定理\ref{theo110}可证.
						
			\begin{theo}\label{theo111}
				\hfill\par
				令$(a_i)_{i\in I}$为基数族,$J\subset I$,则$\sum\limits_{i\in J}a_i\leq \sum\limits_{i\in I}a_i$;如果对任意$i\in I-J$,$a_i\neq 0$,则$\mathop{\mathsf{P}}\limits_{i\in J}a_i\leq \mathop{\mathsf{P}}\limits_{i\in I}a_i$.
			\end{theo}
			证明:令$b_i=a_i$($i\in J$),$b_i=0$(或$b_i=1$)($i\in I-J$),根据定理\ref{theo110}可证.
						
			\begin{theo}\label{theo112}
				\hfill\par
				$a$、$a'$、$b$、$b'$均为基数,$a\leq a'$,$b\leq b'$,则$a^b\leq {a'}^{b'}$.
			\end{theo}
			证明:根据定理\ref{theo111}、定理\ref{theo103}可证.
						
			\begin{theo}\label{theo113}
				\textbf{康托尔定理}
				\par
				$a$为基数,则$2^a>a$.
			\end{theo}
			证明:由于$x\mapsto \{x\}$为$a$到$\mathcal{P}(a)$的单射,故$a\leq 2^a$.
			\par
			对任意$a$到$\mathcal{P}(a)$的映射$f$,令$X$为$a$的元素中满足$x\notin f(x)$的元素集合,如果$x\notin X$,则$f(x)\neq X$,如果$x\in f(x)$,则$x\in a-X$,故$f(x)\neq X$,因此,$X\notin f(a)$,因此不存在$a$到$\mathcal{P}(a)$的满射,故$2^a>a$.
						
			\begin{theo}\label{theo114}
				\textbf{所有基数不能组成集合}
				\par
				$\text{非}Coll_x(x\text{为基数})$.	
			\end{theo}
			证明:如果存在这样的集合$U$,令$S$为$(X)_{X\in U}$的和,则任何基数都和$S$的某个子集等势.令$s=Card(S)$,则$s<2^s$,但$2^s$也是基数,矛盾,
			
			\begin{cor}\label{cor299}
				\hfill\par
				(1)1+1=2.
				\par
				(2)a+a=2a.
			\end{cor}
			证明:
			\par
			(1)令$A=\{x\}$,$B=\{y\}$,其中$x\neq y$,则$A\cup B=\{x, y\}$,$A\cap B=\varnothing$,根据定理\ref{theo94}(2)可证.
			\par
			(2)根据补充定理\ref{cor299}(1)、定理\ref{theo96}(5)可证.
						
			\begin{cor}\label{cor300}
				\hfill\par
				令$(a_i)_{i\in I}$、$(b_i)_{i\in I}$为基数族,对任意$i\in I$,$b_i\geq 2$:
				\par
				(1)对任意$i\in I$,$a_i\leq b_i$,则$\sum\limits_{i\in I}a_i\leq \mathop{\mathsf{P}}\limits_{i\in I}b_i$.
				\par
				(2)对任意$i\in I$,$a_i<b_i$,则$\sum\limits_{i\in I}a_i<\mathop{\mathsf{P}}\limits_{i\in I}b_i$.
			\end{cor}
			证明:
			\par
			(1)对任意$i\in I$,令$p_i$为$a_i$到$b_i$的单射,$p$为映射$x\mapsto p_{pr_2x}(pr_1x)$.
			\par
			如果$Card(I)=0$,根据补充定理\ref{cor296},命题成立.
			\par
			如果$Card(I)=1$,令$I=\{i\}$,则$\sum\limits_{i\in I}a_i=a_i$,$\mathop{\mathsf{P}}\limits_{i\in I}b_i=b_i$,命题成立.
			\par
			如果$Card(I)=2$,令$I=\{i, j\}$,令$c_i=b_{i-1}$,$c_j=b_{j-1}$,则$b_ib_j\geq c_ic_j+c_i+c_j+1$,大于等于$b_i+b_j$,大于等于$a_i+a_j$,命题成立.
			\par
			如果$Card(I)>2$,令$f$为映射$i\mapsto \tau_x(x\in b_i)$,$g$为映射$i\mapsto \tau_x(x\in b_i-f(i))$.令$A$\\为$(a_i)_{i\in I}$的和,对任意$x\in A$,如果$p(x)\neq f(pr_2x)$,令$h(x)=(\bigcup\limits_{i\in I-pr_2x}\{(f(i), i)\})\cup\\\{f(pr_2x), pr_2x\}$,如果$p(x)=f(pr_2x)$,令$h(x)=(\bigcup\limits_{i\in I-pr_2x}\{(g(i), i)\})\cup\{f(pr_2x), pr_2x\}$,根据定义,$h$为$A$到$\prod\limits_{i\in I}a_i$的单射,故$\sum\limits_{i\in I}a_i\leq\mathop{\mathsf{P}}\limits_{i\in I}b_i$.
			\par
			(2)类似补充定理\ref{cor300}(1)可证.

			\begin{sign}
				\textbf{序数的第一射影的基数(cardinal de la première projection d'une ordinal)}
				\par
				令$a$为序数,在没有歧义的情况下,$Card(pr_1a)$可以简记为$Card(a)$.
			\end{sign}
			
			\begin{cor}\label{cor301}
				\hfill\par
				(1)$E$、$F$为良序集,且$E$同构于$F$,则$Card(E)=Card(F)$.
				\par
				(2)$Card(\text{序数}0)=\text{基数}0$;$Card(\text{序数}1)=\text{基数}1$.
				\par
				(3)$(l_i)_{i\in I}$为序数族,$I$为良序集,则$Card(\sum\limits_{i\in I}l_i)= \sum\limits_{i\in I}Card(l_i)$,$Card(\mathop{\mathsf{P}}\limits_{i\in I}l_i)=\\\mathop{\mathsf{P}}\limits_{i\in I}Card(l_i)$.
				\par
				(4)$Card(\text{序数}2)=\text{基数}2$.
				\par
				(5)$a$、$b$为序数,且$Card(a)<Card(b)$,则$a<b$.
				\par
				(6)序数$a\leq b$,则$Card(a)\leq Card(b)$.
			\end{cor}
			证明:
			\par			
			(1)根据定义可证.
			\par
			(2)根据补充定理\ref{cor248}(2)、补充定理\ref{cor248}(3)、补充定理\ref{cor290}(2)、补充定理\ref{cor290}(3)可证.
			\par
			(3)根据定义可证.
			\par
			(4)根据补充定理\ref{cor299}(1)、补充定理\ref{cor301}(3)可证.
			\par
			(5)根据定义可证.
			\par
			(6)根据定义可证.
			
			\begin{cor}\label{cor302}
				\hfill\par
				$(\forall E)(\exists X)(X\subset E\text{与}X \notin E)$.
			\end{cor}
			证明:根据定理\ref{theo113}可证.
			
			\begin{exer}\label{exer121}
				\hfill\par
				$f$为$E$到$F$的映射,$g$为$F$到$E$的映射,求证:存在$E$的子集$A$、$B$以及$F$的子集$A'$、$B'$,使$B=E-A$,$B'=F-A'$,且$A'=f\langle A\rangle$,$B=g\langle B'\rangle$.
			\end{exer}
			证明:考虑$X\mapsto E-g\langle F-f\langle X\rangle\rangle(X\in \mathcal{P}(E))$,根据习题\ref{exer63},存在$A$使$A= E-g\langle F-f\langle A\rangle\rangle$,令$A'=f\langle A\rangle$,$B'=F-f\langle A\rangle$,$B=E-A$,则$B=g\langle B'\rangle$.
			\par
			注:原书习题\ref{exer121}中$f$为单射的条件,是多余的.
			
			\begin{exer}\label{exer122}
				\hfill\par
				$E$和$F$是不同的集合,求证:$E^F\neq F^E$,并且,如果$Card(E)=2$,$Card(F)=2+2$,则$E^F$、$F^E$至少有一个不是基数.
			\end{exer}
			证明;如果$E$、$F$其中一个为$\varnothing$,显然成立.
			\par
			如果均不为$\varnothing$,且$E^F=F^E$,对任意$G\in F^E$,$pr_1G=E$,由于$G\in E^F$,故$pr_1G=F$,因此$E=F$,矛盾.
			\par
			如果$Card(E)=2$,$Card(F)=2+2$,由于$Card(E^F)=Card(F^E)$,故$E^F$、$F^E$至少有一个不是基数.其中至少有一个不是基数.
			
			\begin{exer}\label{exer123}
				\hfill\par
				令$(a_i)_{i\in I}$、$(b_i)_{i\in I}$为基数族,对任意$i\in I$,$b_i\geq 2$:
				\par
				(1)对任意$i\in I$,$a_i\leq b_i$,求证:$\sum\limits_{i\in I}a_i\leq \mathop{\mathsf{P}}\limits_{i\in I}b_i$.
				\par
				(2)对任意$i\in I$,$a_i<b_i$,求证:$\sum\limits_{i\in I}a_i<\mathop{\mathsf{P}}\limits_{i\in I}b_i$.
			\end{exer}
			证明:即补充定理\ref{cor300}.
			
			\begin{exer}\label{exer124}
				\hfill\par
				令$f$为$\mathcal{P}(E)-\{\varnothing\}$到$E$的映射,并且,对任意$X\subset E$且$X\neq \varnothing$,$f(X)\in X$:
				\par
				(1)$b$为基数,$A=\{x|x\in E\text{与}Card(f^{-1}\langle x\rangle))\leq b\}$,令$a=Card(A)$,求证:$2^a\leq 1+ab$.
				\par
				(2)$B=\{x|x\in E\text{与}(X\in f^{-1}\langle x\rangle)\Rightarrow Card(X) \leq b)\}$,求证:$Card(B)\leq b$.
			\end{exer}
			证明:
			\par
			(1)令$Y=\bigcup\limits_{x\in A}f^{-1}\langle x\rangle$,则$Card(Y)\leq ab$,对任意$y\in \mathcal{P}(A)-\{\varnothing\}$,$f(y)\in A$,故$y\in Y$,因此$\mathcal{P}(A)-\{\varnothing\}\subset Y$,所以$Card(\mathcal{P}(A))\leq 1+ab$,得证.
			\par
			(2)令$x=f(B)$,则$x\in B$,故$Card(B)\leq b$.
			
			\begin{exer}\label{exer125}
				\hfill\par
				$(l_i)_{i\in I}$为序数族,$I$为良序集,求证:$Card(\sum\limits_{i\in I}l_i)= \sum\limits_{i\in I}Card(l_i)$,$Card(\mathop{\mathsf{P}}\limits_{i\in I}l_i)=\\\mathop{\mathsf{P}}\limits_{i\in I}Card(l_i)$.
			\end{exer}
			证明:即补充定理\ref{cor301}(3).
			
			\begin{exer}\label{exer126}
				\hfill\par
				求证:$(\forall E)(\exists X)(X\subset E\text{与}X \notin E)$.
			\end{exer}
			证明:即补充定理\ref{cor302}.

		\section{自然数,有限集合(Entiers naturels, ensembles finis)}		
			\begin{de}
				\textbf{有限基数(cardinal fini),自然数(entier naturel),有限集合(ensemble finie),元素数目(nombre d'éléments),数目(nombre),有限族(famille finie)
				}
				\par
				对于基数$a$,如果$a+1\neq a$,则称$a$为有限基数,又称自然数.如果$Card(E)$为自然数,则称$E$为有限集合,此时,$Card(E)$称为$E$的元素数目.在没有歧义的情况下,某类项的集合的元素数目,也可以简称为该类项的数目.如果族的指标集是有限集,则称该族为有限族.
			\end{de}
			注:原书常用“entier”表示自然数,在第二卷引入“整数”时,则称作“entier rationnel”.
			
			\begin{theo}\label{theo115}
				\hfill\par
				当且仅当$a+1$为自然数时,$a$为自然数.
			\end{theo}
			证明:根据补充定理\ref{cor297},$a+1=b+1\Leftrightarrow a=b$,因此$a\neq a+1\Leftrightarrow a+1\neq a+1+1$,得证.
						
			\begin{theo}\label{theo116}
				\hfill\par
				$n$为自然数,基数$a\leq n$,则$a$为自然数;并且,如果$n\neq 0$,则存在唯一的自然数$m$使$n\\=m+1$,且$a<n\Leftrightarrow a\leq m$.
			\end{theo}
			证明:由于$a\leq n$,因此存在$b$使$n=a+b$,由于$n\neq n+1$,因此$a+1+b\neq a+b$,因此$a+1\neq a$,故$a$为自然数.如果$n\neq 0$,则$n\geq 1$,根据定理\ref{theo109},存在自然数$m$使$n=m+1$;根据定理\ref{theo101},$m$是唯一的.
			\par
			令$a<n$时,设$n=a+b$,且$b\neq 0$,由于$b\leq n$,故$b$为自然数,因此存在$c$使$b=c+1$,因此$m=a+c$,因此$a\leq m$,得证.
						
			\begin{theo}\label{theo117}
				\hfill\par
				有限集合的的子集是有限集合.
			\end{theo}
			证明:根据定理\ref{theo116}可证.
									
			\begin{theo}\label{theo118}
				\hfill\par
				$X$为有限集合$E$的子集,且$X\neq E$,则$Card(X)<Card(E)$.
			\end{theo}
			证明:设$a\in E-X$,令$Y=E-\{a\}$,则$Card(X)\leq Card(Y)$,同时$Card(Y)+1=Card(E)$,根据定理\ref{theo116}得证.
									
			\begin{theo}\label{theo119}
				\hfill\par
				$f$为有限集合$E$到$F$的映射,则$f(E)$为有限集合.
			\end{theo}
			证明:根据定理\ref{theo93},$Card(f(E))\leq Card(E)$,根据定理\ref{theo116}得证.
									
			\begin{theo}\label{theo120}
				\hfill\par
				$E$和$F$的元素数目相同,$f$为$E$到$F$的映射,则以下三个公式等价:
				\par
				第一,f是单射;
				\par
				第二,f是满射;
				\par
				第三,f是双射.
			\end{theo}
			证明:若$f$是单射,则$Card(f(E))=Card(E)$,等于$Card(F)$,根据定理\ref{theo118},$f$为满射;反过来,若$f$不是单射,设存在$x\neq x'$使$f(x)=f(x')$,则$f(E-\{x\})=f(E)$,故$Card(f(E-\{x\}))\leq Card(E-\{x\})$,又因为$Card(E-\{x\})<Card(E)$,故$f(E)<Card(E)$,因此$f$不是满射.综上得证.
			
			\begin{C}\label{C61}
				\textbf{数学归纳法}
				\par
				包含$2$元特别符号$\in$ 、显式公理\ref{ex1}、显式公理\ref{ex2}、显式公理\ref{ex3}和公理模式\ref{Sch8}的等式理论$M$中,$R$为公式,$n$不是常数,如果“$(0|n)R\text{与}(\forall n)(n\text{为自然数}\text{与}R\Rightarrow (n+1|n)R)$”是$M$的定理,则$(\forall n)(n\text{为自然数}\Rightarrow R)$是$M$的定理.
			\end{C}
			证明:假设$(\exists n)(n\text{为自然数}\text{与}\text{非}R)$为真,令$q$为自然数,且$\text{非}(q|n)R$.由于集合\\$\{n|n\text{为自然数}\text{与}n\leq q\text{与}R\}$为非空良序集,故有最小元$s$,如果$s=0$,矛盾;如果$s>0$,令$s=s'+1$,根据定理\ref{theo116},$s'<s$,因此$(s'|n)R$为真,故$(s|n)R$为真,矛盾.
			
			\begin{Ccor}\label{Ccor87}
				\hfill\par
				包含$2$元特别符号$\in$ 、显式公理\ref{ex1}、显式公理\ref{ex2}、显式公理\ref{ex3}和公理模式\ref{Sch8}的等式理论$M$中,$R$为公式,$n$、$p$不是常数,令$S$为$(\forall p)(n\text{为自然数}\text{与}p\text{为自然数}\text{与}p<n\Rightarrow (p|n)R)$,如果$S\Rightarrow R$是$M$的定理,则$(\forall n)(n\text{为自然数}\Rightarrow R)$是$M$的定理.
			\end{Ccor}
			证明:$(0|n)S$为真,故$(0|n)R$为真.设$R$对$n$成立,由于$(n+1|n)S\Leftrightarrow S\text{与}R$,故$(n+1|n)S$为真,因此$(n+1|n)R$为真,根据证明规则\ref{C61}得证.
						
			\begin{Ccor}\label{Ccor88}
				\hfill\par
				包含$2$元特别符号$\in$ 、显式公理\ref{ex1}、显式公理\ref{ex2}、显式公理\ref{ex3}和公理模式\ref{Sch8}的等式理论$M$中,$R$为公式,$n$不是常数,$k$为自然数,如果$(k|n)R\text{与}(\forall n)(n\text{为自然数}\text{与}n\geq k\text{与}R\Rightarrow (n+1|n)R)$是M的定理,则$(\forall n)(n\text{为自然数}\text{与}n\geq k\Rightarrow R)$是$M$的定理.
			\end{Ccor}
			证明:令$m=n-k$,根据证明规则\ref{C61}可证.
			
			\begin{Ccor}\label{Ccor89}
				\hfill\par
				包含$2$元特别符号$\in$ 、显式公理\ref{ex1}、显式公理\ref{ex2}、显式公理\ref{ex3}和公理模式\ref{Sch8}的等式理论$M$中,$R$为公式,$n$不是常数,$a$、$b$为自然数,如果$(a|n)R\text{与}(\forall n)(n\text{为自然数}\text{与}a\leq n\text{与}n<b\text{与}R \Rightarrow (n+1|n)R)$是$M$的定理,则$(\forall n)(n\text{为自然数}\text{与}a\leq n\text{与}n\leq b \Rightarrow R)$是$M$的定理.
			\end{Ccor}
			证明:令$S$为$a\leq n\text{与}n\leq b \Rightarrow R$,根据证明规则\ref{C61},$S$为真,故$(\forall n)(n\text{为自然数}\text{与}a\leq n\text{与}n\leq b \Rightarrow R)$.
						
			\begin{Ccor}\label{Ccor90}
				\hfill\par
				包含$2$元特别符号$\in$ 、显式公理\ref{ex1}、显式公理\ref{ex2}、显式公理\ref{ex3}和公理模式\ref{Sch8}的等式理论$M$中,$R$为公式,$n$不是常数,$a$、$b$为自然数,如果$(b|n)R\text{与}(\forall n)(n\text{为自然数}\text{与}a\leq n\text{与}n<b\text{与}(n+1|n)R \Rightarrow R)$是$M$的定理,则$(\forall n)(n\text{为自然数}\text{与}a\leq n\text{与}n\leq b \Rightarrow R)$是$M$的定理.
			\end{Ccor}
			证明:$(\forall n)(n\text{为自然数}\text{与}a\leq n\text{与}n<b\text{与}(n+1|n)R \Rightarrow R)\Leftrightarrow (\forall n)(n\text{为自然数}\text{与}a\leq n\text{与}n<b\text{与}\text{非}R \Rightarrow \text{非}(n+1|n)R)$.如果存在$k$,$a\leq k\text{与}k\leq b$,并且$\text{非}(k|n)R$为真,则根据补充证明规则\ref{Ccor89},$\text{非}(b|n)R$为真,矛盾.
						
			\begin{theo}\label{theo121}
				\hfill\par
				令$E$为右方有向集(或左方有向集、格、全序集),则$E$的非空有限子集有上界(或有下界、有最小上界和最大下界、有最大元和最小元).
			\end{theo}
			证明:考虑右方有向集的情况,令$n$为$1$,命题显然成立.设公式对$n$成立,对于设元素数目为$n+1$的子集$X$,设$x\in X$,令$Y=X-\{x\}$,则$Y$有上界$y$,又由于$\{x, y\}$有上界,该上界也是$X$的上界,得证.其他情况同理可证.
						
			\begin{theo}\label{theo122}
				\hfill\par
				任何非空全序有限集合是有最大元的良序集.
			\end{theo}
			证明:根据定理\ref{theo121}可证.
						
			\begin{theo}\label{theo123}
				\hfill\par
				任何非空偏序有限集合有极大元.
			\end{theo}
			证明:根据定理\ref{theo121},该集合为归纳集,根据定理\ref{theo80}可证.
			
			\begin{cor}\label{cor303}
				\hfill\par
				任何非空偏序有限集合有极小元.
			\end{cor}
			证明:考虑按相反关系排序的偏序集,根据定理\ref{theo123}可证.
						
			\begin{de}
				\textbf{有限性(caractère fini)}
				\par
				$F$的元素都是$E$的子集,如果$X\in F\Leftrightarrow (\forall Y)((Y\subset X)\text{与}(Y\text{为有限集合})\Rightarrow (Y\in F))$,则称$F$具有有限性.
			\end{de}
						
			\begin{theo}\label{theo124}
				\hfill\par
				$F$的元素都是$E$的子集,且具有有限性,则$F$按包含关系排序的非空偏序集有极大元.
			\end{theo}
			证明:令$G$为$F$的全序子集,设$G$的元素的并集为$X$,$X$的任意有限子集$Y$当中的元素$y$,存在$Z_y\in G$使$y\in Z_y$,根据定理\ref{theo122},$Z_y$($y\in Y$)的集合为全序有限集合,因此有最大元$S$,故$Y\subset S$且$S\in G$,由于$S\in F$且$F$具有有限性,因此$Y\in F$,由于$F$具有有限性,因此$X\in F$.根据定理\ref{theo82},$F$有极大元.
						
			\begin{cor}\label{cor304}
				\hfill\par
				当且仅当任意按包含关系排序的$\mathcal{P}(E)$的非空子集均有极大元时,$E$为有限集合.
			\end{cor}
			证明:必要性根据定理\ref{theo123}可证.
			\par
			充分性:令$F$为$E$的有限子集的集合,假设其有极大元$X$,如果$X\neq E$,则存在$a\in X-E$,$X\cup\{a\}$也是$E$的有限子集,矛盾,故$X=E$,得证.
						
			\begin{cor}\label{cor305}
				\hfill\par
				$E$为良序集,$E$按相反关系排序的偏序集也是良序集,则$E$为有限集合.
			\end{cor}
			证明:如果$E=\varnothing$,命题显然成立;如果$E\neq \varnothing$,令$S=\{x|]\gets, x[\text{为有限集合}\}$,则$S\neq \varnothing$.设$S$的最大元为$y$,则$]\gets, y[gcup\{y\}$为有限集合,如果$E-(]\gets, y[\cup\{y\})\neq \varnothing$,则其有最小元$z$,则$] \gets, z[$为有限集合,故$z\in S$,且$z>y$,矛盾.因此,$E=]\gets, y[\cup\{y\}$,得证.
			
			\begin{de}
				\textbf{最长反链的长度(longueur de l'antichaîne la plus longue),最长反链(antichaîne la plus longue)}
				\par
				$E$为偏序集,$A=\{a|(\exists X)((X\text{为}E\text{的自由子集})\text{与}Card(X)=a)\}$,如果$A$有最大元自然数$k$,则称$A$为$E$的最长反链的长度.对$E$的任意自由子集$X$,如果$Card(X)=k$,则称$X$为$E$\\的最长反链.
			\end{de}
				
			\begin{cor}\label{cor306}
				\textbf{狄尔沃斯定理}
				\par
				$E$为偏序集,其最长反链的长度为自然数$k$,则存在$E$的划分$\Delta_F$使$Card(F)=k$,并且,$F$的元素均为$E$的全序子集.
			\end{cor}
			证明:
			\par
			如果$E$为有限集合,设$E$的元素数目为$n$,对$n$用数学归纳法:
			\par
			令$a$为$E$的极小元.对于$E-\{a\}$:
			\par
			如果$E$的最长反链的长度为$k$,根据归纳假设,存在$E-\{a\}$的划分$\Delta_F$使$Card(F)=k$,且$F$的个元素均为全序集.令$X={x|(\exists G)(G\in F\text{与}x\text{为}G\text{的最小元})}$,则$Card(X)=k$,故$a$和$X$的某个元素$y$是可比较的,设$y\in G$、$G\in F$,则将$a$加入$G$,可得到符合条件的划分.
			\par
			如果果$E$的最长反链的长度为$k-1$,根据归纳假设,存在$E-\{a\}$的划分$\Delta_F$使$Card(F)=k-1$,且$F$的个元素均为全序集.则将$\{a\}$单独作为一个集合,加入$F$中,可得到符合条件的划分.
			\par
			根据证明规则\ref{C61},对有限集合命题得证.
			\par
			如果$E$不是有限集合,设命题对$k-1$成立,令$G$为满足下列条件的集合$C$的集合:
			\par
			对$E$的任意有限子集$F$,均存在$F$的划分$\Delta_G$使$Card(G)\leq k$,且$G$的个元素均为全序集,并使$C\cap F$是$G$的一个元素的子集.
			\par
			根据定义,这样的集合均为全序集,同时,根据定理\ref{theo82},$C$有极大元$C_0$.
			\par
			如果存在$E-C_0$的自由子集,其元素数目为$k$,令其为$\bigcup\limits_{i\in [1, k]}\{a_i\}$(对任意$i\in [1, k]$、$j\in [1, k]$、$i \neq j$,均有$a_i\neq a_j$),则对任意$i\in [1, k]$,均存在有限集合$F_i$,使其任何划分$\Delta_G$,只要$Card(G)\leq k$,则$(C_0\cup\{a_i\})\cap F_i$均不是$G$的任何一个元素的子集,故$a_i\in F_i$.
			\par
			令$H=\bigcup\limits_{i\in [1, k]}F_i$,则$\bigcup\limits_{i\in [1, k]}\{a_i\}\subset H$.同时,存在$H$的某个划分$\Delta_G$,其满足$Card(G)=k$.
			\par
			此时,令$E_i$为$G$的元素且满足$a_i\in E_i$,则存在$j\in [1, k]$,使$C_0\cap H\subset E_j$,令$P_i=E_i\cap F_j$,则$(P_i)_{i\in [1, k]}$是$F_j$的划分,故$(C_0\cup\{a_j\})\cap F_j\subset P_j$,矛盾.
			\par
			因此,$E-C_0$的最长反链的长度小于$k$,由于$C_0$为全序集,$E-C_0$的最长反链的长度为$k-1$,根据归纳假设,存在$E-C_0$的划分$\Delta_G$使$Card(G)\leq k-1$,将$C_0$加入该划分,即可证得命题对$k$也成立.
			
			\begin{cor}\label{cor307}
				\hfill\par
				$A$为集合,$(X_i)i\in [1, m]$,$(Y_j)j\in [m+1, m+n]$均为$A$的有限子集族.自然数$h$满足下列条件:
				\par
				第一,$m\leq n+h$,$h<m$;
				\par
				第二,对任意自然数$r\in [1, m-h]$,以及$[1, m]$的元素数目为$r+h$的子集$\bigcup\limits_{k\in [1, r+h]}\{i_k\}$(对任意$x\in [1, r+h]$、$y \in [1, r+h]$、$x\neq y$,均有$i_k\neq i_y$),存在$[m+1, m+n]$的元素数目为$r$的子集$\bigcup\limits_{l\in [1, r]}\{j_l\}$(对任意$x\in [m+1, m+n]$、$y \in [m+1, m+n]$、$x\neq y$,均有$j_k\neq j_y$),使对任意$l\in [1, r]$,$\bigcup\limits_{k\in [1, r+h]}X_{i_k}$均和$Y_{j_l}$相交,
				\par
				则存在$A$的有限集合$B$,$Card(B)\leq n+h$,并且所有的$X_i$($i\in [1, m]$)和所有的$Y_j$($j\in [m+1, m+n]$)均和$B$相交.
			\end{cor}
			证明:令$G=\Delta_{[1, m+n]}\cup\{(x, y)|(x\in [1, m])\text{与}(y\in [m+1, m+n])\text{与}(X_x\cap Y_y\neq \varnothing)\}$,则$G$为在$[1, m+n]$上的偏序图.
			\par
			令$[1, m+n]$按$(x, y)\in G$排序,则其最长反链的长度为$n+h_0$,根据补充定理\ref{cor306}可证.
						
			\begin{cor}\label{cor308}
				\textbf{霍尔定理}
				\par
				$E$、$F$为有限集合,$x\mapsto A(x)$为$E$到$\mathcal{P}(F)$的映射.则当且仅当对$E$的任意子集$H$均有\\$Card(\bigcup\limits_{x\in H}A(x))\geq Card(H)$时,存在$E$到$F$的单射$f$,使对任意$x\in E$均有$f(x)\in A(x)$.	
			\end{cor}
			证明:
			\par
			必要性根据定义可证.
			\par
			充分性:
			\par
			令$a$、$b$互不相等,$G=\Delta_{(E\times \{a\})\cup(F\times \{b\})}\cup\{(x, y)|(x\in (E\times \{a\}))\text{与}(y\in (A(x)\times \{b\}))\text{与}y\in A(x)\}$,根据补充定理\ref{cor306},$(E\times \{a\})\bigcup\limits_(F\times \{b\})$存在$Card(F)$个全序集的划分\\$(x_i)_{i\in [0, Card(F)]}$.
			\par
			对任意$x_i$,$Card(x_i\cap(F\times \{b\}))\leq 1$,$Card(x_i\cap(E\times \{a\}))\leq 1$.因此,对任意$x_i$,$Card(x_i\cap(F\times \{b\}))=1$,令$g$为映射$v\mapsto x_i(v\in E)$,其中$x_i$为$(v, a)$所在的集合,$h$为映射$x_i\mapsto pr_1(x_i\cap(F\times \{b\}))$,令$f=h\circ g$,$f$即满足要求.
						
			\begin{cor}\label{cor309}
				\hfill\par
				$E$、$F$为有限集合,$x\mapsto A(x)$为$E$到$\mathcal{P}(F)$的映射,$G\subset F$.则当且仅当对$E$的任意子集$H$均有$Card(\bigcup\limits_{x\in H}A(x))\geq Card(H)$,并且对任意$L\subset G$均有$Card(\{x|x\in E\text{与}A(x)\cap L\neq \varnothing\}\geq Card(L))$时,存在$E$到$F$的单射$f$,使对任意$x\in E$均有$f(x)\in A(x)$,且$G\subset f(E)$.
			\end{cor}
			证明:
			\par
			必要性根据定义可证.
			\par
			充分性:
			\par
			对$Card(E)$用数学归纳法:
			\par
			$Card(E)=0$时,命题显然成立,
			\par
			设命题对$Card(E)\leq n$成立,当$Card(E)=n+1$时:
			\par
			根据补充定理\ref{cor308},存在$G$到$E$的单射$f$、$E$到$F$的单射$g$,令其图分别为$P$、$Q$,如果$f\\\langle G\rangle=E$,则命题成立;否则,令$x\in E-f\langle G\rangle$,如果$g(x)\in F-G$,根据归纳假设命题对$E-x$、$F-g(x)$、$G$成立,故命题对$E$、$F$、$G$成立;如果$g(x)\in G$,根据归纳假设命题对$E-x$、$F-g(x)$、$G-g(x)$成立,故命题对$E$、$F$、$G$成立.综上,得证.
						
			\begin{de}
				\textbf{流动性(mobile)}
				\par
				$A$为集合,$R$的元素都是$A$的有限子集,如果$R$满足下列条件,则称$R$具有流动性:
				\par
				对$R$的任何两个不同元素$X$、$Y$,如果$z\in X\cap Y$,则存在$Z\in R$,使$Z\subset X\cup Y$且$z\notin Z$.
			\end{de}
			
			\begin{de}
				\textbf{纯子集(partie pure)}
				\par
				$Q\subset A$,$R\subset \mathcal{P}(A)$,如果$Q$的任何子集都不是$R$的元素,则称$Q$为$A$关于$R$的纯子集.
			\end{de}

			\begin{exer}\label{exer127}
				\hfill\par
				(1)$E$为集合,$F(E)$是$E$的有限子集集合.令$H$为满足下列条件的$\mathcal{P}(E)$的子集$G$的集合:
				\par
				第一,$\varnothing\in G$;
				\par
				第二,对任意$X\in G$、$x\in E$,均有$X\cup\{x\}\in G$.
				\par
				求证:$F(E)$是按包含关系排序的$H$的最小元.
				\par
				(2)求证:$E$的任何两个有限子集$A$、$B$的并集是$E$的有限子集.
				\par
				(3)$E$为有限集合,求证:$\mathcal{P}(E)$为有限集合.
			\end{exer}
			证明:
			\par
			(1)根据定义可证$F(E)\in H$.反过来,$\varnothing\in G$,因此基数为$0$的$E$的子集都是$G$的元素,设基数为$n$的$E$的子集都是$G$的元素,对任意基数为$n+1$的$E$的子集$X$,由于$X\neq \varnothing$,故令$x\in X$,根据定理\ref{theo94}(2)、定理\ref{theo101},$Card(X-\{x\})=n$,故$X-\{x\}\in G$,因此$X\in G$,根据证明规则\ref{C61}得证.
			\par
			(2)令$H=\{X|X\subset E\text{与}X\bigcup\limits_A\text{为有限集合}\}$,则$\varnothing\in H$,并且,对任意$X\in G$、$x\in E$,$Card(X\cup A\cup\{x\})$为$Card(X\cup A)$或$Card(X\cup A)+1$,根据定理\ref{theo115},$X\cup\{x\}\in G$,根据习题\ref{exer127}(1),$F(E)\subset H$,故$B\in H$,得证.
			\par
			(3)当$Card(E)=0$时,命题显然成立,设命题对$Card(E)=n$成立,当$Card(E)=n+1$时:
			\par
			令$x\in E$,$E'=E-\{x\}$,则$\mathcal{P}(E)=\mathcal{P}(E')\cup g(\mathcal{P}(E'))$,其中$g$为$\mathcal{P}(E')$到$\mathcal{P}(E)$的映射$X\mapsto X\bigcup\limits_\{x\}$.由于$\mathcal{P}(E)$为有限集合,$g$为$\mathcal{P}(E')$到$g(\mathcal{P}(E')$的双射,故$g(\mathcal{P}(E')$为有限集合,根据习题\ref{exer127}(2),$\mathcal{P}(E)$为有限集合,根据证明规则\ref{C61}得证.
			
			\begin{exer}\label{exer128}
				\hfill\par
				求证:当且仅当任何按包含关系排序的$\mathcal{P}(E)$的非空子集均有极大元时,$E$为有限集合.
			\end{exer}
			证明:即补充定理\ref{cor304}.
			
			\begin{exer}\label{exer129}
				\hfill\par
				$E$为良序集,$E$按相反关系排序的偏序集也是良序集,求证:$E$为有限集合.
			\end{exer}
			证明:即补充定理\ref{cor305}.
			
			\begin{exer}\label{exer130}
				\hfill\par
				$E$为有限集合,其元素数目$n\geq 2$,$C\subset E\times E$,对任意$x\neq y$,$(x, y)$和$(y, x)$有且只有一个是$C$的元素.求证:存在$[1, n]$到$E$的映射$f$,对任意$i\in [1, n-1]$,$(f(i), f(i+1))\in C$.
			\end{exer}
			证明:将命题加强为存在双射.
			\par
			$n=2$,命题显然成立.
			\par
			设命题对$n$成立,对于元素数目为$n+1$的集合$E$,设$a\in E$,令$E'=E-\{a\}$,$C'=C\cap E'\times E'$,设$f'$为$[1, n]$到$E'$的双射并且关于图$C'$满足条件,令$A=\{x|x\in [1, n]\text{与}(f'(x), a)\in C\}$,$m=sup_EA$,则当$i\in [1, m]$时,$f(i)=f'(i)$,当$i=m+1$时,$f(i)=a$,如果$m<n$,则当$i\in [m+2, n+1]$时,$f(i)=f'(i-1)$.则$f$为满足条件的双射.根据补充证明规则\ref{Ccor88}得证.
			
			\begin{exer}\label{exer131}
				\hfill\par
				$E$为偏序集,其最长反链的长度为自然数$k$,求证:存在$E$的划分$\Delta_F$使$Card(F)=k$,并且,$F$的元素,均为按$E$的偏序在该集合上导出的偏序排序的全序集.
			\end{exer}
			证明:即补充定理\ref{cor306}.
			
			\begin{exer}\label{exer132}
				\hfill\par
				(1)$A$为集合,$(X_i)i\in [1, m]$,$(Y_j)j\in [m+1, m+n]$均为$A$的有限子集族.存在自然数$h$满足下列条件,且其中最小的为$h_0$:
				\par
				第一,$m\leq n+h$,$h<m$;
				\par
				第二,对任意自然数$r\in [1, m-h]$,以及$[1, m]$的元素数目为$r+h$的子集$\bigcup\limits_{k\in [1, r+h]}\{i_k\}$(对任意$x\in [1, r+h]$、$y \in [1, r+h]$、$x\neq y$,均有$i_k\neq i_y$),存在$[m+1, m+n]$的元素数目为$r$的子集$\bigcup\limits_{l\in [1, r]}\{j_l\}$(对任意$x\in [m+1, m+n]$、$y \in [m+1, m+n]$、$x\neq y$,均有$j_k\neq j_y$),使对任意$l\in [1, r]$,$\bigcup\limits_{k\in [1, r+h]}X_{i_k}$均和$Y_{j_l}$相交,
				\par
				则存在$A$的有限集合$B$,$Card(B)\leq n+h_0$,并且所有的$X_i$($i\in [1, m]$)和所有的$Y_j$($j\in [m+1, m+n]$)均和$B$相交.
				\par
				求证:存在$A$的有限集合$B$,其元素数目小于等于$n+h$,并且所有的$X_i$($i\in [1, m]$)和所有的$Y_j$($j\in [m+1, m+n]$)均和$B$相交.
				\par
				(2)$E$、$F$为有限集合,$x\mapsto A(x)$为$E$到$\mathcal{P}(F)$的映射.求证:当且仅当对$E$的任意子集$H$均有$Card(\bigcup\limits_{x\in H}A(x))\geq Card(H)$时,存在$E$到$F$的单射$f$,使对任意$x\in E$均有\\$f(x)\in A(x)$.
				\par
				(3)$E$、$F$为有限集合,$x\mapsto A(x)$为$E$到$\mathcal{P}(F)$的映射,$G\subset F$.求证:当且仅当对$E$的任意子集$H$均有$Card(\bigcup\limits_{x\in H}A(x))\geq Card(H)$,并且对任意$L\subset G$均有$Card(\{x|x\in E\text{与}A(x)\cap L\neq \varnothing\}\geq Card(L))$时,存在$E$到$F$的单射$f$,使对任意$x\in E$均有$f(x)\in A(x)$,且$G\subset f(E)$.
			\end{exer}
			证明:
			\par
			(1)根据补充定理\ref{cor307}可证.
			\par
			(2)即补充定理\ref{cor308}.
			\par
			(3)即补充定理\ref{cor309}.
			
			\begin{exer}\label{exer133}
				\hfill\par
				(1)$E$为有限格,求证:$E$的任意元素$a$,都是有限个不可约元素的最小上界.
				\par
				(2)$E$为有限格,$J$为其不可约元素集合.令$S(x)=\{y|y\in J\text{与}y\leq x\}$,求证:$x\mapsto S(x)$为$E$到按包含关系排序的$\mathcal{P}(J)$的一个子集的同构,并且,$S(inf(x, y))=S(x)\cap S(y)$.
			\end{exer}
			证明:
			\par
			(1)令$J=\{v|v\in E\text{与}y\text{为}E\text{的不可约元素}\}$,$H=\{u|u\in E\text{与}(\exists X)(X\subset J\text{与}sup\ X=u)\}$.如果$E-H\neq \varnothing$,则有极小元$z$,故$z=sup(x, y)$,且$x<z$、$y<z$,故$x\in H$、$y\in H$,因此$x$、$y$均为有限个不可约元素的最小上界,故$z\in H$,矛盾.
			\par
			(2)	根据习题\ref{exer133}(1)可证.
			
			\begin{exer}\label{exer134}
				\hfill\par
				(1)$E$为分配格,$a$是$E$的不可约元素,求证:$a\leq sup(x, y)\Rightarrow a\leq x\text{或}a\leq y$.
				\par
				(2)$E$为有限分配格,$J$为其不可约元素集合.令$S(x)=\{y|y\in J\text{与}y\leq x\}$,求证:$x\mapsto S(x)$为$E$到按包含关系排序的$\mathcal{P}(J)$的一个子集的同构,并且,$S(sup(x, y))=S(x)\cup S(y)$;同时,令$J^*$为按在$J$上的偏序关系的相反关系排序的偏序集,$I=\{0, 1\}$,$A(J^*, I)$为\\$J^*$到$I$的单增映射的集合,按$f\in A(J^*, I)\text{与}g\in A(J^*, I)\text{与}(\forall x)(x\in J^*\Rightarrow f(x)\leq g(x))$排序,则$E$同构于$A(J^*, I)$.
				\par
				(3)$E$为有限分配格,$J$为其不可约元素集合.令$S(x)=\{y|y\in J\text{与}y\leq x\}$.令$(y_i)i\in [1, k]$为$]x, \to [$的所有两两不相等的极小元组成的元素族.对任意$i\in [1, k]$,令$q_i$为$S(y_i)-S(x)$的一个元素,求证:对任意$i\in [1, k]$、$j\in [1, k]$且$i\neq j$,$q_i$和$q_j$是不可比较的.
				\par
				(4)$E$为有限分配格,$J$为其不可约元素集合.令$S(x)=\{y|y\in J\text{与}y\leq x\}$,$a$为$E$的最小元,$P=J-\{a\}$.令$(q_i)i\in [1, k]$为$P$的元素族,且对任意$i\in [1, k]$、$j\in [1, k]$且$i\neq j$,$q_i$和$q_j$是不可比较的.令$u=\mathop{sup}\limits_{i\in [1, k]}q_i$,$v_j=\mathop{sup}\limits_{i\in [1, k]-\{i\}}q_i$($j\in [1, k]$),$x=\mathop{inf}\limits_{i\in [1, k]}v_i$,$y_j=\mathop{inf}\limits_{i\in [1, k]-\{i\}}v_i$,求证:对任意$i\in [1, k]$,$v_i<u$,$x<y_i$,并且,区间$]x, \to [$有$k$个两两不相等的极小元.
			\end{exer}
			证明:
			\par
			(1)即补充定理\ref{cor199}(4).
			\par
			(2)根据补充定理\ref{cor199}(4)可证.
			\par
			(3)如果$q_i=q_j$,则$y_i=y_j$,矛盾;如果$q_i<q_j$,则$q_i$、$x$均为$\{yi, yj\}$的下界,故$x<q_i$,因此$q_i<y_j$,与$y_j$是最小元矛盾.
			\par
			(4)由于$u=sup(v_i, q_i)$,因此$v_i<u$.如果$y_i=v_i$,则对任意$j\in [1, k]$且$i\neq j$,$v_i\leq v_j$,则$v_j\geq u$,矛盾,故$x<yi$.令$A_i=\{s|s>x\text{与}s\leq yi\}$,如果$A_i\cap A_j\neq \varnothing$($i\neq j$),设$t\in A_i$、$t\in A_j$,则$t\leq x$,矛盾,故$A_i\cap A_j=\varnothing$.令$A_i$的其中一个极小元为$z_i$,故$z_1$、$z_2$、$\cdots$、$z_k$为两两不相等的极小元.
			
			\begin{exer}\label{exer135}
				\hfill\par
				(1)令$(C_i)_{i\in [1, n]}$为全序集有限族,$E$为其乘积.$A$为$E$的内部格,求证:$A$最多有$n$个两两不可比较的不可约元素.
				\par
				(2)令$F$为有限分配格,$J$为其不可约元素集合.$a$为$F$的最小元,$P=J-\{a\}$.令$A=\{a|(\exists X)((X\text{为}P\text{的自由子集})\text{与}Card(X)=a)\}$,$A$的最大元为$n$,求证:$F$同构于全序集有限族的乘积的某个内部格.
			\end{exer}
			证明:
			\par
			(1)若$A$有$r$个两两不可比较的不可约元素且$r>n$,设其组成元素族$(a_i)_{i\in [1, r]}$.令$u=\mathop{sup}\limits_{i\in [1, r]}a_i$,$v_j=\mathop{sup}\limits_{i\in [1, r]-\{j\}}a_i$.则对任意$i\in [1, n]$,$pr_i(v_j)$之中最多有一个不等于$pr_i(u)$,因此,存在$j\in [1, r]$,使$v_j=u$,故$a_i\leq v_i$.根据补充定理\ref{cor199}(1)、补充定理\ref{cor199}(2)、补充定理\ref{cor199}(3),$A$为分配格.但根据补充定理\ref{cor199}(4),运用数学归纳法可得,$a_i\leq v_i$为假,矛盾.
			\par
			(2)根据补充定理\ref{cor306},存在的$P$的划分$\Delta_G$,其中$G$的元素均为全序集并且元素数目为$n$.将最小元$a$加入$G$的每个元素,得到全序集族$(Ci)_{i\in [1, n]}$.对任意$x\in F$,令$x_i$为$\{b|b\in Ci\text{与}b\leq x\}$在$C_i$上的最小上界,根据定义可证,$x\mapsto \{(i, c)|(i, c)\text{为有序对}\text{与}(\exists i)(i\in [1, n]\text{与}c\\=x_i)\}$为$F$到$\prod\limits_{i\in [1, n]}C_i$的某个内部格的同构.同时,令该映射为$f$,令$f_i=pr_if$,当$x\in F$、$y\in F$时,令$z=sup(x, y)$,则对任意$i\in [1, n]$,$f_i(z)\geq sup(f_i(x), f_i(y))$,同时,$f_i(z)\leq sup(x, y)$,根据补充定理\ref{cor199}(4),$fi_(z)\leq x\text{或}f_i(z)\leq y$,故$f_i(z)\leq sup(f_i(x), f_i(y))$,因此$f_i(z)=\\sup(f_i(x), f_i(y))$,故$f(z)=sup(f(x), f(y))$,同理可证最大下界的情况.故该映射的值域为内部格.得证.
			
			\begin{exer}\label{exer136}
				\hfill\par
				(1)	求证:当且仅当$E$的偏序图是$n$个在$E$上的全序图的交集时,E同构于$n$个全序集的乘积的某个子集.
				\par
				(2)$E$为偏序集,其偏序为$F$,求证:当且仅当存在在$E$上的偏序$F'$,使$E$的任何两个不同元素$x$、$y$,仅在按其中一个偏序排序的$E$上是可比较的时,$E$同构于两个全序集的乘积的子集.
				\par
				(3)$A$是元素数目为$n$的有限集合,$E=\{X|(\exists x)(x\in A\text{与}X=\{x\})\}\cup\{X|(\exists x)(x\in A\text{与}X=A-\{x\})\}$,求证:$E$同构于$n$个全序集的乘积的某个子集,并且,当$m<n$时,$E$不能同构于任意$m$个全序集的乘积的任意子集,.
			\end{exer}
			(1)充分性:
			\par
			设$E$的偏序图为$G$,$G=\bigcap\limits_{i\in [1, n]}G_i$,其中$G_i$均为在$E$上的全序图.令$E_i$为按$(x, y)\in G$排序的全序集,则$E$同构于$\prod\limits_{i\in [1, n]}E_i$的一个子集.根据定义可证.
			\par
			必要性:设$E$同构于$\prod\limits_i\in [1, n]F_i$的一个子集$A$,$F=\prod\limits_i\in [1, n]F_i$.令$(f_i)_{i\in [1, n]}$为$[1, n]$的排列族,其中$f_i(1)=i$,$G_i$为集族$(F_{f_i(j)})_{j\in [1, n]}$的字典式乘积在$E$上导出的全序的全序图,根据定义可证,$E$的偏序图为$\bigcap\limits_{i\in [1, n]}G_i$.
			\par
			(2)根据习题\ref{exer136}(1)可证.
			\par
			(3)设$E$的偏序图为$G$,根据习题\ref{exer136}(1),$G$为某个全序图族$(G_i)_{i\in [1, m]}$的并集,且$m<n$.令$F=\{X|(\exists x)(x\in A\text{与}X=\{x\})\}$,令全序图族$G_i\cap F\times F$相应的最大元为$a_i$,故存在$\{x\}\in F$,使$\{x\}$不是任何$G_i\cap F\times F$的最大元,则$(\{x\}, A-\{x\})\in G$,矛盾.
			\par
			另一方面,令$(a_i)_{i\in [1, m]}$为所有$A$的单个元素的集合组成的族,$b_i=A-a_i$,全序图$G_1$按照$a_1$、$a_2$、$\cdots$、$a_{n-1}$、$b_n$、$a_n$、$b_{n-1}$、$\cdots$、$b_1$排序,$G_2$按照$a_2$、$a_3$、$\cdots$、$a_n$、$b_1$、$a_1$、$b_n$、$\cdots$、$b_2$排序,以此类推,则全序图族$(G_i)_{i\in [1, m]}$的并集为$G$,根据习题\ref{exer136}(1),存在性得证.
			
			\begin{exer}\label{exer137}
				\hfill\par
				令$R\subset \mathcal{P}(A)$:
				\par
				(1)求证:$A$关于$R$的纯子集集合有极大元,并且,$A$关于$R$的任何纯子集都是某个极大元的子集.
				\par
				(2)令$M$为$A$关于$R$的纯子集集合的某个极大元,求证:对任意$x\in \complement_AM$,存在唯一的有限集合$E_M(x)$,满足$E_M(x)\subset M$并且$E_M(x)\cup\{x\}\in R$.并且,对任意$y\in E_M(x)$,$E_M(x)\cup\{x\}-\{y\}$是$A$关于$R$的纯子集集合的某个极大元.
				\par
				(3)$M$、$N$均为$A$关于$R$的纯子集集合的极大元,并且$N\cap\complement_AM$为有限集合,求证:$Card(M)=Card(N)$.
				\par
				(4)	$M$、$N$均为$A$关于$R$的纯子集集合的极大元,令$N'=N\cap\complement_AM$,$M'=M\cap\complement_AN$,求证:$M'\subset EM(x)$,并且$Card(M)=Card(N)$.
			\end{exer}
			证明:
			\par
			(1)对于$A$关于$R$的任何纯子集E,令$X=\{M|E\subset M\text{与}M\text{为}A\text{关于}R\text{的纯子集}\}$,根据定理\ref{theo82},$X$有极大元,得证.
			\par
			(2)前半部分根据定义可证.根据定义可证,$E_M(x)\cup\{x\}-y$是纯子集.设它不是极大元,其是极大元$N$的子集,则$E_M(x)\cup\{x\}\in R$、$N\cup\{y\}\in R$,矛盾.
			\par
			(3)根据习题\ref{exer137}(2),对$Card(N\cap\complement_AM)$运用数学归纳法可证.
			\par
			(4)对任意$x\in N'$,存在$y\in M'$且$y\notin E_M(x)$,如果存在$z\in M'$且$z\notin E_M(x)$,则$E_M(x)\cup\{x\}-\{y\}$是$A$关于$R$的纯子集集合的某个极大元,故$(E_M(x)\cup\{x\}-\{y\})\cup\{z\}$有一个子集是$R$的元素,矛盾,因此,$M'\subset E_M(x)$,且$M'$为有限集合,同理$N'$为有限集合,因此$Card(M)=Card(N)$.
			\par
			注:原书习题\ref{exer137}(4)有误.

		\section{自然数的运算(Calcul sur les entiers)}		
			\begin{theo}\label{theo125}
				\textbf{有限个自然数的和、积均为自然数}
				\par
				令$(a_i)_{i\in I}$为自然数有限族,则$\sum\limits_{i\in I}a_i$、$\mathop{\mathsf{P}}\limits_{i\in I}a_i$均为自然数.
			\end{theo}
			证明:
			\par
			先证自然数$a$、$b$的和是自然数:
			\par
			$b=0$,$a+b=a$,命题显然成立,假设命题对$b$成立,即$a+b$为自然数,则$a+(b+1)=(a+b)+1$,根据定理\ref{theo115},其为自然数,得证.
			\par
			令$n=Card(I)$,当$n=0$,$\sum\limits_{i\in I}a_i=0$,命题显然成立;设命题对$n$成立,则对于$Card(I)\\=n+1$,令$I=J\cup\{k\}$,其中$k\notin J$,则$Card(J)=n$,故$\sum\limits_{i\in J}a_i$为自然数,而$\sum\limits_{i\in I}a_i=(\sum\limits_{i\in J}a_i)+a_k$,因此也是自然数.
			\par
			对于乘法,根据定理\ref{theo99}及上述证明,自然数$a$、$b$的乘积为自然数.类似加法可证.				
						
			\begin{theo}\label{theo126}
				\hfill\par
				令$(X_i)_{i\in I}$为有限集族,则$\bigcup\limits_{i\in I}X_i$为有限集合.
			\end{theo}
			证明:根据定理\ref{theo125},$(X_i)_{i\in I}$的和$S$为有限集合.根据补充定理\ref{cor116}(1),存在$S$到$\bigcup\limits_{i\in I}X_i$\\的满射,得证.
			
			\begin{theo}\label{theo127}
				\hfill\par
				令$(X_i)_{i\in I}$为有限集族,则$\prod\limits_{i\in I}X_i$为有限集合.
			\end{theo}
			证明:根据定理\ref{theo125}可证.
			
			\begin{theo}\label{theo128}
				\textbf{自然数的自然数次幂为自然数}
				\par
				令$a$、$b$为自然数,则$a^b$为自然数.
			\end{theo}
			证明:根据定理\ref{theo127}、定理\ref{theo103}可证.
			
			\begin{theo}\label{theo129}
				\hfill\par
				有限集合的子集集合是有限集合.
			\end{theo}
			证明:根据定理\ref{theo128}、定理\ref{theo108}可证.
			
			\begin{theo}\label{theo130}
				\hfill\par
				$a$、$b$为自然数,则$a<b\Leftrightarrow (\exists c)(c>0\text{与}b=a+c)$.
			\end{theo}
			证明:由于$a<b$,故存在$c$使$b=a+c$,由于$a\neq b$,故$c>0$;反过来,如果$b=a+c$,由于$c\geq 1$,故$a<b$.
			
			\begin{theo}\label{theo131}
				\hfill\par
				令$(a_i)_{i\in I}$、$(b_i)_{i\in I}$为自然数有限族,$(\forall I)(i\in I\Rightarrow a_i\leq b_i)$,且$(\exists I)(i\in I\text{与}a_i<b_i)$,则$\sum\limits_{i\in I}a_i<\sum\limits_{i\in I}b_i$;如果$(\forall I)(i\in I\Rightarrow b_i\neq 0)$,则$\mathop{\mathsf{P}}\limits_{i\in I}a_i<\mathop{\mathsf{P}}\limits_{i\in I}b_i$.
			\end{theo}
			证明:设$a_j<b_j$,则对$I-\{j\}$,根据定理\ref{theo110},$\sum\limits_{i\in I-\{j\}}a_i\leq \sum\limits_{i\in I-\{j\}}b_i$;令$b_j=a_j+c$,则$c>0$,故$c+\sum\limits_{i\in I}a_i\leq \sum\limits_{i\in I}b_i$,因此$\sum\limits_{i\in I}a_i<\sum\limits_{i\in I}b_i$.
			根据定理\ref{theo110},$\mathop{\mathsf{P}}\limits_{i\in I-\{j\}}a_i\leq \mathop{\mathsf{P}}\limits_{i\in I-\{j\}}b_i$,因此$\mathop{\mathsf{P}}\limits_{i\in I}a_i\leq c\cdot \mathop{\mathsf{P}}\limits_{i\in I-\{j\}}b_i+\mathop{\mathsf{P}}\limits_{i\in I}b_i$.根据定理\ref{theo100},$c\cdot \mathop{\mathsf{P}}\limits_{i\in I-\{j\}}b_i>0$,得证.
			
			\begin{theo}\label{theo132}
				\hfill\par
				$a$、$a'$、$b$为自然数,$b>0$,$a<a'$,则$a^b<{a'}^b$.
			\end{theo}
			证明:根据定理\ref{theo103}、定理\ref{theo131}可证.
			
			\begin{theo}\label{theo133}
				\hfill\par
				$a$、$b$、$b'$为自然数,$b<b'$,$a>1$,则$a^b<a^{b'}$.
			\end{theo}
			证明:设$b'=b+c$,则$c>0$.$a^{b'}=a^b\cdot a^c$.由于$c>0$,故$a^c\geq a$,因此$a^c>1$,故$ab<ab'$.
			
			\begin{theo}\label{theo134}
				\hfill\par
				$a$、$b$、$b'$为自然数,则$a+b=a+b'\Leftrightarrow b=b'$,如果$a>0$,则$ab=ab'\Leftrightarrow b=b'$.
			\end{theo}
			证明:若$b<b'$,根据定理\ref{theo131},$a+b<a+b'$,$ab<ab'$;若$b>b'$,根据定理\ref{theo131},$a+b>a+b'$,$ab>ab'$;若$b=b'$,则$a+b=a+b'$,$ab=ab'$,得证.
			
			\begin{cor}\label{cor310}
				\hfill\par
				$a$、$b$、$b'$为自然数,则$a+b<a+b'\Leftrightarrow b<b'$,如果$a>0$,则$ab<ab'\Leftrightarrow b<b'$.
			\end{cor}
			证明:类似定理\ref{theo134}可证.
			
			\begin{cor}\label{cor311}
				\hfill\par
				$a$、$b$是自然数,$b>1$,则$a<b^a$.
			\end{cor}
			证明:根据补充定理\ref{cor310}可证.
			
			\begin{theo}\label{theo135}
				\textbf{自然数的差的唯一性}
				\par
				$a$、$b$为自然数,$a\leq b$,则存在唯一的自然数$c$,使$b=a+c$.
			\end{theo}
			证明:根据定理\ref{theo109},存在性成立.若$b=a+c$、$b=a+c'$,根据定理\ref{theo134},$c=c'$,唯一性成立.
			
			\begin{de}
				\textbf{自然数的差(différence de entiers)}
				\par
				$a$、$b$、$c$为自然数,$b=a+c$,则称$c$为$b$和$a$的差,记作$b-a$.
			\end{de}
			
			\begin{cor}\label{cor312}
				\hfill\par
				$a$、$b$为自然数,$a\leq b$,$c$为$b$和$a$的差,则$c$小于$a$,并且$a$是$b$和$c$的差.
			\end{cor}
			证明:根据定理\ref{theo95}(1)可证.
			
			\begin{theo}\label{theo136}
				\hfill\par
				$a$、$b$为自然数,则映射$x\mapsto x+a$为区间$[0, b]$到区间$[a, a+b]$的同构,且为严格单增映射,$y\mapsto y-a$为其逆同构.
			\end{theo}
			证明:令$y=x+a$,则$x=y-a$.根据补充定理\ref{cor310},$x\in [0, b]\Leftrightarrow y\in [a, a+b]$.因此,$x\mapsto x+a$、$y\mapsto y-a$均为双射,且互为逆映射.同时,$x\mapsto x+a$、$y\mapsto y-a$均为严格单增映射.根据定义得证.
			
			\begin{theo}\label{theo137}
				\hfill\par
				$a$、$b$为自然数,$a\leq b$,则区间$[a, b]$的元素数目为$b-a+1$.
			\end{theo}
			证明:如果$a=0$,$b=0$,命题显然成立.如果命题对$[0, b]$成立,则$[0, b+1]=[0, b]\cup\{b+1\}$,显然命题对$[0, b+1]$成立.
			\par
			如果$a>0$,则区间$[a, b]$同构于$[0, b-a]$,故命题也成立.
			
			\begin{theo}\label{theo138}
				\hfill\par
				非空有限全序集的基数为$n$,则该集合同构于区间$[1, n]$.
			\end{theo}
			证明:根据定理\ref{theo84},该非空有限全序集同构于区间$[1, n]$的某个区间,或者区间$[1, n]$同构于该非空有限全序集的某个区间.由于同构集合等势,根据定理\ref{theo118},得证.
			
			\begin{de}
				\textbf{有限序列(suite finie),序列的长度(longueur d'une suite)}
				\par
				如果族的指标集的是自然数有限集,则称该族为有限序列;指标集的基数,称为该序列的长度.
			\end{de}
			
			\begin{cor}\label{cor313}
				\hfill\par
				有限序列的长度为$n$,则区间$[1, n]$同构于指标集.
			\end{cor}
			证明:根据定理\ref{theo138}可证.
			
			\begin{de}
				\textbf{第k项(k-éme terme),首项(premier terme)}
				\par
				令$f$为区间$[1, n]$到有限序列$(t_i)_{i\in I}$的指标集$I$的同构,则$t_{f(k)}$称为该有限序列的第$k$项,\\$t_{f(1)}$称为该有限序列的首项.
			\end{de}
			
			\begin{de}
				\textbf{特征函数(fonction caractéristique)}
				\par
				$A\subset E$,$E$到$\{0, 1\}$的映射$f$满足下列条件:
				\par
				当$x\in A$时,$f(x)=1$;当$x\in E-A$时,$f(x)=0$,
				\par
				则称$f$为$E$的子集$A$的特征函数,记作$\psi_A$.
			\end{de}
			
			\begin{cor}\label{cor314}
				\hfill\par
				对于$E$的子集$A$、$B$,$A=B\Leftrightarrow \psi_A=\psi_B$.
			\end{cor}
			证明:根据定义可证.
			
			\begin{theo}\label{theo139}
				\hfill\par
				对于$E$的子集$A$、$B$:
				\par
				(1)$\psi_E-A(x)=1-\psi_A(x)$.
				\par
				(2)$\psi_A\cap B(x)=\psi_A(x)\psi_B(x)$.
				\par
				(3)$\psi_A\cup B(x)+\psi_A\cap B(x)= \psi_A(x)+\psi_B(x)$.
			\end{theo}
			证明:根据定义可证.
			
			\begin{theo}\label{theo140}
				\textbf{自然数的商和余数的唯一性}\par
				$a$、$b$为自然数,$b>0$,则存在唯一的一对$q$、$r$,使$q$、$r$均为自然数,$a=bq+r$,并且$r<b$.
			\end{theo}
			证明:由于$ba\geq a$,因此$\{q|q为自然数\text{与}bq\leq a\}\subset [1, a]$,故$\{q|q为自然数\text{与}bq\leq a\}$为全序有限集,令其最大元为$q$,则$bq\geq a$,令$r=a-bq$,$r<b$,存在性得证.
			令$q$、$r$和$q'$、$r'$都符合条件,则$bq+r=bq'+r'$,如果$q<q'$,设$q'=q+c$,$c>0$,则$r=bc+r'$,因此$r>b$,矛盾,同理$q'<q$也矛盾,因此$q=q'$,故$r=r'$,唯一性得证.
			
			\begin{de}
				\textbf{自然数的商(quotient de entiers),自然数的余数(reste de entiers),整除(divisible),约数(diviseur)}
				\par
				$a$、$b$为自然数,$b>0$,$q$、$r$均为自然数,$a=bq+r$,并且$r<b$,则$q$称为$a$除以$b$的商,记作$[a/b]$,$r$称为$a$除以$b$的余数.如果$r=0$,则称$a$能被$b$整除,或称$b$整除$a$,或称$b$是$a$的约数,此时商记作$a/b$.
			\end{de}
			
			\begin{cor}\label{cor315}
				\hfill\par
				$a$、$b$、$c$均为自然数,$b>0$,$c>0$,且$b$整除$a$、$c$整除$b$,则$c$整除$a$,且$a/c=(a/b)(b/c)$.
			\end{cor}
			证明:根据定义可证.
			
			\begin{cor}\label{cor316}
				\hfill\par
				$a$、$b$、$c$均为自然数,$c>0$,且$c$整除$a$、$c$整除$b$,则$c$整除$a+b$,且$(a+b)/c=a/c+b/c$;如果$a\geq b$,则$c$整除$a-b$,且$(a-b)/c=a/c-b/c$.
			\end{cor}
			证明:根据定义可证.
			
			\begin{cor}\label{cor317}
				\hfill\par
				(1)$a$为自然数,$a>0$,则$a$整除$0$.
				\par
				(2)$a$为自然数,则$1$整除$a$.
			\end{cor}
			证明:根据定义可证.
			
			\begin{cor}\label{cor318}
				\hfill\par
				$a$、$b$、$c$为自然数,$a$除以$c$的商为$q$,余数为$r$,$b$除以$c$的商为$q'$,余数为$r'$,则$a<b\Leftrightarrow(q<q')\text{或}((q=q')\text{与}(r<r'))$.
			\end{cor}
			证明:由于$a=cq+r$,$b=cq'+r'$,如果$a<b$,且$q>q'$,则$cq\geq cq'+c$,由于$r '<c$,故$cq+r>cq'+r'$,矛盾.如果$q=q'$,则$r<r'$.
			反过来,如果$q<q'$,则$q+1\leq q'$,又因为$r<c$,故$cq+r<c(q+1)$,因此$a<cq'$,故$a<b$.
			
			\begin{de}
				\textbf{偶数(pairs)、奇数(impairs)}
				\par
				$a$为自然数,如果$2$整除$a$,则称$a$为偶数,否则,称$a$为奇数.
			\end{de}
			
			\begin{cor}\label{cor319}
				\hfill\par
				$a$为自然数,则$a\text{为偶数}\Leftrightarrow (\exists n)(n\text{为自然数}\text{与}a=2n)$,$a\text{为奇数}\Leftrightarrow (\exists n)(n\text{为自然数}\text{与}a\\=2n+1)$.
			\end{cor}
			证明:根据定义可证.
			
			\begin{theo}\label{theo141}
				\hfill\par
				令$b$、$k$、$h$为自然数,$b>1$、$k>0$,$(J_h)_{h\in [0, k-1]}$,对任意$h\in [0, k-1]$, $J_h \in [0, b-1]$,令$E_k$为该族的字典式乘积,令$r$为$(r_h)_{h\in [0, k-1]}$,则映射$r\mapsto \sum\limits_{h\in [0, k-1]}r_hb^{k-h-1}$为$E_k$到$[0, b^k-1]$的同构.
			\end{theo}
			证明:
			\par
			当$k=1$时,命题显然成立.
			\par
			如果命题对$k$成立,令$E_k$到$[0, b^k-1]$的映射为$f$:
			\par
			令$(r_h)_{h\in [0, k]}\in E_{k+1}$,则$\sum\limits_{h\in [0, k]}r_hb^{k-h}\leq r_0b^k+b^k-1$,由于$r_0+1\leq b$,故$\sum\limits_{h\in [0, k]}r_hb^{k-h}<b^{k+1}-1$,即$r\mapsto \sum\limits_{h\in [0, k]}r_hb^{k-h}$为$E_{K+1}$到$[0, b^{k+1}-1]$的映射,设该映射为$g$.
			\par
			对于$r\in E_{k+1}$($r=(r_h)_{h\in [0, k]}$),$r'\in E_{k+1}$($r'=({r'}_h)_{h\in [0, k]}$),令$s=(r_{i+1})_{i\in [0, k-1]}$,$s'=({r'}_{i+1})_{i\in [0, k-1]}$,则$g(r)=r_0b^k+f(s)$,$g(r')={r'}_0b^k+f(s')$:
			\par
			如果$g(r)=g(r')$,由于$f(s)<b^k$、$f(s')<b^k$,根据定理\ref{theo140},$r=r'$,故$g$为单射.
			\par
			同时,对任意$x\in E_{k+1}$,根据定理\ref{theo140},存在$q<b$、$t<b^k$,使$x=qb^k+t$,令$r_0=q$,$s=f^{-1}(t)$,$r_i=s_{i-1}$($i\in [1, k]$),则$g(r)=x$,故$g$为满射.
			\par
			如果$r<r'$,则:
			\par
			若$r_0<{r'}_0$,则$g(r)<r_0b^k+b^k$,$g(r')\geq {r'}_0b^k$,又因为$r_0+1\leq {r'}_0$,因此$g(r)\leq g(r')$;
			\par
			若$r_0={r'}_0$,则令$j$为$\{i|i\in [1, k]\text{与}r_i\neq {r'}_i\}$的最小元,故$r_j<{r'}_j$,因此$s<s'$,根据归纳假设,$f(s)<f(s')$,因此$g(r)\leq g(r')$.
			\par
			故$g$为单增函数.
			\par
			综上,根据补充定理\ref{cor166},得证.
			
			\begin{cor}\label{cor320}
				\textbf{自然数展开的唯一性}
				\par
				令$a$、$b$为自然数,$b>1$、$a>0$,则存在唯一的自然数$k$以及族$(r_h)_{h\in [0, k-1]}$,使$(\forall h)(h\in [0, k-1]\Rightarrow r_k\in [0, k-1])$,$r_0\neq 0$,并且$a=\sum\limits_{h\in [0, k^-1]}r_hb^{k-h-1}$.
			\end{cor}
			证明:根据补充定理\ref{cor311},$a<b^a$,因此$\{x|x\leq a\text{与}a<b^x\}$有最小元$k$,因此$b^{k-1}\leq a$,$a<b^k$.根据定理\ref{theo141},存在唯一的族,$(r_h)_{h\in [0, k-1]}$,使$(\forall h)(h\in [0, k-1]\Rightarrow r_k\in [0, k-1])$ 并且$a=\sum\limits_{h\in [0, k-1]}r_hb^{k-h-1}$.
			此时,如果$r_0=0$,则令$s_i=r_{i+1}$($i\leq k-2$),根据定理\ref{theo141},$\sum\limits_{h\in [0, k-2]}s_hb^{k-h-2}\leq b^{k-1}$,故$a\leq b^{k-1}-1$,矛盾,因此$r0\neq 0$.
			\par
			如果另有$k'$满足条件,若$k'<k$,则$a\leq b^{k'}-1$,故$a\leq b^{k-1}-1$,矛盾;若$k<k'$,则$a\geq b^k$,同样矛盾.得证.
			
			\begin{de}
				\textbf{自然数的展开(développement d'un entier)}
				\par
				令$a$、$b$为自然数,$b>1$、$a>0$,如果自然数$k$以及族$(r_h)_{h\in [0, k-1]}$,使$(\forall h)(h\in [0, k-1]\Rightarrow r_k\in [0, k-1])$,$r_0\neq 0$,并且$a=\sum\limits_{h\in [0, k^-1]}r_hb^{k-h-1}$,则称$(r_h)_{h\in [0, k-1]}$为$a$基于$b$的展开.
			\end{de}
			注:原书将$\sum\limits_{h\in [0, k-1]}r_hb^{k-h-1}$称为$a$基于$b$的展开,但这一式子本身等于$a$,不应作为单独的概念,故修改.
			
			\begin{theo}\label{theo142}
				\textbf{乘法原理}
				\par
				$a$、$b$为基数,$Card(E)=a$,$Card(E)=b$,$f$为$E$到$F$的满射,对任意$y\in F$,\\$Card(f^{-1}\langle y\rangle)=c$,则$a=bc$.
			\end{theo}
			证明:对任意$y\neq y'$,$f^{-1}\langle y\rangle\cap f^{-1}\langle y'\rangle=\varnothing$,因此$(f^{-1}\langle y\rangle)_{y\in F}$是$E$的划分.根据定理\ref{theo94}(2)、定理\ref{theo99}可证.
			
			\begin{de}
				\textbf{阶乘(factorielle)}
				\par
				$n$为自然数,则$\mathop{\mathsf{P}}\limits_{i\in [1, n]}i$称为$n$的阶乘,记作$n!$.
			\end{de}
			
			\begin{cor}\label{cor321}
				\hfill\par
				$0!=1$;$1!=1$.
			\end{cor}
			证明:根据定义可证.
			
			\begin{theo}\label{theo143}
				\hfill\par
				$m$、$n$为自然数,$m\leq n$,$A$的元素数目为$m$,$B$的元素数目为$n$,$A$到$B$的单射的数目为$n!/(n-m)!$.
			\end{theo}
			证明:$m=0$,则$A=\varnothing$,故仅有单射$(\varnothing, \varnothing, B)$,因此元素数目为$1$,故命题对$0$成立.
			\par
			设命题对$m$成立,考虑$m+1$:
			\par
			设$a\in A$,令$A'=A-\{a\}$.$F$为$A$到$B$的单射,$F'$为$A'$到$B$的单射,$g$为映射$f\mapsto f'|A$,由于$f(a)\in B-f'(A')$,其元素数目为$n-m$,故对任意$f'$,$g^{-1}\langle f'\rangle$的元素数目为$n-m$,根据定理\ref{theo142},$F$的数目为$(n!/(n-m)!)\cdot (n-m)$,得证.
			
			\begin{theo}\label{theo144}
				\hfill\par
				$A$的排列的数目为$n!$.
			\end{theo}
			证明:根据定理\ref{theo143}可证.
			
			\begin{cor}\label{cor322}
				\hfill\par
				$E$为元素数目为$n$的集合,$(p_i)_{i\in [1, h]}$为自然数有限序列,$\sum\limits_{i\in [1, h]}p_i=n$.则存在两两不相交的集族$(X_i)i\in[1, h]$,其为$E$的覆盖且满足$(\forall i)(i\in [1, h]\Rightarrow Card(X_i)=p_i)$.
			\end{cor}
			证明:若$h=0$,则$n=0$,$E=\varnothing$,$\varnothing$显然满足条件.
			\par
			设命题对$h$成立,则对$h+1$:
			\par
			由于$E$存在元素数目为$\sum\limits_{i\in [1, h]}p_i$的子集$E'$,其存在满足条件的覆盖($X_i)_{i\in [1, h]}$.因此$E-E'$的元素数目为$p_{h+1}$,则$(p_i)_{i\in [1, h]}\cup(h+1, E-E')$是满足条件的覆盖.得证.
			
			\begin{theo}\label{theo145}
				\hfill\par
				$E$的元素数目为$n$,$(p_i)_{i\in [1, h]}$为自然数有限序列,$\sum\limits_{i\in [1, h]}p_i=n$.满足$(\forall i)(i\in [1, h]\Rightarrow Card(X_i)=p_i)$且两两不相交的$E$的覆盖$(X_i)_{i\in [1, h]}$的数目为$n!/\mathop{\mathsf{P}}\limits_{i\in [1, h]}(p_i!)$.
			\end{theo}
			证明:令$G$为$E$的排列集合,$P$为满足条件的覆盖的集合.根据补充定理\ref{cor322},$P$不是空集.
			\par
			令$(A_i)_{i\in [1, h]}$为符合条件的一个覆盖,对$G$的任何一个元素$f$,则$(f(A_i))_{i\in [1, h]}$也是$P$的元素.令该覆盖为$g(f)$.
			\par
			则对任意$(X_i)_{i\in [1, h]}\in P$,如果$g(f)=(X_i)_{i\in [1, h]}$,则对任意$i\in [1, h]$,$f(A_i)=X_i$.根据定理\ref{theo33},集族$(f_i)_{i\in I}$(其中$f_i=f|A_i$)与$f$一一对应,令$T_i$表示$A_i$到$X_i$的双射的集合,则其数目为$p_i!$,因此使$g(f)=f(X_i)_{i\in [1, h]}$的f,与$\prod\limits_{i\in [1, h]}T_i$的元素一一对应,故其数目为$\mathop{\mathsf{P}}\limits_{i\in [1, h]}(p_i!)$,由于$G$的元素数目为$n!$,根据定理\ref{theo142}得证.
			
			\begin{theo}\label{theo146}
				\hfill\par
				$A$的元素数目为$n$,$p\leq n$,则元素数目为$p$的$A$的元素的子集数目为$n!/(p!(n-p)!)$.
			\end{theo}
			证明:根据定理\ref{theo145}可证.
			
			\begin{de}
				\textbf{组合数(coefficient binomial)}
				\par
				$n$、$p$为自然数,且$p\leq n$,则$n!/(p!(n-p)!)$称为$n$取$p$的组合数,记作$\binom{n}{p}$.
			\end{de}
			
			\begin{cor}\label{cor323}
				\hfill\par
				$n$、$p$为自然数,且$p\leq n$,则$\binom{n}{p}=\binom{n}{n-p}$.
			\end{cor}
			证明:根据定义可证.
			
			\begin{theo}\label{theo147}
				\hfill\par
				$E$、$F$均为全序有限集,元素数目分别是$p$、$n$,$p\leq n$,则$E$到$F$的严格单增映射的数目为$\binom{n}{p}$.
			\end{theo}
			证明:全序有限集为良序集,根据定理\ref{theo84},$E$到$F$的任何元素数目为$p$的子集的严格单增映射唯一,根据定理\ref{theo146}可证.
			
			\begin{theo}\label{theo148}
				\hfill\par
				$\sum\limits_{p\in [0, n]}\binom{n}{p}=2^n$.
			\end{theo}
			证明:根据定理\ref{theo108}可证.
			
			\begin{cor}\label{cor324}
				\hfill\par
				$n$为自然数,则$\binom{n}{0}=1$.
			\end{cor}
			证明:根据定义可证.
			
			\begin{cor}\label{cor325}
				\hfill\par
				$n$为自然数且$n>0$,则$\binom{n}{1}=n$.
			\end{cor}
			证明:令$A$的元素数目为$n$,$A$的元素数目为$1$的子集和$A$的元素一一对应,得证.
			
			\begin{theo}\label{theo149}
				\hfill\par
				$\binom{n+1}{p+1}=\binom{n}{p+1}+\binom{n}{p}$.
			\end{theo}
			证明:
			\par
			令$E$的元素数目为$n+1$,其中一个元素为$a$,令$E'=E-\{a\}$,$F$为$E$的元素数目为$p+1$的子集的集合,$G$为$E'$的元素数目为$p$或$p+1$的子集的集合.
			\par
			对任意$A\in F$,$A\cap E'\in G$,因此,$A\mapsto A\cap E'$为$F$到$G$的映射,令其为$f$;且对任意$B\in G$,如果$B$的元素数目为$p$,则$f(B\cup\{a\})=B$,如果B的元素数目为$p+1$,则$f(B)=B$.令$A\neq A'$,如果$a\notin A$、$a\notin A'$,则$f(A)=A$,$f(A')=A'$,$f(A)\neq f(A')$;如果$a\in A$、$a\in A'$,则$f(A)=A-\{a\}$,$f(A')=A'-\{a\}$,$f(A)\neq f(A')$;如果$a\in A$、$a\notin A'$,则$f(A)$元素数目为$p$,$f(A')$元素数目为$p+1$,$f(A)\neq f(A')$ ;如果$a\in A'$、$a\notin A$,同理$f(A)\neq f(A')$.
			\par
			综上,$f$是$F$到$G$的双射.得证.
			
			\begin{cor}\label{cor326}
				\hfill\par
				$n$为自然数且$n>1$,则$\binom{n}{2}=n(n-1)2$.
			\end{cor}
			证明:$n=2$时,命题显然成立.设命题对$n$成立,根据定理\ref{theo149},$\binom{n+1}{2}=\binom{n}{2}+\binom{n}{1}$,得证.
			
			\begin{cor}\label{cor327}
				\hfill\par
				$n$、$m$为自然数,则$\sum\limits_{i\in [0, m-1]}\binom{n+i}{n}=\binom{n+m}{n+1}$.
			\end{cor}
			证明:根据定理\ref{theo149},对$i$用数学归纳法可证.
			
			\begin{cor}\label{cor328}
				\hfill\par
				(1)$t$、$r$、$q$为自然数,则$\sum\limits_{i\in [0, r]}\binom{t+r-i}{t}\binom{q+i}{q}=\binom{q+t+r+1}{r}$.
				\par
				(2)$p$、$q$、$n$为自然数,且$p\leq n$,$q<p$,则$\sum\limits_{k\in [q+1, n-p+q+1]}\binom{n-k}{p-q-1}\binom{k-1}{q}=\binom{n}{p}$.
			\end{cor}
			证明:
			\par
			(1)如果$t=0$,对$i$使用数学归纳法可证命题成立.如果$r=0$,根据定义,命题成立.根据定理\ref{theo149},对$r+t$使用数学归纳法可证.
			\par
			(2)令$t=p-q-1$,$i=k-q-1$,$r=n-p$,根据补充定理\ref{cor328}(1)、补充定理\ref{cor323}可证.
			
			\begin{theo}\label{theo150}
				\hfill\par
				$n$为自然数,且$n>0$,则满足$i\in [1, n]$、$j\in [1, n]$且$i<j$(或$i\leq j$)的有序对$(i, j)$的数目是$n(n-1)/2$(或$n(n+1)/2$).
			\end{theo}
			证明:满足$i\in [1, n]$、$j\in [1, n]$且$i<j$的有序对$(i, j)$,与$[1, n]$的元素数目为$2$的子集一一对应,故为$n(n-1)/2$;满足$i\in [1, n]$、$j\in [1, n]$且$i<j$的有序对$(i, j)$,与$[1, n]$的元素数目为$2$或$1$的子集一一对应,故为$n(n+1)/2$.
			
			\begin{theo}\label{theo151}
				\hfill\par
				$n$为自然数,且$n>0$,则$\sum\limits_{i\in [1, n]}i=n(n+1)/2$.
			\end{theo}
			证明:对于满足$i\in [1, n]$、$j\in [1, n]$且$i\leq j$的有序对$(i, j)$,对任意$j\in [1, n]$,有序对$(i, j)$的数目为$j$,根据定理\ref{theo150}得证.
			
			\begin{theo}\label{theo152}
				\hfill\par
				$n$、$h$为自然数,$E$为元素数目为$h$的集合,$E$到$[0, n]$且满足$\sum\limits_{x\in E}u(x)\leq n$(或$\sum\limits_{x\in E}u(x)=n$且$h>0$)的映射$u$的数目是$\binom{n+h}{n}$(或$\binom{n+h-1}{h-1}$).
			\end{theo}
			证明:
			\par
			设$g$为$[1, h]$到$E$的双射.
			\par
			对于$E$到$[0, n]$且满足$\sum\limits_{x\in E}u(x)\leq n$的映射$u$,定义$f_u$如下:$f_u(1)=u(g(1))+1$,当$1\leq i\text{与}i\leq h-1$时,$f_u(i+1)=f_u(i)+1+u(g(i+1))$;则$f_u$为$[1, h]$到$[1, n+h]$的严格单增映射.根据定义可证$u\mapsto f_u$为双射,故$u$的集合的元素数目为$\binom{n+h}{n}$.
			\par
			满足$\sum\limits_{x\in E}u(x)=n$的$u$的集合的元素数目为$\binom{n+h}{n}-\binom{n+h-1}{n-1}$,故等于$\binom{n+h-1}{h-1}$.
			
			\begin{cor}\label{cor329}
				\textbf{二项式定理}
				\par
				$a$、$b$、$n$为自然数:
				\par
				(1)$(a+b)^n=\sum\limits_{i\in [0, n]}\binom{n}{i}a^{n-i}b^i$.
				\par
				(2)如果$a\geq b$,则$(a-b)^n=\sum\limits_{i\in[0, n]\text{与}i\text{为偶数}}\binom{n}{i}a^{n-i}b^i-\sum\limits_{i\in[0, n]\text{与}i\text{为奇数}}\binom{n}{i}a^{n-i}b^i$.
				\par
				(3)$n$为自然数,则:
				$\sum\limits_{i\in[0, n]\text{与}i\text{为偶数}}\binom{n}{i}=\sum\limits_{i\in[0, n]\text{与}i\text{为奇数}}\binom{n}{i}$.
				
			\end{cor}
			证明:
			\par
			(1)根据定理\ref{theo149},对$n$用数学归纳法可证.
			\par
			(2)根据定理\ref{theo149},对$n$用数学归纳法可证.
			\par
			(3)令$a=1$,$b=1$,根据补充定理\ref{cor329}(2)可证.
			
			\begin{cor}\label{cor330}
				\hfill\par
				$(a_i)_{i\in [0, n]}$、$(b_i)_{i\in [0, n]}$为自然数族,则$(\forall x)(x\text{为自然数}\Rightarrow \sum\limits_{i\in [0, n]}a_ix^i=\sum\limits_{i\in [0, n]}b_ix^i)\Leftrightarrow (\forall i)(i\in [0, n]\Rightarrow a_i=b_i)$.
			\end{cor}
			证明:根据公理模式\ref{Sch7},右边$\Rightarrow$ 左边;令$x=sup(sup(a_i)_{i\in I}, sup(b_i)_{i\in I})+1$,根据补充定理\ref{cor320},左边$\Rightarrow$右边.
			
			\begin{cor}\label{cor331}
				\hfill\par
				$a$、$b$、$n$、$p$为自然数:
				\par
				(1)$\binom{n}{p}(a+b)^p=\sum\limits_{i\in [0, p]}\binom{n}{i}\binom{n-i}{p-i}a^{p-i}b^i$;
				\par
				(2)$\binom{n}{p}(a-b)^p=\sum\limits_{i\in[0, p]\text{与}i\text{为偶数}}\binom{n}{i}\binom{n-i}{p-i}a^{p-i}b^i-\sum\limits_{i\in[0, n]\text{与}i\text{为奇数}}\binom{n}{i}\binom{n-i}{p-i}a^{p-i}b^i$.
				\par
				(3)$\sum\limits_{i\in [0, p]}\binom{n}{i}\binom{n-i}{p-i}=2^p\binom{n}{p}$.
				\par
				(4)$\sum\limits_{i\in[0, p]\text{与}i\text{为偶数}}\binom{n}{i}\binom{n-i}{p-i}=\sum\limits_{i\in[0, n]\text{与}i\text{为奇数}}\binom{n}{i}\binom{n-i}{p-i}$.
			\end{cor}
			证明:
			\par
			(1)根据补充定理\ref{cor329}(1),$(x+a+b)^n=\sum\limits_{p\in [0, n]}\binom{n}{p}(a+b)^px^{n-p}$;
			\par
			同时,$(x+a+b)^n=\sum\limits_{i\in [0, n]}\binom{n}{i}(x+a)^{n-i}b^i$,
			\par
			即$\sum\limits_{i\in [0, n]}\binom{n}{i}b^i(\sum\limits_{q\in [0, n-i]}\binom{n-i}{q}x^qa^{n-i-q})$,
			\par
			即$\sum\limits_{(i, q)\in \{(x, y)|x\in [0, n]\text{与}y\in [0, n-x]\}}\binom{n}{i}\binom{n-i}{q}x^qa^{n-i-q}b^i$,
			\par
			即$\sum\limits_{q\in [0, n]}(\sum\limits_{i\in [0, n-q]}\binom{n}{i}\binom{n-i}{q}x^qa^{n-i-q}b^i)$,
			\par
			即$\sum\limits_{p\in [0, n]}(\sum\limits_{i\in [0, p]}\binom{n}{i}\binom{n-i}{p-i}x^{n-p}a^{p-i}b^i)$.
			\par
			根据补充定理\ref{cor330},得证.
			\par
			(2)考虑$(x+a-b)^n$,类似补充定理\ref{cor331}(1)可证.
			\par
			(3)令$a=1$,$b=1$,根据补充定理\ref{cor331}(1)可证.
			\par
			(4)令$a=1$,$b=1$,根据补充定理\ref{cor331}(2)可证.

			\begin{exer}\label{exer138}
				\hfill\par
				$p$、$q$、$n$为自然数,且$p\leq n$,$q<p$,求证:$\sum\limits_{k\in [q+1, n-p+q+1]}\binom{n-k}{p-q-1}\binom{k-1}{q}=\binom{n}{p}$.
			\end{exer}
			证明:即补充定理\ref{cor328}(2).
			
			\begin{exer}\label{exer139}
				\hfill\par
				$n$为自然数,求证:
				$(a-b)^n=\sum\limits_{i\in[0, n]\text{与}i\text{为偶数}}\binom{n}{i}=\sum\limits_{i\in[0, n]\text{与}i\text{为奇数}}\binom{n}{i}$.
			\end{exer}
			证明:即补充定理\ref{cor329}(3).
									
			\begin{exer}\label{exer140}
				\hfill\par
				$n$、$p$为自然数,求证:
				\par
				(1)$\sum\limits_{i\in [0, p]}\binom{n}{i}\binom{n-i}{p-i}=2^p\binom{n}{p}$;
				\par
				(2)$\sum\limits_{i\in[0, p]\text{与}i\text{为偶数}}\binom{n}{i}\binom{n-i}{p-i}=\sum\limits_{i\in[0, n]\text{与}i\text{为奇数}}\binom{n}{i}\binom{n-i}{p-i}$.
			\end{exer}
			证明:
			\par
			(1)即补充定理331(3).
			\par
			(2)即补充定理331(4).
			
			\begin{exer}\label{exer141}
				\hfill\par
				定义$[1, h]$到$[0, n]$且满足$\sum\limits_{x\in E}u(x)\leq n$的映射$u$的集合,到$[1, h]$到$[1, n+h]$的严格单增映射的集合的双射,从而证明定理\ref{theo152}.
			\end{exer}
			证明:参见定理\ref{theo152}的证明.
			
			\begin{exer}\label{exer142}
				\hfill\par
				(1)$E$为分配格,$f$为$E$到具有运算法则“$+$”的可交换幺半群$M$的映射,并且对任意$x\in E$、$y\in E$,$f(x)+f(y)=f(sup(x, y))+f(inf(x, y))$,求证:对$E$的任意有限子集$I$,$f(sup(I))+\sum\limits_{n\in \{k|k>0\text{与}2k\leq Card(I)\}}(\sum\limits_{H\in \{K|K\subset I\text{与}Card(K)=2n\}}f(inf(H)))=\\\sum\limits_{n\in \{k|2k+1\leq Card(I)\}}(\sum\limits_{H\in \{K|K\subset I\text{与}Card(K)=2n+1\}}f(inf(H)))$.
				\par
				(2)$A$为集合,$(B_i)_{i\in I}$为$A$的有限子集族,$B=\bigcup\limits_{i\in I}B_i$,对任意$H\subset I$,令$B_H=\bigcap\limits_{i\in H}B_i$,求证:$Card(B) +\sum\limits_{n\in \{k|k>0\text{与}2k\leq Card(I)\}}(\sum\limits_{H\in \{K|K\subset I\text{与}Card(K)=2n\}}Card(B_H))=\\\sum\limits_{n\in \{k|2k+1\leq Card(I)\}}(\sum\limits_{H\in \{K|K\subset I\text{与}Card(K)=2n+1\}}Card(B_H)))$.
			\end{exer}
			证明:
			\par
			(1)	对$Card(I)$运用数学归纳法可证.
			\par
			(2)	根据习题\ref{exer142}(1)可证.
			\par
			注:习题\ref{exer142}涉及尚未介绍的“可交换幺半群”知识.
			
			\begin{exer}\label{exer143}
				\hfill\par
				$n$、$h$为自然数,求证:$\sum\limits_{i\in \{k|k\text{为偶数}\text{与}k \in [0, n]\}}\binom{h}{i}\binom{n+h-i}{h}=\sum\limits_{i\in \{k|k\text{为奇数}\text{与}k \in [0, n]\}}\binom{h}{i}\binom{n+h-i}{h}+1$.
			\end{exer}
			证明:令$B=\{u|u\text{为}[1, h]\text{到}[0, n]\text{的映射}\text{与}\sum\limits_{x\in [1, h]}u(x)\leq n\}-\{u|u\text{为}[1, h]\text{到}[0, n]\text{的映射}\text{与}\\(\forall x)(x\in [1, h]\Rightarrow u(x)=0\}$,$B_i=\{u|u\in B\text{与}u(i)\geq 1\}$($i\in [1, h]$),根据习题\ref{exer142}(2)可证.
			
			\begin{exer}\label{exer144}
				\hfill\par
				令$S_{n,p}$为$[1, n]$到$[1, p]$的满射的数目:
				\par
				(1)求证:$\sum\limits_{i\in \{k|k\text{为偶数}\text{与}k \in [0, n]\}}\binom{p}{i}(p-i)^n=S_{n,p}+\sum\limits_{i\in \{k|k\text{为奇数}\text{与}k \in [0, n]\}}\binom{p}{i}(p-i)^n$.
				\par
				(2)求证:$S_{n,p}=p(S_{n-1,p}+S_{n-1,p-1})$.
				\par
				(3)求证:Sn+1, n=n((n+1)!)/2,Sn+2, n=n(3n+1)((n+2)!)/24.
				\par
				(4)	令$P_{n, p}$为满足下列条件的$G$的数目:
				\par
				第一,$\Delta_G$为$[1, n]$的划分;
				\par
				第二,$Card(G)=p$.
				\par
				求证:$S_{n,p}=p!P_{n,p}$.
			\end{exer}
			证明:
			\par
			(1)由于$p^n=\sum\limits_{i\in [0, n]}S_{n,p-i}\binom{p}{i}$,根据补充定理\ref{cor331}(2)可证.
			\par
			(2)	根据习题\ref{exer144}(1)、定理\ref{theo149}可证.
			\par
			(3)	考虑$[1, n+1]$、$[1, n+2]$的$n$个集合的划分,根据定理\ref{theo142}可证.
			\par
			(4)	根据定理\ref{theo142}可证.
						
			\begin{exer}\label{exer145}
				\hfill\par
				$E$的元素数目为$n$,$p_n$为$\{u|u\text{为}E\text{的排列}\text{与}(\forall x)(x\in E\Rightarrow u(x)\neq x)\}$,求证:\\$\sum\limits_{i\in \{k|k\text{为偶数}\text{与}k \in [0, n]\}}\binom{n}{i}(n-i)!=p_n+\sum\limits_{i\in \{k|k\text{为奇数}\text{与}k \in [0, n]\}}\binom{n}{i}(n-i)!$,并且,当$n$趋向无穷大时,$p_n$趋向$n!/e$.
			\end{exer}
			证明:根据习题\ref{exer142}(2)可证.
			\par
			注:习题\ref{exer145}涉及尚未介绍的“实数”和“数列极限”知识.
			
			\begin{exer}\label{exer146}
				\hfill\par
				(1)$E$的元素数目为$qn$,求证:满足下列条件的集合$G$的数目,为$(qn)!/(n!(q!)n)$:
				\par
				第一,$G$的元素数目为$n$;
				\par
				第二,$G$的任何元素的元素数目均为$p$;
				\par
				第三,$\Delta_G$为$E$的划分.
				\par
				(2)$E=[1, qn]$,令$A$为满足下列条件的集合$G$的数目,为(qn)!/(n!(q!)n):
				\par
				第一,$G$的元素数目为$n$;
				\par
				第二,$G$的任何元素的元素数目均为$p$;
				\par
				第三,$\Delta_G$为$E$的划分;
				\par
				第四,$G$的任何元素都不是区间.
				\par
				则$\sum\limits_{i\in \{k|k\text{为偶数}\text{与}k \in [0, n]\}}(qn-i(q-1))!/(i!(n-i)!q^{n-i})=A+\sum\limits_{i\in \{k|k\text{为奇数}\text{与}k \in [0, n]\}}\\(qn-i(q-1))!/(i!(n-i)!q^{n-i})$.
			\end{exer}
			证明:
			\par
			(1)根据定理\ref{theo142}可证.
			\par
			(2)根据习题\ref{cor142}(2)可证.
			
			\begin{exer}\label{exer147}
				\hfill\par
				令$q_{n,k}$为$[1, k]$到$[1, n]$的满足下列条件的严格单增映射u的数目:
				对任意奇数(或偶数)$x\in [1, k]$,$u(x)$为偶数(或奇数),则$q_{n,k}=C([(n+k)/2], k)$.
			\end{exer}
			证明:令$u'(x)=u(x)+x$,则$u\mapsto u'$为双射,故所求映射数目等于$[1, k]$到$[1, n+k]$中的偶数组成的集合的映射数目,得证.
			
			\begin{exer}\label{exer148}
				\hfill\par
				$E$为$n$个符号组成的集合,$S$为将符号$f$添加到$E$得到的集合.设$f$的权重为$2$,$E$的元素的权重为$0$.
				\par
				(1)$M$为$L_0(S)$中满足下列条件的有意义的单词的集合:
				\par
				$E$的各元素均出现且仅出现一次.令$u_n$为$M$的元素数目,求证:$u_{n+1}=(4n-2)u_n$,并且当$n\geq 2$时,$u_n=\mathop{\mathsf{P}}\limits_{i\in [2, n]}(4n-6)$.
				\par
				(2)	令$x_i$为$E$的第$i$个符号,求证:如果给定$E$的符号顺序,则$M$的单词数目$v_n=\\\binom{2n-2}{n-1}/n$,并且$v_{n+1}=\sum\limits_{i\in [1, n]}(v_iv_{n+1-i})$.
			\end{exer}
			证明:
			\par
			(1)长度为$2n-2$的字符串,其中$n-1$个符号为$f$(权重为$2$)、$n-1$个符号为$g$(权重为$0$),其数目为$\binom{2n-2}{n-1}$);而其中“不合法”即存在$m<n-1$,使前$m$个字符权重之和小于$m$,对任意不合法的字符串$S$,对其中满足条件的最小$m$,从$m+1$个字符开始,将$f$和$g$调换,则得到长度$2n-2$的字符串$S'$,其中$n-2$个符号为$f$,$n$个符号位$g$.$S\mapsto S'$为双射,故“合法”的字符串为$\binom{2n-2}{n-1}-\binom{2n-2}{n-2}=\binom{2n-2}{n-1}/n$,因此$u_n=(2n-2)!/(n-1)!$,因此$u_{n+1}=(4n-2)u_n$,用数学归纳法可证当$n\geq 2$时,$u_n=\mathop{\mathsf{P}}\limits_{i\in [2, n]}(4n-6)$.
			\par
			(2)根据习题\ref{exer148}(1)可证.
			
			\begin{exer}\label{exer149}
				\hfill\par
				(1)$p$、$q$为自然数且$p\geq 1$、$q\geq 1$,$n=2p+q$,$E$的元素数目为$n$,令$N=\binom{n}{p}$,$(X_i)_{i\in [1, n]}$为$E$的所有元素数目为$p$的子集按某种顺序排成的序列,$(Y_i)_{i\in [1, n]}$为$E$的所有元素数目为$p+q$的子集按某种顺序排成的序列,
				\par
				求证:存在一个$[1, N]$到$[1, N]$的双射$f$,使对任意$i\in [1, N]$,均有$X_{f(i)}\subset Y_i$.
				\par
				(2)$h$、$k$为自然数且$p\geq 1$、$q\geq 1$,$n$为自然数且$2h+k<n$,$E$的元素数目为$n$,
				\par
				$(X_i)_{i\in [1, r]}$为$E$的若干不同的各元素数目为$h$的子集按某种顺序排成的序列,求证:存在$E$的若干不同的各元素数目为$h+k$的子集按某种顺序排成的序列$(Y_i)i\in [1, r+1]$,并且,对任意$i\in [1, r+1]$,均存在$j\in [1, r]$,使$X_j\subset Y_i$,对任意$i\in [1, r]$,均存在$j\in [1, r+1]$,使$X_i\subset Y_j$.
			\end{exer}
			证明:
			\par
			(1)根据补充定理\ref{cor308}可证.
			\par
			(2)对$n$用数学归纳法,分别考虑含有某个元素$a$的集合,和不含某个元素$a$的集合,根据习题\ref{exer149}(1)可证.
						
			\begin{exer}\label{exer150}
				\hfill\par
				$m$、$q$为自然数,$E$的元素数目为$2m$,且$q<m$,F是满足下列条件的$\mathcal{P}(E)$的子集$G$的集合:令$X$、$Y$为$G$的两个不同元素,且$X\subset Y$,则$Card(Y-X)\leq 2q$.
				\par
				(1)设$\{k|(\exists M)(M\in F\text{与}k=Card(M))\}$的最大元为$p$,并且,$p=Card(M)$,$M\in F$,求证:对任意$A\in M$,$Card(A)\geq m-q$,$Card(A)\leq m+q$.
				\par
				(2)对任意$M\in F$,求证:$Card(M)\leq \sum\limits_{k\in [0, 2q]}\binom{2m}{m-q+k}$.
				\par
				(3)将$2m$和$2q$均替换为$2m+1$、$2q+1$,试给出相应的结论.
			\end{exer}
			证明:
			\par
			(1)设$\{x|(\exists A)(A\in M\text{与}x=Card(A))\}$的最小元是$m-q-s$($s\geq 1$),令$X=\{A|A\in M\text{与}Card(A)=m-q-s\}$,$Card(X)=r$,根据习题\ref{exer149}(2),存在$r+1$个基数为$m+q-s+1$个$E$的子集组成集合$Y$,使$X$的任何元素都是某个$Y$的元素的子集,且$Y$的任何元素都包含$X$的某个元素.则$(M-X)\cup Y \in F$,且$Card((M-X)\cup Y)>Card(M)$,矛盾.
			\par
			(2)根据习题\ref{exer150}(1)可证.
			\par
			(3)类似习题\ref{exer150}(1)、\ref{exer150}(2)可知,相应的基数为$\sum\limits_{k\in [0, 2q+1]}\binom{2m+1}{m-q+k)}$.
			
			\begin{exer}\label{exer151}
				\hfill\par
				$E$为有限集合,元素数目为$n$,$(a_i)_{i\in [1, n]}$为$E$的各元素按某种顺序排成的序列,$(A_i)i\in [1, m]$为E的若干子集按某种顺序排成的序列,
				\par
				(1)令$k_j=Card(\{i|i\in [1, m]\text{与}a_j\in Ai\})$,$s_i=Card(A_i)$,求证:$\sum\limits_{j\in [1, n]}k_j=\sum\limits_i\in [1, m]s_i$.
				\par
				(2)对任意$x\in E$、$y\in E$,均存在唯一的$i\in [1, n]$,使$x\in A_i$、$y\in A_i$,求证:如果$a_j\notin A_i$,则$s_i\leq k_j$.
				\par
				(3)对任意$x\in E$、$y\in E$,均存在唯一的$i\in [1, n]$,使$x\in A_i$、$y\in A_i$,求证:存在$i\in [1, n]$使$A_i=E$,或者$m\geq n$.
				\par
				(4)对任意$x\in E$、$y\in E$,均存在唯一的$i\in [1, n]$,使$x\in A_i$、$y\in A_i$,且$m=n$,求证:$(A_i)_{i\in [1, m]}$必然符合下列两种情况之一:
				\par
				第一,$A_1=\{a_1, a_2, \cdots, a_{n-1}\}$,以及$A_i=\{a_i, a_n\}(i\in [1, n-1])$;第二,每个子集都有$k$个元素,每个元素都属于$k$个子集.
			\end{exer}
			证明:
			\par
			(1)根据定义可证.
			\par
			(2)根据定义可证.
			\par
			(3)设$k$是$\{a|a=k_i\text{与}i\in [1, n]\}$的最小元,相应的元素为$a$,子集为$A_1$、$A_2$、$\cdots$、$A_k$,令$B_i=A_i-\{a\}$,令$b_i=Card(B_i)$,设其中最大元为$b_k$,如果$b_k\geq k$,根据习题\ref{exer151}(2),对任意$a_i\notin A_k$,$1+b_k\leq k_i$,根据习题\ref{exer151}(2),$\sum\limits_{i\in [1, m]}s_i\geq (1+b_k)\sum\limits_{i\in [1, k-1]}b_i+k(b_k+1)$.同时,$\sum\limits_{i\in [1, m]}s_i\leq k+\sum\limits_{i\in [1, k]}b_i+(m-k)k$.令$y=b_k$,$\sum\limits_{i\in [1, k]}b_i=x$,由于$(y-k)(x-1)+k(k-2)\geq 0$,故$m\geq 1+x+y$.如果$b_k<k$,根据习题\ref{exer151}(2),$(1+\sum\limits_{i\in [1, k]}b_i)k\leq k+\sum\limits_{i\in [1, k]}b_i+(m-k)k$,由于$\sum\limits_{i\in [1, k]}b_i\leq k(k-1)$,故$m\geq 1+\sum\limits_{i\in [1, k]}b_i$.
			\par
			(4)根据习题\ref{exer151}(3)证明过程中等号成立的条件可证.
			\par
			注:原书习题\ref{exer151}(3)有误.
			
			\begin{exer}\label{exer152}
				\hfill\par
				$E$为有限集合,$X$、$Y$为$\mathcal{P}(E)$的两个非空子集,$a$、$b$、$c$、$d$是四个不等于$0$的自然数,并且满足下列条件:
				\par
				第一,对任意$A\in X$、$B\in Y$,$Card(A\cap B)\geq a$;
				\par
				第二,对任意$A\in X$,$Card(A)\leq b$;
				\par
				第三,对任意$B\in Y$,$Card(B)\leq c$;
				\par
				第四,对任意$x\in E$,$X\cup Y$的元素中,含有$x$的数目均为$d$.
				\par
				求证:$Card(E)\leq bc/a$,并且,当且仅当满足下列条件时,等号成立:
				\par
				第一,对任意$A\in X$、$B\in Y$,$Card(A\cap B)=a$;
				\par
				第二,对任意$A\in X$,$Card(A)=b$;
				\par
				第三,对任意$B\in Y$,$Card(B)=c$;
				\par
				第四,存在$r\leq d$,使对任意$x\in E$,$X$的元素中,含有$x$的元素数目均为$r$.
			\end{exer}
			证明:令$Card(X)=s$,$Card(Y)=t$,$E$的元素为$a_1$、$a_2$、$\cdots$、$a_n$,$r_i$为含有$a_i$的$X$的元素的数目,则$\sum\limits_{i\in [1, n]}r_i\leq sb$,$\sum\limits_{i\in [1, n]}(d-r_i)\leq tc$,$\sum\limits_{i\in [1, n]}(r_i(d-r_i))\geq ast$,对$n$用数学归纳法,可证$n\sum\limits_{i\in [1, n]}(r_i(d-r_i))\leq (\sum\limits_{i\in [1, n]}r_i)(\sum\limits_{i\in [1, n]}(d-r_i))$,故$Card(E)\leq bc/a$.其中等号成立的条件是所有$r_i$都相等、且对任意$A\in X$、$B\in Y$,$Card(A\cap B)=a$;对任意$A\in X$,$Card(A)=b$;对任意$B\in Y$,$Card(B)=c$.
			
			\begin{exer}\label{exer153}
				\hfill\par
				$E$为有限集合,元素数目为$n$,$X$为$\mathcal{P}(E)$的非空子集,$a$、$b$、$c$是三个不等于$0$的自然数,并且满足:
				\par
				第一,对$X$的任意不同的元素$A$、$B$,$Card(A\cap B)=a$;
				\par
				第二,对任意$A\in X$,$Card(A)\leq b$;
				\par
				第三,对任意$x\in E$,$X$的元素中,含有$x$的元素数目均为$c$.
				\par
				求证:$n(a-1)\leq b(b-1)$.
			\end{exer}
			证明:令$x\in E$,$Y=\{Q|(\exists A)(A\in X\text{与}x\in A\text{与}Q=A-\{x\})\}$,$Z=\{Q|Q\in X\text{与}x\notin Q\}$,根据习题\ref{exer152},$a(n-1) \leq b(b-1)$,又因为$a\leq n$,故$n(a-1)\leq b(b-1)$.
						
			\begin{exer}\label{exer154}
				\hfill\par
				$i$、$h$、$k$是自然数,$i\geq 1$,$h\geq i$,$k\geq i$,求证:存在满足下列条件的自然数$m_i(h, k)$:
				\par
				对任意有限集合$E$,令$F_i(E)$为$E$的元素数目为i的子集的集合,如果$E$的元素数目不小于$m_i(h, k)$,且对满足$X\subset F_i(E)$的任意集合$X$,令$Y= F_i(E)-X$,则$E$的任意$h$个元素的子集都包含一个$X$的元素,E的任意$k$个元素的子集都包含一个$Y$的元素,不可能同时出现.
			\end{exer}
			证明:根据定义,$m_1(h, k)=h+k-1$,$m_i(i, k)=k$,$m_i(h, i)=h$.
			\par
			下面证明$m_i(h, k)=m_{i-1}(m_i(h-1, k), m_i(h, k-1))+1$.
			\par
			如果元素数目为该数的$E$不满足条件,即存在$X\subset F_i(E)$、$Y= F_i(E)-X$,使$E$的任意$h$个元素的子集都包含一个$X$的元素,$E$的任意$k$个元素的子集都包含一个$Y$的元素.令$a\in E$,$E'=E-\{a\}$,$E'$的任意$m_i(h-1, k)$个元素的子集,其任意$k$个元素的子集都包含一个$Y\cup\mathcal{P}(E')$的元素,故存在某个$h-1$个元素的子集$Z$,其所有$i$个元素的子集都是$Y\cup\mathcal{P}(E')$的元素,由于$Z\cup\{a\}$包含一个$X$的元素$U$,故$a\in U$,令$X'=\{G|G\subset E'\text{与}G\cup\{a\}\in X\}$,故$U-\{a\}\in X'$,即$E'$的任意$m_i(h-1, k)$个元素的子集,包含一个$X'$的元素,
			\par
			同理,令$Y'=\{G|G\subset E'\text{与}G\cup\{a\}\in Y\}$,则$E'$的任意$m_i(h, k-1)$个元素的子集,包含一个$Y'$的元素.
			\par
			此外,$X'\subset F_{i-1}(E')$,$Y'=F_{i-1}(E')-X'$,矛盾.
				
			\begin{exer}\label{exer155}
				\hfill\par
				(1)$E$为有限偏序集,元素数目为$p$,$m$、$n$为自然数且$mn<p$,求证:$E$有$m$元全序子集,或者有$n$元自由子集.
				\par
				(2)令$h$、$k$为不等于$0$的自然数,令$r(h, k)=(h-1)(k-1)$,$I$为有限集,其元素均为自然数,且元素数目不小于$r(h, k)$,$E$为全序集,$(X_i)_{i\in I}$为$E$的元素的有限序列,求证:存在$I$的元素数目为$h$的子集$H$使$(x_i)_{i\in H}$为单增序列,或者存在$I$的元素数目为$k$的子集$K$使\\$(x_i)_{i\in K}$为单减序列.
			\end{exer}
			证明:
			\par
			(1)根据补充定理\ref{cor306}可证.
			\par
			(2)根据补充定理\ref{cor306}可证.
			\par
			注:根据序列的定义,习题\ref{exer151}(2)的条件中,集合$I$的性质做了修改.

		\section{无穷集合(Ensembles infinis)}		
			\begin{de}
				\textbf{无穷集合(ensemble infini)}
				\par
				如果一个集合不是有限集合,则称其为无穷集合.
			\end{de}
			
			\begin{de}
				\textbf{无穷基数(cardinal infini)}
				\par
				无穷集合的基数称为无穷基数.
			\end{de}
			
			\begin{cor}\label{cor332}
				\textbf{无穷公理和自然数集合的存在性等价}
				\par
				$(x\text{为自然数})\text{是}x\text{上的集合化公式}\Leftrightarrow (\exists X)(X\text{为无穷集合})$.
			\end{cor}
			证明:
			\par
			若存在无穷集合,令$a$为无穷集合的基数,根据定理\ref{theo116},对任意自然数$n$,$a>n$.根据补充定理\ref{cor293},存在集合$\{x|x\text{为基数}\text{与}x<a\}$,则任意自然数n均属于该集合,根据证明规则52可证.
			\par
			反过来,如果$(x\text{为自然数})\text{是}x\text{上的集合化公式}$,令该集合为$E$,对任意自然数$n$,\\$Card([0, n])=n+1$,因此$Card(E)\neq n$,故$E$为无穷集合.
			
			\begin{ex}\label{ex4}
				\text{无穷公理}
				\par
				$(\exists X)(X\text{为无穷集合})$.
			\end{ex}
			
			\begin{metadef}
				\textbf{集合论(théorie des ensembles)}
				\par
				包含$2$元特别符号$\in$ 、显式公理\ref{ex1}、显式公理\ref{ex2}、显式公理\ref{ex3}、显式公理\ref{ex3}、显式公理\ref{ex4}和公理模式\ref{Sch8}的等式理论,称为集合论.
			\end{metadef}
			
			\begin{theo}\label{theo153}
				\textbf{存在自然数集合}
				\par
				$(x\text{为自然数})\text{是}x\text{上的集合化公式}$.
			\end{theo}
			证明:根据显式公理\ref{ex4}、补充定理\ref{cor332}可证.
						
			\begin{de}
				\textbf{自然数集合(ensemble des entiers)}
				\par
				$\{n|n\text{为自然数}\}$称为自然数集合,记作$N$.
			\end{de}
			
			\begin{cor}\label{cor333}
				\hfill\par
				(1)$N$为良序集.
				\par
				(2)$a$为有限基数,$E$为无穷集合,则$a<Card(E)$.
			\end{cor}
			证明:
			\par
			(1)根据定理\ref{theo89}可证.
			\par
			(2)根据定理\ref{theo116}可证.
			
			\begin{de}
				\textbf{序列(suite),元素序列(suite d'éléments),无穷序列(suite infinie)}
				\par
				指标集是$N$的子集的族(或$E$的元素族),称为序列(或元素序列);序列$(x_n)_{n\in N\text{与}R}$也可以记作$(x_n)_R$;在没有歧义的情况下,$(x_n)_{n\geq 0}$和$(x_n)_{n\geq 1}$均可简记为$(x_n)$.指标集是$N$的无穷子集的序列,称为无穷序列.
			\end{de}
			
			\begin{de}
				\textbf{单增序列(suite croissante),单减序列(suite décroissante),单调序列(suite monotone),严格单增序列(suite strictement croissante),严格单减序列(suite strictement décroissante),严格单调序列(suite strictement monotone)}
				\par
				如果序列相应的映射为单增映射(或单减映射、单调映射、严格单增映射、严格单减映射、严格单调映射),则称该序列为单增序列(或单减序列、单调序列、严格单增序列、严格单减序列、严格单调序列).
			\end{de}

			\begin{de}
				\textbf{仅顺序不同的序列(suite qui ne diffèrent que par l'ordre des termes)
				}
				\par
				如果$(x_n)_{n\in I}$和$(y_n)_{n\in I}$的指标集相同,且存在$I$的排列$f$,使$(\forall n)(n\in I\Rightarrow x_{f(n)}=y_n)$,则称$(x_n)_{n\in I}$和$(y_n)_{n\in I}$为仅顺序不同的序列.
			\end{de}
			
			\begin{C}\label{C62}
				\hfill\par
				集合论中,$u$为字母,$T$为项,则存在唯一的项$U$和$N$到$U$的满射$f$,使对任意自然数$n$,令$f(n)$为$[0, n[$到$f\langle[0, n[\rangle$的满射,并在$[0, n[$上和$f$重合,使$(f(n)|u)T=f(n)$.
			\end{C}
			证明:根据证明规则\ref{C60}可证.
						
			\begin{C}\label{C63}
				\hfill\par
				集合论中,$S$和$a$为项,则存在唯一的项$V$和$N$到$V$的满射$f$,使$f(0)=a$,并且对任意$n\geq 1$,$f(n)=(f(n-1)|n)S$.
			\end{C}
			证明:
			令$T$为$\tau_y((u=(\varnothing, \varnothing, \varnothing)\text{与}(y=a))\text{或}(u\neq (\varnothing, \varnothing, \varnothing)\text{与}y=(u(M(u))|v)S))$,其中$M(u)=(u\text{的定义域的最小上界})$,由于$f(0)=(\varnothing, \varnothing, \varnothing)$,所以$(u|f(0))T=a$;$n>0$时,$(u|f(n))T=(f(n-1)|n)S$,根据证明规则\ref{C62},得证.
			
			\begin{theo}\label{theo154}
				\hfill\par
				任何无穷集合$E$,均存在和$N$等势的子集.
			\end{theo}
			证明:根据定理\ref{theo78},在$E$上存在良序,令$E$为按该良序排序的良序集.假设$E$不包含和$N$等势的子集,则$E$和$N$的片段等势,故$E$为有限集合,矛盾.
						
			\begin{theo}\label{theo155}
				\hfill\par
				$Card(N\times N)=Card(N)$.
			\end{theo}
			证明:考虑映射$(m, n)\mapsto (m+n)(m+n)+m$,如果$(m'+n')(m'+n')+m'=(m+n)(m+n)+m$,假设$m'+n'>m+n$,则$(m'+n')(m'+n')\geq (m+n+1)(m+n+1)$,故$(m'+n')(m'+n')>(m+n)(m+n)+m$,矛盾;同理$m'+n'<m+n$也矛盾,故$m'+n'=m+n$,因此$m=m'$,$n=n'$,故该映射为$N\times N$到$N$的单射.故$Card(N\times N)\leq Card(N)$.考虑映射$n\mapsto (n, 0)$,其为$N$到$N\times N$的单射.故$Card(N)\leq Card(N\times N)$.得证.

			\begin{theo}\label{theo156}
				\hfill\par
				$a$为无穷基数,则$a^2=a$.
			\end{theo}
			证明:
			\par
			令$card(E)=a$,$D$为$E$的子集且$D$和$N$等势.根据定理\ref{theo155},$Card(D\times D)=Card(D)$.
			\par
			令$f_0$为$D$到$D\times D$的双射,令$M=\{z|z\text{为有序对}\text{与}pr_1z\subset E\text{与}D\subset pr_1z\text{与}(pr_2z\text{为}pr_1z\text{到}\\pr_1z\times pr_1z\text{的双射})\text{与}(pr_2z\text{为}f_0\text{在}pr_1z\text{上的延拓})\}$,$R$为$z\in M\text{与}z'\in M\text{与}(pr_1z\subset pr_1z')\text{与}\\(pr_2z'\text{为}pr_2z\text{在}pr_1z'\text{上的延拓})$,则$R$为偏序关系.
			\par
			因此,根据补充定理\ref{cor226}(2),$M$是归纳集,因此令$M$的极大元是$(F, f)$.
			\par
			令$b=Card(F)$,则$b=b^2$,由于$b$为无穷基数,因此$b\leq 2b$,$2b\leq 3b$,$3b\leq b^2$,故$b=2b$,$2b=3b$.
			\par
			假设$b<a$,若$Card(E-F)\leq b$,则$Card(E)\leq 2b$,即$Card(E)\leq b$,矛盾,故$Card(E-F)>b$.
			\par
			因此存在$Y\subset E-F$,且$Y$和$F$等势,令$Z=F\cup Y$,则$Z\times Z=(Y\times Y)\cup(F\times Y)\cup(Y\times F)\big(F\times F)$,由于$Card(Y\times Y)=b$,$Card(Y\times F)=b$,$Card(F\times Y)=b$,故$Card(Z\times Z-F\times F)=b$,因此存在$Y$到$Z\times Z-F\times F$的双射,故存在$Z$到$Z\times Z$的双射,其为$f$的延拓,与$(F, f)$是极大元矛盾.
			\par
			因此$b=a$,故$a^2=a$.
			
			\begin{theo}\label{theo157}
				\hfill\par
				$a$为无穷基数,$n$为自然数且$n>0$,则$a^n=a$.
			\end{theo}
			证明:根据定理\ref{theo156},用数学归纳法可证.
			
			\begin{theo}\label{theo158}
				\hfill\par
				如果基数有限族$(a_i)_{i\in I}$满足$(\forall i)(i\in I\Rightarrow a_i\neq 0)$,$\bigcup\limits_{i\in I}\{a_i\}$的最大元$a$为无穷基数,则该基数族的积等于$a$.
			\end{theo}
			证明:设其积为$b$,$I$的元素数目为$n$,由于$b\leq a^n$,则$b\leq a$,同时,由于$(\forall i)(i\in I\Rightarrow a_i\geq 1)$,因此$b\geq a$,故$b=a$.
			
			\begin{theo}\label{theo159}
				\hfill\par
				$a$为无穷基数,基数族$(a_i)_{i\in I}$满足$(\forall i)(i\in I\Rightarrow a_i\leq a)$,并且$Card(I)\leq a$,则$\sum\limits_{i\in I}a_i\leq a$,如果$(\exists i)(i\in I\text{与}a_i=a)$,则$\sum\limits_{i\in I}a_i=a$.
			\end{theo}
			证明:$\sum\limits_{i\in I}a_i\leq Card(I)\cdot a$,因此$\sum\limits_{i\in I}a_i\leq a^2$,故$\sum\limits_{i\in I}a_i\leq a$.如果存在$i\in I$使$a_i=a$,因为$a_i\leq \sum\limits_{i\in I}a_i$,故$\sum\limits_{i\in I}a_i=a$.
			
			\begin{theo}\label{theo160}
				\hfill\par
				$a$、$b$均不为0,且其中至少有一个为无穷基数,则$a+b=sup(a, b)$,$ab=sup(a, b)$.
			\end{theo}
			证明:根据定理\ref{theo158}、\ref{theo159}可证.
						
			\begin{cor}\label{cor334}
				\hfill\par
				$a$、$b$、$c$、$d$均为基数,如果$a<c$,$b<d$,则$a+b<c+d$,$ab<cd$.
			\end{cor}
			证明:如果$c$、$d$均为有限集合,根据定理\ref{theo131}可证.
			如果$c$、$d$至少有一个为无穷集合,则$c+d=sup(c, d)$,$cd=sup(c, d)$.如果$a$、$b$均为有限集合,则$a+b$、$ab$为有限集合,命题成立;如果$a$、$b$至少有一个是无穷集合,则$a+b=sup(a, b)$,$ab=sup(a, b)$,命题同样成立.
			
			\begin{cor}\label{cor335}
				\hfill\par
				$E$为无穷集合,则 :
				\par
				(1)$Card(E)Card(N)=Card(E)$,$Card(E)+Card(N)=Card(E)$;
				\par
				(2)$n$为自然数,$nCard(E)=Card(E)$,$n+Card(E)=Card(E)$.
			\end{cor}
			证明:根据定理\ref{theo160}可证.
			
			\begin{cor}\label{cor336}
				\hfill\par
				$a$为基数,$b$、$c$为无穷基数:
				\par
				(1)$(2^b)^b=2^b$.
				\par
				(2)如果$a\leq 2^b$且$a\geq 2$,则$a^b=2^b$.
				\par
				(3)如果$a<b$且$b<a^c$,则$b^c=a^c$.
				\par
				(4)$b^b=2^b$.
			\end{cor}
			证明:
			\par
			(1)根据定理\ref{theo156}、定理\ref{theo106}可证.
			\par
			(2)根据定理\ref{theo112}、补充定理\ref{cor336}(1)可证.
			\par
			(3)根据定理\ref{theo112},$b^c\geq a^c$,$b^c\leq a^{cc}$,根据定理\ref{theo156},$b^c\leq a^c$,得证.
			\par
			(4)根据补充定理\ref{cor336}(2)可证.
			
			\begin{de}
				\textbf{可数集合(ensemble dénombrable),不可数集合(ensemble non \\dénombrable)}
				\par
				如果$A$和$N$的某个子集等势,则称$A$为可数集合,否则,称$A$为不可数集合.
			\end{de}
			
			\begin{theo}\label{theo161}
				\hfill\par
				(1)可数集合的子集是可数集合.
				\par
				(2)有限个可数集的集族的积是可数集合.
				\par
				(3)各项均为可数集的序列的并集是可数集合.
			\end{theo}
			证明:根据定理\ref{theo158}、定理\ref{theo159}可证.
			
			\begin{theo}\label{theo162}
				\hfill\par
				可数无穷集合和$N$等势.
			\end{theo}
			证明:根据定理\ref{theo154}可证.
			
			\begin{theo}\label{theo163}
				\hfill\par
				$E$为无穷集合,可数无穷集族$(X_i)_{i\in I}$是$E$的划分,则$E$和$I$等势.
			\end{theo}
			证明:$Card(E)=Card(N)Card(I)$,根据定理\ref{theo160}可证.
			
			\begin{theo}\label{theo164}
				\hfill\par
				$f$为$E$到$F$的满射,$F$为无穷集合,如果对任意$y\in F$,$f^{-1}\langle y\rangle$均为可数集合,则$F$和$E$等势.
			\end{theo}
			证明:根据补充定理\ref{cor115},$(f^{-1}\langle y\rangle)_{)y\in F}$是$E$的划分,则$Card(E)\leq Card(F)Card(N)$,根据补充定理\ref{cor335},$Card(E)\leq Card(F)$,又因为$f$为$E$到$F$的满射,因此$Card(E)\geq \\Card(F)$,得证.
			
			\begin{theo}\label{theo165}
				\hfill\par
				无穷集合E的有限子集集合F,和E等势.
			\end{theo}
			证明:令$F_n$为$E$的元素数目为$n$的子集的集合,对任意$X\in F_n$,均存在$[1, n]$到$X$的双射,则$F_n$的基数小于等于区间$[1, n]$到$E$的映射数目,即$(Card(E))^n$,等于$Card(E)$.因此\\$Card(F)\leq Card(E)Card(N)$,等于$Card(E)$;
			\par
			另一方面,$x\mapsto \{x\}$是$E$到$F$的映射,故$Card(E) \leq Card(F)$,得证.
			
			\begin{theo}\label{theo166}
				\hfill\par
				无穷集合$E$的元素的有限序列集合$S$,和$E$等势.
			\end{theo}
			证明:令$F$为$N$的有限子集集合.$S$即所有$E^I$的并集,其中$I\subset N$.
			由于$I\subset N$的子集,故$Card(E^I)=Card(E)$,根据定理\ref{theo165},$Card(F)=Card(N)$,因此$Card(S)\leq Card(E)$ .
			\par
			同时,$E$的每个元素可以单独组成有限序列,故$Card(E)\leq Card(S)$,得证.
			
			\begin{de}
				\textbf{连续统(puissance du continu)}
				\par
				如果$E$和$\{X|X\subset N\}$等势,则称$E$为连续统.
			\end{de}
			
			\begin{theo}\label{theo167}
				\hfill\par
				连续统是不可数集合.
			\end{theo}
			证明:根据定理\ref{theo113}可证.
			
			\begin{de}
				\textbf{稳定序列(suite stationnaire)}
				\par
				$E$的元素序列$(x_n)_{n\in N}$,如果存在自然数$m$,对任意$n\geq m$,$x_m=x_n$,则称其为稳定序列.
			\end{de}
			
			\begin{theo}\label{theo168}
				\hfill\par
				$E$为偏序集,则下列两个公式等价:
				\par
				第一,$E$的一切非空子集都有极大元;
				\par
				第二,$E$的一切单增元素序列$(x_n)_{n\in N}$都是稳定序列.
			\end{theo}
			证明:
			如果第一个公式为真,设$(x_n)$各项组成的集合为$X$,则$X$有极大元,设为$x_m$,则当$n\geq m$时,$x_m=x_n$,故其为稳定序列.
			\par
			如果第二个公式为真,假设$E$的非空子集$A$没有极大元,对任意$x\in A$,令$T_x=\{y|y>x\text{与}y\in A\}$,故$T_x\neq \varnothing$.根据定理\ref{theo41},存在$A$到$A$的映射$f$,使$f(x)>x$.设$a\in A$,则递归定义$x_0=a$,$x_{n+1}=f(x_n)$,该序列是单增序列且不是稳定序列.
			
			\begin{theo}\label{theo169}
				\hfill\par
				$E$为全序集,当且仅当$E$的一切单减元素序列$(x_n)_{n\in N}$都是稳定序列时,$E$为良序集.
			\end{theo}
			证明:类似定理\ref{theo168}可证.
			
			\begin{theo}\label{theo170}
				\hfill\par
				偏序有限集的一切序列$(x_n)_{n\in N}$都是稳定序列.
			\end{theo}
			证明:根据定理\ref{theo123},偏序有限集有极大元,根据定理\ref{theo168}可证.
			
			\begin{cor}\label{cor337}
				\hfill\par
				(1)$E$为集合,如果$Card(E)>1$,则存在$E$的排列,该排列没有不动点.
				\par
				(2)$E$为无穷集合,则$E$的排列的集合,和$\mathcal{P}(E)$等势.
				\par
				(3)$E$、$F$均为无穷集合,$Card(E)=Card(F)$,则$E$到$F$的满射的集合,和$\mathcal{P}(E)$等势.
				\par
				(4)$F$为无穷集合,$Card(E)<Card(F)$,则$Card(\{X|X\subset F\text{与}Card(X)=E\})=Card(F^E)$;
				\par
				(5)$F$为无穷集合,$Card(E)<Card(F)$,则$Card(\{f|f\text{为}E\text{到}F\text{的单射}\})=\\Card(F^E)$;
				\par
				(6)$E$为无穷集合,$m$为自然数,则$Card(\{X|X\subset F\text{与}Card(X)=m\})=Card(E)$;
				\par
				(7)$E$为无穷集合,则$Card(\{X|X\subset F\text{与}Card(X)\text{为自然数}\})=Card(E)$;
				\par
				(8)$F$为无穷集合,$Card(E)<Card(F)$,则$Card(\{X|X\subset F\text{与}Card(X)\leq E\})=Card(F^E)$.
			\end{cor}
			证明:
			\par
			(1)对于有限集合,对其基数运用数学归纳法可证.
			\par
			对于无穷集合,令$F=E\times \{0, 1\}$,则$Card(F)=Card(E)$,则存在$E$到$F$的双射$f$.
			\par
			令$g$为:
			\par
			如果$pr_2x=0$,则$g(x)=(pr_1x, 1)$;
			\par
			如果$pr_2x=1$,则$g(x)=(pr_1x, 0)$.
			\par
			则$g$为$F$到$F$的双射,故排列$f^{-1}\circ g\circ f$没有不动点,得证.
			\par
			(2)令$E$的排列的集合为$F$,对任意$f\in F$,均有$f\subset E\times E$,所以$Card(F)\leq \\Card(\mathcal{P}(E))$.
			\par
			同时,对$E$的任意子集$A$,如果$Card(E-A)>1$,或者$E=A$,根据补充定理\ref{cor337}(1),存在以$A$为不动点的$E$的排列,故$Card(F)+Card(E)\geq Card(\mathcal{P}(E))$,因此$Card(F)\geq Card(\mathcal{P}(E))$,得证.
			\par
			(3)该集合的元素,为$E\times F$的子集,故其势均小于等于$\mathcal{P}(F)$.
			\par
			令$a\in F$,对任意$A\subset E$且$Card(A)=Card(E)$,设$A$到$F-\{a\}$的双射为$f$,当$x\in A时$,$g(x)=f(x)$,当$x\in E-A$时,$g(x)=a$,则$g$是$E$到$F$的满射,根据补充定理\ref{cor337}(2),以上映射的集合,势为$\mathcal{P}(F)$.
			\par
			(4)	对任意$X$,如果$X\subset F\text{与}Card(X)=E$,则存在$E$到$X$的双射$f$,则$f$在$F$上的延拓,是$F^E$的元素,故$Card(\{X|X\subset F\text{与}Card(X)=E\})\leq Card(F^E)$.反过来,任意$E$到$F$的映射,对应$x\mapsto (x, f(x))$,其函数图是$E\times E\times F$的子集且势为$Card(E)$,根据定理\ref{theo160},$Card(E\times E\times F)=Card(F)$,故$Card(\{X|X\subset F\text{与}Card(X)=E\})\geq Card(F^E)$,得证.
			\par
			(5)	根据定义,$Card(\{f|f\text{为}E\text{到}F\text{的单射}\})\leq Card(F^E)$.反过来,对任意$E$到$F$的映射,对应$x\mapsto (x, f(x))$,是$E$到$E\times F$的单射.得证.
			\par
			(6)	令$Card(\{X|X\subset F\text{与}Card(X)=m\})=A$,根据定理\ref{theo165},$A\leq E$,同时,令$B$为$E$的子集,且$Card(B)=m-1$,则$Card(E-B)=Card(E)$.且对任意$x\in E-B$,$Card(B\cup\{x\})=m$,故$A\geq E$,得证.
			\par
			(7)	根据补充定理\ref{cor337}(6),$Card(\{X|X\subset F\text{与}Card(X)\text{为自然数}\})=Card(E)\\Card(N)$,得证.
			\par
			(8)	根据补充定理\ref{cor337}(4),$Card(\{X|X\subset F\text{与}Card(X)\leq E\})\leq Card(F^E)Card(E)$,得证.
			
			\begin{cor}\label{cor338}
				\hfill\par
				(1)在有限集合上的任何两个良序均同构.
				\par
				(2)在无穷集合$E$上的良序的序数的集合,没有最大元.
				\par
				(3)在无穷集合$E$上的良序的序数的集合的最小上界,不是$E$上的良序.
				\par
				(4)$a$为在无穷集合$E$上的良序,则$Ord(N)\leq Ord(a)$.
			\end{cor}
			证明:
			\par
			(1)对元素数目运用数学归纳法可证.
			\par
			(2)设$a$为该集合最大元,根据补充定理\ref{cor301}(3),$Card(a+1)=Card(a)+1$.
			\par
			根据补充定理\ref{cor301}(1),$Card(a)=Card(E)$,又因为$E$为无穷集合,根据定理\ref{theo160},$Card(a\\+1)=Card(E)$,故存在$E$上的良序,其序数为$a+1$,而$a+1>a$,矛盾.
			\par
			(3)根据补充定理\ref{cor338}(2)可证.
			\par
			(4)如果$Ord(a)<Ord(n)$,则$E$同构于$N$的一个片段,故$E$为有限集合,矛盾.
			
			\begin{de}
				\textbf{有限序数(ordinal fini),无穷序数(ordinal infini),可数序数(ordinal dénombrable),极限序数(ordinal limite)}
				\par
				如果序数是在有限集合上的良序,称其为有限序数.
				\par
				如果序数是在无穷集合上的良序,称其为无穷序数.
				\par
				如果序数是可数集合上的良序,称其为可数序数.
				\par
				如果序数不是零也没有前导,称其为极限序数.
			\end{de}
			
			\begin{cor}\label{cor339}
				\hfill\par
				(1)有限序数小于无穷序数.
				\par
				(2)$a$为有限序数,$b$为没有前导的序数,则对任意序数$c<b$,$c+a<b$.
				\par
				(3)有限个有限序数的和、有限个有限序数的积,均为有限序数.
				\par
				(4)$a$、$b$为有限序数,则$a^b$为有限序数,并且,令$A=Card(a)$,$B= Card(b)$,则$A^B\\=Card(a^b)$.
				\par
				(5)$a$为有限序数,且$a>0$,则$a$有前导.
			\end{cor}
			证明:
			\par
			(1)根据补充定理\ref{cor333}(2)可证.
			\par
			(2)对$a$用数学归纳法可证.
			\par
			(3)根据定理\ref{theo125}、补充定理\ref{cor301}(3)可证.
			\par
			(4)根据定理\ref{theo103},对$b$用数学归纳法可证.
			\par
			(5)根据定理\ref{theo135}、补充定理\ref{cor301}(3)可证.
			
			\begin{cor}\label{cor340}
				\hfill\par
				$a$为基数,则“$x\text{为序数}\text{与}Card(x)<a$”为x上的集合化公式.
			\end{cor}
			证明:令$p$为在$a$上的良序,如果$p\leq x$,则$Card(p)\leq Card(x)$,故$a\leq Card(x)$.因此,$x\text{为序数}\text{与}Card(x)<a\Rightarrow x<p$,根据补充定理\ref{cor252}(1)可证.
			
			\begin{cor}\label{cor341}
				\hfill\par
				令序数$a>0$,$O'(a)$为良序集$\{x|x\text{为序数}\text{与}x\leq a\}$,并按下列方式定义定义域为$O'(a)$的函数$f_a$:
				\par
				$f_a(0)=Ord(N)$,
				\par
				对$x>0$且$x\leq a$,令$f_a(x)$为$\{y|y\text{为序数}\text{与}(\exists z)(z\text{为序数}\text{与}z<x\text{与}Card(y)\leq \\Card(f_a(z)))\}$的最小上界.
				\par
				则:
				\par
				(1)令$x\leq a$、$y\leq a$,如果$x<y$,则$Card(f_a(x))<Card(f_a(y))$;
				\par
				(2)如果$x\leq a$、$a\leq b$,则$f_a(x)=f_b(x)$.
			\end{cor}
			证明:
			\par
			(1)根据补充定理\ref{cor338}(3)可证.
			\par
			(2)如果$x\leq a$、$a\leq b$,$Card(f_a(x))\neq Card(f_b(x))$,令$\{y|y\leq a\text{与}Card(f_a(y))\neq Card(f_b(y))\}$的最小元是$t$,则$t\neq 0$,且对任意$z<t$,$f_a(z)=f_b(z)$,因此$f_a(t)=f_b(t)$,矛盾,因此$Card(f_a(x))=Card(f_b(x))$.
			
			\begin{de}
				\textbf{初始序数(ordinal initial),阿列夫(aleph)}
				\par
				令序数$a\geq 0$,$O'(a)$为良序集$\{x|x\text{为序数}\text{与}x\leq a\}$,并按下列方式定义定义域为$O'(a)$的函数$f_a$:
				\par
				$f_a(0)=Ord(N)$,
				\par
				对$x>0$且$x\leq a$,令$f_a(x)$为$\{y|y\text{为序数}\text{与}(\exists z)(z\text{为序数}\text{与}z<x\text{与}Card(y)\leq \\Card(f_a(z)))\}$的最小上界.
				\par
				则称$f_a(a)$为指标$a$的初始序数,记作$\omega_a$,称$Card(\omega_a)$为指标$a$的阿列夫,记作$\aleph_a$.其中,在没有歧义的情况下,$\omega_0$也可以简记为$\omega$.
			\end{de}
			
			\begin{cor}\label{cor342}
				\hfill\par
				(1)$\aleph_0=Card(N)$.
				\par
				(2)如果$a<b$,则$\omega_a<\omega_b$.
				\par
				(3)初始序数都是无穷序数.
			\end{cor}
			证明:
			\par
			(1)根据定义可证.
			\par
			(2)根据补充定理\ref{cor341}(1)、补充定理\ref{cor301}(5),$f_b(a)<f_b(b)$,根据补充定理\ref{cor341}(2),$f_a(a)<f_b(b)$,得证.
			\par
			(3)根据补充定理\ref{cor342}(2)、补充定理\ref{cor339}(1)可证.
			
			\begin{cor}\label{cor343}
				\hfill\par
				令$a$为无穷基数,$W(a)$为$\{y|y\text{为序数}\text{与}Card(y)<a\}$的最小上界,则:
				\par
				(1)$W(a)$为$\{y|y\text{为序数}\text{与}Card(y)=a\}$的最小元;
				\par
				(2)是$\{y|y\text{为序数}\text{与}Card(y)=a\}$的最小元是初始序数,并且,令其为$\omega_x$,则$a=\aleph_x$.
			\end{cor}
			证明:
			\par
			(1)如果$a=Card(N)$,根据补充定理\ref{cor338}(4)、补充定理\ref{cor339}(1),$W(a)=Ord{N}$.
			\par
			如果$a>Card(N)$,根据补充定理\ref{cor338}(3)可证.
			\par
			(2)如果$a=Card(N)$,根据补充定理\ref{cor343}(1),$W(a)=Ord{N}$,故为初始序数.
			\par
			如果$a>Card(N)$:
			\par
			令$H=\{z|z\text{为序数}\text{与}\omega_z<u\}$,则$H\neq \varnothing$.
			\par
			如果$sup\ H\in H$,则令$x=sup\ H+1$,如果$sup\ H\notin H$,则令$x=sup\ H$.故对任意$w<x$,$Card(\omega_w)<a$,因此$\omega_x\leq u$;如果$\omega_x<W(a)$,则$x\in H$,矛盾,因此$\omega_x=W(a)$.
			\par
			进而,根据定义可证$a=\aleph_x$.
					
			\begin{cor}\label{cor344}
				\hfill\par
				令$a$为序数:
				\par
				(1)$O'(a)=\{x|x\text{为序数}\text{与}x\leq a\}$,$W'(a)=\{x|x\text{为无穷基数}\text{与}x\leq \aleph_a\}$,则映射$x\mapsto \aleph_x(x\in O'(a))$为$O'(a)$到$W'(a)$的同构.
				\par
				(2)$a\leq \omega_a$.
				\par
				(3)$Card([0, \aleph_a[)\leq \aleph_a$;
				\par
				(4)$Card(a)\leq \aleph_a$.
			\end{cor}
			证明:
			\par
			(1)根据补充定理\ref{cor341}(1)、补充定理\ref{cor343}(2)可证.
			\par
			(2)根据补充定理\ref{cor344}(1)可证.
			\par
			(3)$Card([0, \aleph_a[)=\aleph_0+Card([0, a[)$,因此$Card([0, \aleph_a[)=\aleph_0+Card(a)$,故\\$Card([0, \aleph_a[)\leq \aleph_0+\aleph_a$,得证.
			\par
			(4)	根据补充定理\ref{cor344}(2)可证.
			
			\begin{cor}\label{cor345}
				\hfill\par
				$a$为序数,则$(\forall x)(x\text{为基数}\Rightarrow x\leq \aleph_a\text{或}x\geq \aleph_{a+1})$.
			\end{cor}
			证明:根据补充定理\ref{cor344}(1)可证.
			
			\begin{cor}\label{cor346}
				\hfill\par
				令$a$为无穷序数、$b$为序数,$a$没有前导,则对任意定义域为$\{y|y\text{为序数}\text{与}y<b\}$、值域的元素都是序数的严格单增映射$f$,如果$a=\mathop{sup}\limits_{x\in \{y|y\text{为序数}\text{与}y<b\}}f(x)$,则:
				\par
				(1)$\sum\limits_{x\in \{y|y\text{为序数}\text{与}y<b\}}\aleph_{f(x)}=\aleph_a$.
				\par
				(2)$\sum\limits_{x\in \{y|y\text{为序数}\text{与}y<b\}}\omega_{f(x)}=\omega_a$.
			\end{cor}
			证明:
			\par
			(1)根据定理\ref{theo159},$\sum\limits_{x\in \{y|y\text{为序数}\text{与}y<b\}}\aleph_f(x)\leq \aleph_a$.
			\par
			如果$\{y|y\text{为序数}\text{与}y<b\}$有最大元$v$,则$a=f(v)$,根据定理\ref{theo159}可证;如果$\{y|y\text{为序数}\text{与}\\y<b\}$没有最大元,则对任意序数$c<a$,均存在$x\in \{y|y\text{为序数}\text{与}y<b\}$,使$f(x)>c$,故$\sum\limits_{x\in \{y|y\text{为序数}\text{与}y<b\}}\aleph_f(x)>\aleph_c$,根据补充定理\ref{cor344}(1),$\sum\limits_{x\in \{y|y\text{为序数}\text{与}y<b\}}\aleph_f(x)=\aleph_a$.
			\par
			(2)根据补充定理\ref{cor346}(1)、补充定理\ref{cor301}(3)可证.
			
			\begin{cor}\label{cor347}
				\hfill\par
				$\omega_a$为标准序数函数符号.
			\end{cor}
			证明:
			\par
			对任意序数族$(x_i)_{i\in I}$且$i=\varnothing$,令$a=\mathop{sup}\limits_{i\in I}x_i$,$b=\mathop{sup}\limits_{i\in I}\omega_{x_i}$.
			\par
			根据补充定理\ref{cor342}(2),$\omega_a\geq b$.
			\par
			根据补充定理\ref{cor301}(6),对任意$i\in I$均有$\aleph_{x_i}\leq Card(b)$.
			\par
			根据补充定理\ref{cor343}(2),存在序数$x$使$Card(b)=\aleph_x$.
			\par
			根据补充定理\ref{cor341}(1),对任意$i \in I$,$x_i\leq x$.
			\par
			因此,$x\geq a$.
			\par
			根据补充定理\ref{cor342}(2),$\omega_a\leq \omega_x$.
			\par
			根据补充定理\ref{cor343}(1),$\omega_x\leq b$.
			\par
			综上,$\omega_a=b$,得证.
			
			\begin{cor}\label{cor348}
				\hfill\par
				(1)如果有限序数$a>0$,则$a$有前导.
				\par
				(2)如果有限序数$a>1$,则$a$为可约的序数.
			\end{cor}
			证明:
			\par
			(1)令$a$为在$E$上的良序,则$Card(E)$为非空有限集;令其最大元为$y$,$b=\\Ord(Card(E)-\{y\})$,则$a=b+1$,得证.
			\par
			(2)根据补充定理\ref{cor348}(1)可证.
			
			\begin{cor}\label{cor349}
				\hfill\par
				(1)$a$为序数,则$\omega_a$没有前导.
				\par
				(2)如果序数$x$没有前导,且$x>0$,则$x\geq \omega_0$.
				\par
				(3)$\omega_0$是不可约的序数.
				\par
				(4)$a$为序数,当且仅当$a<\omega_0$时,$a$为有限序数.
				\par
				(5)$\{a|a\text{为有限序数}\}$的最小上界为$\omega_0$.
				\par
				(6)如果序数$x$没有前导,则存在序数$y$,使$x=\omega_0y$.
				\par
				(7)$a$为有限序数,则$a\omega_0=\omega_0$.
			\end{cor}
			证明:
			\par
			(1)	根据定义可证.
			\par
			(2)	根据补充定理\ref{cor348}(1)可证.
			\par
			(3)	如果$\omega_0=x+y$,其中$x$、$y$为序数,则$Card(x)$、$Card(y)$均为有限基数,但根据补充定理\ref{cor301}(3), $Card(N)=Card(x)+Card(y)$,矛盾.
			\par
			(4)	如果$a$为有限序数,根据定义可证$a<\omega_0$.反过来,如果$a<\omega_0$,则$a$同构于$N$的区间上的良序,故$a$为有限序数.
			\par
			(5)	设最小上界为$x$,则$x\leq \omega_0$.如果$x$为有限序数,则$x+1$也是有限序数,矛盾.
			\par
			(6)	根据补充定理\ref{cor262},存在$y$、$z$使$x=\omega_0y+z$,且$z<\omega_0$.由于$x$没有前导,故$z=0$,得证.
			\par
			(7)	根据定义可证.
			
			\begin{cor}\label{cor350}
				\hfill\par
				令序数$a>0$,则:
				\par
				(1)$a\omega_0$是不可约的序数;
				\par
				(2)$a\omega_0>a$;
				\par
				(3)如果序数$x$不可约,且$x>a$,则$x\geq a\omega_0$.
			\end{cor}
			证明:
			\par
			(1)根据补充定理\ref{cor265}、补充定理\ref{cor349}(3)可证.
			\par
			(2)根据补充定理\ref{cor257}(3)可证.
			\par
			(3)根据补充定理\ref{cor262}可证.
			
			\begin{cor}\label{cor351}
				\hfill\par
				令序数$a>0$,则$(a+1)\omega_0=a\omega_0$.
			\end{cor}
			证明:根据补充定理\ref{cor246}(1),$(a+1)\omega_0\geq a\omega_0$.
			\par
			另一方面,根据补充定理\ref{cor350}(1),$a\omega_0$不可约,同时,根据补充定理\ref{cor350}(2),$a\omega_0>a$,因此$a\omega_0\geq a+1$,由于$a\omega_0$不可约,故$a\omega_0>a+1$,根据补充定理\ref{cor350}(3),$a\omega_0\geq (a+1)\omega_0$,得证.
			
			\begin{cor}\label{cor352}
				\hfill\par
				(1)$a$为序数,则当且仅当存在序数$b$,使$a={\omega_0}^b$时,$a$为不可约的序数.
				\par
				(2)$a$、$b$为序数,$a<b$,则${\omega_0}^a+{\omega_0}^b={\omega_0}^b$.
			\end{cor}
			证明:
			\par
			(1)根据补充定理\ref{cor350}(1)、补充定理\ref{cor276}可证.
			\par
			(2)根据补充定理\ref{cor352}(1)、补充定理\ref{cor263}可证.
			
			\begin{cor}\label{cor353}
				\hfill\par
				(1)对任意序数$a$,以及序数$c>1$,存在唯一的一对序数有限序列$(l_i)_{i\in [1, k]}$、$(m_i)_{i\in [1, k]}$,使$a=\sum\limits_{i\in [1, k]}c^{l_i}m_i$,其中,对任意$i\in [1, k]$,$m_i>0\text{与}m_i<c$,对任意$i\in [1, k-1]$,$li>l_{i+1}$.
				\par
				(2)对任意序数$a$,存在唯一的单减有限序列$(b_i)i\in [1, k]$,使$a=\sum\limits_{i\in [1, k]}{\omega_0}^{b_i}$.
			\end{cor}
			证明:
			\par
			(1)命题对$a=0$、$a=1$显然成立.假设命题对$[0, a[$成立,根据补充定理\ref{cor276},存在唯一的序数$e$、$f$、$g$,使$a=c^ef+g$,则$g<a$,根据证明规则\ref{C59}可证.
			\par
			(2)根据补充定理\ref{cor353}(1)可证.
			
			\begin{de}
				\textbf{序数的展开(développement d'un ordinal),序数的展开的最大指数(plus grand indice d'une développement d'un ordinal)}
				\par
				对任意序数$a>0$,如果单减有限序列$(b_i)_{i\in [1, k]}$,使$a=\sum\limits_{i\in [1, k]}{\omega_0}^{b_i}$,则称$(b_i)_{i\in [1, k]}$为$a$的展开,$b_1$为$a$的展开的最大指数,记作$\psi(a)$.
			\end{de}
			
			\begin{cor}\label{cor354}
				\hfill\par
				$a$、$b$为序数:
				\par
				(1)$a<{\omega_0}^{\psi(a)+1}$.
				\par
				(2)如果$\psi(a)<\psi(b)$,则$a<b$.
				\par
				(3)如果$a<b$,则$\psi(a)\leq \psi(b)$.
				\par
				(4)$a$、$b$为序数,$a<{\omega_0}^b$,则$a+{\omega_0}^b={\omega_0}^b$.
			\end{cor}
			证明:
			\par
			(1)根据补充定理\ref{cor257}(3)可证.
			\par
			(2)设$a$的展开有$n$项,则$a\leq {\omega_0}^{\psi(a)}n$,又因为$a<\omega_0\psi(a)+1$,因此$a<\omega_0\psi(b)$,故$a<b$.
			\par
			(3)	根据补充定理\ref{cor354}(1)可证.
			\par
			(4)	根据补充定理\ref{cor352}(2)、补充定理\ref{cor353}(2)可证.
			
			\begin{cor}\label{cor355}
				\hfill\par
				$w(x)$为定义在$x\geq a_0$上的序数函数符号,对任意序数$x\geq a_0$、$y>x$,均有$w(x)<w(y)$.则对任意序数$x\geq a_0$和序数$y$,$w(x+y)\geq w(x)+y$.进而,存在序数$a$,对任意序数$x\geq a$,均有$w(x)\geq x$.
			\end{cor}
			证明:假设存在$y$使$w(x+y)<w(x)+y$,设其中最小的为$y_0$,则$w(x+y_0)<w(x)+y_0$,令$b$满足$w(x+y_0)=w(x)+b$,则$b<y_0$且$w(x+b)<w(x)+b$,矛盾,故$w(x+y)\geq w(x)+y$.
			\par
			令$a=a_0\omega_0$,根据补充定理\ref{cor350}(1),$a$不可约,根据补充定理\ref{cor263},$a=a_0+a$,故$w(a)\geq a$,进而,当$x\geq a$时,均有$w(x)\geq x$.
			
			\begin{de}
				\textbf{临界序数(ordinal critique)}
				\par
				令$f(x, y)$为定义在$x\geq a_0$、$y\geq b_0$上的序数函数符号,$c$为无穷序数且$c>a_0$、$c>b_0$,如果对任意序数$x\geq a_0$、$x<c$,均有$f(x, c)=c$,则称$c$为$f(x, y)$的临界序数.
			\end{de}
			
			\begin{cor}\label{cor356}
				\hfill\par
				\par
				$w(x)$为定义在$x\geq a_0$上的序数函数符号,对任意序数$x\geq a_0$,$w(x)\geq x$,并且,对任意序数$x$、$y$,$x<y\text{与}x\geq a_0\Rightarrow w(x)<w(y)$.
				\par
				令$g(x, y)$为定义在$x\geq a_0$、$y\geq a_0$上的序数函数符号,并满足:
				\par
				第一,$(x\text{为序数}\text{与}y\text{为序数}\text{与}x\geq a_0\text{与}y\geq a_0)\Rightarrow g(x, y)>x$;
				\par
				第二,$a_0\leq x\text{与}x\leq x'\text{与}a_0\leq y\text{与}y\leq y'\Rightarrow g(x, y)\leq g(x', y')$.
				\par
				$f(x, y)$为定义在$x\geq a_0$、$y\geq 1$上的序数函数符号,其按下列方式定义:
				\par
				第一,对任意序数$x\geq a_0$,$f(x, 1)=w(x)$;
				\par
				第二,对任意序数$x\geq a_0$, $y>1$,$f(x, y)=\mathop{sup}\limits_{z\in ]0, y[}g(f(x, z), x)$.
				\par
				则:
				\par
				(1)对任意序数$b$,最多存在有限个序数$y$,使$f(x, y)=b$至少有一个解.
				\par
				(2)$f(x, y)$的临界序数没有前导.
				\par
				(3)如果存在集合$A$,对任意$x\in A$,$(x\text{为序数})\text{与}f(x, c)=c$,并且,$c$为$A$的最小上界,则$c$为$f(x, y)$的临界序数.
				\par
				(4)令$h(x)=f(x, x)$($x\geq a_0$),序列$(a_n)_{n\in N=\{0\}}$满足$a_1=a_0+2$、$a_{n+1}=h(a_n)$,则序列$(a_n)_{n\in N=\{0\}}$的最小上界,为$f(x, y)$的临界序数.
				\par
				(5)如果集合的元素都是$f(x, y)$的临界序数,则该集合最小上界是$f(x, y)$临界序数.
				\par
				(6)$f(x, y)$的临界序数是不可约的.
			\end{cor}
			证明:
			\par
			(1)设$y$组成的集合为$A$,$f(y)$为$\{x|f(x, y)=b\}$的最小元.如果$y<y'$,根据补充定理\ref{cor270}(6),$f(y)>f(y')$,令$f(y)$的值域的最小元为$x_0$,则相应的$y_0$为$A$的最大元.由于$A$有最小元也有最大元,故$A$为有限集合.
			\par
			(2)根据补充定理\ref{cor270}(8),$f(y, y+1)\geq y+y$,得证.
			\par
			(3)对任意$z<c$,存在$x\in A$使$x>z$,由于$f(x, c)=c$,故$f(z, c)\leq c$,根据补充定理\ref{cor270}(3)可证.
			\par
			(4)令$(a_n)_{n\in N=\{0\}}$的最小上界为$z$,对任意$i\in N-\{0\}$,$f(a_i, z)\geq z$,同时,$f(a_i, a_{i+1})\\\leq f(a_{i+1}, a_{i+1})$,故$f(a_i, a_{i+1})\leq z$,因此$f(a_i, z)=z$,根据补充定理\ref{cor356}(3)可证.
			\par
			(5)根据补充定理\ref{cor356}(3)可证.
			\par
			(6)如果$r$为$f(x, y)$的临界序数,且$r=\sum\limits_{i\in [1, k]}{\omega_0}^{b_i}$($k\geq 2$),则$f({\omega_0}^{b_1}, r)\geq {\omega_0}^{b_1}+r$,故$f({\omega_0}^{b_1}, r)>r$,矛盾.
			
			\begin{cor}\label{cor357}
				\hfill\par
				$a$、$b$为序数,$a\geq 2$,$b$没有前导,则$a^b$不可约.
			\end{cor}
			证明:假设$a^b=x+y$,根据补充定理\ref{cor353}(1),$x$、$y$对$a$均有唯一的展开.设最大的指数分别是$p$、$q$,相应的系数分别是$m$、$n$.则$p<b$,$q<b$.如果$p\geq q$,则$x+y\leq a^p(m+n)$,故$x+y\leq a^{p+1}2$,进而$x+y\leq a^{p+2}$,由于$b$没有前导,故$p+2<b$,根据补充定理\ref{cor272},矛盾.如果$p<q$,则$x+y\leq a^{q+2}$,同样矛盾.
			
			\begin{cor}\label{cor358}
				\hfill\par
				(1)$a$为有限序数,$a\geq 2$,则$a^{\omega_0}=\omega_0$.
				\par
				(2)$a$为无穷序数,则$a^{\omega_0}={\omega_0}^{\psi(a)\omega_0}$.
			\end{cor}
			证明:
			\par
			(1)令$b$为有限序数,根据补充定理\ref{cor311}、补充定理\ref{cor339}(4),$a^b>b$,因此$a^{\omega_0}\geq \omega_0$.同时,由于$a^b$为有限序数,故$a^{\omega_0}\leq \omega_0$.得证.
			\par
			(2)根据补充定理\ref{cor273},$a^{\omega_0}\geq {\omega_0}^{\psi(a)\omega_0}$,$a^{\omega_0}\leq {\omega_0}^{(\psi(a)+1)\omega_0}$,根据补充定理\ref{cor351}可证.
			
			\begin{cor}\label{cor359}
				\hfill\par
				$a$为有限序数,$a\geq 2$,$b=\omega_0c$,则$a^b={\omega_0}^c$.
			\end{cor}
			证明:根据补充定理\ref{cor358}(1)、补充定理\ref{cor274}可证.
			
			\begin{cor}\label{cor360}
				\hfill\par
				$a>0$,则$\{x|x\text{为不可约的序数}\text{与}x\leq a\}$的最大元为${\omega_0}^{\psi(a)}$.
			\end{cor}
			证明:根据补充定理\ref{cor352}(2)、补充定理\ref{cor354}(3)可证.
			
			\begin{cor}\label{cor361}
				\hfill\par
				$a$为无穷序数,$p$为$\{x|x\text{为不可约的序数}\text{与}x\leq a\}$的最大元,序数$b$没有前导,则$a^b=p^b$.
			\end{cor}
			证明:根据补充定理\ref{cor358}(2)、补充定理\ref{cor274}可证.
			
			\begin{cor}\label{cor362}
				\hfill\par
				当且仅当存在序数$b$使$c$等于${\omega_0}^{{\omega_0}^b}$时,$c$为$xy$的临界序数.
			\end{cor}
			证明:如果$c$是$xy$的临界序数.由于${\omega_0}^{\psi(c)c}>c$,故$c={\omega_0}^{\psi(c)}$.根据补充定理\ref{cor263},$\psi(c)$\\不可约,根据补充定理\ref{cor352}(1),存在序数$b$使$c$等于${\omega_0}^{{\omega_0}^b}$.
			\par
			反过来,如果存在序数$b$使$c$等于${\omega_0}^{{\omega_0}^b}$,根据补充定理\ref{cor352}(1),$\psi(c)$不可约;由于$a<c$时,$a<{\omega_0}^{\psi(a)+1}$,因此$ac\leq {\omega_0}^{\psi(a)+1+\psi(c)}$,又因为$\psi(a)+1<\psi(c)$,根据补充定理\ref{cor354}(4),$ac\leq c$,同时,由于$ac\geq c$,因此$ac=c$.
			
			\begin{cor}\label{cor363}
				\hfill\par
				令$c$为序数,则当且仅当存在序数$b$使$c$等于${\omega_0}^{{\omega_0}^b}$时,对任意序数$a>1$、$a\leq c$,均存在$x$使$c=a^x$.
			\end{cor}
			证明:
			如果对任意序数$a>1$、$a\leq c$,均存在$x$使$c=a^x$,则令$a={\omega_0}^{\psi(c)}$,故$c={\omega_0}^{\psi(c)x}$,因此$\psi(c)x =\psi(c)$,故$c={\omega_0}^{\psi(c)}$.
			\par
			如果$\psi(c)$可约:若$\psi(c)=a+b$,且$a>b$,则${\omega_0}^a<c$,${\omega_0}^{a2}>c$,矛盾;若$\psi(c)=a+b$,且$a<b$,则${\omega_0}^b<c$,${\omega_0}^{b2}>c$,矛盾;若$\psi(c)=2$,则$\omega_0+1<{\omega_0}^2$,$(\omega_0+1)^2>{\omega_0}^2$,矛盾;若$\psi(c)=a+a$,且$a>1$,则${\omega_0}^{a+1}<c$,${\omega_0}^{(a+1)2}>c$,矛盾.
			\par
			因此,$\psi(c)$不可约.根据补充定理\ref{cor352}(1),存在序数$b$使$\psi(c)= {\omega_0}^b$.
			\par
			反过来,如果$c={\omega_0}^{\psi(c)}$,$\psi(c)= {\omega_0}^b$,根据补充定理\ref{cor358}(1)、补充定理\ref{cor358}(2)、补充定理\ref{cor266},对任意序数$a>1$、$a\leq c$,均存在$x$使$c=ax$.
			
			\begin{cor}\label{cor364}
				\hfill\par
				$b$为序数,$c={\omega_0}^{{\omega_0}^b}$,则存在唯一的$x$使$c=ax$,并且$x$不可约.
			\end{cor}
			证明:根据补充定理\ref{cor272},$x$具有唯一性.如果$a$为有限序数,根据补充定理\ref{cor358}(1),$x={\omega_0}^{1+b}$,故$x$不可约;如果$a$为无穷序数,根据补充定理\ref{cor266},$x$不可约.
			
			\begin{cor}\label{cor365}
				\hfill\par
				$x^y$的最小的临界序数是$\omega_0$.
			\end{cor}
			证明:根据补充定理\ref{cor349}(5)、补充定理\ref{cor339}(4)可证.
			
			\begin{de}
				\textbf{艾普塞朗数(nombre epsilon),艾普塞朗序数(ordinal epsilon)}
				\par
				如果$a$为序数,${\omega_0}^a=a$,则称$a$为艾普塞朗数,或称$a$为艾普塞朗序数.
			\end{de}
			
			\begin{cor}\label{cor366}
				\hfill\par
				艾普塞朗数为无穷序数.
			\end{cor}
			证明:由于$a=0$不满足要求,故$a>0$,因此$a\geq \omega_0$,故$a$为无穷序数.
			
			\begin{cor}\label{cor367}
				\textbf{艾普塞朗数的构建}
				\par
				$(x_i)_{i\in N}$为序数序列,其中对任意$i\in N$,$x_{i+1}={\omega_0}^{x_i}$,则:
				\par
				 (1)$\mathop{sup}\limits_{i\in N}x_i$是艾普塞朗数.
				 \par
				 (2)如果$x_0=\omega_0$,则$\mathop{sup}\limits_{i\in N}x_i$是最小的艾普塞朗数.
			\end{cor}
			证明:
			\par
			(1)令$a=\mathop{sup}\limits_{i\in N}x_i$.
			\par
			如果$x_0=x_1$,则对任意$i\in N$,$x_0=x_1$,故$a=x_0$,命题得证.
			\par
			如果$x_0\neq x_1$,则$(x_i)_{i\in N}$为严格单增序列.设$\psi(a)=b$,则对任意$i\in N$,如果$b<a$,则存在$x_i\geq b+1$,故$x_{i+1}\geq a$,因此$a=x_{i+1}$,故$a<x_{i+2}$,矛盾.因此$a=b$,即$a\geq {\omega_0}^a$.同时,根据补充定理\ref{cor275},$a\leq {\omega_0}^a$,得证.
			\par
			(2)如果无穷序数$b<\mathop{sup}\limits_{i\in N}x_i$,则存在$i\in N$,使$x_i\leq b$,$x_{i+1}>b$,故$b$不是艾普塞朗数,得证.
			
			\begin{cor}\label{cor368}
				\hfill\par
				艾普塞朗数是$xy$的临界序数.
			\end{cor}
			证明:根据补充定理\ref{cor362}可证.
			
			\begin{cor}\label{cor369}
				\hfill\par
				当且仅当序数$a$是艾普塞朗数或$\omega_0$时,$a$是$x^y$($x\geq 2$、$y\geq 1$)的临界序数.
			\end{cor}
			证明:
			\par
			根据定义可证$\omega_0$是$x^y$($x\geq 2$、$y\geq 1$)的临界序数.令$a$为艾普塞朗数,$x<a$且$x\geq 2$,则$x^a\leq {\omega_0}^{(\psi(x)+1)a}$,由于$a$不可约,故$\psi(x)+1<a$,因此$x^a\leq {\omega_0}^a$,进而可得$x^a=a$.
			\par
			反过来,根据定义,如果$a$是$x^y$($x\geq 2$、$y\geq 1$)的临界序数,则$a$是艾普塞朗数或$\omega_0$.
			
			\begin{de}
				\textbf{共尾性(caractère final),正则序数(ordinal régulier),奇异序数\\(ordinal singulier)}
				\par
				$E$为全序集,其共尾良序子集的偏序类的集合的最小元,称为$E$的共尾性;令序数$a$为在$E$上的良序集,则$E$的共尾性也称为$a$的共尾性.令$a$为序数,如果$a$的共尾性为$a$,则称$a$为正则序数,否则,称$a$为奇异序数.
			\end{de}
			
			\begin{cor}\label{cor370}
				\hfill\par
				$a$、$b$为序数,则$b+a$、$ba$的共尾性都等于$a$的共尾性.
			\end{cor}
			证明:令$pr_1b=E$,$pr_1a=F$,$E$和$F$的和集为$G$,令$G$的偏序类最小的共尾良序子集为$G'$,$F$的偏序类最小的共尾良序子集为$F'$,由于$F\times \{1\}$是G的共尾子集,故$Ord(G')\leq Ord(F')$;同时,$G'\cap(F\times \{1\})$是$G$的共尾子集,故$Ord(G')= Ord(G'\cap(F\times \{1\}))$.由于$pr_1(G'\cap(F\times \{1\}))$是$F$的共尾子集,故$Ord(F')\leq Ord(G')$,$b+a$的情形得证.类似可证$ba$的情形.
			
			\begin{cor}\label{cor371}
				\hfill\par
				(1)有前导的序数的共尾性为$1$.
				\par
				(2)$0$、$1$是正则序数.
				\par
				(3)$a$有限序数且$a>0$,则$a$的共尾性是$1$.
				\par
				(4)$\omega_0$是正则序数.
				\par
				(5)序数$a<b$,则$a$的共尾性不大于$b$的共尾性.
				\par
				(6)$a>0$,则$a$的共尾性大于$0$.
				\par
				(7)$a$为无穷序数,如果$a$没有前导,则$a$的共尾性是无穷序数.
				\par
				(8)$b$为序数,$I=\{y|y\text{为序数}\text{与}y<b\}$,$(x_i)_{i\in I}$为序数族,且$x\mapsto x_i$为单增函数,则$\mathop{sup}\limits_{i\in I}(x_i)$的共尾性小于等于$b$,
				\par
				(9)$b$为序数,$I=\{y|y\text{为序数}\text{与}y<b\}$,$(x_i)_{i\in I}$为序数族,则$\sum\limits_{i\in I}x_i$的共尾性小于等于$b$,
			\end{cor}
			证明:
			\par
			(1)	根据补充定理\ref{cor370}可证;
			\par
			(2)	根据定义可证;
			\par
			(3)	根据补充定理\ref{cor371}(1)、补充定理\ref{cor339}(5)可证.
			\par
			(4)	令$pr_1\omega_0=E$,如果$E$有有限共尾子集$F$,设$F$的最大元为$a$,则$a$为$E$的最大元,故$\omega_0$有前导,矛盾.
			\par
			(5)	根据定义可证.
			\par
			(6)	根据定义可证.
			\par
			(7)	令$a$为没有前导的序数,且$a>0$,如果其共尾性为有限序数,设$pr_1a$的共尾良序子集种,偏序类最小的是$E$,则$E$有最大元,故$pr_1a$有最大元,故$a$有前导,矛盾.
			\par
			(8)令$a=\mathop{sup}\limits_{i\in I}(x_i)$,$O_a=\{x|x\text{为序数}\text{与}x<a\}$,$A=\bigcup\limits_{i\in I}x_i$,则根据补充定理\ref{cor253}(1),$Ord(O_a)=a$,$Ord(I)=b$.且$A$是$(O_a)$的共尾子集.
			\par
			同时,令$g$为映射$i\mapsto x_i$,$f$为映射$y\mapsto (g^{-1}\langle y \rangle \text{的最小元})$,则$f$为$A$到$I$的子集的同构,故$Ord(A)\geq b$,得证.
			\par
			(9)根据补充定理\ref{cor371}(8)可证.
			
			\begin{cor}\label{cor372}
				\hfill\par
				无穷正则序数都是初始序数.
			\end{cor}
			证明:令$a$为无穷正则序数,$E=Card(a)$,$E$上的良序集合的最小元为$b$.
			令按$b$排序的$E$的最小元为$p$,定义$E$到$\{0, 1\}$的映射$f$:
			\par
			$f(p)=1$;
			\par
			对任意$x\in E$,$x>_bp$,如果$(\forall i)(i<_bx\Rightarrow i<_ax)$,则令$f(x)=1$,否则$f(x)=0$.
			\par
			令$A=\{z|z\in E\text{与}f(z)=1\}$,按$b$在$A$上导出的偏序排序,故$Ord(A)\leq b$.如果存在$x\in E$,对任意$y\in A$,均有$x>_ay$,则$f(x)=0$,故$\{i|i\in E\text{与}i<_bx\text{与}i>_ax\}\neq \varnothing$,设其最小元为$m$,因此$(\forall i)(i<_bm\Rightarrow i<_am)$,故$m\in A$,且$m>_ax$,矛盾.因此,$A$为按$a$排序的$E$的共尾良序子集.故$a$的共尾性小于等于$b$,故$a=b$,根据补充定理\ref{cor343}(2)得证.
			
			\begin{cor}\label{cor373}
				\hfill\par
				$a$为序数,如果$a=0$或$a$有前导,则初始序数$\omega_a$是正则序数.
			\end{cor}
			证明:
			\par
			如果$a=0$,根据补充定理\ref{cor371}(4),命题成立.
			\par
			如果$a$有前导,令$a=b+1$,$F=pr_1\omega_a$,设初始序数$\omega_a$的偏序类最小的共尾良序子集为$E$,$r\notin E$,令良序集$E'=E\cup\{r\}$,其中$r$为最小元.如果$Ord(E)<\omega_a$,根据补充定理\ref{cor345},$Card(E)\leq \aleph_b$,故$Card(E')\leq \aleph_b$.当$i\in E$时,令$X_i=\{z|z\in F\text{与}z\geq i\text{与}(\forall j)(j\in E\text{与}j>i\Rightarrow j>z)\}$,$X_r=\{z|z\in F\text{与}(\forall j)(j\in E\Rightarrow j>z)\}$,故$\sum\limits_{i\in E'}Card(X_i)=\aleph_b+1$,因此存在$m\in E'$,使$Card(X_m)=\aleph_a$.设$\omega_a$在$X_i$上导出的偏序的偏序类是$x_i$,则$\sum\limits_{i\in E'}x_i=\omega_a$,同时$x_m\geq \omega_a$,$x_m>0$,矛盾.
			
			\begin{cor}\label{cor374}
				\hfill\par
				$a>0$,$a$没有前导,且$a<\omega_a$,则初始序数$\omega_a$是奇异序数.
			\end{cor}
			证明:根据补充定理\ref{cor346}(1),$\sum\limits_{x\in \{y|y\text{为序数}\text{与}y<a\}}\aleph_x=\aleph_a$.根据补充定理\ref{cor371}(9),$\omega_a$的共尾性小于等于$a$,得证.
			
			\begin{cor}\label{cor375}
				\hfill\par
				$\omega_{\omega_0}$是最小的奇异序数.
			\end{cor}
			证明:根据补充定理\ref{cor373}、补充定理\ref{cor374}可证.
			
			\begin{de}
				\textbf{不可达序数(ordinal inaccessible)}
				\par
				如果序数$a$没有前导,且$\omega_a$是正则序数,则称$\omega_a$为不可达序数.
			\end{de}
			
			\begin{cor}\label{cor376}
				\hfill\par
				$a>0$,且$\omega_a$为不可达序数,则$a=\omega_a$.
			\end{cor}
			证明:根据补充定理\ref{cor344}(2)、补充定理\ref{cor374}可证.
			
			\begin{cor}\label{cor377}
				\hfill\par
				$\omega_0$是不可达序数.
			\end{cor}
			证明:根据定义可证.
			
			\begin{cor}\label{cor378}
				\hfill\par
				(1)令$k$为最小的艾普塞朗数,则$\omega_k$的共尾性是$\omega_0$.
				\par
				(2)当$a>0$、$a\leq k$时,$\omega_a$不是不可达序数.
			\end{cor}
			证明:
			\par
			(1)令$x_0=\omega_0$,对于$i\in N$,令$x_{i+1}=\omega_{x_i}$,则$k=\mathop{sup}\limits_{i\in N}(x_i)$,根据补充定理\ref{cor367}(2),$k$为最小的艾普塞朗数.
			\par
			同时,$\omega_k$的共尾性不大于$\omega_0$.而根据补充定理\ref{cor349}(1),$k$没有前导,根据补充定理\ref{cor371}(7),$k$的共尾性为无穷序数,故$k$的共尾性是$\omega_0$.
			\par
			(2)根据补充定理\ref{cor376}可证.
			\par
			注:不能确定是否存在$\omega_0$以外的不可达序数.
			
			\begin{cor}\label{cor379}
				\hfill\par
				$E$为全序集,则:
				\par
				(1)$E$的共尾性是正则序数;
				\par
				(2)如果$E$非空且没有最大元,则E的共尾性是初始序数.
			\end{cor}
			证明:
			\par
			(1)根据定义可证.
			\par
			(2)根据补充定理\ref{cor371}(7)、补充定理\ref{cor372}可证.
			
			\begin{cor}\label{cor380}
				\hfill\par
				(1)$a$、$a'$为序数,$\omega_a$为的共尾性为$\omega_{a'}$,$I$为良序集且$Ord(I)<\omega_{a'}$,$(x_i)_{i\in I}$为序数族,如果对任意$i\in I$均有$x_i<\omega_a$,则$\sum\limits_{i\in I}x_i<\omega_a$.
				\par
				(2)	$a$为序数,$\omega_a$为正则序数,$I$为良序集且$Ord(I)<\omega_a$,$(x_i)_{i\in I}$为序数族,如果对任意$i\in I$均有$x_i<\omega_a$,则$\sum\limits_{i\in I}x_i<\omega_a$.
			\end{cor}
			证明:
			\par
			(1)设$[0, Ord(I)[$到$I$的同构为$f$,对任意序数$a$,令$I_a=f([0, a[)$,$s_a=\sum\limits_{i\in I}axi$.考虑集合$\bigcup\limits_{i\in [0, Ord(I)[}\{si\}$,由于$Ord(I)<\omega_{a'}$,故存在$y\in [0, \omega_a[$,对任意$i\in [0, Ord(I)[$,均有$s_i<y$,即$y>\sum\limits_{i\in I}x_i$,得证.
			\par
			(2)根据补充定理\ref{cor380}(1)可证.
			
			\begin{de}
				\textbf{正则基数(cardinal régulier),奇异基数(cardinal singulier)}
				\par
				$a$为序数,如果$\omega_a$为正则序数(或奇异序数),则称$\aleph_a$为正则基数(或奇异基数).
			\end{de}
			
			\begin{cor}\label{cor381}
				\hfill\par
				(1)$a$、$a'$为序数,$\omega_a$为的共尾性为$\omega_{a'}$,$Card(I)<\aleph_{a'}$,$(x_i)_{i\in I}$为基数族,如果对任意$i\in I$均有$x_i<\aleph_a$,则$\sum\limits_{i\in I}x_i<\aleph_a$.
				\par
				(2)$a$、$a'$为序数,$\omega_a$为的共尾性为$\omega_{a'}$,$Card(I)=\aleph_{a'}$,则存在基数族$(x_i)_{i\in I}$使$\sum\limits_{i\in I}x_i=\aleph_a$.
				\par
				(3)$a$为序数,当且仅当对任意基数族$(x_i)_{i\in I}$,若$Card(I)<\aleph_a$,且对任意$i\in I$均有$x_i<\aleph_a$,则$\sum\limits_{i\in I}(x_i)_{i\in I}<\aleph_a$时,$\aleph_a$为正则基数.
			\end{cor}
			证明:
			\par
			(1)根据补充定理\ref{cor380}(1)、补充定理\ref{cor201}可证.
			\par
			(2)类似补充定理\ref{cor373}的证明可证.
			\par
			(3)根据补充定理\ref{cor381}(1)、补充定理\ref{cor381}(2)可证.
			
			\begin{cor}\label{cor382}
				\hfill\par
				$a$为序数,基数$m\neq 0$,则:
				\par
				(1)${\aleph_{a+1}}^m={\aleph_a}^m\aleph_{a+1}$.
				\par
				(2)序数$c$满足$Card(c)\leq m$,则${\aleph_{a+c}}^m={\aleph_a}^m{\aleph_{a+c}}^{Card(c)}$.
				\par
				(3)$Card(a)\leq m$,则${\aleph_a}^m=2^m{\aleph_a}^{Card(a)}$.
			\end{cor}
			证明:
			\par
			(1)如果$m\geq \aleph_{a+1}$,根据补充定理\ref{cor336}(2),)${\aleph_{a+1}}^m=2^m$、${\aleph_a}^m=2^m$、$2^m>\aleph_{a+1}$,根据定理\ref{theo160}得证.
			\par
			如果$m<\aleph_{a+1}$,则$m\leq \aleph_a$,由于${\aleph_{a+1}}^m\geq {\aleph_a}^m$,${\aleph_{a+1}}^m\geq \aleph_{a+1}$,根据定理\ref{theo160},${\aleph_{a+1}}^m\geq {\aleph_a}^m\aleph_{a+1}$.
			\par
			同时,令$\aleph_{a+1}$按其最小良序排序,该最小良序同构于$\omega_{a+1}$.对任意$m$到$\aleph_{a+1}$的映射$f$,根据补充定理\ref{cor373},$f\langle m\rangle$不是$\aleph_{a+1}$共尾子集,即$\{x|(\forall y)(y\in f(m)\Rightarrow y<x)\}$非空,令其最小元为$n_f$.
			\par
			根据定理\ref{theo85},$Ord(S_{n_f})< \omega_{a+1}$,故$Card(S_{n_f})<\aleph_a$.因此$Card(S_{n_f})\leq \aleph_a$.
			\par
			令映射$g$为$f\mapsto n_f$,对任意$p\in \aleph_{a+1}$,$Card(g^{-1}\langle p \rangle=			)=Card(S_{n_f})^m$,因此\\$Card(g^{-1}\langle p \rangle)\\={\aleph_a}^m$.
			\par
			因此,$m$到$\aleph_{a+1}$的映射$f$的数目,小于等于${\aleph_a}^m\aleph_{a+1}$.
			\par
			综上,${\aleph_{a+1}}^m={\aleph_a}^m\aleph_{a+1}$.
			\par
			(2)使用超限归纳法:
			\par
			命题对$c=0$显然成立,对$1$,根据补充定理\ref{cor382}(1)可证.
			\par
			假设命题对小于$c=1$的序数($c>1$)成立:
			\par
			如果$c$有前导,根据归纳假设可证;
			\par
			如果$c$为极限序数,根据定理\ref{theo160},${\aleph_{a+c}}^m\geq {\aleph_a}^m{\aleph_{a+c}}^{Card(c)}$.同时,根据补充定理\ref{cor346}(1)、补充定理\ref{cor300}(1)、定理\ref{theo105},${\aleph_{a+c}}^m\leq \mathop{\mathsf{P}}\limits_{d\in [0, c[}{\aleph_{a+d}}^m$,根据归纳假设,${\aleph_{a+c}}^m\leq\\\mathop{\mathsf{P}}_{d\in [0, c[}({\aleph_a}^m{\aleph_{a+d}}^{Card(d)})$,根据补充定理\ref{cor253}(1),${\aleph_{a+c}}^m\leq {\aleph_a}^{m\cdot Card(c)}{\aleph_{a+d}}^{Card(c) Card(c)}$,根据定理\ref{theo160}可证.
			\par
			(3)根据补充定理\ref{cor382}(2)、补充定理\ref{cor336}(2)可证.
			
			\begin{cor}\label{cor383}
				\hfill\par
				$a$、$b$为序数,$a$没有前导,$x\mapsto s_x$为$[0, \omega_b[$到$[0, a[$的严格单增映射,且$\mathop{sup}\limits_{x\in [0, \omega_b[}s_x=a$,则${\aleph_a}^{\aleph_b}=\mathop{\mathsf{P}}\limits_{x\in [0, \omega_b[}\aleph_{s_x}$.
			\end{cor}
			证明:
			\par
			$x\mapsto s_x$为$[0, \omega_b[$到$[0, a[$的严格单增映射,故$[0, a[\neq \varnothing$,因此$a$为无穷序数.同时,对任意$x\in [0, \omega_b[$、$y\in [0, \omega_b[$、$x\neq y$、均有$s_x\neq s_y$.
			\par
			对任意$[0, \omega_b[$到$[0, \omega_a[$的映射$f$,用通过超限归纳法定义$[0, \omega_b[$到$[0, \omega_b[$的映射$h_f$:
			\par
			第一,如果$\{x|x\in [0, \omega_b[\text{与}f(0)\leq \omega_{s_x}\}=\varnothing$,令$Card(f(0))=\aleph_z$,则对任意$x\in [0, \omega_b[$,$s_x\leq z$,故$z\geq a$,$f(0)\geq \omega_a$,矛盾,因此,$\{x|x\in [0, \omega_b[\text{与}f(0)\leq \omega_{s_x}\}\neq\varnothing$,令$h_f(0)$为其最小元.
			\par
			第二,对$t\in [0, \omega_b[$,同理可证$\{x|x\in [0, \omega_b[\text{与}f(0)\leq \omega_{s_x}\}\neq\varnothing$,令$k$为其最小元.由于$Card(k)<aleph_b$,同时$Card([0, k[)=Card(k)$,故$Card[k, \omega_b[=\aleph_b$.因此,$\{x|x\in [[k, \omega_b[\text{与}(\forall)(i\in [0, t[\Rightarrow f(i)\neq x)\}\neq \varnothing$.令$h_f(t)$为其最小元.
			\par
			故$h_f$为单射.
			\par
			令$g$为映射$f\mapsto h_f(f\in \mathcal{F}([0, \omega_b[; [0, \omega_a[, h_f\in \mathcal{F}([0, \omega_b[); [0, \omega_b[))$,
			\par
			对任意$t\in [0, \omega_b[$,$Card([0, \omega_{s_{h_f(t)}}])=\aleph{s_{h_f(t)}}+1$,等于$\aleph{s_{h_f(t)}}$,则对任意$u\in\\\mathcal{F}([0, \omega_b[; [0, \omega_b[)$,$g^{-1}\langle z\rangle=\mathop{\mathsf{P}}\limits_{t\in [0, \omega_b[}\aleph_{s_{h_t}}$,小于等于$\mathop{\mathsf{P}}\limits_{x\in [0, \omega_b[}\aleph_{s_x}$,
			\par			
			故${\aleph_a}^{\aleph_b}\leq {\aleph_b}^{\aleph_b}\mathop{\mathsf{P}}\limits_{x\in [0, \omega_b[}\aleph_{s_x}$.
			\par
			又因为对任意$x\in [0, \omega_b[$,$2\leq \aleph_{s_x}$,同时,${\aleph_b}^{\aleph_b}=2^{\aleph_b}$,因此,$2^{\aleph_b}\leq\mathop{\mathsf{P}}\limits_{x\in [0, \omega_b[}\aleph_{s_x}$,故${\aleph_a}^{\aleph_b}\leq 2^{\aleph_b}\mathop{\mathsf{P}}\limits_{x\in [0, \omega_b[}\aleph_{s_x}$,因此${\aleph_a}^{\aleph_b}\leq \mathop{\mathsf{P}}\limits_{x\in [0, \omega_b[}\aleph_{s_x}$.
			\par
			同时,根据定理\ref{theo110},故${\aleph_a}^{\aleph_b}\geq\mathop{\mathsf{P}}\limits_{x\in [0, \omega_b[}\aleph_{s_x}$,得证.
			
			\begin{cor}\label{cor384}
				\hfill\par
				$a$、$a'$为序数,令$\omega_{a'}$为$\omega_a$的共尾性,则${\aleph_a}^{\aleph_{a'}}>\aleph_a$.
			\end{cor}
			证明:
			如果$a=0$或$a$有前导,根据补充定理\ref{cor373},$a=a'$,根据定理\ref{theo113}可证.
			\par
			如果$a$没有前导,设$\omega_a$的共尾子集为$A$,且$Ord(A)=\omega_{a'}$.
			\par
			令$g$为$A$到$[0, a[$的映射:
			\par
			对任意$x\in A$,如果$Card(x)$为有限基数,则$g(x)=0$;如果$Card(x)$为无穷基数,令其为$\aleph_y$,则$g(x)=\omega_y$.
			\par
			令$g\text{的值域}=G$,则$Card(G)\leq \aleph_{a'}$,同时,$\bigcup\limits_{i\in G}\{\omega_i\}$也是$\omega_a$的共尾子集,故则$Card(G)=\aleph_{a'}$.
			令$h$为$[0, \omega_{a'}[$到$G$的同构,则$\mathop{sup}\limits_{x\in [0, \omega_a'[}h(x)=a$.根据补充定理\ref{cor383},${\aleph_a}^{\aleph_{a'}}=\mathop{\mathsf{P}}\limits_{x\in [0, \omega_{a'}[}\aleph_{h(x)}$.根据补充定理\ref{cor300}(2)、补充定理\ref{cor346}(1)可证.
			
			\begin{cor}\label{cor385}
				\hfill\par
				$a$、$a'$为序数,令$\omega_{a'}$为$\omega_a$的共尾性,$c$为序数,如果存在基数$n$使$\aleph_a=n^{\aleph_c}$,则$c<a'$.
			\end{cor}
			证明:假设$c\geq a'$,则$\aleph_c\aleph_{a'}=\aleph_c$,故${\aleph_a}^{\aleph_{a'}}=\aleph_a$,矛盾.
			
			\begin{cor}\label{cor386}
				\hfill\par
				$a$、$a'$为序数,令$\omega_{a'}$为$\omega_a$的共尾性,序数$b<a'$,则${\aleph_a}^{\aleph_b}=\sum\limits_{c\in [0, a[}{\aleph_c}^{\aleph_b}$.
			\end{cor}
			证明:$\sum\limits_{c\in [0, a[}{\aleph_c}^{\aleph_b}\leq Card(a){\aleph_a}^{\aleph_b}$,由于$Card(a)\leq\aleph_a$,故$\sum\limits_{c\in [0, a[}{\aleph_c}^{\aleph_b}\leq {\aleph_a}^{\aleph_b}$.
			\par
			另一方面,对任意$[0, \omega_b[$到$[0, \omega_a[$的映射$f$,均存在$c\in [0, a[$,使$f\langle[0, \omega_b[\rangle\subset [0, \omega_c[$,因此$\sum\limits_{c\in [0, a[}{\aleph_c}^{\aleph_b}\geq {\aleph_a}^{\aleph_b}$,得证.
			
			\begin{cor}\label{cor387}
				\hfill\par
				$a$、$a'$为序数,令$\omega_{a'}$为$\omega_a$的共尾性,基数$b<\aleph_{a'}$且$b\neq 0$,则${\aleph_a}^b=\aleph_a\sum\limits_{m\in [0, \aleph_a[}m^b$.
			\end{cor}
			证明:对任意$b$到$\aleph_a$的映射$f$,$Card(f(b))<\aleph_a$,同时,当$m<\aleph_a$时,如果$m$为无穷基数,根据补充定理\ref{cor339}(4)、补充定理\ref{cor337}(6),$Card(\{x|x\subset \aleph_a\text{与}Card(x)=m\})\geq \aleph_a$,因此${\aleph_a}^b\geq \aleph_a\sum\limits_{m\in [0, \aleph_a[}m^b$.
			\par
			如果$b$为自然数,则${\aleph_a}^b=\aleph_a$,故${\aleph_a}^b\leq \aleph_a\sum\limits_{m\in [0, \aleph_a[}m^b$,命题成立.
			\par
			如果$b\geq \aleph_0$且$b<a$,根据补充定理\ref{cor386},${\aleph_a}^b\leq \aleph_a\sum\limits_{m\in [0, \aleph_a[}m^b$,命题成立.
			
			\begin{cor}\label{cor388}
				\hfill\par
				$a$、$a'$为序数,令$\omega_{a'}$为$\omega_a$的共尾性,基数$b\geq \aleph_{a'}$,则${\aleph_a}^b=(\mathop{sup}\limits_{m\in [0, \aleph_a[}m^b)^{\aleph_{a'}}$.
			\end{cor}
			证明:
			\par
			令$(x_i)_{i\in \aleph_{a'}}$为$\omega_a$的共尾子集,则$\aleph_a=\bigcup\limits_{i\in \aleph_{a'}}([0, x_i+1])$,令$y_i=Card([0, x_i+1])$,则$\aleph_a\leq \sum\limits_{i\in \aleph_{a'}}y_i$,根据补充定理\ref{cor300}(1),${\aleph_a}^b\leq \mathop{\mathsf{P}}\limits_{i\in \aleph_{a'}}y_i^b$,因此${\aleph_a}^b\leq (\mathop{sup}\limits_{m\in [0, \aleph_a[}m^b)^{\aleph_{a'}}$.
			\par
			另一方面,$(\mathop{sup}\limits_{m\in [0, \aleph_a[}m^b)^{\aleph_{a'}}\leq {\aleph_a}^{b\aleph_{a'}}$,故$(\mathop{sup}\limits_{m\in [0, \aleph_a[}m^b)^{\aleph_{a'}}\leq {\aleph_a}^b$,得证.
			
			\begin{cor}\label{cor389}
				\hfill\par
				$a$为无穷基数,基数$b\geq a$,则$a^b=a\sum\limits_{m\in [0, a[}m^b$.
			\end{cor}
			证明:根据补充定理\ref{cor336}(2),$a^b=2^b$,$a\sum\limits_{m\in [0, a[}m^b=a(Card([0, a[)2^b$.由于$Card([0, a[\\\leq a$,又因为$a\leq 2^b$,故$a^b=a\sum\limits_{m\in [0, a[}m^b$.
			
			\begin{cor}\label{cor390}
				\hfill\par
				$a$为正则基数,基数$b\neq 0$,则$a^b=a\sum\limits_{m\in [0, a[}m^b$.
			\end{cor}
			证明:根据补充定理\ref{cor387}、补充定理\ref{cor389}可证.
			
			\begin{cor}\label{cor391}
				\hfill\par
				$a$、$a'$为序数,令$\omega_{a'}$为$\omega_a$的共尾性,$y$为基数,并且,$(\forall z)(z\text{为基数}\text{与}z<\aleph_a\Rightarrow z^y\leq \aleph_a)$:
				\par
				(1)如果$y>0\text{与}y<\aleph_{a'}$,则${\aleph_a}^y=\aleph_a$;
				\par
				(2)如果$y\geq \aleph_{a'}$,则${\aleph_a}^y={\aleph_a}^{\aleph_{a'}}$.
			\end{cor}
			证明:
			\par
			(1)如果$y$为自然数,根据定理\ref{theo157}可证.
			\par
			如果$y$为无穷基数,根据补充定理\ref{cor386},${\aleph_a}^y\leq \aleph_aCard(a)\aleph_a$,故${\aleph_a}^y\leq \aleph_a$,得证.
			\par
			(2)根据补充定理\ref{cor388}可证.
			
			\begin{cor}\label{cor392}
				\textbf{广义连续统假设下的定理1}
				\par
				如果$(\forall k)(k\text{为序数}\Rightarrow 2^{\aleph_k}=\aleph_k+1)$,则:
				\par
				(1)$x$、$y$均为无穷基数,则$y<x\Rightarrow 2^y\leq x$;
				\par
				(2)$a$为序数,$x$、$y$均为基数,$x<\aleph_a$、$y<\aleph_a$,则$x^y\leq \aleph_a$;
				\par
				(3)$a$、$a'$为序数,令$\omega_{a'}$为$\omega_a$的共尾性,$y$为基数,$y>0\text{与}y<\aleph_{a'}$,则${\aleph_a}^y=\aleph_a$;
				\par
				(4)$a$、$a'$为序数,令$\omega_{a'}$为$\omega_a$的共尾性,$y$为基数,$y\leq \aleph_a\text{与}y\geq \aleph_{a'}$,则${\aleph_a}^y=\aleph_{a+1}$;
				\par
				(5)$a$为基数且$a>0$,如果对任意基数$b\neq 0$均有$a^b=a\sum\limits_{m\in [0, a[}m^b$,则$a$为正则基数.
			\end{cor}
			证明:
			\par
			(1)令$y=\aleph_a$,$x=\aleph_b$($a$、$b$为序数),则$a<b$,故$a+1\leq b$,得证.
			\par
			(2)根据补充定理\ref{cor392}(1),$2^x\leq \aleph_a$、$2^y\leq \aleph_a$,则$2^{xy}\leq \aleph_a$,又因为$x<2^x$,故$x^y\leq \aleph_a$.
			\par
			(3)根据补充定理\ref{cor392}(2)、补充定理\ref{cor391}(1)可证.
			\par
			(4)根据补充定理\ref{cor384},${\aleph_a}^y>\aleph_a$,故${\aleph_a}^y\geq \aleph_a+1$.同时,$\aleph_ay^\leq {\aleph_a}^{\aleph_a}$,故${\aleph_a}^y\leq 2^{\aleph_a}$,得证.
			\par
			(5)令$a=\omega_x$,$\omega_{x'}$为$\omega_x$的共尾性,如果$x'<x$,令$b=\omega_{x'}$,则$a^b=\aleph_{x+1}$.根据补充定理\ref{cor392}(2),$a\sum\limits_{m\in [0, a[}m^b\leq aaa$,故$a\sum\limits_{m\in [0, a[}m^b\leq a$,因此$a\sum\limits_{m\in [0, a[}m^b<a^b$,矛盾.
			
			\begin{cor}\label{cor393}
				\hfill\par
				$a$为基数且$a>2$,对任意基数$m\in ]0, a[$,均有$a^m=a$,则$a$为正则基数.
			\end{cor}
			证明:根据补充定理\ref{cor384}可证.
			
			\begin{cor}\label{cor394}
				\textbf{广义连续统假设的等价命题}
				\par
				$(\forall a)(\forall m)((a\text{为正则基数})\text{与}(m\text{为基数})\text{与}(m\in ]0, a[)\Rightarrow a^m=a)\Leftrightarrow (\forall k)(k\text{为序数}\Rightarrow 2^{\aleph_k}=\aleph_{k+1})$.
			\end{cor}
			证明:如果广义连续统假设成立,根据补充定理\ref{cor392}(3),左边公式为真.
			\par
			反过来,如果左边公式为真,由于$\aleph_{k+1}$为正则基数,故${\aleph_{k+1}}^{\aleph_k}=\aleph_{k+1}$,由于$\aleph_{k+1}\leq 2^{\aleph_k}$,故广义连续统假设成立.
			
			\begin{de}
				\textbf{支配基数(cardinal dominant)}
				\par
				$a$为无穷基数,如果对任意基数$m<a$、$n<a$均有$m^n<a$,则称$a$为支配基数.
			\end{de}
			
			\begin{cor}\label{cor395}
				\hfill\par
				(1)$\aleph_0$为支配基数.
				\par
				(2)$a$为序数,如果$\aleph_a$为支配基数,则$a$没有前序.
			\end{cor}
			证明:根据定义可证.
			
			\begin{cor}\label{cor396}
				\hfill\par
				基数$a>0$,对任意基数$m<a$,均有$2^m<a$,则$a$为支配基数.
			\end{cor}
			证明:如果$a$为有限基数,$1$、$2$显然不符合条件;$a>2$时,$2^{a-1}\geq a$,矛盾,故$a$为无穷基数.
			\par
			如果$a=\aleph_0$,$a$为支配基数.$a>\aleph_0$时,令$p=sup\{m, n, \aleph_0\}$.则无穷基数$p<a$,$p^p=2^p$,故$p^p<a$,因此$m^n<a$.得证.
			
			\begin{cor}\label{cor397}
				\hfill\par
				递归定义基数序列$(a_i)_{i\in N}$:
				\par
				$a_0=\aleph_0$,
				\par
				对任意$i\in N$,$a_{i+1}=2^{a_i}$.
				\par
				令$b=\sum\limits_{i\in N}a_i$,则$b$是大于$\aleph_0$的最小的支配基数.
			\end{cor}
			证明:由于$a_1>\aleph_0$,故$b>\aleph_0$.
			\par
			对任意基数$m<b$,存在自然数$n$使$m<a_n$,故$2^m<b$,根据补充定理\ref{cor396},$b$为支配基数.
			\par
			对于支配基数$x$,如果$x>\aleph_0$,且$x<b$,则存在自然数$n$使$x\leq a_n$,设其中最小的为$n_0$,那么$2^{a_{n_0-1}}\geq x$,矛盾.故$x\geq b$.
			
			\begin{cor}\label{cor398}
				\hfill\par
				递归定义基数序列$(a_i)_{i\in N}$:
				\par
				$a_0=\aleph_0$,
				\par
				对任意$i\in N$,$a_{i+1}=2^{a_i}$.
				\par
				令$b=\sum\limits_{i\in N}a_i$,则:
				\par
				(1)$b^{\aleph_0}=2^b$;
				\par
				(2)${\aleph_0}^b=b^{\aleph_0}$;
				\par
				(3)$b^{\aleph_0}=(2^b)^b$.
			\end{cor}
			证明:
			\par
			(1)根据补充定理\ref{cor336}(4),$b^b=2^b$,由于$b>\aleph_0$,故$b^{\aleph_0}\leq 2^b$;同时,$2^b=2^{\sum\limits_{i\in N}a^i}$,等于$\mathop{\mathsf{P}}\limits_{i\in N-\{0\}}a_i$,故$2^b\leq b^{\aleph_0}$,得证.
			\par
			(2)根据补充定理\ref{cor398}(1)可证.
			\par
			(3)由于$(2^b)^b=2^b$,得证.
			
			\begin{de}
				\textbf{不可达基数(cardinal inaccessible),强不可达基数(cardinal fortement inaccessible)}
				\par
				$a$为序数,如果$\omega_a$为不可达序数,则称$\aleph_a$为不可达基数.不可达基数同时是支配基数的,称为强不可达基数.
			\end{de}
			
			\begin{cor}\label{cor399}
				\textbf{广义连续统假设下的定理2}
				\par
				如果$(\forall k)(k\text{为序数}\Rightarrow 2^{\aleph_k}=\aleph_k+1)$,则不可达基数均为强不可达基数.
			\end{cor}
			证明:根据补充定理\ref{cor396}可证.
			
			\begin{cor}\label{cor400}
				\hfill\par
				基数$a\geq 3$,当且仅当对任意基数族$(a_i)_{i\in I}$均有$Card(I)<a\text{与}(\forall i)(i\in I\Rightarrow a_i<a)\Rightarrow \mathop{\mathsf{P}}\limits_{i\in I}a_i<a$时,$a$为强不可达基数.
			\end{cor}
			证明:
			\par
			充分性:根据定义,$a$为支配基数;令$a=\aleph_k$,如果$k=m+1$,则$2^{\aleph_m}\geq a$,矛盾,故$k$没有前导;同时,根据补充定理\ref{cor381}(3),$a$是正则基数.充分性得证.
			\par
			必要性:令$s=\sum\limits_{i\in I}(a_i)$,由于$a$为正则基数,根据补充定理\ref{cor381}(3),$s<a$;因此 $\mathop{\mathsf{P}}\limits_{i\in I}a_i\leq s^{Card(I)}$,由于$a$为支配基数,必要性得证.
			
			\begin{cor}\label{cor401}
				\hfill\par
				$a$为基数,则:
				\par
				(1)如果对任意基数$b>0$且$b<a$,均有$a^b=a$,则对任意基数$b>0$,均有$a^b=a2^b$.
				\par
				(2)$a$为支配基数,如果对任意基数$b>0$,均有$a^b=a2^b$,则对任意基数$b>0$且$b<a$,均有$a^b=a$.
			\end{cor}
			证明:
			\par
			(1)当$2^b>a$时,$a^b=2^b$,$a2^b=2^b$;当$a\geq 2^b$时,$b<a$,故$a^b=a$,$a2^b=a$,故$a^b=a2^b$.
			\par
			(2)当$b<a$时,$2b<a$,故$a^b=a$.
			
			\begin{cor}\label{cor402}
				\hfill\par
				$a$为无穷基数,当且仅当$a$为支配基数并且对任意基数$b>0$且$b<a$均有$a^b=a$时,$a$为强不可达基数.
			\end{cor}
			证明:充分性根据补充定理\ref{cor393}可证.必要性根据补充定理\ref{cor391}(1)可证.
			
			\begin{de}
				\textbf{发散映射(application divergente)}
				\par
				序数$a>0$,$[0, a[$到$[0, a[$的映射$f$,如果满足对任意序数$l_0<a$,均存在序数$m_0<a$,使$(x\text{为序数}\text{与}x\geq m_0\text{与}x<a)\Rightarrow f(x)\geq l_0$,则称$f$为发散映射.
			\end{de}
			
			\begin{cor}\label{cor403}
				\hfill\par
				$a$、$b$为序数,$g$为$[0, b[$到$[0, a[$的严格单增映射,并且,对任意序数$c$,$g(\mathop{sup}\limits_{d\in [0, c[}d)= \\\mathop{sup}\limits_{d\in [0, c[}g(d)$,$\mathop{sup}\limits_{d\in [0, b[}g(d)=a$.则当且仅当存在$[0, b$[到$[0, b[$的发散映射$h$满足$(\forall x)(x\in ]0, b[\Rightarrow h(x)<x)$时,存在$[0, a[$到$[0, a[$的发散映射$f$满足$(\forall x)(x\in ]0, a[\Rightarrow f(x)<x)$.
			\end{cor}
			证明:
			\par
			设$h$存在,对任意序数$x<a$,由于$\mathop{sup}\limits_{d\in [0, b[}g(d)=a$,故$sup\{z|z\text{为序数}\text{与}z<b\text{与}g(z)\leq x\}<b$,令$f(x)=g(h(sup\{z|z\text{为序数}\text{与}z<b\text{与}g(z)\leq x\}))$,当$x>0$时,由于$\{z| z\text{为序数}\text{与}z\\<b\text{与}g(z)>x\}\neq \varnothing$,故可令其最小元为$c$,则$\{z|z\text{为序数}\text{与}z<b\text{与}g(z)\leq x\}=[0, c[$,由于$g(\mathop{sup}\limits_{d\in [0, c[}d)=\mathop{sup}\limits_{d\in [0, c[}g(d)$,故$g(\mathop{sup}\limits_{d\in [0, c[}d)\leq x$,因此$g(h(sup\{z|z\text{为序数}\text{与}z<b\text{与}g(z)\leq x\}))<x$,即$(\forall x)(x\in ]0, a[\Rightarrow f(x)<x)$.
			\par
			同时,对任意$y_0<a$,$\{z|z\text{为序数}\text{与}z<b\text{与}g(z)\geq y_0\}\neq \varnothing$,令$l_0$为其最小元,则存在$m_0<b$使$(x\text{为基数}\text{与}x\geq m_0\text{与}x<b)\Rightarrow h(x)\geq l_0$,令$x_0=g(m_0)$,当$x\geq x_0$时,$sup\{z|z\text{为序数}\text{与}z<b\text{与}g(z)\leq x\}\geq m_0$,故$h(sup\{z|z\text{为序数}\text{与}z<b\text{与}g(z)\leq x\})\geq l_0$,因此$g(h(sup\{z|z\text{为序数}\text{与}z<b\text{与}g(z)\leq x\}))\geq y_0$,故$f$为$[0, a[$到$[0, a[$的发散映射.
			\par
			反过来,设$f$存在,对任意序数$x<b$,由于$\mathop{sup}\limits_{d\in [0, b[}g(d)=a$,故$\{z|z\text{为序数}\text{与}z<b\text{与}g(z)\\\leq f(g(x))\}\neq b$,令$h(x)=sup\{z|z\text{为序数}\text{与}z<b\text{与}g(z)\leq f(g(x))\}$,当$x>0$时,由于\\$\{z|z\text{为序数}\text{与}z<b\text{与}g(z)> f(g(x))\}\neq \varnothing$,故可令其最小元为$c$,则$\{z|z\text{为序数}\text{与}z<b\text{与}\\g(z)\leq f(g(x))\}=[0, c[$,如果$h(x)\geq x$,即$\mathop{sup}\limits_{d\in [0, c[}d\geq x$,因此$g(\mathop{sup}\limits_{d\in [0, c[}d)>f(g(x))$,但\\$g(\mathop{sup}\limits_{d\in [0, c[}d)=\mathop{sup}\limits_{d\in [0, c[}(g(d))$,故$g(\mathop{sup}\limits_{d\in [0, c[}d)\leq f(g(x))$,矛盾,即$(\forall x)(x\in ]0, a[\Rightarrow f(x)<x)$.
			\par
			同时,对任意$y_0<b$,令$l_0=g(y_0)$,则存在$m_0<a$使$(x\text{为基数}\text{与}x\geq m_0\text{与}x<a)\Rightarrow f(x)\geq l_0$,$\{z|z\text{为序数}\text{与}z<b\text{与}g(z)\geq m_0\}\neq \varnothing$,令$x_0$为其最小元,当$x\geq x_0$时,$g(x)\geq m_0$,故$f(g(x))\geq l_0$,因此$h(x)\geq y_0$.故$h$为$[0, b[$到$[0, b[$的发散映射.
			
			\begin{cor}\label{cor404}
				\hfill\par
				$a$为序数,则当且仅当$\omega_a$的共尾性为$\omega_0$时,存在$[0, \omega_a[$到$[0, \omega_a[$的发散映射$f$满足$(\forall x)(x\in [0, \omega_a[\Rightarrow f(x)<x)$.
			\end{cor}
			证明:如果$\omega_a$的共尾性为$\omega_0$,令$E$为$[0, \omega_a[$的共尾子集,令$g(x)=\{z|z\in E\text{与}z>x\}$的最小元,根据补充定理\ref{cor403}可证.
			\par
			反过来,设$f$为$[0, \omega_a[$到$[0, \omega_a[$的满足条件的发散映射,令$x_0=1$,$x_{n+1}=\{z|(\forall y)\\(y\text{为序数}\text{与}y<\omega_a\text{与}y\geq z\Rightarrow f(z)>x_n\}$的最小元.假设存在序数$u<\omega_a$,且对任意$i\in N$,均有$x_i<u$.设满足条件的最小序数为$u_0$,则对任意$i\in N$,$f(u_0)>x_i$,矛盾.故$\bigcup\limits_{i\in N}\{x_i\}$为$a$\\的共尾子集,得证.
			
			\begin{cor}\label{cor405}
				\hfill\par
				$a$、$a'$为为序数,且$a'>0$,$\omega_a$的共尾性为$\omega_{a'}$,$f$为$[0, \omega_a[$到$[0, \omega_a[$的映射,且满足$(\forall x)\\(x\in [0, \omega_a[\Rightarrow f(x)<x)$,则存在序数$l_0$,使$Card(\{x|x\in [0, \omega_a[\text{与}f(x)=l0\})\geq \aleph_{a'}$.
			\end{cor}
			证明:假设对任意$x\in [0, \omega_a[$,均存在$y\in [0, \omega_a[$使$f(y)>x$,则令$x_0=1$,$x_{n+1}=\{z|(\forall y)(y\text{为序数}\text{与}y<\omega_a\text{与}y\geq z\Rightarrow f(z)>x_n\}$的最小元,故$\bigcup\limits_{i\in N}\{x_i\}$为$a$的共尾子集,矛盾,故存在$x\in [0, \omega_a[$,使$(\forall x)(x\in [0, \omega_a[\Rightarrow f(x)\in [0, x])$,因此$Card(f\langle[0, \omega_a[\rangle)<\aleph_a$,令$x_i=Card(\{x|x\in [0, \omega_a[\Rightarrow f(x)=i\}$,则$\sum\limits_{f\langle[0, \omega_a[\rangle x_i}=\aleph_a$.根据补充定理\ref{cor381}(1)可证.
			
			\begin{cor}\label{cor406}
				\hfill\par
				$F$为无穷集合,其元素都是$E$的子集,对任意$A\in F$均有$Card(A)=Card(F)$,则:
				\par
				(1)存在$E$的子集$P$使$Card(P)=Card(F)$,并且$F$的所有元素都不是$P$的子集.
				\par
				(2)如果对$F$的任意子集$G$,$Card(G)<Card(F)$,均有$Card(E-\bigcup\limits_{A\in G}A)\geq Card(F)$,则存在$E$的子集$P$使$Card(P)=Card(F)$,并且对任意$A\in F$均有$Card(A\cap P)<Card(F)$.
			\end{cor}
			证明:
			\par
			(1)令$Card(F)=\aleph_a$,其中$a$为序数,$h$为$\{z|z\text{为序数}\text{与}z<\omega_a\}$到$F$的同构.令$f(0)=\tau_z(z\in h(0))$,$g(0)=\tau_z(z\in h(0)-\{f(0)\})$,对任意序数$x>0$、$x<\omega_a$,令$u=h(x)-h(x)\cap(\bigcup\limits_{b\text{为序数}\text{与}b<x}\{f(b), g(b)\})$,由于$2^Card(a)< \aleph_a$,故$u\neq \varnothing$,因此令$f(x)=\tau_z(z\in u)$,$g(0)=\tau_z(z\in u-\{f(x)\})$.$f\langle\omega_a\rangle$、$g\langle\omega_a\rangle$不相交,其中至少有一个符合条件,得证.
			\par
			(2)令$Card(F)=\aleph_a$,其种$a$为序数,$h$为$\{z|z\text{为序数}\text{与}z<F\}$到$F$的同构.令$f(0)=\tau_z(z\in E-h(0))$,对任意序数$x>0$、$x<\omega_a$,令$u=E-\bigcup\limits_{b\text{为序数}\text{与}b\leq x}h(b)$,则$u\neq \varnothing$,因此令$f(x)=\tau_z(z\in u)$,$f\langle\omega_a\rangle$即符合条件,得证.
			
			\begin{de}
				\textbf{不相交度(degré de disjonction)}
				\par
				$E$为无穷集合,集族$(X_i)_{i\in I}$为$E$的覆盖,且对任意$i\in I$、$j\in I$且$i\neq j$,均有$X_i\neq X_j$.如果存在最小基数$c$,使对任意$i\in I$、$j\in I$且$i\neq j$,均有$Card(X_i\cap X_j)<c$,则称$c$为$(X_i)_{i\in I}$的不相交度.
			\end{de}
			
			\begin{cor}\label{cor407}
				\hfill\par
				$E$为无穷集合,集族$(X_i)_{i\in I}$为$E$的覆盖,且对任意$i\in I$、$j\in I$且$i\neq j$,均有$X_i\neq X_j$,$c$为$(X_i)_{i\in I}$的不相交度,则$Card(I)\leq Card(E)^c$.
			\end{cor}
			证明:当$c=0$时,命题显然成立.
			\par
			$c>0$时,$E$的任意势为$c$的子集,最多只能是$\{A|(\exists i)(i\in I\text{与}Card(X_i)\geq c\text{与}A=X_i)\}$当中一个元素的子集,根据补充定理\ref{cor337}(4),$Card(\{A|(\exists i)(i\in I\text{与}Card(X_i)\geq c\text{与}A=X_i)\})\leq Card(E)^c$.
			\par
			同时,$\{A|(\exists i)(i\in I\text{与}Card(X_i)<c\text{与}A=X_i)\}\leq \sum\limits_{d\in [0, c[}E^d$,而$\sum\limits_{d\in [0, c[}E^d\leq cE^c$,故$\sum\limits_{d\in [0, c[}E^d\leq E^c$.综上,$Card(I)\leq Card(E)^c$.			

			\begin{de}
				\textbf{诺特集(ensemble noethérien)}
				\par
				$E$为偏序集,如果$E$的任何非空子集都有极大元,则称$E$为诺特集.
			\end{de}
			
			\begin{theo}\label{theo171}
				\hfill\par
				$E$为诺特集,$F\subset E$,并且,$((a\in E)\text{与}(x>a\Rightarrow x\in F))\Rightarrow(a\in F)$,则$F=E$.
			\end{theo}
			证明:如果$F\neq E$,则$E-F$有极大元$b$,因此$b\in F$,矛盾.
			
			\begin{cor}\label{cor408}
				\hfill\par
				(1)	诺特集的子集也是诺特集.
				\par
				(2)	$E$为偏序集,则有限个按导出的偏序排序的E的诺特子集的并集,按导出的偏序排序,也是诺特集.
			\end{cor}
			证明:
			\par
			(1)根据定义可证.
			\par
			(2)对子集数目用数学归纳法可证.
			
			\begin{cor}\label{cor409}
				\hfill\par
				$E$为偏序集,当且仅当对任意$a\in E$,区间$]a, \to [$均为诺特集时,$E$为诺特集.
			\end{cor}
			证明:根据定义可证.
						
			\begin{Ccor}\label{Ccor91}
				\textbf{按诺特集的偏序的相反关系排序的偏序集的超限归纳法}
				\par
				集合论中,$E$为偏序集,其按相反关系排序得到的偏序集为诺特集.$u$为字母,$T$为项.则存在集合$U$和$E$到$U$的满射$f$,使对任意$x\in E$,均有$f(x)=(f(x)|u)T$,其中$f(x)$是$f(x)$在区间$]\gets, x[$上的限制,并且,满足条件的$U$和$f$是唯一的.
			\end{Ccor}
			证明:设$U$和$f$、$U'$和$f'$均满足条件,$A=\{x|x\in E\text{与}f(x)\neq f'(x)\}$,如果$A\neq \varnothing$,则其有极小元$a$,则$f(a)=f'(a)$,矛盾.唯一性得证.
			\par
			设所有存在$U$和$f$的$E$的子集的并集为$A$,则$A$也存在$U$和$f$.如果$A\neq E$,则$E-A$有极小元$a$,故$A\cup\{a\}$存在$U$和$f$,矛盾,存在性得证.
			
			\begin{cor}\label{cor410}
				\hfill\par
				(1)$E$为诺特集,并且$E$的任何有限子集均有最小上界,则$E$有最大元.
				\par
				(2)$E$为诺特集,并且$E$的任何有限子集均有最小上界,则$E$为完备格.
			\end{cor}
			证明:
			\par
			(1)令$x$为$E$的极大元,如果存在$y\in E$且$x$、$y$不可比较,$\{x, y\}$没有上界,矛盾,故$x$为\\$E$的最大元.
			\par
			(2)对$E$的任意非空子集$F$,令$a_0$为$F$的任意元素,对任意$i\in N$,如果存在$x>a_i$并且$x$是$F$的某个有限子集的最小上界,则令$a_{i+1}$为任何一个$x$,否则令$a_{i+1}=a_i$.根据定理\\\ref{theo168},存在自然数$m$,对任意$n\geq m$,$a_m=a_n$,因此$a_m$为$F$的最小上界.得证.

			\begin{exer}\label{exer156}
				\hfill\par
				当且仅当对任意$E$到$E$的映射$f$,存在$E$的非空子集$S$使$S\neq E\text{与}f(S)\subset S$时,$E$为无穷集合.
			\end{exer}
			证明:对于元素数目为$n$有限集合,令$g$为$E$到$[1, n]$的双射,$f=((\bigcup\limits_{i\in [1, n-1]}\{(i, i+1)\})\cup\{(n, 1)\}, [1, n], [1, n])$.则考虑映射$g^{-1}\circ f\circ g$.对任意$S\subset E$,令$T=g(S)$,如果$g^{-1}\circ f\circ g(S)\subset S$,则$f(T)\subset T$,设$a$为$[1, n]-T$的最小元,如果$a=1$,则$n\notin T$,设$T$的最大元为$b$,则$b<n$,$f(b)=b+1$,而$b+1\notin T$,矛盾.如果$a\neq 1$,则$a-1\in T$,$f(a-1)=a$,而$a\notin T$,矛盾.故不存在$E$的非空子集$S$使$S\neq E\text{与}f(S)\subset S$.
			\par
			对于无穷集合$E$,对任意映射$f$,令$x\in E$,$y_0=x$,$y_n=f(y_{n-1})$,$S=\bigcup\limits_{i\in N}\{y_i\}$,则$f(S)\subset S$.假设$S=E$,则存在$y_n=x$,则$S=\bigcup\limits_{i\in [0, n-1]}\{y_i\}$,故$Card(S)\leq n$,矛盾.
			
			\begin{exer}\label{exer157}
				\hfill\par
				$a$、$b$、$c$、$d$均为基数,如果$a<c$,$b<d$,求证:$a+b<c+d$,$ab<cd$.
			\end{exer}
			证明:即补充定理\ref{cor334}.
			
			\begin{exer}\label{exer158}
				\hfill\par
				$E$为无穷集合,则$Card(\{X|X\subset E\text{与}Card(X)=Card(E)\}=Card(\mathcal{P}(E))$.
			\end{exer}
			证明:
			\par
			设$a\neq b$,则$Card(E\times \{a\}\cup E\times \{b\})=E$,设$f$为$(E\times \{a\})\cup (E\times \{b\})$到$E$的双射,令$g$为$X\mapsto f\langle (X\times \{a\})\bigcup\limits_(E\times \{b\})\rangle (X\in \mathcal{P}(E))$,
			\par
			由于$Card((X\times \{a\})\cup(E\times \{b\}))=Card(E)$,故$Card f\langle (X\times \{a\})\bigcup\limits_(E\times \{b\})\rangle)=Card(E)$,因此$g$为$\mathcal{P}(E)$到$\{X|X\subset E\text{与}Card(X)=Card(E)\}$的单射,故$Card(\{X|X\subset E\text{与}\\Card(X)=Card(E)\}\geq Card(\mathcal{P}(E))$,得证,
			
			\begin{exer}\label{exer159}
				\hfill\par
				$E$为无穷集合,$F=\{G|\Delta_G\text{为}E\text{的划分}\}$,求证:$Card(F)=Card(\mathcal{P}(E))$.
			\end{exer}
			证明:
			\par
			令$H=\{Y|(\exists I)(I\subset E\text{与}Y=\{E, E-I\})\}$,根据定理\ref{theo142},$2^H=\mathcal{P}(E)$,根据定理\ref{theo160},$Card(H)=Card(\mathcal{P}(E))$.
			\par
			由于$H\subset F$,故$Card(F)\geq Card(\mathcal{P}(E))$.
			\par
			反过来,令$f$为$X\mapsto \{\{x, y\}|x=y\text{与}x\in X\}$,则$f(X)\subset E\times E$,故$Card(F)\leq Card(\mathcal{P}(E\times E))$,即$Card(F)\leq Card(\mathcal{P}(E))$,得证.
			
			\begin{exer}\label{exer160}
				\hfill\par
				$E$为无穷集合,求证:$E$的排列的集合,和$\mathcal{P}(E)$等势.
			\end{exer}
			证明:即补充定理\ref{cor337}(2).
			
			\begin{exer}\label{exer161}
				\hfill\par
				$E$、$F$均为无穷集合,$Card(E)\leq Card(F)$,求证:$E$到$F$的满射的集合(如果$Card{E}=Card(F)$),$E$到$F$的映射的集合,$E$的子集到$F$的映射的集合,均和$\mathcal{P}(F)$等势.
			\end{exer}
			证明:
			根据补充定理\ref{cor337}(3)、补充定理\ref{cor336}(2)可证.
			注:原书习题\ref{exer161}第一部分有误.
			
			\begin{exer}\label{exer162}
				\hfill\par
				$E$、$F$均为无穷集合,$Card(E)<Card(F)$,求证:$Card(\{X|X\subset F\text{与}Card(X)=E\})=Card(F^E)$,$Card(\{f|f\text{为}E\text{到}F\text{的单射}\})=Card(F^E)$.
			\end{exer}
			证明:根据补充定理\ref{cor337}(4)、补充定理\ref{cor337}(5)可证.
			\par
			注:习题\ref{exer162}中,$E$为无穷集合的条件可以去掉.
			
			\begin{exer}\label{exer163}
				\hfill\par
				$E$为无穷集合,求证:在$E$上的良序集合,和$\mathcal{P}(E)$等势.
			\end{exer}
			证明:任何良序的图,都是$E\times E$的子集.
			\par
			另一方面,令$F$为在$E$上的良序,对任意$E$的排列$f$,$f(x)\leq (y)\text{与}x\in E\text{与}y\in E$也是良序关系,对于不同的$f$,其相应的良序也互不相同.根据习题\ref{exer160}可证.
			
			\begin{exer}\label{exer164}
				\hfill\par
				令$E$为非空良序集合,对任意$x\in E$,如果$x$不是$E$的最小元,则$]\gets, x[$有最大元,求证:$E$同构于$N$或者$E$同构于$[0, n]$($n$为自然数).
			\end{exer}
			证明:如果命题为假,根据定理\ref{theo84},存在$x\in E$,使$]\gets, x[$同构于$N$,矛盾.			
			
			\begin{exer}\label{exer165}
				\hfill\par
				(1)$a$为基数,求证:“$x\text{为序数}\text{与}Card(x)<a$”为x上的集合化公式.
				\par
				(2)令序数$a\geq 0$,$O'(a)$为良序集$\{x|x\text{为序数}\text{与}x\leq a\}$,并按下列方式定义定义域为$O'(a)$的函数$f_a$:
				\par
				$f_a(0)=Ord(N)$,
				\par
				对$x>0$且$x\leq a$,令$f_a(x)$为$\{y|y\text{为序数}\text{与}(\exists z)(z\text{为序数}\text{与}z<x\text{与}Card(y)\leq \\Card(f_a(z)))\}$的最小上界.
				\par
				求证:
				\par
				令$x\leq a$、$y\leq a$,如果$x<y$,则$Card(f_a(x))<Card(f_a(y))$;
				\par
				如果$x\leq a$、$a\leq b$,则$f_a(x)=f_b(x)$;
				\par
				特别是,$\aleph_0=Card(N)$.
				\par
				(3)求证:
				\par
				令$a$为无穷基数,$W(a)$为$\{y|y\text{为序数}\text{与}Card(y)<a\}$的最小上界,则$W(a)$是初始序数,并且,设其为$\omega_x$,则$a=\aleph_x$;进而,$\omega_x$为$\{y|y\text{为序数}\text{与}Card(y)=a\}$的最小元.
				\par
				令$a$为序数,$O'(a)=\{x|x\text{为序数}\text{与}x\leq a\}$,$W'(a)=\{x|x\text{为无穷基数}\text{与}x\leq \aleph_a\}$,则:
				\par
				映射$x\mapsto \aleph_x(x\in O'(a))$为$O'(a)$到$W'(a)$的同构;
				\par
				特别是,令$a$为序数,则$(\forall x)(x\text{为基数}\Rightarrow x\leq \aleph_a\text{或}x\geq \aleph_a+1)$.
				\par
				令$a$为无穷序数、$b$为序数,$a$没有前导,则:
				\par
				对任意定义域为$\{y|y\text{为序数}\text{与}y<b\}$、值域的元素都是序数的严格单增映射$f$,如果$a=\mathop{sup}\limits_{x\in \{y|y\text{为序数}\text{与}y<b\}}f(x)$,即有$\sum\limits_{x\in \{y|y\text{为序数}\text{与}y<b\}}\aleph_{f(x)}=\aleph_a$.
				\par
				(4)	求证:$\omega_a$为标准序数函数符号.
			\end{exer}
			证明:
			\par
			(1)	即补充定理\ref{cor340}.
			\par
			(2)	即补充定理\ref{cor341}、补充定理\ref{cor342}(1).
			\par
			(3)	第一部分根据补充定理\ref{cor343}(1)、补充定理\ref{cor343}(2)可证,后面得部分即补充定理\ref{cor344}(1)、补充定理\ref{cor345}、补充定理\ref{cor346}(1).
			\par
			(4)	即补充定理\ref{cor347}.
			\par
			注:习题\ref{exer165}(2)开头应改为“$a\geq 0$”.
			
			\begin{exer}\label{exer166}
				\hfill\par
				(1)求证:
				\par
				$a$为序数,则$\omega_a$没有前导.
				\par
				如果序数$x$没有前导,且$x>0$,则$x\geq \omega_0$.
				\par
				$\omega_0$是不可约的序数.
				\par
				令序数$a>0$,则:$a\omega_0$是不可约的序数;$a\omega_0>a$;如果序数$x$不可约,且$x>a$,则$x\geq a\omega_0$.
				\par
				令序数$a>0$,则$(a+1)\omega_0=a\omega_0$.
				\par
				(2)$a$为序数,求证:当且仅当存在序数$b$,使$a={\omega_0}^b$时,$a$为不可约的序数.
			\end{exer}
			证明:
			\par
			(1)即补充定理\ref{cor349}(1)、补充定理\ref{cor349}(2)、补充定理\ref{cor349}(3)、补充定理\ref{cor350}、补充定理\ref{cor351}.
			\par
			(2)即补充定理\ref{cor352}(1).
			
			\begin{exer}\label{exer167}
				\hfill\par
				(1)
				求证:
				\par
				对任意序数$a$,以及序数$c>1$,存在唯一的一对序数有限序列$(l_i)_{i\in [1, k]}$、$(m_i)_{i\in [1, k]}$,使$a=\sum\limits_{i\in [1, k]}c^{l_i}m_i$,其中,对任意$i\in [1, k]$,$m_i>0\text{与}m_i<c$,对任意$i\in [1, k-1]$,$li>l_{i+1}$.
				\par
				对任意序数$a$,存在唯一的单减有限序列$(b_i)i\in [1, k]$,使$a=\sum\limits_{i\in [1, k]}{\omega_0}^{b_i}$.
				\par			
				(2)令$n$为大于$0$的自然数,按下列方式定义$N-\{0\}$到$N$的映射$f$:
				\par
				$f(1)=1$;
				\par
				$f(n)=\mathop{sup}\limits_{k\in [1, n-1]}(k2^{k-1}+1)f(n-k)$.
				\par
				求证:
				\par
				对任意序数族$(a_i)_{i\in [1, n]}$,$Card(\{x|(\exists g)(g\text{为}[0, n]\text{的排列}\text{与}x=\sum\limits_{i\in [1, n]}a_{g(i)}\})\leq f(n)$,并且存在某个序数族$(a_i)_{i\in [1, n]}$使等号成立.
				\par
				同时,当$n\geq 20$时,$f(n)=81f(n-5)$.
				\par
				(3)令$g$为$[0, n]$的排列,求证:对不同的$g$,$\sum\limits_{i\in [1, n]}(\omega_0+g(i))$各不相同.
			\end{exer}
			证明:
			\par
			(1)即补充定理\ref{cor353}.
			\par
			(2)考虑$a_i$的展开的最大指数,设其中最大指数等于最小值的有$k$个序数.在排列中,另外$n-k$个序数的序数和多对有$f(n-k)$种可能.考虑这些序数后面的序数:如果有序数,最后一个序数有$k$种可能,前面的序数有$2^{k-1}$种可能,不等式得证.
			\par
			同时,设$f(n)=(k2^{k-1}+1)f(n-k)$,则令$a_i=\omega_0i+i$($i\in [1, k]$),$a_{k+1}$、$a_{k+2}$、$\cdots$、$a_n$为使等号成立的$n-k$个序数展开所有指数加$2$后得到的结果,则此时等号成立.
			\par
			通过数学归纳法可证当$n\geq 20$时,$f(n)=81f(n-5)$.
			\par
			(3)	用数学归纳法可证$\sum\limits_{i\in [1, n]}(\omega_0+g(i))= {\omega_0}^n+\sum\limits_{i\in [1, n]} {\omega_0}^{n-i}g(i)$,得证.
			
			\begin{exer}\label{exer168}
				\hfill\par
				(1)$w(x)$为定义在$x\geq a_0$上的序数函数符号,对任意序数$x\geq a_0$、$y>x$,均有$w(x)<w(y)$.求证:对任意序数$x\geq a_0$和序数$y$,$w(x+y)\geq w(x)+y$.进而,存在序数$a$,对任意序数$x\geq a$,均有$w(x)\geq x$.
				\par
				(2)$w(x)$为定义在$x\geq a_0$上的序数函数符号,对任意序数$x\geq a_0$,$w(x)\geq x$,并且,对任意序数$x$、$y$,$x<y\text{与}x\geq a_0\Rightarrow w(x)<w(y)$.
				\par
				令$g(x, y)$为定义在$x\geq a_0$、$y\geq a_0$上的序数函数符号,并满足:
				\par
				第一,$(x\text{为序数}\text{与}y\text{为序数}\text{与}x\geq a_0\text{与}y\geq a_0)\Rightarrow g(x, y)>x$;
				\par
				第二,$a_0\leq x\text{与}x\leq x'\text{与}a_0\leq y\text{与}y\leq y'\Rightarrow g(x, y)\leq g(x', y')$.
				\par
				$f(x, y)$为定义在$x\geq a_0$、$y\geq 1$上的序数函数符号,其按下列方式定义:
				\par
				第一,对任意序数$x\geq a_0$,$f(x, 1)=w(x)$;
				\par
				第二,对任意序数$x\geq a_0$, $y>1$,$f(x, y)=\mathop{sup}\limits_{z\in ]0, y[}g(f(x, z), x)$.
				\par
				求证:
				\par
				对任意序数$b$,最多存在有限个序数$y$,使$f(x, y)=b$至少有一个解.
				\par
				(3)$f(x, y)$的临界序数没有前导.
				\par
				同时,如果存在集合$A$,对任意$x\in A$,$(x\text{为序数})\text{与}f(x, c)=c$,并且,$c$为$A$的最小上界,则$c$为$f(x, y)$的临界序数.
				\par
				(4)令$h(x)=f(x, x)$($x\geq a_0$),序列$(a_n)_{n\in N=\{0\}}$满足$a_1=a_0+2$、$a_{n+1}=h(a_n)$,则序列$(a_n)_{n\in N=\{0\}}$的最小上界,为$f(x, y)$的临界序数.
				\par
				(5)如果集合的元素都是$f(x, y)$的临界序数,则该集合最小上界是$f(x, y)$临界序数.
				\par
				并且,$f(x, y)$的临界序数是不可约的.
			\end{exer}
			证明:
			\par
			(1)	即补充定理\ref{cor355}.
			\par
			(2)	即补充定理\ref{cor356}(1).
			\par
			(3)	即补充定理\ref{cor356}(2)、补充定理\ref{cor356}(3).
			\par
			(4)	即补充定理\ref{cor356}(4).
			\par
			(5)	即补充定理\ref{cor356}(5)、补充定理\ref{cor356}(6).
			
			\begin{exer}\label{exer169}
				\hfill\par
				(1)求证:
				$a$、$b$为序数,$a\geq 2$,$b$没有前导,则$a^b$不可约.
				\par
				$a$为有限序数,$a\geq 2$,$b=\omega_0c$,则$a^b={\omega_0}^c$.
				\par
				$a$为无穷序数,$p$为$\{x|x\text{为不可约的序数}\text{与}x\leq a\}$的最大元,序数$b$没有前导,则$a^b=p^b$.
				\par
				(2)求证:
				\par
				令序数函数符号$f(x, y)=x^y$,$c$为序数,则当且仅当对任意序数$a>1$、$a\leq c$,均存在$x$使$c=a^x$时,$c$为$f$的临界序数;并且,使$c=a^x$成立的$x$是唯一且不可约的.
				\par
				反过来,对任意$a>1$和不可约的序数$p$,$a^p$为$f$的临界序数.
				\par
				进而,当且仅当存在序数$b$使$c$等于${\omega_0}^{{\omega_0}^b}$时,$c$为$f$的临界序数.
				\par
				(3)	令序数函数符号$f(x, y)=xy$,则$f$的最小的临界序数是可数序数.
			\end{exer}
			证明:
			\par
			(1)	即补充定理\ref{cor357}、补充定理\ref{cor359}、补充定理\ref{cor361}.
			\par
			(2)	根据补充定理\ref{cor362}、补充定理\ref{cor363}、补充定理\ref{cor364}可证.
			\par
			(3)	根据补充定理\ref{cor365}可证.
			
			\begin{exer}\label{exer170}
				\hfill\par
				令序数$a>0$、$c>1$,序数有限序列$(l_i)_{i\in [1, k]}$、$(m_i)_{i\in [1, k]}$,使$a=\sum\limits_{i\in [1, k]}c^{l_i}m_i$,其中,对任意$i\in [1, k]$,$m_i>0\text{与}mi<c$,对任意$i\in [1, k-1]$,$li>l_{i+1}$.令$L(a)=\{x|(\exists i)(i\in [1, k]\text{与}x=l_i\}$,求证:
				\par
				(1)对任意$i\in [1, k]$,$l_i\leq a$,并且,如果存在$i\in [1, k]$使$l_i=a$,则$a$为$x^y$($x\geq 2$)的临界序数.
				\par
				(2)	令$L_1(a)=L(a)$,对于自然数$n>1$,$L_n(a)=\bigcup\limits_{b\in L_{n-1}(a)}L(b)$,求证:存在自然数$n_0$,对任意自然数$n\geq n_0$,均有$L_{n+1}(a)=L_n(a)$,并且,$L_n(a)$的元素均为$x^y$的临界序数.
			\end{exer}
			证明:
			\par
			(1)根据补充定理\ref{cor275},$c^{l_i}\geq cl_i$,故$c^{l_i}\geq l_i$,又因为$a\geq c^{l_i}$,因此$l_i\leq a$.如果存在$i\in [1, k]$使$l_i\leq a$,则$a=c^a$.
			\par
			如果$c$为有限序数,设$a=\omega_0d+e$($e$为有限序数),则$a={\omega_0}^dc^e$,设$d=1+k$,则$e=0$,$d={\omega_0}^k$,因此$d$为不可约的序数,故$d=k$或$d=1$.如果$d=1$,则$a=\omega_0$,故$a$为$x^y$($x\geq 2$)的临界序数.如果$d=k$,则$a={\omega_0}^a$,根据补充定理\ref{cor369},$a$为$x^y$($x\geq 2$)的临界序数.
			\par
			如果$c$为无穷序数,则$a\geq {\omega_0}^a$,故$a={\omega_0}^a$,根据补充定理\ref{cor369},$a$为$x^y$($x\geq 2$)的临界序数.
			\par
			(2)令$M_n(a)=\{b|b\in Ln(a)\text{与}b\notin L(b)\}$,如果对任意$n\in N$,$M_n(a)\neq \varnothing$,令$y_n$为\\$M_n(a)$的最大元.则$\{z|(\exists y)(y\in N\text{与}z=y_n)\}$有最小元$y_m$,但$y_m\notin L(ym_)$,故对任意$z\in M_{n+1}(a)$,均有$z<y_m$,即$y_{m+1}<y_m$,矛盾.
			\par
			注:原书习题\ref{exer170}关于“$a=0$”的内容有误.
			
			\begin{exer}\label{exer171}
				\hfill\par
				(1)求证:
				\par
				无穷正则序数都是初始序数.
				\par
				$a$为序数,如果$a=0$或$a$有前导,则初始序数$\omega_a$是正则序数.
				\par
				$a>0$,$a$没有前导,且$a<\omega_a$,则初始序数$\omega_a$是奇异序数.
				\par		
				$\omega_{\omega_0}$是最小的奇异序数.
				\par
				(2)求证:
				$a>0$,且$\omega_a$为不可达序数,则$a=\omega_a$.
				\par
				令$k$为最小的艾普塞朗数,则$\omega_k$的共尾性是$\omega_0$.
				\par
				进而,当$a>0$、$a\leq k$时,$\omega_a$不是不可达序数.
				\par
				(3)$E$为全序集,求证:$E$的共尾性是正则序数;如果$E$非空且没有最大元,则E的共尾性是初始序数;进而,令$\omega_a$的共尾性为$\omega_{a'}$,则$a'\leq a$,并且,当且仅当$a'=a$时,$\omega_a$为正则序数.
				\par
				(4)$a$为序数,$\omega_a$为正则序数,$I$为良序集且$Ord(I)<\omega_a$,$(x_i)_{i\in I}$为序数族,如果对任意$i\in I$均有$x_i<\omega_a$,求证:$\sum\limits_{i\in I}x_i<\omega_a$.
			\end{exer}
			证明:
			\par
			(1)即补充定理\ref{cor372}、补充定理\ref{cor373}、补充定理\ref{cor374}、补充定理\ref{cor375}.
			\par
			(2)即补充定理\ref{cor376}、补充定理\ref{cor378}.
			\par
			(3)前两部分即补充定理\ref{cor379},最后一部分根据定义可证.
			\par
			(4)即补充定理\ref{cor380}(2).
			
			\begin{exer}\label{exer172}
				\hfill\par
				$a$为序数,求证:当且仅当对任意基数族$(x_i)_{i\in I}$,若$Card(I)<\aleph_a$,且对任意$i\in I$均有$x_i<\aleph_a$,则$\sum\limits_{i\in I}(x_i)_{i\in I}<\aleph_a$时,$\aleph_a$为正则基数.
			\end{exer}
			证明:即补充定理\ref{cor381}(3).
			
			\begin{exer}\label{exer173}
				\hfill\par
				$a$为序数,基数$m\neq 0$,则:
				\par
				(1)${\aleph_{a+1}}^m={\aleph_a}^m\aleph_{a+1}$.
				\par
				(2)序数$c$满足$Card(c)\leq m$,则${\aleph_{a+c}}^m={\aleph_a}^m{\aleph_{a+c}}^{Card(c)}$.
				\par
				(3)$Card(a)\leq m$,则${\aleph_a}^m=2^m{\aleph_a}^{Card(a)}$.
			\end{exer}
			证明:即补充定理\ref{cor382}.
			
			\begin{exer}\label{exer174}
				\hfill\par
				(1)$a$、$b$为序数,$a$没有前导,$x\mapsto s_x$为$[0, \omega_b[$到$[0, a[$的严格单增映射,且$\mathop{sup}\limits_{x\in [0, \omega_b[}s_x=a$,求证:$\aleph_a\aleph_b=\mathop{\mathsf{P}}\limits_{x\in [0, \omega_b[}\aleph_{s_x}$.
				\par
				(2)$a$、$a'$为序数,令$\omega_{a'}$为$\omega_a$的共尾性,则${\aleph_a}^{\aleph_{a'}}>\aleph_a$.并且,令$c$为序数,如果存在基数$n$使$\aleph_a=n^{\aleph_c}$,则$c<a'$.
				\par
				(3)$a$、$a'$为序数,令$\omega_{a'}$为$\omega_a$的共尾性,序数$b<a'$,则${\aleph_a}^{\aleph_b}=\sum\limits_{c\in [0, a[}{\aleph_c}^{\aleph_b}$.
			\end{exer}
			证明:
			\par
			(1)	即补充定理\ref{cor383}.
			\par
			(2)	即补充定理\ref{cor384}、补充定理\ref{cor385}.
			\par
			(3)	即补充定理\ref{cor386}.
			\par
			注:习题\ref{exer174}(1)中“严格单增”的条件可以去掉.
			
			\begin{exer}\label{exer175}
				\hfill\par
				(1)	求证:$a$为正则基数,基数$b\neq 0$,则$a^b=a\sum\limits_{m\in [0, a[}m^b$.并且,如果$(\forall k)(k\text{为序数}\Rightarrow 2^{\aleph_k}=\aleph_k+1)$,$a$为基数且$a>0$,如果对任意基数$b\neq 0$均有$a^b=a\sum\limits_{m\in [0, a[}m^b$,则$a$为正则基数.
				\par
				(2)$a$为基数且$a>2$,对任意基数$m\in ]0, a[$,均有$a^m=a$,求证:$a$为正则基数.
				\par
				(3)求证:$(\forall a)(\forall m)((a\text{为正则基数})\text{与}(m\text{为基数})\text{与}(m\in ]0, a[)\Rightarrow a^m=a)\Leftrightarrow (\forall k)\\(k\text{为序数}\Rightarrow 2^{\aleph_k}=\aleph_{k+1})$.
			\end{exer}
			证明:
			\par
			(1)	即补充定理\ref{cor390}、补充定理\ref{cor392}(5).
			\par
			(2)	即补充定理\ref{cor393}.
			\par
			(3)	即补充定理\ref{cor394}.
			
			\begin{exer}\label{exer176}
				\hfill\par
				(1)基数$a>0$,对任意基数$m<a$,均有$2^m<a$,求证:$a$为支配基数.
				\par
				(2)递归定义基数序列$(a_i)_{i\in N}$:
				\par
				$a_0=\aleph_0$,
				\par
				对任意$i\in N$,$a{i+1}=2^{a_i}$.
				\par
				令$b=\sum\limits_{i\in N}a_i$,求证:$b$是大于$\aleph_0$的最小的支配基数.
				\par
				(3)递归定义基数序列$(a_i)_{i\in N}$:
				\par
				$a_0=\aleph_0$,
				\par
				对任意$i\in N$,$a{i+1}=2^{a_i}$.
				\par
				令$b=\sum\limits_{i\in N}a_i$,求证:
				\par
				$b^{\aleph_0}=2^b$;
				\par
				${\aleph_0}^b=b^{\aleph_0}$;
				\par
				$b^{\aleph_0}=(2^b)^b$.
			\end{exer}
			证明:
			\par
			(1)即补充定理\ref{cor396}.
			\par
			(2)即补充定理\ref{cor397}.
			\par
			(3)即补充定理\ref{cor398}.
			
			\begin{exer}\label{exer177}
				\hfill\par
				(1)求证:如果$(\forall k)(k\text{为序数}\Rightarrow 2^{\aleph_k}=\aleph_k+1)$,则不可达基数均为强不可达基数.
				\par
				(2)基数$a\geq 3$,求证:当且仅当对任意基数族$(a_i)_{i\in I}$均有$Card(I)<a\text{与}(\forall i)(i\in I\Rightarrow a_i<a)\Rightarrow \mathop{\mathsf{P}}\limits_{i\in I}a_i<a$时,$a$为强不可达基数.
				\par
				(3)$a$为无穷基数,求证:当且仅当a为支配基数并且满足下列条件之一时,a为强不可达基数:
				\par
				第一,对任意基数$b>0$且$b<a$均有$a^b=a$;
				\par
				第二,如果对任意基数$b>0$,均有$a^b=a2^b$,则对任意基数$b>0$且$b<a$,均有$a^b=a$.
			\end{exer}
			证明:
			\par
			(1)即补充定理\ref{cor399};
			\par
			(2)即补充定理\ref{cor400};
			\par
			(3)根据补充定理\ref{cor401}、补充定理\ref{cor402}可证.
			
			\begin{exer}\label{exer178}
				\hfill\par
				(1)$a$、$b$为序数,$g$为$[0, b[$到$[0, a[$的严格单增映射,并且,对任意序数$c$,$g(\mathop{sup}\limits_{d\in [0, c[}d)\\=\mathop{sup}\limits_{d\in [0, c[}g(d)$,$\mathop{sup}\limits_{d\in [0, b[}g(d)=a$.求证:当且仅当存在$[0, b$[到$[0, b[$的发散映射$h$满足$(\forall x)\\(x\in ]0, b[\Rightarrow h(x)<x)$时,存在$[0, a[$到$[0, a[$的发散映射$f$满足$(\forall x)(x\in ]0, a[\Rightarrow f(x)<x)$.
				\par
				(2)$a$为序数,求证:当且仅当$\omega_a$的共尾性为$\omega_0$时,存在$[0, \omega_a[$到$[0, \omega_a[$的发散映射$f$满足$(\forall x)(x\in [0, \omega_a[\Rightarrow f(x)<x)$.
				\par
				(3)$a$、$a'$为为序数,且$a'>0$,$\omega_a$的共尾性为$\omega_{a'}$,$f$为$[0, \omega_a[$到$[0, \omega_a[$的映射,且满足$(\forall x)\\(x\in [0, \omega_a[\Rightarrow f(x)<x)$,求证:存在序数$l_0$,使$Card(\{x|x\in [0, \omega_a[\text{与}f(x)=l0\})\geq \aleph_{a'}$.
			\end{exer}
			证明:
			\par
			(1)即补充定理\ref{cor403}.
			\par
			(2)即补充定理\ref{cor404}.
			\par
			(3)即补充定理\ref{cor405}.
			
			\begin{exer}\label{exer179}
				\hfill\par
				$F$为无穷集合,其元素都是$E$的子集,对任意$A\in F$均有$Card(A)=Card(F)$,求证:
				\par
				(1)存在$E$的子集$P$使$Card(P)=Card(F)$,并且$F$的所有元素都不是$P$的子集.
				\par
				(2)如果对$F$的任意子集$G$,$Card(G)<Card(F)$,均有$Card(E-\bigcup\limits_{A\in G}A)\geq Card(F)$,则存在$E$的子集$P$使$Card(P)=Card(F)$,并且对任意$A\in F$均有$Card(A\cap P)<Card(F)$.
			\end{exer}
			证明:即补充定理\ref{cor406}.
			
			\begin{exer}\label{exer180}
				\hfill\par
				(1)$E$为无穷集合,集族$(X_i)_{i\in I}$为$E$的覆盖,且对任意$i\in I$、$j\in I$且$i\neq j$,均有$X_i\neq X_j$,$c$为$(X_i)_{i\in I}$的不相交度,求证:$Card(I)\leq Card(E)^c$.
				\par
				(2)$a$为序数,集合$F$满足$Card(F)\geq 2$且$Card(F)<\aleph_a$,$E$为$[0, \omega_a[$不同于自身的片段集合到$F$的映射的集合.对任意$[0, \omega_a[$到$F$的映射$f$,令$K_f$为$f$在$[0, \omega_a[$的不同于自身的片段上的限制的集合.令$G=(K_f)_{f\in \{z|z\text{为}[0, \omega_a[\text{到}F\text{的映射}\}}$.求证:$Card(E)\leq Card(F)\aleph_a$,$G$为$E$的覆盖,$Card(\{z|\text{为}[0, \omega_a[\text{到}F\text{的映射}\})=Card(F)\aleph_a$,并且其不相交度等于$\aleph_a$.
				\par
				(3)$E$为无穷集合,$Card(E)=a$,基数$p>1$、$c>p$,并且,对任意基数$m<c$,均有$p^m<a$,同时$a=\sum\limits_{m\in [0, c[}p^m$.求证:存在集族$(X_i)_{i\in I}$,对任意$i\in I$,$Card(X_i)=c$,$Card(I)=p^c$,且其不相交度等于$c$.特别是,$E$为可数无穷集合时,存在$E$的覆盖$(X_i)_{i\in I}$,$Card(I)=2^{\aleph_0}$,且对任意$i\in I$、$j\in I$,$X_i\cap X_j$为有限集合.
			\end{exer}
			证明:
			\par
			(1)	即补充定理407.
			\par
			(2)	$Card(E)=\sum\limits_{m\in [0, \aleph_a[}Card(F)^m$,故$Card(E)\leq \aleph_aCard(F)^{\aleph_a}$,因此$Card(E)\leq Card(F)^{\aleph_a}$.
			\par
			根据定义可证$G$为$E$的覆盖、$Card(\{z|z\text{为}[0, \omega_a[\text{到}F\text{的映射}\})=Card(F)^{\aleph_a}$.
			\par
			对任意$[0, \omega_a[$到$F$的映射$f$、$g$,$K_f\cap K_g=\bigcup\limits_{x\in \{x|x\in [0, \omega_a[\text{与}f|Sx=g|Sx\}}\{f|Sx\}$,故$Card(K_f\cap Kg)<\aleph_a$.同时,对任意基数$c<\aleph_a$,令$u$、$v$是$F$的不同元素,$h$为任何一个$[0, c[$到$F$的映射,令$f$、$g$为$h$在$[0, \omega_a[$上的延拓,其中,当$x\in [c, \omega_a[$时,$f(x)=u$,$g(x)=v$,则$Card(K_f\cap K_g)=c$,故$G$的不相交度等于$\aleph_a$.
			\par
			(3)	类似习题\ref{exer180}(2)可证.
			
			\begin{exer}\label{exer181}
				\hfill\par
				$E$为无穷集合,$F$的元素均为$E$的子集,$A\in F$,基数$a\geq \aleph_0$,$Card(E)=a$、$Card(A)\\=a$、$Card(F)=a$,求证:存在$E$的划分$(B_i)_{i\in I}$,使$Card(I)=a$,对任意$i\in I$均有\\$Card(B_i)=a$,并且对任意$i\in I$、$A\in F$均有$A\cap B_i\neq \varnothing$.
			\end{exer}
			证明:令$a=\aleph_c$,$h_0$为$[0, \omega_c[$到$F$的同构.令$f_0(0)=\tau_x(x\in h_0(0))$,对于$i\in [1, \omega_c[$,$h_0(i)\cap(\bigcup\limits_{j\in [0, i[}\{f0(j)\})\neq \varnothing$,故令$f_0(i)=\tau_x(x\in h_0(i)\cap(\bigcup\limits_{j\in [0, i[}\{f0(j)\}))$,$g(0)=f_0\langle[0, \omega_c[\rangle$;对$k\in [1, \omega_c[$,$x\in [0, \omega_c[$,令$h_k(x)=h_0(x) \cap(\bigcup\limits_{j\in [0, k[}g(j))$,则$h_k(x)\neq \varnothing$,此时,令$f_k(0)=\tau_x(x\in h_k(0))$,对于$i\in [1, \omega_c[$,$h_k(i)\cap(\bigcup\limits_{j\in [0, i[}\{fk(j)\})\neq \varnothing$,故令$f_k(i)=\tau_x(x\in h_k(i)\cap(\bigcup\limits_{j\in [0, i[}\{fk(j)\}))$,$g(k)=f_k\langle[0, \omega_c[\rangle$.
			\par
			最后令$B_0=E-\bigcup\limits_{j\in [1, \omega_c[}g(j)$,对$i\in [1, \omega_c[$,$B_i=g(i)$,$I=[0, \omega_c[$,则$(B_i)_{i\in I}$满足要求.
			
			\begin{exer}\label{exer182}
				\hfill\par
				$L$为无穷集合,$(E_l)_{l\in L}$为非空族,对任意自然数$n>0$,$Card(\{l|l\in L\text{与}Card(E_l)>n\})=Card(L)$,$E=\prod\limits_{l\in L}E_l$,求证:存在$F\subset E$, $Card(F)=2^{Card(E)}$,并且,对$F$的元素组成的任意有限序列$(f_k)_{k\in [1, n]}$,均存在$l\in L$,使所有的$f_k(l)$各不相同.
			\end{exer}
			证明:
			\par
			对任意自然数$n$,令$A_n=\{l|l\in L\text{与}Card(E_l)=n\}$;如果$A_n$为有限集合,令$g$为$A_n$到\\$[0, Card(A_n)-1]$的双射,当$l\in A_n$时,$f(l)=\{g(l)\}$;如果$A_n$为无穷集合,令$h$为$A_n$到$N\times A_n$的双射,当$l\in A_n$时,$f(l)=pr_1(h(l))$.
			\par
			对任意自然数$n$,令$B_n=\{l|l\in L\text{与}f(l)=n\}$,对任意自然数$j$,令$S_j=[2^j, 2^{j+1}-1]$,$L_j=\bigcup\limits_{k\in S_j}B_k$,则对任意$l\in L_j$,$Card(E_l)\geq 2^j$,$Card(L_j)=Card(L)^j$,并且,$(L_j)_{j\in N}$是$L$\\的划分.
			\par
			令$p_j$为$L_j$到$\prod\limits_{i\in [1, j]}L$的双射,$q_l$为$\prod\limits_{i\in [1, j]}\{0, 1\}$到$E_l$的单射,$G=\mathcal{F}(L; \{0, 1\})$,对任意$g\in G$,设$l\in L_j$,$p_j(l)=(x_i)_{i\in [1, j]}$,则令$f_g(l)=q_l(g(x_i)_{i\in [1, j]})$.令$F=\bigcup\limits_{g\in G}\{f_g\}$.
			\par
			对$F$的元素组成的任意有限序列$(f_k)_{k\in [1, n]}$,令$A=\bigcup\limits_{(i, j)\in [1, n]\times [1, n]-\Delta_{[1, n]}}\{\tau_z(g_i(z)\neq g_j(z))\}$,则$A$为有限集合,令$l={p_{Card(A)}}^{-1}((i)_{i\in A})$,则所有的$f_k(l)$各不相同,得证.
			\par
			注:原书习题\ref{exer182}遗漏“非空”的条件.
			
			\begin{exer}\label{exer183}
				\hfill\par
				$E$为无穷集合,$n$为自然数,$F_n(E)$为$E$的元素数目为n的子集的集合,$m$为自然数,\\$(x_i)_{i\in [1, m]}$为$F_n(E)$的划分,求证:存在$i\in [1, m]$和$E$的无穷子集$F$,使$F$的元素数目为$n$的子集,都是$x_i$的元素.
			\end{exer}
			证明:
			\par
			命题对$n=1$显然成立;假设命题对$[1, n-1]$成立,则对任意$a\in E$,均存在$j(a)\in [1, m]$和$E-\{a\}$的无穷子集$M(a)$,使$M(a)$任意元素数目为$n-1$的子集$A$,都满足$A\cup\{a\}$为、、$X_j(a)$的元素.
			\par
			令$a_1$为$E$的任意元素,$a_2$为$M(a_1)$的任意元素,$a_3$为$M(a_1)$按照同样方法确定的无穷子集的任意元素,以此类推.则$\bigcup\limits_{i\in N}\{a_i\}$符合要求,得证.
			
			\begin{exer}\label{exer184}
				\hfill\par
				(1)$E$为偏序集,求证:有限个按导出的偏序排序的$E$的诺特子集的并集,按导出的偏序排序,也是诺特集.
				\par
				(2)$E$为偏序集,求证:当且仅当对任意$a\in E$,区间$]a, \to [$均为诺特集时,$E$为诺特集.
				\par
				(3)$E$为偏序集,其按相反关系排序得到的偏序集为诺特集.$u$为字母,$T$为项.求证:存在集合$U$和$E$到$U$的映射$f$,使对任意$x\in E$,均有$f(x)=(f(x)|u)T$,其中$f(x)$是$f(x)$在区间$]\gets, x[$上的限制,并且,满足条件的$U$和$f$是唯一的.
				\par
				(4)$E$为诺特集,并且$E$的任何非空有限子集均有最小上界.求证:如果$E$有最小元,则$E$为完备格;如果$E$没有最小元,则将$a$添加到集合$E$并使$a$为最小元得到的偏序集$E'$,为完备格.
			\end{exer}
			证明:
			\par
			(1)	即补充定理\ref{cor408}(2).
			\par
			(2)	即补充定理\ref{cor409}.
			\par
			(3)	即补充证明规则\ref{Ccor91}.
			\par
			(4)	根据补充定理\ref{cor410}(2)可证.
			\par
			注:原书习题\ref{exer184}(4)遗漏“非空”的条件.
			
			\begin{exer}\label{exer185}
				\hfill\par
				$E$为格,其按相反关系排序得到的偏序集为诺特集.
				\par
				求证:
				\par
				对任意$a\in E$,$a$可以表示为$\mathop{sup}\limits_{i\in [1, n]}e_i$,其中,$n$为自然数,并且,对任意$i\in [1, n]$,$e_i$为$E$\\的不可约元素.
				\par
				$J$为其不可约元素集合.令$S(x)=\{y|y\in J\text{与}y\leq x\}$,则$x\mapsto S(x)$为$E$到按包含关系排序的$\mathcal{P}(J)$的一个子集的同构,并且,$S(inf(x, y))=S(x)\cap S(y)$.
				\par
				$E$为分配格,$J$为其不可约元素集合.令$S(x)=\{y|y\in J\text{与}y\leq x\}$,则$x\mapsto S(x)$为$E$\\到按包含关系排序的$\mathcal{P}(J)$的一个子集的同构,并且,$S(sup(x, y))=S(x)\cup S(y)$;同时,令$J^*$为按在$J$上的偏序关系的相反关系排序的偏序集,$I=\{0, 1\}$,$A(J^*, I)$为$J^*$到$I$的单增映射的集合,按$f\in A(J^*, I)\text{与}g\in A(J^*, I)\text{与}(\forall x)(x\in J^*\Rightarrow f(x)\leq g(x))$排序,则$E$同构于$A(J^*, I)$.
				\par
				$E$为分配格,$J$为其不可约元素集合.$a$为$E$的最小元,$P=J-\{a\}$.令$A=\\\{a|(\exists X)((X\text{为}P\text{的自由子集})\text{与}Card(X)=a)\}$,$A$的最大元为$n$,则$E$同构于全序集有限族的乘积的某个内部格.
			\end{exer}
			证明:
			\par
			第一部分:设$a$不是不可约元素,则存在$b$、$c$使$sup(b, c)=a$且$b<a$,$c<a$.则$b$、$c$是不可比较的,令$inf(b, c)=d$.如果$\{x|x\in E\text{与}x<b\text{与}(\text{非}x\leq d)\}$为空,则$b$为不可约元素,令$e=b$;否则$\{x|x\in E\text{与}x<b\text{与}(\text{非}x\leq d)\}$存在极小元$e$,$e$为不可约元素.在两种$i$情况下均有$a=sup(e, c)$.根据定理\ref{theo168}可证.
			\par
			其他部分,类似习题\ref{exer133}(2)、习题\ref{exer134}(2)、习题\ref{exer135}(2)可证.
			
			\begin{exer}\label{exer186}
				\hfill\par
				$A$为无穷集合,$E$为$A$的无穷子集集合,并按包含关系的相反关系排序.求证:$E$为完全右方分支集,但不是右方无向集.
			\end{exer}
			证明:根据定义可以证明E为完全右方分支集.对任意无穷集合$y\subset A$且$y\neq A$,$a\in A-y$,令$x=y\cup\{a\}$,则对任意无穷集合$z\subset x$,$z\cap y\subset A$且为无穷集合.根据补充定理\ref{cor212},$E$不是右方无向集.
			
			\begin{exer}\label{exer187}
				\hfill\par
				$(M_n)_{n\in Z}$、$(P_n)_{n\in Z}$均为两两不相交且不全为空集的有限集合序列,其中,指标集$Z$为整数集.令$a_n=Card(M_n)$,$b_n=Card(P_n)$.如果存在自然数$k>0$,使对任意$n\in Z$和自然数$l\geq 1$,均有:
				\par
				$a_n+a_{n+1}+\cdots+a_{n+l}\leq b_{n-k}+b_{n-k+1}+\cdots +b_{n+k+l}$;
				\par
				$b_n+b_{n+1}+\cdots+b_{n+l}\leq a_{n-k}+a_{n-k+1}+\cdots +a_{n+k+l}$.
				\par
				令$M=\bigcup\limits_{n\in Z}M_n$,$P=\bigcup\limits_{n\in Z}P_n$.
				\par
				求证:存在$M$到$P$的双射$f$,使对任意$n\in Z$,均有:
				\par
				$f(M_n)\subset \bigcup\limits_{i\in [n-k-1, n+k+1]}P_i$,$f^{-1}(P_n)\subset \bigcup\limits_{i\in [n-k-1, n+k+1]}M_i$.
			\end{exer}
			证明:令$M$、$P$按各集合全序的序数和排序,设$M_{m_0}\neq \varnothing$,其最小元为$x$;$d$为所有\\$b_{n-k}+b_{n-k+1}+\cdots +b_{n+k+l}-(a_n+a_{n+1}+\cdots+a_{n+l})$、$a_{n-k}+a_{n-k+1}+\cdots +a_{n+k+l}-(b_n+b_{n+1}+\cdots+b_{n+l})$($n\in Z$、自然数$l\geq 1$)的最小值,令$g$为$M$到$P$的同构,其中$g(x)$为$\bigcup\limits_{i\in [m0-k, \to [}P_i$的最小元素.
			\par
			令$S_0=0$,$S_i=\sum\limits_{j\in [-k, i-k-1]}b_{m_0+j}-\sum\limits_{j\in [0, i-1]}a_{m_0+j}$($i>0$),$S_i=\sum\limits_{j\in [-1, i]}a_{m_0+j}-\\\sum\limits_{j\in [-k-1, i-k]}b_{m_0+j}$($i<0$),$T_0=\sum\limits_{j\in [-k, k-1]}b_{m_0+j}$,$T_i=\sum\limits_{j\in [-k, k+i-1]}b_{m_0+j}-\sum\limits_{j\in [0, i-1]}a_{m_0+j}$($i>0$),$T_i=\sum\limits_{j\in [-1, i]}a_{m_0+j}+\sum\limits_{j\in [-k, k+i-1]}b_{m_0+j}$($i<0$且$i>-2k$),$T_{-2k}=\sum\limits_{j\in [-1, -2k]}a_{m_0+j}$ ,$T_i=\sum\limits_{j\in [-1, i]}a_{m_0+j}-\sum\limits_{j\in [-k-1, i+k]}b_{m_0+j}$($i<-2k$).则对任意$i\in Z$、$j\in Z$,$S_i\leq T_j$,因此存在$d$,对于任意$i\in Z$,均有$S_i\leq d$,$T_i\geq d$,令$f$为$M$到$P$的同构,其中$f(x)为\bigcup\limits_{i\in [m_0-k, \to [}P_i$的第$d+1$个元素.则$f$满足要求.
			\par
			注:
			\par
			习题\ref{cor187}的结论可加强为$f(M_n)\subset \bigcup\limits_{i\in [n-k, n+k]}P_i$,$f^{-1}(P_n)\subset \bigcup\limits_{i\in [n-k, n+k]}M_i$.
			\par
			同时,习题\ref{cor187}涉及尚未介绍的“整数”知识.
			
			\begin{exer}\label{exer188}
				\hfill\par
				$a$、$b$为基数,$a\geq 2$、$b\geq 1$,其中至少有一个是无穷基数.$E$为集合,$F\subset \mathcal{P}(E)$,$Card(F)>a^b$,并且对任意$X\in F$,均有$Card(X)\leq b$.求证:存在$G\subset F$,$Card(G)>a^b$,并且$G$的任何两个元素都有相同的交集.
			\end{exer}
			证明:
			\par
			令$c$为大于$a^b$的最小基数,$G\subset F$且$Card(G)=c$,$M=\bigcup\limits_{X\in G}X$.假设$M\leq a^b$,若$b$为有限基数,根据补充定理\ref{cor337}(7),$Card(G)\leq a^b$,若$b$为无穷基数,根据补充定理\ref{cor337}(8),$Card(G)\leq a^b$,矛盾,因此$Card(M)\geq c$,同时,$Card(M)\leq \sum\limits_{X\in G}Card(X)$,故$Card(M)=c$.
			\par
			令$M$按最小良序排序,在$b$上的最小良序的偏序类为$r$.对任意$X\in G$,令其偏序类为$t_X$,$f_X$为$[0, t_X[$到$X$的同构,对任意序数$i\in [0, r[$,$y_i=\bigcup\limits_{X\in \{Y|Y\in G\text{与}i<t_Y\}}\{f_X(i)\}$,由于\\$\bigcup\limits_{i\in [0, r[}y_i=M$,故存在序数$i\in [0, r[$,使$Card(y_i)=c$,设满足条件的最小序数为$j$,故\\$Card(\bigcup\limits_{i\in [0, j[}y_i)<c$.
			\par
			进而,存在$N\subset G$,使$Card(N)=c$且对任意$Y\in N$,$f_Y(j)$各不相同.令$N_0\subset N$,使$Card(N_0)=c$且对任意$Y\in N_0$,$f_Y(0)$全部相等,进而,递归定义$N_i$($i<j$),令$N_i\subset \bigcap\limits_{k\in [0, i[}N_k$,且对任意$Y\in N_0$,$f_Y(i)$全部相等.
			\par
			令$Q=\bigcup\limits_{i\in [0, j[}f_Y(i)$,$R=\bigcap\limits_{i\in [0, j[}N_i$,$s$为在$c$上的最小良序的偏序类.令$X_0\in R$且$f_{X_0}(j)$是所有$f_Y(j)$($Y\in R$)的最小元,进而,递归定义$X_i$($i<s$),令$X_i$为所有$f_Y(j)$($Y\in R$且$Y-Q\subset M-\bigcup\limits_{j\in [0, i[}X_j$)的最小元.因此,对任意$i\in [0, s[$、$j\in [0, s[$且$i\neq j$,均有$X_i\cap X_j=Q$,得证.

		\section{射影极限和归纳极限(Limites projectives et limites inductives)}		
			\begin{de}
				\textbf{集合射影系统(système projectif d'ensembles),集族的射影极限(limite projective de famille d'ensembles),集族的射影极限到集合的规范映射(application canonique de la limite projective da famille d'ensembles dans un ensemble)}
				\par
				$I$为预序集,$(E_a)_{a\in I}$为集族,$(f_{ab})_{a\in I\text{与}b\in I\text{与}a\leq b}$为函数族,其中$f_{ab}$为$E_b$到$E_a$的映射,并且满足下列条件:
				\par
				第一,如果$a\leq b$,$b\leq c$,则$f_{ab}\circ f_{bc}=f_{ac}$;
				\par
				第二,$f_{aa}=Id_{E_a}$,
				\par
				则$((E_a)_{a\in I}, (f_{ab})_{a\in I\text{与}b\in I\text{与}a\leq b})$称为关于$I$的集合射影系统,在没有歧义的情况下可以简记为$((E_a), (f_{ab}))$或$(E_a, f_{ab})$.
				\par
				令$G=\prod\limits_{a\in I}E_a$,$E=\{x|x\in G\text{与}(\forall a)(\forall b)(a\in I\text{与}b\in I\text{与}a\leq b\Rightarrow pr_ax=f_{ab}(pr_bx))\}$,则称$E$为集族$(E_a)_{a\in I}$对于函数$(f_{ab})_{a\in I\text{与}b\in I\text{与}a\leq b}$的射影极限,记作$\lim\limits_{\gets a}(E_a, f_{ab})$,在没有歧义的情况下可以简记为$\lim\limits_\gets(E_a, f_{ab})$或$\lim\limits_\gets E_a$.$pr_a$在$E$上的限制称为$E$到$E_a$的规范映射.
			\end{de}
			
			\begin{cor}\label{cor411}
				\hfill\par
				$((E_a)_{a\in I}, (f_{ab})_{a\in I\text{与}b\in I\text{与}a\leq b})$为关于预序集$I$的集合射影系统,$E=\lim\limits_\gets(E_a, f_{ab})$,对任意$a\\\in I$,令$E$到$E_a$的规范映射为$f_a$,则当$a\in I$、$b\in I$、$a\leq b$时,$f_a=f_{ab}\circ f_b$.
			\end{cor}
			证明:根据定义可证.
			
			\begin{cor}\label{cor412}
				\hfill\par
				$I$为按$x=y\text{与}x\in I$排序的预序集,$(E_a)_{a\in I}$为集族,$(f_{aa})_{a\in I}$为函数族,且对任意$a\in I$,$f_{aa}=Id_{E_a}$.则$\lim\limits_\gets E_a= \prod\limits_{a\in I}E_a$.
			\end{cor}
			证明:由于$a\in I\text{与}b\in I\text{与}a\leq b\Rightarrow a=b$.令$G=\prod\limits_{a\in I}E_a$,因此,对任意$x\in G$,$pr_ax=f_{aa}(pr_ax)$为真,得证.
			
			\begin{cor}\label{cor413}
				\hfill\par
				关于$\varnothing$的集合射影系统的射影极限为$\{\varnothing\}$.
			\end{cor}
			证明:根据定义可证.
			
			\begin{cor}\label{cor414}
				\hfill\par
				$(Ea, f_{ab})$为关于$I$的集合射影系统,其中$I$为右方有向集,$\lim\limits_\gets E_a=E$,对任意$a\in I$,令$f_a$为$E$到$E_a$的规范映射,如果对任意$a\in I$、$b\in I$,$f_{ab}$均为单射,则对任意$a\in I$,$f_a$为单射.
			\end{cor}
			证明:
			\par
			设$x\in E$、$y\in E$,且$f_a(x)=f_a(y)$,对任意$b\in I$,存在$c\in I$使$a\leq c$、$b\leq c$,由于$f_{ac}$为单射,故$f_c(x)=f_c(y)$,故$f_b(x)=f_b(y)$.即对任意$b\in I$,$pr_bx=pr_by$,因此$x=y$,得证.
			
			\begin{cor}\label{cor415}
				\textbf{限制指标集可以得到集合射影系统}
				\par
				$I$为预序集,$((E_a)_{a\in I}, (f_{ab})_{a\in I\text{与}b\in I\text{与}a\leq b})$为关于$I$的集合射影系统,$(E_a)_{a\in I}$对于\\$(f_{ab})_{a\in I\text{与}b\in I\text{与}a\leq b}$的射影极限为$E$,$J$为$I$的预序子集,则$((E_a)_{a\in J}, (f_{ab})_{a\in J\text{与}b\in J\text{与}a\leq b})$是关于$J$的集合射影系统,并且,令$E'$为$(E_a){a\in J}$对于$(f_{ab})_{a\in J\text{与}b\in J\text{与}a\leq b}$的射影极限,对任意$a\in I$,$E$到$E_a$的规范映射为$f_a$,则对任意$x\in E$,$(f_a(x))_{a\in J}\in E'$.
			\end{cor}
			证明:根据定义可证.
			
			\begin{de}
				\textbf{通过限制得到的集合射影系统(système projectif d'ensembles obtenu par restriction),集族的射影极限的之间的规范映射(application canonique entre limites projectives de familles d'ensembles)}
				\par
				$I$为预序集,$((E_a)_{a\in I}, (f_{ab})_{a\in I\text{与}b\in I\text{与}a\leq b})$为关于$I$的集合射影系统,$J$为$I$的预序子集,则\\$((E_a)_{a\in J}, (f_{ab})_{a\in J\text{与}b\in J\text{与}a\leq b})$称为通过将指标集限制在$J$上得到的集合射影系统.
				\par
				令$(E_a)_{a\in I}$对于$(f_{ab})_{a\in I\text{与}b\in I\text{与}a\leq b}$的射影极限为$E$,为$(E_a){a\in J}$对于$(f_{ab})_{a\in J\text{与}b\in J\text{与}a\leq b}$的射影极限为$E'$,对任意$a\in I$,$E$到$E_a$的规范映射为$f_a$,则函数$x\mapsto (f_a(x))_{a\in J}$称为$E$到$E'$的规范映射.
			\end{de}
			
			\begin{cor}\label{cor416}
				\hfill\par
				$I$为预序集,$J$为$I$的预序子集,$J'$为$J$的预序子集,$(E_a)_{a\in I}$对于$(f_{ab})_{a\in I\text{与}b\in I\text{与}a\leq b}$的射影极限为$E$,$(E_a)_{a\in J}$对于$(f_{ab})_{a\in J\text{与}b\in J\text{与}a\leq b}$的射影极限$E'$,$(E_a)_{a\in J'}$对于$(f_{ab})_{a\in J'\text{与}b\in J'\text{与}a\leq b}$的射影极限为$E''$,$E$到$E'$的规范映射为$g$,$E'$到$E''$的规范映射为$g'$,$E$到$E''$的规范映射为$g''$,则\\$g''=g'\circ g$.
			\end{cor}
			证明:根据定义可证.
					
			\begin{theo}\label{theo172}
				\textbf{集合到集族的映射族的映射族的射影极限唯一存在}
				\par
				$I$为预序集,$(E_a, f_{ab})$为关于$I$的集合射影系统,$E= \lim\limits_\gets E_a$,对任意$a\in I$,$E$到$E_a$的规范映射为$f_a$.对任意$a\in I$,令$u_a$为$F$到$E_a$的映射,并且$(\forall a)(\forall b)(a\in I\text{与}b\in I\text{与}a\leq b\Rightarrow f_{ab}\circ u_b=u_a)$.则存在唯一的$F$到$E$的映射$u$,使$(\forall a)(a\in I\Rightarrow u_a=f_a\circ u)$,此时,该映射$u$满足$(\forall y)(y\in F\Rightarrow u(y)=(u_a(y))a\in I)$,并且,$(\exists y)(\exists z)(\exists a)(y\in F\text{与}z\in F\text{与}a\in I\text{与}u_a(y)\neq u_a(z))\Leftrightarrow (u\text{为单射})$.
			\end{theo}
			证明:
			\par
			$u_a=f_a\circ u$,即对任意$y\in F$,$pr_a(u(y))=u_a(y)$,因此,当且仅当$u(y)=(u_a(y))_{a\in I}$时,$u_a=f_a\circ u$;同时,当$a\in I\text{与}b\in I\text{与}a\leq b$时,$u_a(y)=f_{ab}(u_b(y))$,因此,$pr_a(u(y))=f_{ab}(pr_b(u(y)))$,故$u(y)\in E$,故$u$为$F$到$E$的映射.根据定义可证$(\exists y)(\exists z)(\exists a)(y\in F\text{与}z\in F\text{与}a\in I\text{与}u_a(y)\neq u_a(z))\Leftrightarrow (u\text{为单射})$.
			
			\begin{de}
				\textbf{集合到集族的映射射影系统(système projectif d'applications d'un ensemble dans la famille d'ensembles),集合到集族的映射族的射影极限(limite projective de famille d'applications d'un ensemble dans la famille d'ensembles)
				}
				\par
				$I$为预序集,$(E_a, f_{ab})$为关于$I$的集合射影系统,$E= \lim\limits_\gets E_a$,对任意$a\in I$,$E$到$E_a$的规范映射为$f_a$.对任意$a\in I$,令$u_a$为$F$到$E_a$的映射,并且$(\forall a)(\forall b)(a\in I\text{与}b\in I\text{与}a\leq b\Rightarrow f_{ab}\circ u_b=u_a)$.则称映射族$(u_a)_{a\in I}$为$F$到$(E_a, f_{ab})$的映射射影系统.如果$F$到$E$的映射$u$使$(\forall a)(a\in I\Rightarrow u_a=f_a\circ u)$,则称$u$为映射族$(u_a)_{a\in I}$的射影极限,记作$\lim\limits_\gets u_a$.
			\end{de}
			
			\begin{cor}\label{cor417}
				\hfill\par
				$I$为预序集,$(E_a, f_{ab})$为关于$I$的集合射影系统,$E= \lim\limits_\gets E_a$,对任意$a\in I$,$E$到$E_a$的规范映射为$f_a$,则$(f_a)_{a\in I}$为$E$到$(_Ea, f_{ab})$的映射射影系统,且$\lim\limits_\gets f_a=Id_E$.
			\end{cor}
			证明:根据补充定理\ref{cor411}可证.
					
			\begin{theo}\label{theo173}
				\textbf{集族之间的映射族的映射族的射影极限唯一存在}
				\par
				$I$为预序集,$(E_a, f_{ab})$、$(F_a, g_{ab})$为关于$I$的集合射影系统,$E= \lim\limits_\gets E_a$,$F= \lim\limits_\gets F_a$,对任意$a\in I$,$E$到$E_a$的规范映射为$f_a$,$F$到$F_a$的规范映射为$g_a$.对任意$a\in I$,令$u_a$为$E_a$到$F_a$的映射,并且$(\forall a)(\forall b)(a\in I\text{与}b\in I\text{与}a\leq b\Rightarrow u_a\circ f_{ab}=g_{ab}\circ u_b)$,则存在唯一的$E$到$F$的映射$u$,使$(\forall a)(a\in I\Rightarrow ua\circ f_a=g_a\circ u)$,此时,该映射满足$u(y)=(u_a(pr_a(y)))_{a\in I}$.
			\end{theo}
			证明:令$v_a=u_a\circ f_a$,则$v_a= u_a\circ f_{ab}\circ f_b$,等于$g_{ab}\circ u_b\circ f_b$,等于$g_{ab}\circ v_b$.根据定理\ref{theo172},存在唯一的$u$,使$(\forall a)(a\in I\Rightarrow v_a=g_a\circ u)$.
			
			\begin{de}
				\textbf{集族之间的映射射影系统(système projectif d'applications entre familles d'ensembles),集族之间的映射族的射影极限(limite projective de famille \\d'applications entre familles d'ensembles)}
				\par
				$I$为预序集,$(E_a, f_{ab})$、$(F_a, g_{ab})$均为关于$I$的集合射影系统,$E= \lim\limits_\gets E_a$,$F= \lim\limits_\gets F_a$,对任意$a\in I$,$E$到$E_a$的规范映射为$f_a$,$F$到$F_a$的规范映射为$g_a$.对任意$a\in I$,令$u_a$为$E_a$到$F_a$\\的映射,并且$(\forall a)(\forall b)(a\in I\text{与}b\in I\text{与}a\leq b\Rightarrow u_a\circ f_{ab}=g_{ab}\circ u_b)$,则称映射族$(u_a)_{a\in I}$为\\$(E_a, f_{ab})$到$(F_a, g_{ab})$的映射射影系统.如果$E$到$F$的映射$u$使$(\forall a)(a\in I\Rightarrow u_a\circ f_a=g_a\circ u)$,则称$u$为映射族族$(u_a)_{a\in I}$的射影极限,记作$\lim\limits_{\gets a} u_a$,在没有歧义的情况下也可以简记为$\lim\limits_\gets u_a$.
			\end{de}
					
			\begin{theo}\label{theo174}
				\hfill\par
				$I$为预序集,$(Ea, f_{ab})$、$(Fa, g_{ab})$、$(G_a, h_{ab})$为关于$I$的集合射影系统,$E= \lim\limits_\gets E_a$,$F= \lim\limits_\gets F_a$,$G= \lim\limits_\gets G_a$,对任意$a\in I$,$E$到$E_a$的规范映射为$f_a$,$F$到$F_a$的规范映射为$g_a$,$G$到$G_a$\\的规范映射为$h_a$.对任意$a\in I$,令$u_a$为$E_a$到$F_a$的映射,$v_a$为$F_a$到$G_a$的映射,则映射族$(v_a\circ u_a)_{a\in I}$为$(E_a, f_{ab})$到$(G_a, h_{ab})$的映射射影系统,并且$\lim\limits_\gets(v_a\circ u_a)=\lim\limits_\gets v_a\circ \lim\limits_\gets u_a$.
			\end{theo}
			证明:令$w_a=v_a\circ u_a$.则$w_a\circ f_{ab}= v_a\circ u_a\circ f_{ab}$,等于$v_a\circ g_{ab}\circ u_b$,等于$h_{ab}\circ v_b\circ u_b$,等于$h_{ab}\circ w_b$,因此映射族$(v_a\circ u_a)_{a\in I}$为$(E_a, f_{ab})$到$(G_a, h_{ab})$的映射射影系统.
			\par
			同时$h_a\circ v\circ u=v_a\circ g_a\circ u$,等于$v_a\circ u_a\circ f_a$,得证.			
						
			\begin{cor}\label{cor418}
				\textbf{子集上的系统为射影系统}
				\par
				$I$为预序集,$(E_a, f_{ab})$为关于$I$的集合射影系统,$E= \lim\limits_\gets E_a$,对任意$a\in I$,$M_a\subset E_a$,如果$(\forall a)(\forall b)(a\in I\text{与}b\in I\text{与}a\leq b\Rightarrow f_{ab}(M_b)\subset M_a)$,当$a\in I\text{与}b\in I\text{与}a\leq b$时,令$g_{ab}$为$f_{ab}$在$M_b$上的限制,则$(M_a, g_{ab})$也是关于$I$的集合射影系统,并且$\lim\limits_\gets M_a=E\cap\prod\limits_{a\in I}M_a$.
			\end{cor}
			证明:根据定义可证.
			
			\begin{de}
				\textbf{子集射影系统(système projectif de parties)}
				\par
				$I$为预序集,$(E_a, f_{ab})$为关于$I$的集合射影系统,对任意$a\in I$,$M_a\subset E_a$,如果$(\forall a)(\forall b)\\(a\in I\text{与}b\in I\text{与}a\leq b\Rightarrow f_{ab}(M_b)\subset M_a)$,则称$(M_a, g_{ab})$为$(E_a)_{a\in I}$的子集射影系统.
			\end{de}
					
			\begin{theo}\label{theo175}
				\hfill\par
				$I$为预序集,$(E_a, f_{ab})$、$({E'}_a, {f'}_{ab})$均为关于$I$的集合射影系统,对任意$a\in I$,$u_a$为$E_a$到\\${E'}_a$的映射,则映射族$(u_a)_{a\in I}$为映射射影系统.令$u=\lim\limits_\gets u_a$,$E'=\lim\limits_\gets {E'}_a$,则对任意\\$({x'}_a)_{a\in I}\in E'$,$(u^{-1}({x'}_a))_{a\in I}$是$(E_a)_{a\in I}$的子集射影系统,并且$\lim\limits_\gets({u_a}^{-1}({x'}_a))=u^{-1}(x')$.
			\end{theo}
			证明:设$x_b\in u^{-1}({x'}_b)$,$u_a(f_{ab}(x_b))={f'}_{ab}(u_b(x_b))$,等于${f'}_{ab}({x'}_b)$,等于${x'}_a$,因此,\\$({u_a}^{-1}({x'}_a))_{a\in I}$是$(E_a)_{a\in I}$的子集射影系统.设$x\in E$且$u(x)=x'$,根据定理\ref{theo172},$u$是唯一的,且$u(x)=(u_a(x))_{a\in I}$,对任意$a\in I$,$u_a(x)={x'}_a$.因此,$x\in E\Leftrightarrow u(x)\in E'$,因此,$x\in u^{-1}(x')\Rightarrow x'\in E'$,并且,$x\in u^{-1}(x')\Rightarrow x\in \prod\limits_{a\in I}u^{-1}({x'}_a)$,根据补充定理\ref{cor418},得证.			
					
			\begin{theo}\label{theo176}
				\hfill\par
				$I$为预序集,$(E_a, f_{ab})$、$({E'}_a, {f'}_{ab})$均为关于$I$的集合射影系统,对任意$a\in I$,$u_a$为$E_a$到\\${E'}_a$的映射,则映射族$(u_a)_{a\in I}$为映射射影系统.令$u=\lim\limits_\gets u_a$,如果对任意$a\in I$,$u_a$为单射(或双射),则$u$是单射(或双射).
			\end{theo}
			证明:根据定理\ref{theo175}可证.
			
			\begin{cor}\label{cor419}
				\hfill\par
				$I$为预序集,$(E_a, f_{ab})$、$({E'}_a, {f'}_{ab})$均均为关于$I$的集合射影系统,对任意$a\in I$,$u_a$为$E_a$到\\${E'}_a$的映射,则映射族$(u_a)_{a\in I}$为映射射影系统.令$u=\lim\limits_\gets u_a$,$E=\lim\limits_\gets E_a$,$E'=\lim\limits_\gets {E'}_a$,如果对任意$a\in I$,$u_a$为满射,则$(u_a(E_a))_{a\in I}$是$(E'a)_{a\in I}$的子集射影系统,$u(E)\subset \lim\limits_\gets u_a(E_a)$.
			\end{cor}
			证明:设${x'}_b\in u_b(E_b)$,令${x'}_b= u_b(x_b)$,${f'}_{ab}({x'}_b)= {f'}_{ab}(u_b(x_b))$,等于$u_a(f_{ab}(u_b))$,根据定义,$(u_a(E_a))_{ a\in I}$是$({E'}_a)_{a\in I}$的子集射影系统.同时,设$x\in E$且$u(x)=x'$,对任意$a\in I$,$u_a(x)={x'}_a$,得证.
					
			\begin{theo}\label{theo177}
				\hfill\par
				$I$为预序集,$(E_a, f_{ab})$为关于$I$的集合射影系统,$E=\lim\limits_\gets E_a$,$F$为$E$的共尾子集,并且是右方有向集,令$(E_a)_{a\in J}$对于$(f_{ab})_{a\in J\text{与}b\in J\text{与}a\leq b}$的射影极限为$E'$,则$E$到$E'$的规范映射$g$为双射.
			\end{theo}
			证明:
			对$a\in J$,令${f'}_a$为$E'$到$E_a$的规范映射,根据定理\ref{theo172},$g$是唯一满足$a\in J\Rightarrow f_a={f'}_a\circ g$的映射.如果$x\in E$、$y\in E$且$x\neq y$,则存在$a\in I$使$f_a(x)\neq f_a(y)$,由于$F$为$E$的共尾子集,因此存在$b\in F$,使$a\leq b$,因此$f_b(x)\neq f_b(y)$,故$f_a$为单射,因此$g$为单射.对任意$x'\in E'$,对任意$a\in I$,存在$b\in J$且$a\leq b$,假设$c\in J$且$a\leq b$,若$b\leq c$,则$f_{ac}({x'}_c)=f_{ab}(f_{bc}({x'}_c))$,等于$f_{ab}({x'}_b)$,因此,$f_{ab}({x'}_b)$和$b$无关,令其为$x_a$.令$x=(x_a)_{a\in J}$,对任意$a\in I$,如果$a\in I$,$b\in I$且$a\leq b$,则存在$b\leq c$,且$c\in J$,因此$f_{ab}(x_b)=f_{ab}(f_{bc}({x'}_c))$,等于$f_{ac}({x'}_c)$,等于$x_a$.因此$x\in E$.当$a\in J$时,由于${x'}_a=f_{aa}({x'}_a)$,因此${x'}_a=x_a$,因此$f_a(x)={x'}_a$,故$g(x)=x'$.因此$g$为满射.
			
			\begin{de}
				\textbf{集族的双重射影极限(double limite projective de famille d'ensembles)}
				\par
				$I$、$L$为预序集,$I\times L$的预序关系为$(x\in I\times L\text{与}y\in I\times L\text{与}pr_1x\leq pr_1y\text{与}pr_2x\leq pr_2y$),$((E_a^x)_{(a, x)\in I\times L}, (f_{ab}^{xy})_{(a, x)\in I\times L\text{与}(b, y)\in I\times L\text{与}(a, x)\leq (b, y)})$为关于$I\times L$的集合射影系统,则其射影极限称为双重射影极限,记作$\lim\limits_{\gets a, x}E_a^x$,在没有歧义的情况下也可以简记为$\lim\limits_\gets E_a^x$.
			\end{de}
			
			\begin{cor}\label{cor420}
				\hfill\par
				$I$、$L$均为预序集,$I\times L$的预序关系为$(x\in I\times L\text{与}y\in I\times L\text{与}pr_1x\leq pr_1y\text{与}pr_2x\leq pr_2y)$,$(E_a^x, f_{ab}^{xy})$为关于$I\times L$的集合射影系统,则$(E_a^x, f_{ab}^{xy})$为关于$I$的集合射影系统,也是关于$L$的集合射影系统.并且,$(\lim\limits_{\gets a}E_a^x, g^{xy})$、$(\lim\limits_{\gets x}E_a^x, h_{ab})$分别是关于$L$的集合射影系统和关于$I$的集合射影系统,其中$g^{xy}= \lim\limits_{\gets a}f_{aa}^{xy}$,$h_{ab}= \lim\limits_{\gets x}f_{ab}^{xx}$.
			\end{cor}
			证明:根据定义,可证$(E_a^x, f_{ab}^{xy})$为关于$I$的集合射影系统,也是关于$L$的集合射影系统.根据定理\ref{theo174},$g^{xz}=g^{xy}\circ g^{yz}$,$h_{ac}=h_{ab}\circ h_{bc}$,因此$(\lim\limits_{\gets a}E_a^x, g^{xy})$、$(\lim\limits_{\gets x}E_a^x, h_{ab})$分别是关于$L$的集合射影系统和关于$I$的集合射影系统.
					
			\begin{theo}\label{theo178}
				\hfill\par
				$I$、$L$均为预序集,$I\times L$的预序关系为$(x\in I\times L\text{与}y\in I\times L\text{与}pr_1x\leq pr_1y\text{与}pr_2x\leq pr_2y)$,$(E_a^x, f_{ab}^{xy})$为关于$I\times L$的集合射影系统,令$\prod\limits_{(a, x)\in I\times L}E_a^x$到$\prod\limits_{x\in L}(\prod\limits_{a\in I}E_a^x)$的规范映射为$f$,$\prod\limits_{(a, x)\in I\times L}E_a^x$到$\prod\limits_{a\in I}(\prod\limits_{x\in L}E_a^x)$的规范映射为$g$,则$f$在$(\lim\limits_{\gets a, x}E_a^x)$上的限制为$(\lim\limits_{\gets a, x}E_a^x)$到$\lim\limits_{\gets x}(\lim\limits_{\gets a}E_a^x)$\\的双射;$g$在$(\lim\limits_{\gets a, x}E_a^x)$上的限制为$(\lim\limits_{\gets a, x}E_a^x)$到$\lim\limits_{\gets a}(\lim\limits_{\gets x}E_a^x)$的双射.
			\end{theo}
			证明:令$g^{xy}= \lim\limits_{\gets a}f_{aa}^{xy}$,$F^x=\lim\limits_{\gets a}E_a^x$,$F= \lim\limits_{\gets x}F^x$,$G=\lim\limits_{\gets a, x}E_a^x$.根据定理\ref{theo46},$f$为$h\mapsto (pr_{I\times \{x\}}h)_{x\in L}$.
			\par
			对任意$u\in F$,如果$x\leq y$,则$pr_xu=g^{xy}(pr_yu)$;如果$a\leq b$,则$pr_a(pr_xu)=\\f_{ab}^{xy}(pr_b(pr_yu))$.由于$h\mapsto (pr_{I\times \{x\}}h)_{x\in L}$为双射,故令$u=(pr_{I\times \{x\}}u')_{x\in L}$,则$pr_{(a, x)}u'=\\f_{ab}^{xy}(pr_{(b, y)}u')$.故$u'\in G$.
			\par
			反过来,如果$u'\in G$,且$a\leq b$、$x\leq y$,则$pr_{(a, x)}u'=f_{ab}^{xy}(pr_{(b, y)}u')$,令$u=\\(pr_{I\times \{x\}}u')_{x\in L}$,因此$pr_a(pr_xu)=f_{ab}^{xy}(pr_b(pr_yu))$,因此,对任意$a\in I$,$pr_a(pr_xu)=\\pr_a(g^{xy}(pr_yu))$,故$pr_xu= g^{xy}(pr_yu)$,因此$u\in F$.故$f$的限制为双射.
			\par
			同理可证$g$的限制为双射.
			
			\begin{cor}\label{cor421}
				\hfill\par
				$I$、$L$均为预序集,$I\times L$的预序关系为$(x\in I\times L\text{与}y\in I\times L\text{与}pr_1x\leq pr_1y\text{与}pr_2x\leq pr_2y)$,$(E_a^x, f_{ab}^{xy})$、$({E'}_{ax}, {f'}_{ab}^{xy})$均为关于$I\times L$的集合射影系统,对于$(a, b)\in I\times L$,$u_a^x$为$E_a^x$到${E'}_a^x$\\的映射,并且$(u_a^x)_{(a, x)\in I\times L}$为$(E_a^x, f_{ab}^{xy})$到$({E'}_a^x, {f'}_{ab}^{xy})$的映射射影系统,则$(u_a^x)_{a\in I}$、$(u_a^x)_{x\in L}$均为\\$(E_a^x, f_{ab}^{xy})$到$({E'}_a^x, {f'}_{ab}^{xy})$的映射射影系统,$(\lim\limits_{\gets a}u_a^x)_{x\in L}$为$(\lim\limits_{\gets a}E_a^x)$到$(\lim\limits_{\gets a}{E'}_a^x)$的映射射影系统,\\$(\lim\limits_{\gets x}u_a^x)_{a\in I}$为$(\lim\limits_{\gets x}E_a^x)$到$(\lim\limits_{\gets x}{E'}_a^x)$的映射射影系统.
			\end{cor}
			证明:
			\par
			根据定义可证$(u_a^x)_{a\in I}$、$(u_a^x)_{x\in L}$均为$(E_a^x, f_{ab}^{xy})$到$({E'}_a^x, {f'}_{ab}^{xy})$的映射射影系统.
			\par
			令$u^x=\lim\limits_{\gets a}u_a^x$,$g^{xy}=\lim\limits_{\gets a}f_{aa}^{xy}$,${g'}^{xy}=\lim\limits_{\gets a}{f'}_{aa}^{xy}$,则当$x\leq y$时,$u_a^x\circ f_{aa}^{xy}={f'}_{aa}^{xy}\circ u_a^y$,根据定理\ref{theo174},$u^x\circ g^{xy}={g'}^{xy}\circ u^y$,因此$(\lim\limits_{\gets a}u_a^x)_{x\in L}$为$(\lim\limits_{\gets a}E_a^x)$到$(\lim\limits_{\gets a}{E'}_a^x)$的映射射影系统,$(\lim\limits_{\gets x}u_a^x)_{a\in I}$为$(\lim\limits_{\gets x}E_a^x)$到$(\lim\limits_{\gets x}{E'}_a^x)$的映射射影系统.
			
			\begin{de}
				\textbf{映射族的双重射影极限(double limite projective de famille \\d'applications)}
				\par
				$I$、$L$均为预序集,$I\times L$的预序关系为$(x\in I\times L\text{与}y\in I\times L\text{与}pr_1x\leq pr_1y\text{与}pr_2x\leq pr_2y)$,$(E_a^x, f_{ab}^{xy})$、$({E'}_{ax}, {f'}_{ab}^{xy})$均为关于$I\times L$的集合射影系统,对于$(a, b)\in I\times L$,$u_a^x$为$E_a^x$到${E'}_a^x$的映射,并且$(u_a^x)_{(a, x)\in I\times L}$为$(E_a^x, f_{ab}^{xy})$到$({E'}_a^x, {f'}_{ab}^{xy})$的映射射影系统,则其射影极限称双重射影极限,记作$\lim\limits_{\gets a, x}u_a^x$.
			\end{de}
					
			\begin{theo}\label{theo179}
				\hfill\par
					$I$、$L$为预序集,$I\times L$的预序关系为$(x\in I\times L\text{与}y\in I\times L\text{与}pr_1x\leq pr_1y\text{与}pr_2x\leq pr_2y)$,$(E_a^x, f_{ab}^{xy})$、$({E'}_{ax}, {f'}_{ab}^{xy})$均为关于$I\times L$的集合射影系统,对于$(a, b)\in I\times L$,$u_a^x$为$E_a^x$到${E'}_a^x$的映射,并且$(u_a^x)_{(a, x)\in I\times L}$为$(E_a^x, f_{ab}^{xy})$到$({E'}_a^x, {f'}_{ab}^{xy})$的映射射影系统,$u_a=\lim\limits_{\gets x}u_a^x$,$u^x=\lim\limits_{\gets a}u_a^x$,$u=\lim\limits_{\gets a, x}u_a^x$.令$\prod\limits_{(a, x)\in I\times L}E_a^x$到$\prod\limits_{x\in L}(\prod\limits_{a\in I}E_a^x)$的规范映射为$f$,$\prod\limits_{(a, x)\in I\times L}{E'}_a^x$到$\prod\limits_{x\in L}(\prod\limits_{a\in I}{E'}_a^x)$的规范映射为$f'$,$\prod\limits_{(a, x)\in I\times L}E_a^x$到$\prod\limits_{a\in I}(\prod\limits_{x\in L}E_a^x)$的规范映射为$g$,$\prod\limits_{(a, x)\in I\times L}{E'}_a^x$到$\prod\limits_{a\in I}(\prod\limits_{x\in L}{E'}_a^x)$的规范映射为$g'$.则$\lim\limits_{\gets a, x}u_a^x={f'}^{-1}\circ \lim\limits_{\gets x}(\lim\limits_{\gets a}u_a^x)\circ f$,$\lim\limits_{\gets a, x}u_a^x={g'}^{-1}\circ \lim\limits_{\gets a}(\lim\limits_{\gets x}u_a^x)\circ g$.
			\end{theo}
			证明:根据定理\ref{theo178}可证.
					
			\begin{theo}\label{theo180}
				\hfill\par
				$I$为预序集,$(E_a^x, f_{ab}^x)_{x\in L}$为关于$I$的集合射影系统族,${f'}_{ab}$是映射族$(f_{ab}^x)_{x\in L}$在乘积上的规范扩展,则$(\prod\limits_{x\in L}E_a^x, {f'}_{ab})$是关于$I$的集合射影系统,且${pr_{\{a\}\times L}}^{-1}(\lim\limits_{\gets a}\prod\limits_{x\in L}E_a^x)=\\{pr_{I\times \{x\}}}^{-1}(\prod\limits_{x\in L}\lim\limits_{\gets a}E_a^x)$.
			\end{theo}
			证明:令$L$为按$(x=y\text{与}x\in L)$排序的预序集,令$g_{ab}^{xy}=f_{ab}^x$,根据补充定理\ref{cor420}、补充定理\ref{cor412},$(\prod\limits_{x\in L}E_a^x, {f'}_{ab})$是关于$I$的集合射影系统,并且,根据定理\ref{theo178}、补充定理\ref{cor412}可证\\${pr_{\{a\}\times L}}^{-1}(\lim\limits_{\gets a}\prod\limits_{x\in L}E_a^x)={pr_{I\times \{x\}}}^{-1}(\prod\limits_{x\in L}\lim\limits_{\gets a}E_a^x)$.
					
			\begin{theo}\label{theo181}
				\hfill\par
				$I$为预序集,且为右方有向集,并有一个可数共尾子集,$(E_a, f_{ab})$为关于$I$的集合射影系统,对任意$a\in I\text{与}b\in I\text{与}a\leq b$,$f_{ab}$为满射,令$E=\lim\limits_\gets E_a$.对任意$a\in I$,令$f_a$为$E$到$E_a$的规范映射,则$f_a$为满射.进而,对任意$a\in I$,$E_a\neq \varnothing$,则$E\neq \varnothing$.
			\end{theo}
			证明:
			令$(a_n)$为$I$的元素序列,并且各项构成$I$的可数共尾子集,则递归构建$a$的元素序列$(b_n)$:令$b_0=a_0$,$b_n=sup(a_n, b_{n-1})$,则$(b_n)$为单增序列,并且各项构成$I$的共尾子集.对任意$x_{b_0}\in E_{b_0}$,由数学归纳法可知存在$x_{b_n}\in E_{b_n}$,使对任意$m<n$,$x_{b_m}=f_{b_mb_n}(x_{b_n})$,对任意$a<b_n$,令$x_a=f_{ab_n}(x_{b_n})$,则$x_a\in E$.故$f_{b_0}$为满射.同理可证,对任意$n\in N$,$f_{b_n}$为满射,进而,当$a\leq b_n$时,$f_a$为满射.因此,对任意$a\in I$,$E_a\neq \varnothing$,则$E\neq \varnothing$.
					
			\begin{theo}\label{theo182}
				\hfill\par
				$I$为预序集,且为右方有向集,$(E_a, f_{ab})$为关于$I$的集合射影系统,对任意$a\in I$,令$F_a$的元素都是$E_a$的子集,并且:
				\par
				第一,$F_a$的任何元素的交集也是$F_a$的元素;
				\par
				第二,如果$G\subset F_a$,并且$G$中任何有限个元素的交集都不是空集(或等价公式:$F_a$为按包含关系排序的偏序集,$G$为左方有向集,且其元素都不是空集),
				\par
				则$\bigcap\limits_{M\in F}G$不是空集.
				\par
				同时,对任意$a$、$b$,如果满足$a\in I\text{与}b\in I\text{与}a\leq b$,均有:
				\par
				第一,对任意$x_a\in E_a$,${f_{ab}}^{-1}(x_a)\in F_b$;
				\par
				第二,对任意$M_b\in F_b$,$f_{ab}(M_b)\in F_a$.令$E=\lim\limits_\gets E_a$,并且对任意$a\in I$,令$f_a$为$E$到$E_a$\\的规范映射,
				\par
				则:
				\par
				第一,对任意$a\in I$,$f_a(E)=\bigcap\limits_{b\geq a\text{与}b\in I}f_{ab}(E_b)$;
				\par
				第二,对任意$a\in I$,$E_a\neq \varnothing$,则$E\neq \varnothing$.
			\end{theo}
			证明:令$S$为符合下列条件的集族$(A_a)_{a\in I}$的集合:对任意$a\in I$,$A_a\neq \varnothing$,$A_a\in F_a$,并且对任意$a$、$b$满足$a\in I\text{与}b\in I\text{与}a\leq b$,$f_{ab}(A_b)\subset A_a$.令$R$为$M\in S\text{与}N\in S\text{与}N\subset M$,则$E$为按$R$排序的偏序集.令$L$为$S$的全序子集,$L_a=\{x|(\exists A)(\exists a)(A\in L\text{与}a\in I\text{与}(a, x)\in A)\}$.$B_a=\bigcap\limits_{x\in L}a_x$.则$(B_a)_{a\in I}\in S$,故$S$为归纳集.
			\par
			设$S$的极大元为$A$,令${A'}_a=\bigcap\limits_{b\geq a\text{与}b\in I}f_{ab}(A_b)$.对任意$a\leq b$,$b\leq c$,$f_{ac}(A_c)=f_{ab}(f_{bc}(A_c)$,因此$f_{ac}(A_c)\subset f_{ab}(A_b)$.如果$a\leq b$,$f_{ab}({A'}_b)\subset \bigcap\limits_{c\geq a\text{与}c\in I}f_{ac}({A'}_c)$,另一方面,对任意$d\geq b$,存在$c\geq d$且$c\geq b$,使$f_{ac}(A_c)\subset f_{ad}(A_d)$,故$\bigcap\limits_{c\geq a\text{与}c\in I}f_{ac}({A'}_c)\subset {A'}_a$,因此$f_{ab}({A'}_b)\subset {A'}_a$.又因为$f_{ab}(A_b)\in F$a,$f_{ab}(A_b)\neq \varnothing$,故$A'\in S$,同时,$A'\subset A$,所以$A=A'$,故对任意$a$、$b$满足$a\in I\text{与}b\in I\text{与}a\leq b$,$f_{ab}(A_b)=A_a$.
			\par
			设$x_a\in A_a$,当$b\geq a$时,令$B_b=A_b\cap {f_{ab}}^{-1}(x_a)$,当$b<a$时,令$B_b=A_b$,如果$b<a$,对任意$c\geq b$,$f_{bc}(B_c)\subset f_{bc}(A_c)$,因此$f_{b_c}(B_c)\subset B_b$;当$b\geq a$时,对任意$c\geq b$,${f_{ac}}^{-1}(x_a)=({f_{bc}}^{-1}({f_{ab}}^{-1}(x_a)))$,因此$f_{bc}({f_{ac}}^{-1}(x_a))\subset {f_{ab}}^{-1}(x_a)$,又因为$f_{bc}(A_c)\subset A_b$,故$f_{bc}(B_c)\subset B_b$;同时,由于$f_{ab}(A_b)=A_a$,故$A_b\cap {f_{ab}}^{-1}(x_a)\neq \varnothing$;此外,由于${f_{ab}}^{-1}(x_a)\in F_b$,$A_b\in F_b$,$B_b\in F_b$.综上,$B\in S$,因此$A=B$,故$A_a=\{x_a\}$,且其对一切$a\in I$都成立.
			\par
			根据补充定理\ref{cor411},$f_a(E)\subset \bigcap\limits_{b\geq a\text{与}b\in I}f_{ab}(E_b)$.另一方面,设$x_a\in \bigcap\limits_{b\geq a\text{与}b\in I}f_{ab}(E_b)$,当$b\geq a$时,令$B_b={f_{ab}}^{-1}(x_a)$,当$b<a$时,令$B_b=E_b$.因此$B_b\neq \varnothing$,并且$b\leq c$时,$f_{bc}(B_c)\subset B_b$,故$B\in S$.根据定理\ref{theo81},$S$有极大元$A$,满足$A\geq B$,设$A_a=\{y_a\}$,令$y=(y_a)$,则$y\in E$.故$f_a(y)=x_a$,因此,$f_a(E)=\bigcap\limits_{b\geq a\text{与}b\in I}f_{ab}(E_b)$.
			如果$I=\varnothing$,则$E=\{\varnothing\}$,故$E\neq \varnothing$.
			如果$I\neq \varnothing$,则当$a\leq b$时,$f_{ab}(E_b)\neq \varnothing$.由于$b\leq c$时,$f_{ab}(E_b)\subset f_{ac}(E_c)$,因此$\bigcap\limits_{b\geq a\text{与}b\in I}f_{ab}(E_b)\neq \varnothing$,故$f_a(E)\neq \varnothing$,因此$E\neq \varnothing$.
			
			\begin{cor}\label{cor422}
				\textbf{集合归纳系统涉及的公式为等价关系}
				\par
				$I$为右方有向集,$(E_a)_{a\in I}$为集族,$(f_{ba})_{a\in I\text{与}b\in I\text{与}a\leq b}$为函数族,其中$f_{ba}$为$E_a$到$E_b$的映射,并且满足下列条件:
				\par
				第一,如果$a\leq b$,$b\leq c$,则$f_{ba}\circ f_{cb}=f_{ca}$.
				\par
				第二,$f_{aa}=Id_{E_a}$.
				\par
				令$G$为集族$(E_a)_{a\in I}$的和,令$R$为公式$x\in G\text{与}y\in G \text{与}(\exists c)(c\in I\text{与}c\geq pr_2x\text{与}c\geq pr_2y\text{与}\\f_{c\ pr_2x}(pr_1x)= f_{c\ pr_2y}(pr_1y)$,则$R$为在$G$上的等价关系.
			\end{cor}
			证明:显然$R$具有反身性和对称性.
			\par
			设$u\in G$、$v\in G$、$w\in G$,令$u=(x, a)$,$v=(y, b)$,$w=(z, c)$,则$x\in E_a$,$y\in E_b$,$z\in E_c$,设$l\geq a$、$l\geq b$,$f_{la}(x)=f_{lb}(y)$,$m\geq b$,$m\geq c$,$f_{mb}(y)=f_{mc}(z)$.由于$I$为右方有向集,则存在$n\geq l$、$n\geq m$,故$f_{na}(x)=f_{nc}(z)$.因此$R$具有传递性.
			
			\begin{de}
				\textbf{集合归纳系统(système inductif d'ensembles),集族的归纳极限(limite inductive de famille d'ensembles),集合到集族的归纳极限的规范映射(application canonique de la limite inductif d'un ensemble dans la famille d'ensembles)}
				\par
				$I$为右方有向集,$(E_a)_{a\in I}$为集族,$(f_{ba})_{a\in I\text{与}b\in I\text{与}a\leq b}$为函数族,其中$f_{ba}$为$E_a$到$E_b$的映射,并且满足下列条件:
				\par
				第一,如果$a\leq b$,$b\leq c$,则$f_{ba}\circ f_{cb}=f_{ca}$.
				\par
				第二,$f_{aa}=Id_{E_a}$,
				则$(E_a)_{a\in I}$, $(f_{ba})_{a\in I\text{与}b\in I\text{与}a\leq b)}$称为关于$I$的集合归纳系统,在没有歧义的情况下可以简记为$((E_a), (f_{ba}))$或$(E_a, f_{ba})$.
				\par
				令$G$为集族$(E_a)_{a\in I}$的和,令$R$为等价关系$x\in G\text{与}y\in G \text{与}(\exists c)(c\in I\text{与}c\geq pr_2x\text{与}c\geq pr_2y\text{与}f_{c\ pr_2x}(pr_1x)= f_{c\ pr_2y}(pr_1y)$,则称商集$G/R$为集族$(E_a)_{a\in I}$对于函数族$(f_{ba})_{a\in I\text{与}b\in I\text{与}a\leq b}$\\的归纳极限,记作$\lim\limits_{\to a}(E_a, f_{ba})$,在没有歧义的情况下可以简记为$\lim\limits_\to (E_a, f_{ba})$或$\lim\limits_\to E_a$.令$f$为\\$G$到$E/R$的规范映射,$g_a$为映射$x\mapsto (x, a)(x\in E_a)$,则映射$f\circ g_a$称为$E_a$到$\lim\limits_\to E_a$的规范映射.
			\end{de}
			
			\begin{cor}\label{cor423}
				\hfill\par
				$I$为右方有向集,$((E_a)_{a\in I}, (f_{ba})_{a\in I\text{与}b\in I\text{与}a\leq b)}$为关于$I$的集合归纳系统,$E=\lim\limits_\to (E_a, f_{ba})$,对任意$a\in I$,令$E_a$到$E$的规范映射为$f_a$,则当$a\in I$、$b\in I$、$a\leq b$时,$f_a=f_b\circ f_{ba}$.
			\end{cor}
			证明:对任意$x\in E_a$,$f_{bb}(f_{ba}(x))=f_{ba}(x)$.由于$f_{ba}(x)\in E_b$,故$f_b(f_{ba}(x))=f_a(x)$,得证.
			
			\begin{cor}\label{cor424}
				\hfill\par
				$I$为右方有向集,$((E_a)_{a\in I}, (f_{ba})_{a\in I\text{与}b\in I\text{与}a\leq b)}$为关于$I$的集合归纳系统,$E=\lim\limits_\to (E_a, f_{ba})$,对任意$a\in I$,令$E_a$到$E$的规范映射为$f_a$,则:
				\par
				(1)对任意$x\in E$,存在$a\in I$、$z\in E_a$,使$f_a(z)=x$.
				\par
				(2)$E=\bigcup\limits_{a\in I}f_a\langle E_a\rangle$.
			\end{cor}
			证明:
			\par
			(1)根据补充证明规则\ref{Ccor35},$G$到$E$的规范映射$f$为满射,因此存在$y\in G$,使$f(y)=x$.令$y=(z, a)$,则$a\in I$、$z\in E_a$,令$g_a$为映射$x\mapsto (x, a)(x\in E_a)$,则$g_a(z) =y$,故$f(g_a(z))=x$,得证.
			\par
			(2)对任意$x\in E_a$,$f_a(x)\in E$;反过来,对任意$x\in E$,根据补充定理\ref{cor424}(1),存在$a\in I$、$z\in E_a$,使$f_a(z)=x$.得证.
			
			\begin{cor}\label{cor425}
				\hfill\par
				$((E_a)_{a\in I}, (f_{ba})_{a\in I\text{与}b\in I\text{与}a\leq b})$为关于$I$的集合归纳系统,$E=\lim\limits_\to (E_a, f_{ba})$,对任意$a\in I$,令$E_a$到$E$的规范映射为$f_a$,则对任意$a\in I$、$x\in E$、$y\in E$,如果$f_a(x)=f_a (y)$,则$(\exists b)(b\in I\text{与}b\geq a\text{与}f_{ba}(x)= f_{ba}(y))$.
			\end{cor}
			证明:令$g_a$为映射$x\mapsto (x, a)(x\in E_a)$,则$f(g_a(x))=f(g_a(y))$,根据证明规则\ref{C55},\\$(\exists b)(b\in I\text{与}b\geq a\text{与}f_{ba}(x)= f_{ba}(y))$.
			
			\begin{cor}\label{cor426}
				\hfill\par
				关于$\varnothing$的集合归纳系统,其归纳极限为$\varnothing$.
			\end{cor}
			证明:根据补充证明规则\ref{Ccor34}(1)可证.
					
			\begin{theo}\label{theo183}
				\hfill\par
				$I$为右方有向集,$(E_a, f_{ba})$为关于$I$的集合归纳系统,$E= \lim\limits_\to E_a$,对任意$a\in I$,$E_a$到$E$\\的规范映射为$f_a$,则:
				\par
				(1)令$n$为自然数,$(x^{(i)})_{i\in [1, n]}$为E的有限元素族,则存在$a\in I$,使$(x^{(i)}_a)_{i\in [1, n]}$为$E_a$的有限元素族,并且,当$i\in [1, n]$时,$x^{(i)}=f_a(x^{(i)}_a)$.
				\par
				(2)令$n$为自然数,$(y^{(i)})_{i\in [1, n]}$为E的有限元素族,如果对任意$i\in [1, n]$、$j\in [1, n]$,均有$f_a(y^{(i)}_a)=fa(y^{(j)}_a)$,则存在$b\geq a$,使得对任意$i\in [1, n]$、$j\in [1, n]$,均有$f_{ba}(y(i)a)=f_{ba}(y(j)a)$.
			\end{theo}
			证明:
			\par
			(1)根据补充定理\ref{cor424}(1),对任意$i\in [1, n]$,存在$b_i\in I$及$z\in E_{b_i}$,使$x^{(i)}=f_{b_i}(z_{b_i})$,取$a\geq b_i$(对任意$i\in [1, n]$),则$x^{(i)}_a=f_{a b_i}(z_{b_i})$,根据补充定理\ref{cor423}可证.
			\par
			(2)根据补充定理\ref{cor425},对任意$i\in [1, n]$、$j\in [1, n]$,存在$c_{ij}\geq a$,使$f_{c_{ij} a}(y^{(i)}_a)=f_{c_{ij} a}(y^{(j)}_a)$,进而,存在$b$,使对任意$i\in [1, n]$、$j\in [1, n]$均有$b\geq c_{ij}$,由于$f_{ba} =f_{b c_{ij}}\circ f_{c_{ij} a}$ ,因此$f_{ba}(y^{(i)}_a)=f_{ba}(y^{(j)}_a)$.
					
			\begin{theo}\label{theo184}
				\textbf{集族到集合的映射族的归纳极限的性质}\par
				$I$为右方有向集,$(E_a, f_{ba})$为关于$I$的集合归纳系统,$E= \lim\limits_\to E_a$,对任意$a\in I$,$E_a$到$E$\\的规范映射为$f_a$,对任意$a\in I$,令$u_a$为$E_a$到$F$的映射,并且$(\forall a)(\forall b)(a\in I\text{与}b\in I\text{与}a\leq b\Rightarrow u_b\circ f_{ba}=u_a)$,则:
				\par
				(1)存在唯一的$E$到$F$的映射$u$,使$(\forall a)(a\in I\Rightarrow u_a=u\circ f_a)$;
				\par
				(2)当且仅当$F=\bigcup\limits_{a\in I}u_a(E_a)$时,$u$为满射;
				\par
				(3)当且仅当$(\forall a)(a\in I\text{与}x\in E_a\text{与}y\in E_a\text{与}u_a(x)=u_a(y)\Rightarrow (\exists b)(b\geq a\text{与}f_{ba}(x)=f_{ba}(y)))$时,$u$为单射.
			\end{theo}
			证明:
			\par
			(1)令$G$到$E$的规范映射为$f$,$g_a$为映射$x\mapsto (x, a)(x\in E_a)$,$F_a=g_a\langle E_a\rangle$,则 $v_a=u_a\circ {g_a}^{-1}$,故$v_a$为$F_a$到$F$的映射,令$v$为$G$到$F$的映射,且对任意$a\in I$,$v$在$F_a$上与$v_a$重合.令$R$为等价关系$x\in G\text{与}y\in G \text{与}(\exists c)(c\in I\text{与}c\geq pr_2x\text{与}c\geq pr_2y\text{与}f_{c\ pr_2x}(pr_1x)= f_{c\ pr_2y}(pr_1y)$,设$x\in F_a$,$y\in F_b$,如果$R$为真,则存在$c$,使$f_{ca}(pr_1x)=f_{cb}(pr_1y)$,由于$v_a=u_a\circ {g_a}^{-1}$,故$v_a(x)=u_a(pr_1x)$,因此$v(x)=u_c(f_{ca}(pr_1x))$,同理$v(y)=u_c(f_{cb}(pr_1y))$,因此$R\Rightarrow v(x)=v(y)$,因此$v$是同等价关系相容的映射,根据证明规则\ref{C57},存在唯一的映射$u$,使$u\circ f=v$.由于$u_a=v\circ g_a$,$f_a=f\circ g_a$,因此$u_a=u\circ f_a$,得证.
			\par
			(2)当$F=\bigcup\limits_{a\in I}u_a(E_a)$时,对任意$x\in F$,存在$a\in I$、$z\in E_a$,使$x=u_a(z)$,因此$x=u(f_a(z))$,故$u$为满射.反过来,若$u$为满射,则对任意$x\in F$,存在$y\in E$使$x=u(y)$,根据补充定理\ref{cor424}(1),存在$a\in I$、$z\in E_a$,使$y=f_a(z)$,故$x=u_a(z)$,因此$x\in u_a(E_a)$.
			\par
			另一方面,对任意$x\in u_a(E_a)$,存在$z\in E_a$,使$x=u_a(z)$,因此$x=u(f_a(z))$,故$x\in F$;综上,$F=\bigcup\limits_{a\in I}u_a(E_a)$.
			\par
			(3)假设$(\forall a)(a\in I\text{与}x\in E_a\text{与}y\in E_a\text{与}u_a(x)=u_a(y)\Rightarrow (\exists b)(b\geq a\text{与}f_{ba}(x)=f_{ba}(y)))$,如果$u(x)=u(y)$,且$x\in E$、$y\in E$,则存在设$a\in I$、$b\in I$、$m\in E_a$、$n\in E_a$,使$f_a(m)=x$、$f_b(n)=y$,因此$u_a(m)=u_b(n)$,故存在$c\in I$,使$c\geq a$,$c\geq b$,且$u_c(f_{ca}(m))=u_c(f_{cb}(n))$,因此存在$d\geq c$,使$f_{da}(m)=f_{db}(n)$.根据补充定理\ref{cor423},$x=\\f_d(f_{da}(m))$,$y=f_d(f_{db}(n))$,因此$x=y$.
			\par
			反过来,如果$u$为单射,且$a\in I\text{与}x\in E_a\text{与}y\in E_a\text{与}u_a(x)=u_a(y)$,则$u(f_a(x))=u(f_a(y))$,则$f_a(x)= f_a(y)$,由于$f_a(x)\in E$,$f_a(y) \in E$,根据补充定理\ref{cor425},$(\exists b)(b\geq a\text{与}f_{ba}(x)=f_{ba}(y))$,得证.
			
			\begin{de}
				\textbf{集族到集合的映射归纳系统(système inductif d'applications da famille d'ensembles dans un ensemble),集族到集合的映射族的归纳极限(limite inductive de famille d'applications da famille d'ensembles dans un ensemble)}
				\par
				$I$为右方有向集,$(E_a, f_{ba})$为关于$I$的集合归纳系统,$E= \lim\limits_\to E_a$,对任意$a\in I$,$E_a$到$E$\\的规范映射为$f_a$,对任意$a\in I$,令$u_a$为$E_a$到$F$的映射,并且$(\forall a)(\forall b)(a\in I\text{与}b\in I\text{与}a\leq b\Rightarrow u_b\circ f_{ba}=u_a)$,则称映射族$(u_a)_{a\in I}$为$(E_a, f_{ba})$到$F$的映射归纳系统.如果$E$到$F$的映射$u$,使$(\forall a)(a\in I\Rightarrow u_a=u\circ f_a)$,则称$u$为映射族$(u_a)_{a\in I}$的归纳极限,记作$\lim\limits_\to u_a$.
			\end{de}
			
			\begin{cor}\label{cor427}
				\hfill\par
				$I$为右方有向集,$(E_a, f_{ba})$为关于$I$的集合归纳系统,$E= \lim\limits_\to E_a$,对任意$a\in I$,$E_a$到$E$\\的规范映射为$f_a$,则$(f_a)_{a\in I}$为$(E_a, f_{ba})$到$E$的映射归纳系统,且$\lim\limits_\to f_a=Id_E$.
			\end{cor}
			证明:根据补充定理\ref{cor423}可证.
					
			\begin{theo}\label{theo185}
				\textbf{集族之间的映射族的归纳极限的存在性和唯一性}
				\par
				$I$为右方有向集,$(E_a, f_{ba})$、$(F_a, g_{ba})$均为关于$I$的集合归纳系统,$E= \lim\limits_\to E_a$,\\$F= \lim\limits_\to F_a$,对任意$a\in I$,$E_a$到$E$的规范映射为$f_a$,$F_a$到$F$的规范映射为$g_a$.对任意$a\in I$,令$u_a$为$E_a$到$F_a$的映射,并且$(\forall a)(\forall b)(a\in I\text{与}b\in I\text{与}a\leq b\Rightarrow u_b\circ g_{ba}=f_{ba}\circ u_a)$,则存在唯一的$E$到$F$的映射$u$,使$(\forall a)(a\in I\Rightarrow u\circ f_a=g_a\circ u_a)$.
			\end{theo}
			证明:令$v_a=g_a\circ u_a$,则$v_b\circ f_{ba}=g_b\circ u_b\circ f_{ba}$,等于$g_b\circ g_{ba}\circ u_a$,等于$g_a\circ u_a$,等于$v_a$.根据定理\ref{theo184}(1),存在唯一的$u$,使$(\forall a)(a\in I\Rightarrow v_a=u\circ f_a)$.
			
			\begin{de}
				\textbf{集族之间的映射归纳系统(système inductif d'applications entre familles d'ensembles),集族之间的映射族的归纳极限(limite inductive de famille \\d'applications entre familles d'ensembles)}
				\par
				$I$为右方有向集,$(E_a, f_{ba})$、$(F_a, g_{ba})$均为关于$I$的集合归纳系统,$E= \lim\limits_\to E_a$,\\$F= \lim\limits_\to F_a$,对任意$a\in I$,$E_a$到$E$的规范映射为$f_a$,$F_a$到$F$的规范映射为$g_a$.对任意$a\in I$,令$u_a$为$E_a$到$F_a$的映射,并且$(\forall a)(\forall b)(a\in I\text{与}b\in I\text{与}a\leq b\Rightarrow u_b\circ g_{ba}=f_{ba}\circ u_a)$,则称映射族$(u_a)_{a\in I}$为$(E_a, f_{ba})$到$(F_a, g_{ba})$的映射归纳系统.如果$E$到$F$的映射$u$使$(\forall a)(a\in I\Rightarrow u\circ f_a=g_a\circ u_a)$,则称$u$为映射族$(u_a)_{a\in I}$的归纳极限,记作$\lim\limits_{\to a} u_a$,在没有歧义的情况下也可以简记为$\lim\limits_\to u_a$.
			\end{de}
					
			\begin{theo}\label{theo186}
				\hfill\par
				$I$为右方有向集,$(E_a, f_{ba})$、$(F_a, g_{ba})$、$(G_a, h_{ba})$均为关于$I$的集合归纳系统,$E= \lim\limits_\to E_a$,$F= \lim\limits_\to F_a$,$G= \lim\limits_\to G_a$,对任意$a\in I$,$E_a$到$E$的规范映射为$f_a$,$F_a$到$F$的规范映射为$g_a$,$G_a$到$G$的规范映射为$h_a$.对任意$a\in I$,令$u_a$为$E_a$到$F_a$的映射,$v_a$为$F_a$到$G_a$的映射,则映射族$(v_a\circ u_a)_{a\in I}$为$(E_a, f_{ba})$到$(G_a, h_{ba}$的映射归纳系统,并且$\lim\limits_\to (v_a\circ u_a)=\lim\limits_\to (v_a\circ (\lim\limits_\to u_a))$.
			\end{theo}
			证明:令$w_a=v_a\circ u_a$.则$h_{ab}\circ w_a=h_{ab}\circ v_a\circ u_a$,等于$v_b\circ g_{ba}\circ u_a$,等于$w_b\circ f_{ba}$,映射族$(v_a\circ u_a)_{a\in I}$为$(E_a, f_{ba})$到$(G_a, h_{ba}$的映射归纳系统.同时$v\circ u\circ f_a=v\circ g_a\circ u_a$,等于$h_a\circ v_a\circ u_a$,得证.
					
			\begin{theo}\label{theo187}
				\hfill\par
				$I$为右方有向集,$(E_a, f_{ba})$、$({E'}_a, {f'}_{ba})$均为关于$I$的集合归纳系统,对任意$a\in I$,$u_a$为$E_a$\\到${E'}_a$的映射,则映射族$(u_a)_{a\in I}$为映射归纳系统.令$u=\lim\limits_\to u_a$,如果对任意$a\in I$,$u_a$为单射(或满射),则$u$是单射(或满射).
			\end{theo}
			证明:令$f_a$为$E_a$到$E$的规范映射,${f'}_a$为${E'}_a$到$E'$的规范映射.
			\par
			如果对任意$a\in I$,$u_a$为单射,令$x\in E_a$,$y\in E_a$,且${f'}_a(u_a(x))={f'}_a(u_a(y))$,根据定理\ref{theo183}(2),存在$b\geq a$,使${f'}_{ba}(u_a(x))={f'}_{ba}(u_a(y))$,故$u_b(f_{ba}(x))=u_b(f_{ba}(y))$,因此$f_{ba}(x)=f_{ba}(y)$,根据定理\ref{theo184}(3),$u$是单射.
			\par
			如果对任意$a\in I$,$u_a$为满射,根据补充定理\ref{cor424}(2),$E'=\bigcup\limits_{a\in I}{f'}_a\langle{E'}_a\rangle$.因此$E=\\\bigcup\limits_{a\in I}u\langle f_a\langle E_a\rangle\rangle$,故$E=u\bigcup\limits_{a\in I}f_a\langle E_a\rangle$,因此$E'=u(E)$,故$u$是满射.
			
			\begin{cor}\label{cor428}
				\textbf{子集上的系统为归纳系统}
				\par
				$I$为右方有向集,$(E_a, f_{ba})$为关于$I$的集合归纳系统,$E= \lim\limits_\to E_a$,对任意$a\in I$,$M_a\subset E_a$,如果$(\forall a)(\forall b)(a\in I\text{与}b\in I\text{与}a\leq b\Rightarrow f_{ba}(M_a)\subset M_b)$,当$a\in I\text{与}b\in I\text{与}a\leq b$时,令$g_{ba}$为$f_{ba}$在$M_a$上的限制,则$(M_a, g_{ba})$也是关于$I$的集合归纳系统,并且,对任意$a\in I$,令$j_a$为$M_a$到$E_a$的规范映射,并且,$\lim\limits_\to j_a$为$\lim\limits_\to M_a$到$E$的单射.
			\end{cor}
			证明:根据定义可证$(M_a, g_{ba})$为关于$I$的集合归纳系统.根据定理\ref{theo187}可证,$\lim\limits_\to j_a$为\\$\lim\limits_\to M_a$到$E$的单射.
			
			\begin{de}
				\textbf{子集归纳系统(système inductif de parties)}
				\par
				$I$为右方有向集,$(E_a, f_{ba})$为关于$I$的集合归纳系统,对任意$a\in I$,$M_a\subset E_a$,如果\\$(\forall a)(\forall b)(a\in I\text{与}b\in I\text{与}a\leq b\Rightarrow f_{ba}(M_a)\subset M_b)$,则称$(M_a)_{a\in I}$为$(E_a)_{a\in I}$的子集归纳系统.
			\end{de}
					
			\begin{theo}\label{theo188}
				\hfill\par
				$I$为右方有向集,$(E_a, f_{ba})$、$({E'}_a, {f'}_{ba})$均为关于$I$的集合归纳系统,对任意$a\in I$,$u_a$为$E_a$\\到${E'}_a$的映射,令$u=\lim\limits_\to u_a$,则:
				\par
				(1)$(M_a)_{a\in I}$为$(E_a)_{a\in I}$的子集归纳系统,则$(u_a (M_a))_{a\in I}$为$({E'}_a)_{a\in I}$的子集归纳系统,并且,$\lim\limits_\to u_a(M_a)=u(\lim\limits_\to M_a)$.
				\par
				(2)$I\neq \varnothing$,$({x'}_a)_{a\in I}$为族,对任意$a\in I$,${x'}_a\in {E'}_a$,当$a\in I\text{与}b\in I\text{与}a\leq b$时,$f_{ba}({x'}_a)={x'}_b$,则${u_a}^{-1}({x'}_a)$为$({E'}_a)_{a\in I}$的子集归纳系统,则存在唯一的$x'\in \lim\limits_\to {E'}_a$,使对任意$a\in I$,$x'={f'}_a({x'}_a)$,并且$\lim\limits_\to {u_a}^{-1}({x'}_a)=u^{-1}(x')$,其中${f'}_a$为$E$到$\lim\limits_\to {E'}_a$的规范映射.				
			\end{theo}
			证明:
			\par
			(1)根据定义,$(u_a (M_a))_{a\in I}$为$({E'}_a)_{a\in I}$的子集归纳系统.令$v_a$为$u_a$通过$E_a$的子集$M_a$和\\${E'}_a$的子集$u_a(M_a)$导出的函数,则$v_a$为满射,根据定理\ref{theo187}可以证明$\lim\limits_\to u_a(M_a)=u(\lim\limits_\to M_a)$.
			\par
			(2)对任意$a\in I$,$b\in I$,$a\leq b$,${f'}_a({x'}_a)={f'}_b({f'}_{ba}({x'}_a))$,因此${f'}_a({x'}_a)={f'}_b({x'}_b)$,令其为$x'$,故$x' \in \lim\limits_\to {E'}_a$.令$N_a={u_a}^{-1}({x'}_a)$,如果$x_a\in N_a$,且$b\geq a$,则${x'}_b={f'}_ba({x'}_a)$,等于${f'}_ba({u'}_a(x_a))$,等于$u_b(f_{ba}(x_a))$,因此$f_{ba}(x_a)\in N_b$,故$(N_a)_{a\in I}$为$(E_a)_{a\in I}$的子集归纳系统.对任意$x\in \lim\limits_\to N_a$,存在$a\in I$、$x_a\in N_a$,使$x=f_a(x_a)$,则$u(x)=u(f_a(x_a))$,等于${f'}_a(u_a(x_a))$,等于${f'}_a({x'}_a)$,等于$x'$,故$x\in u^{-1}(x')$.
			\par
			反过来,如果$x\in u^{-1}(x')$,则存在$a\in I$、$x_a\in N_a$,使$x=f_a(x_a)$.由于${f'}_a({x'}_a)$等于$x'$,等于$u(x)$,等于$u(f_a(x_a))$,等于${f'}_a(u_a(x_a))$,根据定理\ref{theo183}(2),存在$b\geq a$,使${f'}_ba({x'}_a)= {f'}_ba(u_a(x_a))$,即${f'}_ba(u_a(x_a))={x'}_b$,故$u_b(f_{ba}(x_a))={x'}_b$,因此$f_{ba}(x_a)\in N_b$,又因为$x=f_b(f_{ba}(x_a))$,因此$x\in \lim\limits_\to N_a$.综上,$\lim\limits_\to {u_a}^{-1}({x'}_a)=u^{-1}(x')$.
			
			\begin{cor}\label{cor429}
				\textbf{限制指标集可以得到集合归纳系统}
				\par
				$I$为右方有向集,$((E_a)_{a\in I}, (f_{ba})_{a\in I\text{与}b\in I\text{与}a\leq b})$为关于$I$的集合归纳系统,$(E_a)_{a\in I}$对于\\$(f_{ba})_{a\in I\text{与}b\in I\text{与}a\leq b}$的归纳极限为$E$,$J$为$I$的预序子集,则$((E_a)_{a\in J}, (f_{ba})_{a\in J\text{与}b\in J\text{与}a\leq b})$是关于$J$的集合归纳系统.
			\end{cor}
			证明:根据定义可证.
			
			\begin{de}
				\textbf{通过限制得到的集合归纳系统(système inductif d'ensembles obtenu par restriction);集族的归纳极限的之间的规范映射(application canonique entre limites inductives de familles d'ensembles)}
				\par
				$I$为右方有向集,$((E_a)_{a\in I}, (f_{ba})_{a\in I\text{与}b\in I\text{与}a\leq b})$为关于$I$的集合归纳系统,$(E_a)_{a\in I}$对于\\$(f_{ba})_{a\in I\text{与}b\in I\text{与}a\leq b}$的归纳极限为$E$,$J$为$I$的预序子集,则$((E_a)_{a\in J}, (f_{ba})_{a\in J\text{与}b\in J\text{与}a\leq b})$称为通过将指标集限制在J上得到的集合归纳系统.
				\par
				令$(E_a)_{a\in I}$对于$(f_{ba})_{a\in I\text{与}b\in I\text{与}a\leq b}$的归纳极限为$E$,$(E_a)_{a\in J}$对于$(f_{ab})_{a\in J\text{与}b\in J\text{与}a\leq b}$的归纳极限为$E'$,对任意$a\in I$,$E_a$到$E$的规范映射为$f_a$,则映射族$(f_a)_{a\in J}$的归纳极限,称为$E'$到$E$的规范映射.
			\end{de}
			
			\begin{cor}\label{cor430}
				\hfill\par
				$I$为右方有向集,$J$为$I$的预序子集,$J'$为$J$的预序子集,$(E_a)_{a\in I}$对于$(f_{ba})_{a\in I\text{与}b\in I\text{与}a\leq b}$的归纳极限为$E$,$(E_a)_{a\in J}$对于$(f_{ba})_{a\in J\text{与}b\in J\text{与}a\leq b}$的归纳极限$E'$,$(E_a)_{a\in J'}$对于$(f_{ba})_{a\in J'\text{与}b\in J'\text{与}a\leq b}$\\的归纳极限为$E''$,$E'$到$E$的规范映射为$g$,$E''$到$E'$的规范映射为$g'$,$E''$到$E$的规范映射为$g''$,则$g''=g\circ g'$.
			\end{cor}
			证明:根据定义可证.	
					
			\begin{theo}\label{theo189}
				\hfill\par
				$I$为右方有向集,$((E_a)_{a\in I}, (f_{ba})_{a\in I\text{与}b\in I\text{与}a\leq b)}$为关于$I$的集合归纳系统,$J$为$I$的预序子集和共尾子集.令$(E_a)_{a\in I}$对于$(f_{ba})_{a\in I\text{与}b\in I\text{与}a\leq b}$的归纳极限为$E$,$(E_a)_{a\in J}$对于$(f_{ba})_{a\in J\text{与}b\in J\text{与}a\leq b}$\\的归纳极限为$E'$,则$E'$到$E$的规范映射为双射.
			\end{theo}
			证明:令$E'$到$E$的规范映射为$g$,根据定理184(3),$g$为单射.对任意$x\in E$,根据补充定理\ref{cor424}(1),存在$a\in I$、$z\in E_a$,使$f_a(z)=x$.由于$J$为$I$的共尾子集,故存在$b\geq a$,且$b\in J$,根据补充定理\ref{cor423},$f_b(f_{ba}(z))=x$,因此$x\in \bigcup\limits_{a\in J}f_a(E_a)$;反过来,对任意$x\in \bigcup\limits_{a\in J}f_a(E_a)$,均有$x\in E$,因此$E=\bigcup\limits_{a\in J}f_a(E_a)$,根据定理\ref{theo184}(2),$g$为满射.
						
			\begin{de}
				\textbf{集族的双重归纳极限(double limite inductive de famille d'ensembles)}
				\par
				$I$、$L$均为右方有向集,$I\times L$的预序关系为$(x\in I\times L\text{与}y\in I\times L\text{与}pr_1x\leq pr_1y\text{与}pr_2x\leq pr_2y)$,$((E_a^x)(a, x)\in I\times L, (f_{ba}^{yx})_{(a, x)\in I\times L\text{与}(b, y)\in I\times L\text{与}(a, x)\leq (b, y)})$为关于$I\times L$的集合归纳系统,则其归纳极限称为双重归纳极限,记作$\lim\limits_{\to a, x}E_a^x$,在没有歧义的情况下也可以简记为$\lim\limits_\to E_a^x$.
			\end{de}
			
			\begin{cor}\label{cor431}
				\hfill\par
				$I$、$L$均为右方有向集,$I\times L$的预序关系为$(x\in I\times L\text{与}y\in I\times L\text{与}pr_1x\leq pr_1y\text{与}pr_2x\leq pr_2y)$,$((E_a^x)(a, x)\in I\times L, (f_{ba}^{yx})_{(a, x)\in I\times L\text{与}(b, y)\in I\times L\text{与}(a, x)\leq (b, y)})$为关于$I\times L$的集合归纳系统,则$(E_a^x, f_{ba}^{yx})$为关于$I$的集合归纳系统,也是关于$L$的集合归纳系统.并且,$(\lim\limits_{\to a}E_a^x, g^{yx})$、\\$(\lim\limits_{\to x}E_a^x, h_{ba})$分别是关于$L$的集合归纳系统和关于$I$的集合归纳系统,其中$g^{yx}=\lim\limits_{\to a}f_{aa}^{yx}$,$h_{ba}\\=\lim\limits_{\to x}f_{ba}^{xx}$.
			\end{cor}
			证明:根据定义,可证则$(E_a^x, f_{ba}^{yx})$为关于$I$的集合归纳系统,也是关于$L$的集合归纳系统.根据定理\ref{theo186},$g^{zx}=g^{zy}\circ g^{yx}$,$h_{ca}=h_{cb}\circ h_{ba}$,因此,$(\lim\limits_{\to a}E_a^x, g^{yx})$、$(\lim\limits_{\to x}E_a^x, h_{ba})$分别是关于$L$的集合归纳系统和关于$I$的集合归纳系统.
			
			\begin{theo}\label{theo190}
				\hfill\par
				$I$、$L$均为右方有向集,$I\times L$的预序关系为$(x\in I\times L\text{与}y\in I\times L\text{与}pr_1x\leq pr_1y\text{与}pr_2x\leq pr_2y)$,$(E_a^x, f_{ba}^{yx})$为关于$I\times L$的集合归纳系统,令$E_a^x$到$\lim\limits_{\to a}E_a^x$的规范映射为$g_a^x$,$\lim\limits_{\to a}E_a^x$到\\$\lim\limits_{\to x}(\lim\limits_{\to a}E_a^x)$的规范映射为$h_x$,$u=\lim\limits_{\to a, x}(h_x\circ g_a^x)$,则$u$为$\lim\limits_{\to a, x}E_a^x$到$\lim\limits_{\to x}(\lim\limits_{\to a}E_a^x)$的双射;令$E_a^x$到\\$\lim\limits_{\to x}E_a^x$的规范映射为$j_a^x$,$\lim\limits_{\to x}E_a^x$到$\lim\limits_{\to a}(\lim\limits_{\to x}E_a^x)$的规范映射为$k_a$,$v=\lim\limits_{\to a, x}(k_a\circ j_a^x)$,则$v$为\\$\lim\limits_{\to a, x}E_a^x$到$\lim\limits_{\to a}(\lim\limits_{\to x}E_a^x)$的双射.
			\end{theo}
			证明:令$h^{yx}= \lim\limits_{\to a}f_{aa}^{yx}$,$F_x=\lim\limits_{\to a}E_a^x$,$F=\lim\limits_{\to x}F_x$,$E=\lim\limits_{\to a, x}E_a^x$.根据补充定理\ref{cor424}(2),$F=\bigcup\limits_{x\in L}h_x(F_x)$,$F_x=\bigcup\limits_{a\in I}g_a^x(E_a^x)$,因此$F=\bigcup\limits_{x\in L}(h_x\circ g_a^x(E_a^x))$,根据定理\ref{theo184}(2),$u$为满射.
			\par
			另一方面,令$m\in E_a^x$、$n\in E_a^x$,并且$h_x\circ g_a^x(m)=h_x\circ g_a^x(n)$,因此,存在$y\geq x$,使$h^{yx}(g_a^x(m))=h^{yx}(g_a^x(n))$,故$g_a^y(f_{aa}^{yx}(m))=g_a^y(f_{aa}^{yx}(m))$,因此,存在$b\geq a$,使$f_{ba}^{yy}(f_{aa}^{yx}(m))\\=f_{ba}^{yy}(f_{aa}^{yx}(m))$,因此,$f_{ba}^{yx}(m)=f_{ba}^{yx}(n)$,根据定理\ref{theo184}(3),$u$为单射.
			
			\begin{cor}\label{cor432}
				\hfill\par
				$I$、$L$均为右方有向集,$I\times L$的预序关系为$(x\in I\times L\text{与}y\in I\times L\text{与}pr_1x\leq pr_1y\text{与}pr_2x\leq pr_2y)$,$(E_a^x, f_{ba}^{yx})$、$({E'}_a^x, {f'}_ba^{yx})$均为关于$I\times L$的集合归纳系统,对于$(a, b)_{\in I\times L}$,令$u_a^x$为$E_a^x$到\\${E'}_a^x$的映射,并且$(u_a^x)_{(a, x)\in I\times L}$为$(E_a^x, f_{ba}^{yx})$到$({E'}_a^x, {f'}_{ba}^{yx})$的映射归纳系统,则$(u_a^x)_{a\in I}$、$(u_a^x)_{x\in L}$均为$(E_a^x, f_{ba}^{yx})$到$({E'}_a^x, {f'}_{ba}{yx})$的映射归纳系统,$(\lim\limits_{\to a}u_a^x)_{x\in L}$为$(\lim\limits_{\to a}E_a^x)$到$(\lim\limits_{\to a}{E'}_a^x)$的映射归纳系统,$(\lim\limits_{\to x}u_a^x)_{a\in I}$为$(\lim\limits_{\to x}E_a^x)$到$(\lim\limits_{\to x}{E'}_a^x)$的映射归纳系统.
			\end{cor}
			证明:根据定义可证$(u_a^x)_{a\in I}$、$(u_a^x)_{x\in L}$均为$(E_a^x, f_{ba}^{yx})$到$({E'}_a^x, {f'}_{ba}{yx})$的映射归纳系统.
			\par
			令$u_x=\lim\limits_{\to a}u_a^x$,$g^{yx}=\lim\limits_{\to a}f_{aa}^{yx}$,${g'}^{yx}= \lim\limits_{\to a}{f'}_{aa}^{yx}$,则当$x\leq y$时,$u_a^y\circ f_{aa}^{yx}={f'}_{aa}^{yx}\circ u_a^x$,根据定理\ref{theo186},$u_y\circ g^{yx}={g'}^{yx}\circ u_x$,因此,$(\lim\limits_{\to a}u_a^x)_{x\in L}$为$(\lim\limits_{\to a}E_a^x)$到$(\lim\limits_{\to a}{E'}_a^x)$的映射归纳系统.
			\par
			同理可证$(\lim\limits_\to xu_a^x)_{a\in I}$为$(\lim\limits_{\to x}E_a^x)$到$(\lim\limits_{\to x}{E'}_a^x)$的映射归纳系统.
			
			\begin{de}
				\textbf{映射族的双重归纳极限(double limite inductive de famille \\d'applications)}
				\par
				$I$、$L$均为右方有向集,$I\times L$的预序关系为$(x\in I\times L\text{与}y\in I\times L\text{与}pr_1x\leq pr_1y\text{与}pr_2x\leq pr_2y)$,$(E_a^x, f_{ba}^{yx})$、$({E'}_a^x, {f'}_{ba}^{yx})$均为关于$I\times L$的集合归纳系统,对于$(a, b)\in I\times L$,令$u_a^x$为$E_a^x$到\\${E'}_a^x$的映射,并且$(u_a^x)_{(a, x)\in I\times L}$为$(E_a^x, f_{ba}^{yx})$到$({E'}_a^x, {f'}_{ba}^{yx})$的映射归纳系统,则其归纳极限称双重归纳极限,记作$\lim\limits_{\to a, x}u_a^x$.
			\end{de}
					
			\begin{theo}\label{theo191}
				\hfill\par
				$I$、$L$均为右方有向集,$I\times L$的预序关系为$(x\in I\times L\text{与}y\in I\times L\text{与}pr_1x\leq pr_1y\text{与}pr_2x\leq pr_2y)$,$(E_a^x, f_{ba}^{yx})$、$({E'}_a^x, {f'}_{ba}^{yx})$均为关于$I\times L$的集合归纳系统,对于$(a, b)\in I\times L$,令$u_a^x$为$E_a^x$到\\${E'}_a^x$的映射,并且$(u_a^x)_{(a, x)\in I\times L}$为$(E_a^x, f_{ba}^{yx})$到$({E'}_a^x, {f'}_{ba}^{yx})$的映射归纳系统,$u_a=\lim\limits_{\to x}u_a^x$,$u_x=\lim\limits_{\to a}u_a^x$,$u=\lim\limits_{\to a, x}u_a^x$.令$E_a^x$到$\lim\limits_{\to a}E_a^x$的规范映射为$g_a^x$,$\lim\limits_{\to a}E_a^x$到$\lim\limits_{\to x}(\lim\limits_{\to a}E_a^x)$的规范映射为$h_x$,$u=\lim\limits_{\to a, x}(h_x\circ g_a^x)$;令$E_a^x$到$\lim\limits_{\to x}E_a^x$的规范映射为$j_a^x$,$\lim\limits_{\to x}E_a^x$到$\lim\limits_{\to a}(\lim\limits_{\to x}E_a^x)$的规范映射为$k_a$,$v=\lim\limits_{\to a, x}(k_a\circ j_a^x)$;令${E'}_a^x$到$\lim\limits_{\to a}{E'}_a^x$的规范映射为${g'}_a^x$,$\lim\limits_{\to a}{E'}_a^x$到$\lim\limits_{\to x}(\lim\limits_{\to a}{E'}_a^x)$的规范映射为${h'}_x$,$u'=\lim\limits_{\to a, x}({h'}_x\circ {g'}_a^x)$;令${E'}_a^x$到$\lim\limits_{\to x}{E'}_a^x$的规范映射为${j'}_a^x$,$\lim\limits_{\to x}{E'}_a^x$到$\lim\limits_{\to a}(\lim\limits_{\to x}{E'}_a^x)$的规范映射为${k'}_a$,$v'=\lim\limits_{\to a, x}({k'}_a\circ {j'}_a^x)$.则$\lim\limits_{\to a, x}u_a^x={u'}^{-1}\circ \lim\limits_{\to x}(\lim\limits_{\to a}u_a^x)\circ u$,$\lim\limits_{\to a, x}u_a^x={v'}^{-1}\circ \lim\limits_{\to a}(\lim\limits_{\to x}u_a^x)\circ v$.
			\end{theo}
			证明:根据定理\ref{theo190}和可证.
					
			\begin{theo}\label{theo192}
				\hfill\par
				$I$为右方有向集,$(E_a, f_{ba})$、$({E'}_a, {f'}_{ba})$为关于$I$的集合归纳系统,$E=\lim\limits_{\to a}E_a$,$E'=\lim\limits_{\to a}{E'}_a$,对任意$a\in I$,$E_a$到$E$的规范映射为$f_a$,${E'}_a$到$E'$的规范映射为${f'}_a$,则$(E_a\times {E'}_a, f_{ba}\times {f'}_{ba})$也是关于$I$的集合归纳系统,$(fa\times {f'}_a)$是$(E_a\times {E'}_a, f_{ba}\times {f'}_{ba})$到$E\times E'$的映射归纳系统,且$\lim\limits_{\to} (f_a\times {f'}_a)$是$\lim\limits_{\to} (E_a\times {E'}_a)$到$(\lim\limits_\to E_a)\times (\lim\limits_\to {E'}_a)$的双射.
			\end{theo}
			证明:根据定义,$(E_a\times {E'}_a, f_{ba}\times {f'}_{ba})$也是关于$I$的集合归纳系统,$(f_a\times {f'}_a)$是$(E_a\times {E'}_a, f_{ba}\times {f'}_{ba})$到$E\times E'$的映射归纳系统.
			\par
			令$g=\lim\limits_\to (f_a\times {f'}_a)$,由于$E\times E'=\bigcup\limits_{a\in I}f_a(E_a)\times {f'}_a({E'}_a)$,根据定理\ref{theo184}(2),$g$为满射.
			\par
			另一方面,设$(x, x')\in E_a\times {E'}_a$,$(y, y')\in E_a\times {E'}_a$,并且$f_a(x)=f_a(y)$、${f'}_a(x)={f'}_a(y)$,则存在$b\geq a$、$c\geq a$,使$f_{ba}(x)=f_{ba}(y)$、${f'}_{ca}(x)={f'}_{ca}(y)$,进而,存在$d\geq c$、$d\geq b$,故$f_{da}(x)=f_{da}(y)$、${f'}_{da}(x)={f'}_{da}(y)$,根据定理\ref{theo184}(3),$g$为单射.
			
			\begin{de}
				\textbf{乘积的归纳极限到归纳极限的乘积的规范映射(application canonique de la limite inductive d'un produit dans du produit de limites inductives)}
				\par
				$I$为右方有向集,$(E_a, f_{ba})$、$({E'}_a, {f'}_{ba})$为关于$I$的集合归纳系统,$E=\lim\limits_\to E_a$,$E'=\\\lim\limits_\to {E'}_a$,对任意$a\in I$,$E_a$到$E$的规范映射为$f_a$,${E'}_a$到$E'$的规范映射为${f'}_a$,则$(f_a\times {f'}_a)$称为$\lim\limits_\to (E_a\times {E'}_a)$到$(\lim\limits_\to E_a)\times (\lim\limits_\to{E'}_a)$的规范映射.
			\end{de}
					
			\begin{theo}\label{theo193}
				\hfill\par
				$I$为右方有向集,$(E_a, f_{ba})$、$({E'}_a, {f'}_{ba})$、$(F_a, g_{ba})$、$({F'}_a, {g'}_{ba})$均为关于$I$的集合归纳系统,对任意$a\in I$,$u_a$为$E_a$到$F_a$的映射,${u'}_a$为${E'}_a$到${F'}_a$的映射,且$(u_a)$和$({u'}_a)$均为映射归纳系统,则$(u_a\times {u'}_a)$为映射归纳系统,令$\lim\limits_\to (E_a\times {E'}_a)$到$(\lim\limits_\to aE_a)\times (\lim\limits_\to a{E'}_a)$的规范映射为$f$,$\lim\limits_\to (F_a\times {F'}_a)$到$(\lim\limits_\to F_a)\times (\lim\limits_\to {F'}_a)$的规范映射为$g$,则$\lim\limits_\to (u_a\times {u'}_a)= g^{-1}\circ ( (\lim\limits_\to u_a)\times (\lim\limits_\to {u'}_a))\circ f$.
			\end{theo}
			证明:根据定理\ref{theo192}可证.
			
			\begin{exer}\label{exer189}
				\hfill\par
				$I$为右方有向预序集,$(J_l)_{l\in L}$为$I$的子集族,其中$L$为右方有向预序集,并且:
				\par
				第一,$J_l$按在$I$上的预序关系导出的预序关系排序,且为右方有向集;
				\par
				第二,$i\in L\text{与}j\in L\text{与}i\leq j\Rightarrow J_i\subset J_j$;
				\par
				第三,$I=\bigcup\limits_{l\in L}J_l$.
				\par
				$(E_a, f_{ab})$为集合映射系统,$\lim\limits_\gets E_a=E$,对任意$l\in L$,对于$(E_a, f_{ab})$通过将指标集限制在$J_l$上得到的集合射影系统,令$F_l$为其集族对于其函数族的射影极限.当$i\in L\text{与}j\in L\text{与}i\leq j$时,令$g_{ij}$为$F_j$到$F_i$的规范映射.求证:$(F_i, g_{ij})$为关于$I$的集合射影系统,并且,令$F= \lim\limits_\gets F_l$,试定义$F$到$E$的规范双射.
			\end{exer}
			证明:
			\par
			根据补充定理\ref{cor416}可以证明$(F_i, g_{ij})$为关于I的集合射影系统.
			\par
			对任意$x\in E$,根据定义可证$((f_i(x))_{i\in J_j})_{j\in L}\in F$;反过来,对任意$y\in F$、$i\in L$、$j\in L$,如果$a\in J_i\cap J_j$,则$pr_a(pr_iy)=pr_a(pr_jy)$.令$x=(pr_a(pr_{\tau_i(i\in L\text{与}a\in J_i)}y))_{a\in I}$,对任意$a\in I$、$b\in I$,存在$i\in L$,使$a\in J_i$、$b\in J_i$,因此$f_{ab}(pr_b(pr_{\tau_i(i\in L\text{与}a\in J_i)}y))=pr_a(pr_{\tau_i(i\in L\text{与}a\in J_i)}y)$,故$x\in E$,因此,可定义$F$到$E$的规范双射为$y\mapsto (pr_a(pr_{\tau_i(i\in L\text{与}a\in J_i)}y))_{a\in I}$.
			
			\begin{exer}\label{exer190}
				\hfill\par
				$(Ea, f_{ab})$为关于$I$的集合射影系统,其中$I$为右方有向集,$\lim\limits_\gets E_a=E$,对任意$a\in I$,令$f_a$为$E$到$E_a$的规范映射,如果对任意$a\in I$、$b\in I$,$f_{ab}$均为单射,求证:对任意$a\in I$,$f_a$为单射.
			\end{exer}
			证明:即补充定理\ref{cor414}.
			
			\begin{exer}\label{exer191}
				\hfill\par
				$I$为预序集,$(E_a, f_{ab})$、$(F_a, g_{ab})$均为关于$I$的集合射影系统,映射族$(u_a)_{a\in I}$为$(E_a, f_{ab})$到\\$(F_a, g_{ab})$的映射射影系统.对任意$a\in I$,令$G_a$为$u_a$的图,$u=\lim\limits_\gets u_a$.求证:$(G_a)$为某个映射射影系统的集族,并给出其射影极限.
			\end{exer}
			证明:令$h_{ab}$为映射$(x, y)\mapsto (f_{ab}(x), g_{ab}(y))$,则$(G_a, h_{ab})$为映射射影系统,其射影极限为$\bigcup\limits_{z\in u\text{的图}}\{(pr_i(pr_1z), pr_i(pr_2z))_{i\in I}\}$.
			
			\begin{exer}\label{exer192}
				\hfill\par
				$I$为非空右方有向集,且无最大元.$F$为满足下列条件的I的元素序列$x=(a_i)_{i\in [1, 2n]}$(其中$n$为自然数且$n\geq 1$)的集合:
				\par
				第一,当$i\in [1, n]$时,$a_{2i-1}<a_{2i}$;
				\par
				第二,当$j\in [1, n]$、$i\in [1, n]$且$j<i$时,$\text{非}(a_{2i-1}\leq a_{2j-1})$.
				\par
				$F\neq \varnothing$.令$r(x)=a_{2n-1}$,$s(x)=a_{2n}$,$n$称为$x$的长度.
				\par
				(1)对于任意$a\in I$,令$E_a=\{x|x\in F\text{与}r(x)=a\}$.当$a\in I$、$b\in I$、$a\leq b$时,按照下列方式定义$E_b$到$E_a$的函数$f_{ab}$:对于$x\in E_b$,令$x=(a_i)_{i\in [1, 2n]}$,令$j$为$\{i|i\in [1, n]\text{与}a\leq a_{2j-1}\}$的最小元,$f_{ab}(x)=(a_i)_{i\in [1, 2j-2]}\cup\{(2j-1, a), (2j, a_{2j})\}$,求证:对任意$a\in I$、$b\in I$、$a\leq b$,$E_a\neq \varnothing$,$f_{ab}(E_b)=E_a$,并且,$(E_a, f_{ab})$为关于$I$的集合射影系统.
				\par
				(2)令$x_a\in E_a$、$x_b\in E_b$,存在$c\in I$以及$x_c\in E_c$,使$c\geq a$、$c\geq b$,并且,$x_a=f_{ac}(x_c)$,$x_b=f_{bc}(x_c)$,如果$x_a$和$x_b$的长度相等,求证:$s(x_a)=s(x_b)$.
				\par
				(3)$E= \lim\limits_\gets E_a$,且$E\neq\varnothing$,令$(a_i)\in E$,求证:$\{x|(\exists i)(i\in I\text{与}x=s(a_i))\}$可数并且和$I$共尾.
				\par
				(4)令I为不可数集合A的有限子集集合,并按包含关系排序.求证:不存在$I$的可数共尾子集,并且,给出$(E_a, f_{ab})$,对任意$a\in I$、$b\in I$、$a\leq b$,$E_a\neq \varnothing$,$f_{ab}(E_b)$为满射,但$\lim\limits_\gets E_a=\varnothing$.
				\par
				(5)给出关于$I$的集合射影系统$(E_a, f_{ab})$到$({E'}_a, {f'}_ab)$的映射射影系统$(u_a)$,令$u=\\\lim\limits_\gets u_a$,对任意$a\in I$,$u_a$均满射,但$u$不是满射.
			\end{exer}
			证明:
			\par
			(1)对任意$a\in I$,令$d>a$,对任意$i\in [1, n]$,令$a_{2i-1}=a$,$a_{2i}=d$,故$(a_i)_{i\in [1, 2n]}\in E_a$,因此$E_a\neq \varnothing$.当$a\in I$、$b\in I$、$a\leq b$时,对任意$x\in E_a$,均有$x\in E_b$,且$f_{ab}(x)=x$,故$f_{ab}(E_b)=E_a$.根据定义可证$f_{ac}=f_{ab}\circ f_{bc}$,故$(E_a, f_{ab})$为关于$I$的集合射影系统.
			\par
			(2)	根据定义可证.
			\par
			(3)	根据习题\ref{exer192}(2)可证.
			\par
			(4)设$J$为$I$的共尾子集,令$B=\bigcup\limits_{X\in J}X$,则$B$为可数集合,令$x\in A-B$,则不存在$Z\in J$使$\{x\}\subset Z$,矛盾,故不存在I的可数共尾子集.根据习题\ref{exer192}(3)确定的集合射影系统符合条件.
			\par
			(5)令$(E_a, f_{ab})$为根据习题\ref{exer192}(3)确定的集合射影系统,对任意$a\in I$、$b\in I$、$a\leq b$,令${E'}_a$均为单元素集合,${f'}_{ab}$为${E'}_b$到${E'}_a$的双射,$u_a$为$E_a$到${E'}_a$的满射,则$(u_a)$符合条件.
			
			\begin{exer}\label{exer193}
				\hfill\par
				$I$为右方有向集,$(E_a)_{a\in I}$为格族,对任意$a\in I$,$E_a$按相反关系排序,为诺特集.对任意$a\in I$、$b\in I$、$a\leq b$,$f_{ab}$为$E_b$到$E_a$的单增映射,$(E_a, f_{ab})$为关于$I$的集合射影系统.对任意$a\in I$,$G_a$均为$E_a$的非空子集,并且满足下列条件:
				\par
				第一,对任意$a\in I$,$G_a$的任何两个不相同元素都是不可比较的;
				\par
				第二,对任意$a\in I$、$b\in I$、$a\leq b$,$f_{ab}(G_b)=G_a$;
				\par
				第三,对任意$a\in I$、$b\in I$、$a\leq b$、$x_a\in G_a$,${f_{ab}}^{-1}(x_a)$有最大元$M_{ab}(x_a)$;
				\par
				第四,对任意$a\in I$、$b\in I$、$a\leq b$,如果$h_b\in E_b$,且存在$y_b\in G_b$且$y_b\leq h_b$,则对任意$x_a\in G_a$且$x_a\leq f_{ab}(h_b)$,均存在$x_b\in G_b$且$x_b\leq h_b$使$x_a=f_{ab}(x_b)$.
				\par
				求证:子集射影系统$(G_a)$的射影极限不是空集.
			\end{exer}
			证明:
			令$J$为$I$的有限子集,考虑满足下列条件的元素族$(x_a)_{a\in J}$:
			\par
			第一,对任意$a\in J$、$b\in J$、$a\leq b$、$x_a\in G_a$、$x_b\in G_b$,$x_a=f_{ab}(x_b)$;
			\par
			第二,对任意$c\in I$并且$c$是$J$在$I$上的上界,存在$x_c\in c$并且对任意$a\in J$均有$x_a=f_{ac}(x_c)$.
			\par
			因此,对任意$c\in I$并且$c$是$J$在$I$上的上界,$\bigcap\limits_{a\in J}{f_{ac}}^{-1}(x_a)$不为空,且所有元素均为\\$\bigcup\limits_{a\in J}\{M_{ac}(x_a)\}$的下界.由$\bigcup\limits_{a\in J}\{M_{ac}(x_a)\}$为有限集合,且$E_c$为格,故其在$E_c$上有最大下界,令$z\\=\mathop{inf}\limits_{a\in J}(M_{ac}(x_a))$,$u\in \bigcap\limits_{a\in J}{f_{ac}}^{-1}(x_a)$,则$u\leq z$,故对任意$a\in J$,均有$f_{ac}(z)\geq x_a$,同时,由于$z\leq M_{ac}(x_a)$,故$f_{ac}(z)\leq x_a$,因此$f_{ac}(z)=x_a$,故$\mathop{inf}\limits_{a\in J}(M_{ac}(x_a))$为$\bigcap\limits_{a\in J}{f_{ac}}^{-1}(x_a)$的最大元.进而,对任意$y\in G_c$且$y\leq z$,以及任意$d\in J$,$f_{dc}(y)\leq x_d$,由于$f_{dc}(y)\in G_d$、$x_d\in G_d$,故$f_{dc}(y)=x_d$,因此,$\{y|y\in G_c\text{与}y\leq \mathop{inf}\limits_{a\in J}(M_{ac}(x_a))\}=G_c\cap(\bigcap\limits_{a\in J}{f_{ac}}^{-1}(x_a))$.
			\par
			令$J$为$I$的子集,考虑满足下列条件的元素族$(x_a)_{a\in J}$:对$J$的任意有限子集$F$,$(x_a)_{a\in F}$满足上一段的两个条件.如果$J\neq I$,令$b\in I-J$.对$J$的任意有限子集$F$、$F\cup\{b\}$的上界$c$,$\{y|y\in G_c\text{与}y\leq \mathop{inf}\limits_{a\in F}(M_{ac}(x_a))\}=Gc\cap(\bigcap\limits_{a\in F}{f_{ac}}^{-1}(x_a))$.因此$\{y|y\in G_b\text{与}y\leq\\ f_{bc}(\mathop{inf}\limits_{a\in F}(M_{ac}(x_a)))\}=f_{bc}(G_c\cap(\bigcap\limits_{a\in F}{f_{ac}}^{-1}(x_a)))$.根据补充定理\ref{cor410}(1),存在$F_0\subset J$以及$F_0\cup\{b\}$的上界$c_0$,对任意$J$的有限子集$F$,以及任意$F\cup\{b\}$的上界$c$,均有$\mathop{inf}\limits_{a\in F}(M_{ac}(x_a))\geq\\\mathop{inf}\limits_{a\in F_0}(M_{a c_0}(x_a))$.令$x_b$为$\{y|y\in G_b\text{与}y\leq f_{b c_0}(\mathop{inf}\limits_{a\in F_0}(M_{a c_0}(x_a)))\}$的任何一个元素,则$(x_a)_{a\in F\cup\{b\}}$\\符合条件.
			\par
			令$K$为存在满足条件的元素组的I的子集的集合,则$K$非空.$K$按关于$J$、$J'$的偏序关系(存在满足条件的元素族$(x_i)_{a\in J}$和$(y_i)_{a\in J'}$并且前者是后者的子族)的包含关系排序,根据定理\ref{theo80},$K$有极大元,并且其极大元为$I$,故存在元素族$(x_i)_{a\in J}$符合条件,即子集射影系统$(G_a)$的射影极限不是空集.
			
			\begin{exer}\label{exer194}
				\hfill\par
				$I$为右方有向预序集,$(J_l)_{l\in L}$为$I$的子集族,其中$L$为右方有向预序集,并且:
				\par
				第一,$J_l$按在$I$上的预序关系导出的预序关系排序,且为右方有向集;
				\par
				第二,$i\in L\text{与}j\in L\text{与}i\leq j\Rightarrow J_i\subset J_j$;
				\par
				第三,$I=\bigcup\limits_{l\in L}J_l$.$(E_a, f_{ba})$为集合归纳系统.
				\par
				令$\lim\limits_\to E_a=E$,对任意$l\in L$,对于$(E_a, f_{ba})$通过将指标集限制在$J_l$上得到的集合归纳系统,令$F_l$为其集族对于其函数族的归纳极限.当$i\in L\text{与}j\in L\text{与}i\leq j$时,令$g_{ji}$为$F_i$到$F_j$的规范映射.求证:$(F_i, g_{ji})$为关于$I$的集合归纳系统,并且,令$F= \lim\limits_\to F_l$,试定义$F$到$E$的规范双射.
			\end{exer}
			证明:
			\par
			根据补充定理\ref{cor430}可以证明$(F_i, g_{j_i})$为关于$I$的集合归纳系统.
			\par
			令$G$为$(E_a)_{i\in I}$的和,其等价关系为$R$,$H$为$(F_i)_{i\in L}$的和,其等价关系为$S$.对任意$X\in F$,令关于$S$的等价类的代表为$K$,将指标集限制在$J_{pr_2k}$上得到的集合归纳系统的等价关系为$T$,$pr_1k$关于$T$的等价类的代表为$x$,$x$关于$R$的等价类为$G(x)$,则$X\mapsto G(x)$为$F$到$E$的规范双射.			
			
			\begin{exer}\label{exer195}
				\hfill\par
				$I$为右方有向集,$(E_a, f_{ba})$为关于$I$的集合归纳系统,其中,$\lim\limits_\to E_a=E$,对任意$a\in I$,$f_a$为$E$到$E_a$的规范映射,$R_a$为公式$(x\in E_a\text{与}y\in E_a\text{与}f_a(x)=f_a(y))$,求证:对任意$a\in I$、$b\in I$、$a\leq b$,$f_{ba}$是同$R_a$和$R_b$相容的映射;令${E'}_a=E_a/R_a$,${f'}_{ba}$为$f_{ba}$对于$R_a$和$R_b$通过商导出的映射,则${f'}_{ba}$为单射,$({E'}_a, {f'}_{ba})$为关于$I$的集合归纳系统,试定义$E$到$\lim\limits_\to {E'}_a$的规范双射.
			\end{exer}
			证明:
			\par
			对任意$x\in E_a$、$y\in E_a$,如果$f_a(x)= f_a(y)$,则存在$c$使$f_{ac}(x)=f_{ac}(y)$.令$d=sup(b, c)$,则$f_{bd}(f_{ab}(x))=f_{bd}(f_{ab}(y))$,故$f_b(f_{ab}(x))=f_b(f_{ab}(y))$,因此,$f_{ba}$是同$R_a$和$R_b$相容的映射.
			\par
			设${f'}_{ba}(x)={f'}_{ba}(y)$,$x=f_a(u)$,$y=f_a(v)$,则$f_b(f_{ab}(u))= f_b(f_{ab}(v))$,故$f_a(u)=f_a(v)$,因此$x=y$,所以${f'}_{ba}$为单射.同时,根据定义可证$({E'}_a, {f'}_{ba})$为关于$I$的集合归纳系统.$x\mapsto \bigcup\limits_{a\in I}\{pr_1(x\cap(E_a\times \{a\})\}\times \{a\}$为$E$到$\lim\limits_\to {E'}_a$的规范双射.			
			
			\begin{exer}\label{exer196}
				\hfill\par
				$I$为右方有向集,$(E_a, f_{ba})$、$(F_a, g_{ba})$为关于$I$的集合归纳系统,映射族$(u_a)_{a\in I}$为$(E_a, f_{ba})$\\到$(F_a, g_{ba})$的映射归纳系统.对任意$a\in I$,令$G_a$为$u_a$的图,$u=\lim\limits_\to u_a$.求证:$(G_a)$为某个映射归纳系统的集族,并给出其归纳极限.
			\end{exer}
			证明:令$h_{ba}$为映射$(x, y)\mapsto (f_{ba}(x), g_{ba}(y))$,则$(G_a, h_{ba})$为映射归纳系统,其归纳极限为$\bigcup\limits_z\in u\text{的图}\{\{(x, y)|x\in pr_1(pr_1z)\text{与}y=u_{pr_2(pr_1z)}(x)\}\}$.
					
			\begin{exer}\label{exer197}
				\hfill\par
				$I$为预序集,$(E_a)_{a\in I}$为集族,$(f_{ba})_{a\in I\text{与}b\in I\text{与}a\leq b}$为函数族,其中$f_{ba}$为$E_a$到$E_b$的映射,并且满足下列条件:
				\par
				第一,如果$a\leq b$,$b\leq c$,则$f_{ba}\circ f_{cb}=f_{ca}$.
				\par
				第二,$f_{aa}=Id_{E_a}$,
				则$(E_a)_{a\in I}$, $(f_{ba})_{a\in I\text{与}b\in I\text{与}a\leq b)}$称为关于$I$的集合归纳系统,在没有歧义的情况下可以简记为$((E_a), (f_{ba}))$或$(E_a, f_{ba})$.
				\par
				令$R$为公式$(x\in G\text{与}y\in G\text{与}pr_2x\leq pr_2y\text{与}f_{pr_2y pr_2x}(pr_1x)=pr_1y)$,$R'$为公式:
				\par
				“$\text{存在自然数}n>0\text{以及}(x_i)_{i\in [0, n]}\text{(}(\forall i)(i\in [0, n]\Rightarrow x_i\in G)\text{),其中}x_0=x\text{,}x_n=y\text{,并且,}\text{对任意}i\in [0, n-1]\text{,}(x_{i+1}|y)(x_i|x)R\text{或}(x_i|y)(x_{i+1}|x)R\text{为真}$”.
				\par
				令E=G/R',则称$E$为集族\\$(E_a)_{a\in I}$对于函数族$(f_{ba})_{a\in I\text{与}b\in I\text{与}a\leq b}$的归纳极限,记作$\lim\limits_{\to a}(E_a, f_{ba})$,在没有歧义的情况下可以简记为$\lim\limits_\to (E_a, f_{ba})$或$\lim\limits_\to E_a$.令$f$为$G$到$E/R'$的规范映射,$g_a$为映射$x\mapsto (x, a)(x\in E_a)$,则映射$f\circ g_a$称为$E_a$到$\lim\limits_\to E_a$的规范映射.
			\end{exer}
			证明:根据定义可证$I$为右方有向集时,$\lim\limits_\to E_a$,即为集族归纳系统$(E_a, f_{ba})$的归纳极限.
			\par
			类似定理\ref{theo184}(1)、定理\ref{theo185}的证明,可证明$u$的存在性和唯一性.

	\chapter{结构(Structures)}
		\section{结构和同构(Structures et isomorphismes)}		
			\begin{STdef}
				\textbf{阶梯构造模式(schéma de construction d'échelon)}
				\par
				满足下列条件的自然数有序对有限序列$(a_i, b_i)_{i\in [1, m]}$,称为阶梯构造模式:
				\par
				(1)如果$b_i=0$,则$a_i\in [1, i-1]$; 
				\par
				(2)如果$a_i\neq 0$且$b_i\neq 0$,则$a_i\in [1, i-1]$且$b_i\in [1, i-1]$.
			\end{STdef}

			\begin{STdef}
				\textbf{在n个项上的阶梯构造模式(schéma de construction d'échelon sur n termes)}
				\par
				对于阶梯构造模式$(a_i, b_i)_{i\in [1, m]}$,如果$a_1=0$、$b_1>0$且$\{x|i\in [1, m]\text{与}a_i=0\text{与}x=b_i\}$的最大元为$n$,则称其为在$n$个项上的阶梯构造模式.
			\end{STdef}
			
			\begin{STdef}
				\textbf{阶梯构造(construction d'échelon),阶梯(échelon)}
				\par
				令$M$为比集合论强的理论,$E_1$、$E_2$、$\cdots$、$E_n$为$M$的$n$个项,对于在$n$个项上的阶梯构造模式$S=(a_i, b_i)_{i\in [1, m]}$,如果$M$的$m$个项$A_1$、$A_2$、$\cdots$、$A_m$满足下列条件,则称其为阶梯构造模式$S$在$E_1$、$E_2$、$\cdots$、$E_n$上的阶梯构造,并且,其中$A_m$称为阶梯构造模式$S$在基集合$E_1$、$E_2$、$\cdots$、$E_n$上的阶梯,记作$S(E_1, E_2, \cdots, E_n)$:
				\par
				(1)如果$a_i=0$,则$A_i$为项$E_{b_i}$;
				\par				(2)如果$b_i=0$,则$A_i$为项$\mathcal{P}(A_{a_i})$;
				\par
				(3)如果$a_i\neq 0$且$b_i\neq 0$,则$A_i$为项$A_{a_i}\times A_{b_i}$.
			\end{STdef}
			
			\begin{STdef}
				\textbf{映射对模式的规范扩展(extension canonique de schema \\d'applications)}
				\par
				在比集合论强的理论中,令在$n$个项上的阶梯构造模式$S=(a_i, b_i)_{i\in [1, m]}$,$E_1$、$E_2$、$\cdots$、$E_n$、${E'}_1$、${E'}_2$、$\cdots$、${E'}_n$为$M$的项,$f_1$、$f_2$、$\cdots$、$f_n$为映射,且对于$i\in [1, n]$,$f_i$为$E_i$到${E'}_i$\\的映射,设$A_1$、$A_2$、$\cdots$、$A_m$为模式$S$在$E_1$、$E_2$、$\cdots$、$E_n$上的阶梯构造,${A'}_1$、${A'}_2$、$\cdots$、${A'}_m$为模式$S$在${E'}_1$、${E'}_2$、$\cdots$、${E'}_n$上的阶梯构造.如果${g'}_1$、${g'}_2$、$\cdots$、${g'}_m$为映射,且对于$i\in [1, m]$,$g_i$为$A_i$到${A'}_i$的映射,并满足下列条件,则称$g_m$为$f_1$、$f_2$、$\cdots$、$f_n$对模式$S$的规范扩展,记作$\langle f_1, f_2, \cdots, f_n\rangle^S$:
				\par
				(1)如果$a_i=0$,则$g_i$为$f_{b_i}$;
				\par
				(2)如果$b_i=0$,则$g_i$为$g_{a_i}$在子集上的规范扩展;
				\par
				(3)如果$a_i\neq 0$且$b_i\neq 0$,则$g_i$为$g_{a_i}$和$g_{b_i}$在乘积集合上的规范扩展.
			\end{STdef}
			
			\begin{CST}\label{CST1}
				\hfill\par
				在比集合论强的理论中,$S$为在$n$个项上的阶梯构造模式,$f_1$、$f_2$、$\cdots$、$f_n$、${f'}_1$、\\${f'}_2$、$\cdots$、${f'}_n$为映射,且对于$i\in [1, n]$,$f_i$为$E_i$到${E'}_i$的映射,${f'}_i$为${E'}_i$到${E''}_i$的映射,则$\langle {f'}_1\circ f_1, {f'}_2\circ f_2, \cdots, {f'}_n\circ f_n\rangle^S=\langle {f'}_1, {f'}_2, \cdots, {f'}_n\rangle^S\circ \langle f_1, f_2, \cdots, f_n\rangle^S$.
			\end{CST}
			证明:根据补充定理\ref{cor86}、补充定理\ref{cor88}、补充定理\ref{cor119}(1)、补充定理\ref{cor119}(2)可证.
						
			\begin{CST}\label{CST2}
				\hfill\par
				在比集合论强的理论中,$S$为在$n$个项上的阶梯构造模式,$f_1$、$f_2$、$\cdots$、$f_n$均为单射(或满射),则$\langle f_1, f_2, \cdots, f_n\rangle^S$为单射(或满射).
			\end{CST}
			证明:根据补充定理\ref{cor85}、定理\ref{theo36}可证.
									
			\begin{CST}\label{CST3}
				\hfill\par
				在比集合论强的理论中,$S$为在$n$个项上的阶梯构造模式,$f_1$、$f_2$、$\cdots$、$f_n$均为双射,则$(\langle f_1, f_2, \cdots, f_n\rangle^S)^{-1}=\langle {f_1}^{-1}, {f_2}^{-1}, \cdots, {f_n}^{-1}\rangle^S$.
			\end{CST}
			证明:根据结构规则\ref{CST1}、结构规则\ref{CST2}可证.
						
			\begin{CSTcor}\label{CSTcor1}
				\hfill\par
				在比集合论强的理论中,$S$为在$n$个项上的阶梯构造模式,$f_1$、$f_2$、$\cdots$、$f_n$均为恒等映射,则$\langle f_1, f_2, \cdots, f_n\rangle^S$为恒等映射.
			\end{CSTcor}
			证明:根据补充定理\ref{cor87}、补充定理\ref{cor119}(3),运用数学归纳法可证.
						
			\begin{STdef}
				\textbf{类型化(typification)}
				\par
				令$M$为比集合论强的理论,$x_1$、$x_2$、$\cdots$、$x_n$、$s_1$、$s_2$、$\cdots$、$s_p$为互不相同的字母,且都不是$M$的常数.$A_1$、$A_2$、$\cdots$、$A_m$为$M$的项(其中$m$也可以为$0$),并且均不包含$x_1$、$x_2$、$\cdots$、$x_n$、$s_1$、$s_2$、$\cdots$、$s_p$.$S_1$、$S_2$、$\cdots$、$S_p$均为在$n+m$个项上的阶梯构造模式,令公式$T$为$(s_1\in S_1(x_1, x_2, \cdots, x_n, A_1, A_2, \cdots, A_m)\text{与}s_2\in S_2(x_1, x_2, \cdots, x_n, A_1, A_2, \cdots, A_m)\text{与}\cdots\\\text{与}s_p\in S_p(x_1, x_2, \cdots, x_n, A_1, A_2, \cdots, A_m))$,则称$T$为$s_1$、$s_2$、$\cdots$、$s_p$的类型化.
			\end{STdef}
			
			\begin{STdef}
				\textbf{可转换的公式(relation transportable),可转换的公式的主要基集合\\(ensemble de base principal de la relation transportable),可转换的公式的辅助基集合(ensemble de base auxiliaire de la relation transportable)}
				\par
				令$M$为比集合论强的理论,$x_1$、$x_2$、$\cdots$、$x_n$、$s_1$、$s_2$、$\cdots$、$s_p$为互不相同的字母,且都不是$M$的常数.$A_1$、$A_2$、$\cdots$、$A_m$为$M$的项(其中$m$也可以为$0$),并且均不包含$x_1$、$x_2$、$\cdots$、$x_n$、$s_1$、$s_2$、$\cdots$、$s_p$,$T$为$s_1$、$s_2$、$\cdots$、$s_p$的类型化.
				\par
				如果公式$R$满足下列条件,则称公式$R$对类型化T是可转换的,在没有歧义的情况下也可以简称$R$是可转换的,其中,$x_1$、$x_2$、$\cdots$、$x_n$称为主要基集合,$A_1$、$A_2$、$\cdots$、$A_m$称为辅助基集合:
				\par
				设$y_1$、$y_2$、$\cdots$、$y_n$、$f_1$、$f_2$、$\cdots$、$f_p$为互不相同的字母,与$x_1$、$x_2$、$\cdots$、$x_n$、$s_1$、\\$s_2$、$\cdots$、$s_p$也均不相同,且都不是$M$的常数,则$(T\text{与}(f_1\text{是}x_1\text{到}y_1\text{的双射})\text{与}\\(f_2\text{是}x_2\text{到}y_2\text{的双射})\text{与}\cdots \text{与}(f_n\text{是}x_n\text{到}y_n\text{的双射})\Rightarrow (R\Leftrightarrow (y_1|x_1)(y_2|x_2)\cdots (y_n|x_n)\\({s'}_1|s_1)({s'}_2|s_2)\cdots ({s'}_p|s_p)R)$,其中${s'}_j=\langle f_1, f_2, \cdots, f_n, Id_{A_1}, Id_{A_2}, \cdots, Id_{A_m}\rangle ^{S_j}(s_j)$\\($j\in [1, p]$).
			\end{STdef}
						
			\begin{STdef}
				\textbf{结构种类(espèce de structure),结构种类的主要基集合(ensemble de base principal de l'espèce de structure),结构种类的主要基集合(ensemble de base auxiliaire de l'espèce de structure),结构种类的代表特征(caractérisation typique de l'espèce de structure),结构种类的公理(axiome de l'espèce de structure)
				}
				\par
				令$M$为比集合论强的理论,满足下列条件的文本,称为结构种类,其中$x_1$、$x_2$、$\cdots$、$x_n$称为结构种类的主要基集合,$A_1$、$A_2$、$\cdots$、$A_m$称为结构种类的辅助基集合,$T$称为结构种类的代表特征,$R$称为结构种类的公理:
				\par
				第一,$x_1$、$x_2$、$\cdots$、$x_n$、$s$为互不相同的字母,且都不是$M$的常数;
				\par
				第二,$A_1$、$A_2$、$\cdots$、$A_m$为$M$的项,并且均不包含$x_1$、$x_2$、$\cdots$、$x_n$、$s$(其中$m$也可以为$0$);
				\par
				第三,$T$为$s$的类型化;
				\par
				第四,公式$R$对$T$是可转换的.
			\end{STdef}
			注:在原书中,主要讨论$T$为$s$的类型化(即$p=1$)的情况,$T$为多个字母的类型化(即$p>1$)的情况,用同样的方法也可以得到类似的结论.

			\begin{STdef}
				\textbf{通用结构(structure générique)}
				\par
				令$M$为比集合论强的理论,$X$为结构种类,代表特征为$s\in S(x_1, x_2, \cdots, x_n, A_1, A_2, \cdots, \\A_m)$,则在理论$M_X$中,常数$s$称为通用结构.
			\end{STdef}
			
			\begin{STdef}
				\textbf{偏序结构种类(espèce de structure d'ordre)
				}
				\par
				以$A$为主要基集合、没有辅助基集合的结构种类如果满足下列条件,则称为偏序结构种类:
				\par
				第一,结构种类的代表特征为$s\in \mathcal{P}(A\times A)$;
				\par
				第二,结构种类的公理为$(s\circ s=s)\text{与}(s\cap s^{-1}=\Delta_A)$.
			\end{STdef}
						
			\begin{STdef}
				\textbf{代数结构种类(espèce de structure algébriques)}
				\par
				以$A$为主要基集合、没有辅助基集合的结构种类如果满足下列条件,则称为偏序结构种类:
				\par
				第一,结构种类的代表特征为$F\in \mathcal{P}((A\times A)\times A)$;
				\par
				第二,结构种类的公理为“$F\text{是定义域为}A\times A\text{的函数图}$”.
			\end{STdef}
			
			\begin{STdef}
				\textbf{拓扑结构种类(espèce de structure topologique)}
				\par
				以$A$为主要基集合、没有辅助基集合的结构种类如果满足下列条件,则称为偏序结构种类:
				\par
				第一,结构种类的代表特征为$V\in \mathcal{P}(\mathcal{P}(A))$;
				\par
				第二,结构种类的公理为$(\forall V')((V'\subset V)\Rightarrow ((\bigcup\limits_{X\in V'}X)\in V))\text{与}(A\in V)\text{与}(\forall X)(\forall Y)\\((X\in V\text{与}Y\in V)\Rightarrow (X\cap Y\in V))$.
			\end{STdef}
						
			\begin{STdef}
				\textbf{结构种类的理论(théorie de espèce de structure)}
				\par
				令$M$为比集合论强的理论,$X$为结构种类,满足下列条件的理论,称为结构种类$X$的理论,记作$M_X$:
				\par
				第一,显式公理包括$M$的显式公理和“$R\text{与}T$";
				\par
				第二,公理模式、特别符号均和$M$相同.
			\end{STdef}
			
			\begin{STdef}
				\textbf{结构(structure),具有结构(munis de la structure)}
				\par
				令$M$为比集合论强的理论,$X$为结构种类,其代表特征为$T$,公理为$R$.令$M'$为比$M$强的理论,$E_1$、$E_2$、$\cdots$、$E_n$、$U$为$M'$的项,且“$(E_1|x_1)(E_2|x_2)\cdots (E_n|x_n)(U|s)R\text{与}(E_1|x_1)\\(E_2|x_2)\cdots (E_n|x_n)(U|s)T$”是$M'$的定理,则称在理论$M'$中$U$为$X$在主要基集合$E_1$、$E_2$、$\cdots$、$E_n$和辅助基集合$A_1$、$A_2$、$\cdots$、$A_m$上的结构,在没有歧义的情况下也可以简称在理论$M'$中\\$U$为$X$在主要基集合$E_1$、$E_2$、$\cdots$、$E_n$上的结构,或者简称$U$为$X$在主要基集合$E_1$、$E_2$、$\cdots$、$E_n$上的结构.同时,称在理论$M'$中$E_1$、$E_2$、$\cdots$、$E_n$具有$X$的结构$U$,在没有歧义的情况下也可以简称$E_1$、$E_2$、$\cdots$、$E_n$具有$X$的结构$U$,或简称$E_1$、$E_2$、$\cdots$、$E_n$具有结构$U$.
			\end{STdef}
			
			\begin{CSTcor}\label{CSTcor2}
				\hfill\par
				令$M$为比集合论强的理论,$X$为结构种类,其代表特征为$T$,公理为$R$.令$M'$为比$M$强的理论,$E_1$、$E_2$、$\cdots$、$E_n$、$U$为$M'$的项,在理论$M'$中$U$为$X$在主要基集合$E_1$、$E_2$、$\cdots$、$E_n$和辅助基集合$A_1$、$A_2$、$\cdots$、$A_m$上的结构.则对$M_X$的任意定理$B$,$(E_1|x_1)(E_2|x_2)\cdots\\(E_n|x_n)(U|s)B$\\是$M'$的定理.
			\end{CSTcor}
			证明:根据证明规则\ref{C2}可证.
			
			\begin{CSTcor}\label{CSTcor3}
				\hfill\par
				令$M$为比集合论强的理论,$X$为结构种类,其代表特征为$T$,公理为$R$,$S$为其阶梯构造模式.令$M'$为比$M$强的理论,在理论$M'$中$U$为$X$在主要基集合$E_1$、$E_2$、$\cdots$、$E_n$和辅助基集合$A_1$、$A_2$、$\cdots$、$A_m$上的结构.则在理论$M'$中,公式“$U\text{为在理论}M'\text{中}X\text{在主要基集合}E_1\text{、}\\E_2\text{、}\cdots\text{、}E_n\text{上的结构}$”为$U$上的集合化公式.
			\end{CSTcor}
			证明:在理论$M'$中,根据定义,$U\in S(E_1, E_2, \cdots, E_n, A_1, A_2, \cdots, A_m)$,根据证明规则\ref{C52}可证.
			
			\begin{STdef}
				\textbf{结构种类的结构集合(ensemble des structures d'espèce)}
				\par
				令$M$为比集合论强的理论,$X$为结构种类,其代表特征为$T$,公理为$R$,$S$为其阶梯构造模式.令$M'$为比$M$强的理论,则在理论$M'$中,$\{U|U\text{为在理论}M'\text{中}X\text{在主要基集合}E_1\text{、}\\E_2\text{、}\cdots\text{、}E_n\text{上的结构}\}$称为结构种类$X$在主要基集合$E_1$、$E_2$、$\cdots$、$E_n$上的结构集合.
			\end{STdef}
						
			\begin{STdef}
				\textbf{结构的同构(isomorphisme de structures)}
				\par
				令$M$为比集合论强的理论,$X$是理论$M$的结构种类,其主要基集合为$x_1$、$x_2$、$\cdots$、$x_n$,辅助基集合为$A_1$、$A_2$、$\cdots$、$A_m$.$S$为$X$的代表特征中的在$n+m$个项上的阶梯构造模式,$R$为$X$的公理.令$M'$为比$M$强的理论,在理论$M'$中,$U$为$X$在主要基集合$E_1$、$E_2$、$\cdots$、$E_n$\\上的结构,$U'$为$X$在主要基集合${E'}_1$、${E'}_2$、$\cdots$、${E'}_n$上的结构.对任意$i\in [1, n]$,令$f_i$为$E_i$到\\${E'}_i$的双射,如果$\langle f_1, f_2, \cdots, f_n, Id_{A_1}, Id_{A_2}, \cdots, Id_{A_m}\rangle^S(U)=U'$,则称$(f_1, f_2, \cdots, f_n)$为具有结构$U$的集合$E_1$、$E_2$、$\cdots$、$E_n$到具有结构$U'$的集合${E'}_1$、${E'}_2$、$\cdots$、${E'}_n$的同构,在没有歧义的情况下也可以称$(f_1, f_2, \cdots, f_n)$为集合$E_1$、$E_2$、$\cdots$、$E_n$到集合${E'}_1$、${E'}_2$、$\cdots$、${E'}_n$的同构、或称$(f_1, f_2, \cdots, f_n)$为结构$U$到结构$U'$的同构,或称$(f_1, f_2, \cdots, f_n)$为同构,或者称具有结构$U$的集合$E_1$、$E_2$、$\cdots$、$E_n$同构于具有结构$U'$的集合${E'}_1$、${E'}_2$、$\cdots$、${E'}_n$.
			\end{STdef}
			
			\begin{CSTcor}\label{CSTcor4}
				\textbf{逆同构的存在}
				\par
				令$M$为比集合论强的理论,$X$是理论$M$的结构种类,其主要基集合为$x_1$、$x_2$、$\cdots$、$x_n$,辅助基集合为$A_1$、$A_2$、$\cdots$、$A_m$.$S$为$X$的代表特征中的在$n+m$个项上的阶梯构造模式,$R$为$X$的公理.令$M'$为比$M$强的理论,在理论$M'$中,$U$为$X$在主要基集合$E_1$、$E_2$、$\cdots$、$E_n$\\上的结构,$U'$为$X$在主要基集合${E'}_1$、${E'}_2$、$\cdots$、${E'}_n$上的结构.对任意$i\in [1, n]$,令$f_i$为$E_i$到\\${E'}_i$的双射,如果$\langle f_1, f_2, \cdots, f_n, Id_{A_1}, Id_{A_2}, \cdots, Id_{A_m}\rangle^S(U)=U'$,则$\langle {f_1}^{-1}, {f_2}^{-1}, \cdots, {f_n}^{-1}, \\Id_{A_1}, Id_{A_2}, \cdots, Id_{A_m}\rangle^S(U')=U$.
			\end{CSTcor}
			证明:根据补充结构规则\ref{CSTcor3}可证.

			\begin{STdef}
				\textbf{结构的逆同构(isomorphisme réciproques de structures)}
				\par
				令$M$为比集合论强的理论,$X$是理论$M$的结构种类,其主要基集合为$x_1$、$x_2$、$\cdots$、$x_n$,辅助基集合为$A_1$、$A_2$、$\cdots$、$A_m$.$S$为$X$的代表特征中的在$n+m$个项上的阶梯构造模式,$R$为$X$的公理.令$M'$为比$M$强的理论,在理论$M'$中,$U$为$X$在主要基集合$E_1$、$E_2$、$\cdots$、$E_n$\\上的结构,$U'$为$X$在主要基集合${E'}_1$、${E'}_2$、$\cdots$、${E'}_n$上的结构.对任意$i\in [1, n]$,令$f_i$为$E_i$到\\${E'}_i$的双射,如果$\langle f_1, f_2, \cdots, f_n, Id_{A_1}, Id_{A_2}, \cdots, Id_{A_m}\rangle^S(U)=U'$,则称$({f_1}^{-1}, {f_2}^{-1}, \cdots, {f_n}^{-1})$\\为$(f_1, f_2, \cdots, f_n)$的逆同构.
			\end{STdef}

			\begin{CSTcor}\label{CSTcor5}
				\hfill\par
				$(f_1, f_2, \cdots, f_n)$为同构,如果$({f_1}^{-1}, {f_2}^{-1}, \cdots, {f_n}^{-1})$为$(f_1, f_2, \cdots, f_n)$的逆同构,则\\$(f_1, f_2, \cdots, f_n)$为$({f_1}^{-1}, {f_2}^{-1}, \cdots, {f_n}^{-1})$的逆同构.
			\end{CSTcor}
			证明:根据定义可证.
		
			\begin{CSTcor}\label{CSTcor6}
				\hfill\par
				令$M$为比集合论强的理论,$X$是理论$M$的结构种类,其主要基集合为$x_1$、$x_2$、$\cdots$、$x_n$,辅助基集合为$A_1$、$A_2$、$\cdots$、$A_m$.$S$为$X$的代表特征中的在$n+m$个项上的阶梯构造模式,$R$为$X$的公理.令$M'$为比$M$强的理论,在理论$M'$中,$U$为$X$在主要基集合$E_1$、$E_2$、$\cdots$、$E_n$\\上的结构.则$(Id_{E_1}, Id_{E_2}, \cdots, Id_{E_n})$为结构$U$到结构$U$的同构.
			\end{CSTcor}
			证明:根据补充结构规则\ref{CSTcor1}可证.
			
			\begin{CST}\label{CST4}
				\textbf{同构的复合为同构}
				\par
				令$M$为比集合论强的理论,$X$是理论$M$的结构种类.令$M'$为比$M$强的理论,在理论$M'$\\中,$U$、$U'$、$U''$分别为结构种类$X$在主要基集合$E_1$、$E_2$、$\cdots$、$E_n$上、在主要基集合${E'}_1$、${E'}_2$、$\cdots$、${E'}_n$\\上、在主要基集合${E''}_1$、${E''}_2$、$\cdots$、${E''}_n$上的结构,对任意$i\in [1, n]$,$f_i$为$E_i$到${E'}_i$的双射,$g_i$为${E'}_i$到${E''}_i$的双射,如果$(f_1, f_2, \cdots, f_n)$、$(g_1, g_2, \cdots, g_n)$均为同构,则$(g_1\circ f_1, g_2\circ f_2, \cdots, \\g_n\circ f_n)$为同构.
			\end{CST}
			证明:根据结构规则\ref{CST1}可证.
						
			\begin{STdef}
				\textbf{同构的复合(composée de deux isomorphismes)}
				\par
				令$M$为比集合论强的理论,$X$是理论$M$的结构种类.令$M'$为比$M$强的理论,在理论$M'$\\中,$U$、$U'$、$U''$分别为结构种类$X$在主要基集合$E_1$、$E_2$、$\cdots$、$E_n$上、在主要基集合${E'}_1$、${E'}_2$、$\cdots$、${E'}_n$上、在主要基集合${E''}_1$、${E''}_2$、$\cdots$、${E''}_n$上的结构,对任意$i\in [1, n]$,$f_i$为$E_i$到\\${E'}_i$的双射,$g_i$为${E'}_i$到${E''}_i$的双射,如果$(f_1, f_2, \cdots, f_n)$、$(g_1, g_2, \cdots, g_n)$均为同构,则称$(g_1\circ f_1, g_2\circ f_2, \cdots, g_n\circ f_n)$为$(f_1, f_2, \cdots, f_n)$和$(g_1, g_2, \cdots, g_n)$的复合.
			\end{STdef}
			
			\begin{STdef}
				\textbf{自同构(automorphisme)}
				\par
				令$M$为比集合论强的理论,$X$是理论$M$的结构种类.令$M'$为比$M$强的理论,在理论$M'$\\中,具有$X$的结构$U$的集合$E_1$、$E_2$、$\cdots$、$E_n$到具有$X$的结构$U$的集合$E_1$、$E_2$、$\cdots$、$E_n$的同构,称为自同构.
			\end{STdef}
			
			\begin{CSTcor}\label{CSTcor7}
				\hfill\par
				自同构的复合是自同构.自同构的逆同构是自同构.
			\end{CSTcor}
			证明:根据定义可证.
			
			\begin{CST}\label{CST5}
				\textbf{通过转换可以得到结构}
				\par
				令$M$为比集合论强的理论,$X$是理论$M$的结构种类.令$M'$为比$M$强的理论,在理论$M'$\\中,$U$为$X$在主要基集合$E_1$、$E_2$、$\cdots$、$E_n$上的结构,对任意$i\in [1, n]$,$f_i$为$E_i$到${E'}_i$的双射,则在理论$M'$中存在$X$在主要基集合${E'}_1$、${E'}_2$、$\cdots$、${E'}_n$上的结构$U'$,使$(f_1, f_2, \cdots, f_n)$为集合$E_1$、$E_2$、$\cdots$、$E_n$到集合${E'}_1$、${E'}_2$、$\cdots$、${E'}_n$的同构.
			\end{CST}
			证明:令$U'=\langle f_1, f_2, \cdots, f_n, Id_{A_1}, Id_{A_2}, \cdots, Id_{A_m}\rangle^S(U)$,由于$R$是可转换的,因此, $(E_1|x_1)(E_2|x_2)\cdots(E_n|x_n)(U|s)R\Leftrightarrow ({E'}_1|x_1)({E'}_2|x_2)\cdots({E'}_n|x_n)(U'|s)R$,得证.
						
			\begin{STdef}
				\textbf{通过转换得到的结构(structure obtenue en transportant)}
				\par
				令$M$为比集合论强的理论,$X$是理论$M$的结构种类.令$M'$为比$M$强的理论,在理论$M'$\\中,$U$为$X$在主要基集合$E_1$、$E_2$、$\cdots$、$E_n$上的结构,对任意$i\in [1, n]$,$f_i$为$E_i$到${E'}_i$的双射,令$U'=\langle f_1, f_2, \cdots, f_n, Id_{A_1}, Id_{A_2}, \cdots, Id_{A_m}\rangle^S(U)$,则称$U'$为$U$通过映射$f_1$、$f_2$、$\cdots$、$f_n$的转换在${E'}_1$、${E'}_2$、$\cdots$、${E'}_n$上得到的结构.
			\end{STdef}
			
			\begin{STdef}
				\textbf{统一的结构种类(espèce de structure univalente)}
				\par
				如果结构种类的任何两个结构都存在同构,则称该结构种类为统一的.
			\end{STdef}
						
			\begin{STdef}
				\textbf{固有项(terme intrinsèque)}
				\par
				令$M$为比集合论强的理论,$X$是理论$M$的结构种类,其主要基集合为$x_1$、$x_2$、$\cdots$、$x_n$,辅助基集合为$A_1$、$A_2$、$\cdots$、$A_m$,通用结构为常数$s$.$S$是在$n+m$个项上的阶梯构造模式.如果项$V$满足下列条件,则称$V$对于常数$s$是固有的:
				\par
				第一,$V$包含的字母都是理论$M_X$的常数;
				\par
				第二,$V\in S(x_1, x_2, \cdots, x_n, A_1, A_2, \cdots, A_m)$是理论$M_X$的定理;
				\par
				第三,令${M'}_X$为理论$M_X$添加公理“$f_i\text{是}x_i\text{到}y_i\text{的双射}$”($i\in [1, n]$)得到的理论,并且所有的字母$f_i$、$y_i$都不是理论$M_X$的常量且互不相同,$s'$为$s$通过映射$f_1$、$f_2$、$\cdots$、$f_n$的转换在$y_1$、$y_2$、$\cdots$、$y_n$上得到的结构,并且$(y_1|x_1)(y_2|x_2)\cdots(y_n|x_n)(s'|s)V=\langle f_1, f_2, \cdots, f_n, \\Id_{A_1}, Id_{A_2}, \cdots, Id_{A_m}\rangle^S(V)$是理论${M'}_X$的定理.
			\end{STdef}
			
			\begin{STdef}
				\textbf{演绎过程(procédé de déduction)}
				\par
				令$M$为比集合论强的理论,$X$是理论$M$的结构种类,其主要基集合为$x_1$、$x_2$、$\cdots$、$x_n$,辅助基集合为$A_1$、$A_2$、$\cdots$、$A_m$,其代表特征是$s\in S(x_1, x_2, \cdots, x_n, A_1, A_2, \cdots, A_m)$;$Y$也是理论$M$的结构种类,其主要基集合为$y_1$、$y_2$、$\cdots$、$y_r$,辅助基集合为$B_1$、$B_2$、$\cdots$、$B_p$,其代表特征是$t\in T(y_1, y_2, \cdots, y_r, B_1, B_2, \cdots, B_p)$,其公理不包含字母$x_1$、$x_2$、$\cdots$、$x_n$、$s$.
				\par
				如果在理论$M_X$中$P$为$Y$在主要基集合$U_1$、$U_2$、$\cdots$、$U_r$上的结构,且$P$、$U_1$、$U_2$、$\cdots$、$U_r$对于常数$s$是固有的,则称$P$、$U_1$、$U_2$、$\cdots$、$U_r$为从$X$的结构到$Y$的结构的演绎过程,在没有歧义的情况下,也可以称$P$为从$X$的结构到$Y$的结构的演绎过程,或者称$P$为演绎过程.
			\end{STdef}

			\begin{CSTcor}\label{CSTcor8}
				\textbf{从过程中演绎得到结构}
				\par
				令$M$为比集合论强的理论,$X$是理论$M$的结构种类,其主要基集合为$x_1$、$x_2$、$\cdots$、$x_n$,辅助基集合为$A_1$、$A_2$、$\cdots$、$A_m$,其代表特征是$s\in S(x_1, x_2, \cdots, x_n, A_1, A_2, \cdots, A_m)$;$Y$也是理论$M$的结构种类,其主要基集合为$y_1$、$y_2$、$\cdots$、$y_r$,辅助基集合为$B_1$、$B_2$、$\cdots$、$B_p$,其代表特征是$t\in T(y_1, y_2, \cdots, y_r, B_1, B_2, \cdots, B_p)$,其公理不包含字母$x_1$、$x_2$、$\cdots$、$x_n$、$s$.
				\par
				如果在理论$M_X$中$P$为$Y$在主要基集合$U_1$、$U_2$、$\cdots$、$U_r$上的结构,令$M'$为比$M$强的理论,在理论$M'$中,$K$为$X$在主要基集合$E_1$、$E_2$、$\cdots$、$E_n$上的结构,则$(E_1|x_1)(E_2|x_2)\cdots\\(E_n|x_n)(K|s)P$为$Y$在主要基集合${U'}_1$、${U'}_2$、$\cdots$、${U'}_r$上的结构,其中,对任意$j\in [1, r]$,${U'}_j$\\为$(E_1|x_1)(E_2|x_2)\cdots(E_n|x_n)(K|s)U_j$.
			\end{CSTcor}
			证明:根据替代规则\ref{CS2}可证.
						
			\begin{STdef}
				\textbf{从过程中演绎(déduite par le procédé),从属于结构(subordonnée à le structure)}
				\par
				令$M$为比集合论强的理论,$X$是理论$M$的结构种类,其主要基集合为$x_1$、$x_2$、$\cdots$、$x_n$,辅助基集合为$A_1$、$A_2$、$\cdots$、$A_m$,其代表特征是$s\in S(x_1, x_2, \cdots, x_n, A_1, A_2, \cdots, A_m)$;$Y$也是理论$M$的结构种类,其主要基集合为$y_1$、$y_2$、$\cdots$、$y_r$,辅助基集合为$B_1$、$B_2$、$\cdots$、$B_p$,其代表特征是$t\in T(y_1, y_2, \cdots, y_r, B_1, B_2, \cdots, B_p)$,其公理不包含字母$x_1$、$x_2$、$\cdots$、$x_n$、$s$.
				\par
				如果$P$、$U_1$、$U_2$、$\cdots$、$U_r$为从$X$的结构到$Y$的结构的演绎过程,令$M'$为比$M$强的理论,在理论$M'$中,$K$为$X$在主要基集合$E_1$、$E_2$、$\cdots$、$E_n$上的结构,则结构$(E_1|x_1)(E_2|x_2)\\\cdots(E_n|x_n)(K|s)P$称为$K$从过程$P$演绎所得,或称其从属于$K$.
			\end{STdef}
			
			\begin{CST}\label{CST6}
				\hfill\par
				令$M$为比集合论强的理论,$X$是理论$M$的结构种类,其主要基集合为$x_1$、$x_2$、$\cdots$、$x_n$,辅助基集合为$A_1$、$A_2$、$\cdots$、$A_m$,其代表特征是$s\in S(x_1, x_2, \cdots, x_n, A_1, A_2, \cdots, A_m)$;$Y$也是理论$M$的结构种类,其主要基集合为$y_1$、$y_2$、$\cdots$、$y_r$,辅助基集合为$B_1$、$B_2$、$\cdots$、$B_p$,其代表特征是$t\in T(y_1, y_2, \cdots, y_r, B_1, B_2, \cdots, B_p)$,其公理不包含字母$x_1$、$x_2$、$\cdots$、$x_n$、$s$.
				\par
				$P$、$U_1$、$U_2$、$\cdots$、$U_r$为从$X$的结构到$Y$的结构的演绎过程,其中,对任意$j\in [1, r]$,和$U_j$相关的阶梯构造模式为$\mathcal{P}(T_j)$.令$M'$为比$M$强的理论,在理论$M'$中,$(g_1, g_2, \cdots, g_n)$为具有结构$K$的集合$E_1$、$E_2$、$\cdots$、$E_n$到具有结构$K'$的集合${E'}_1$、${E'}_2$、$\cdots$、${E'}_n$的同构.对任意$j\in [1, r]$,令$h_j=\langle g_1, g_2, \cdots, g_n, Id_{A_1}, Id_{A_2}, \cdots, Id_{A_m}\rangle^{T_j}$,$F_j=(E_1|x_1)( E_2|x_2)\cdots(E_n|x_n)\\(K'|s)U_j$,${F'}_j=({E'}_1|x_1)({E'}_2|x_2)\cdots({E'}_n|x_n)(K'|s)U_j$,令$Q$、$Q'$分别从属于结构$K$、$K'$,则\\$(h_1, h_2, \cdots, h_r)$为具有结构$(E_1|x_1)( E_2|x_2)\cdots(E_n|x_n)(K'|s)P$的集合$F_1$、$F_2$、$\cdots$、$F_r$到具有结构$({E'}_1|x_1)({E'}_2|x_2)\cdots({E'}_n|x_n)(K'|s)P$的集合${F'}_1$、${F'}_2$、$\cdots$、${F'}_r$的同构.
			\end{CST}
			证明:根据结构规则\ref{CST2},$h_i$为双射,$\langle h_1, h_2, \cdots, h_r, Id_{B_1}, Id_{B_2}, \cdots, Id_{B_p}\rangle^T$也是双射.由于$U_i$对于常数$s$是固有的,故$h_i\langle F_i\rangle ={F'}_i$.由于$P$对于常数$s$是固有的,故$({E'}_1|x_1)({E'}_2|x_2)\cdots\\({E'}_n|x_n)(K'|s)P =\langle h_1, h_2, \cdots, h_r, Id_{B_1}, Id_{B_2}, \cdots, Id_{B_p}>^T((E_1|x_1)( E_2|x_2)\cdots(E_n|x_n)\\(K'|s)P)$,得证.
			
			\begin{STdef}
				\textbf{等价的结构种类(espèce de structure équivalente)}
				\par
				令$M$为比集合论强的理论,$X$、$Y$是理论$M$的结构种类,其主要基集合均为$x_1$、$x_2$、$\cdots$、$x_n$,通用结构分别为$s$、$t$.令$P$为从$X$的结构到$Y$的结构的演绎过程,$Q$为从$Y$的结构到$X$的结构的演绎过程,在理论$M_X$中,$(P|s)Q=t$是定理,在理论$M_Y$中,$(Q|t)P=s$是定理,则称结构种类$X$和$Y$通过演绎过程$P$和$Q$等价.
			\end{STdef}
			
			\begin{CST}\label{CST7}
				\hfill\par
				令$M$为比集合论强的理论,$X$是理论$M$的结构种类,其主要基集合为$x_1$、$x_2$、$\cdots$、$x_n$,$K$为$X$在主要基集合$E_1$、$E_2$、$\cdots$、$E_n$上的结构,$K'$为$X$在主要基集合${E'}_1$、${E'}_2$、$\cdots$、${E'}_n$上的结构.$K_0$、${K'}_0$均为结构种类$Y$的结构,并且分别和$K$、$K'$等价,则当且仅当\\$(g_1, g_2, \cdots, g_n)$为$K$到$K'$的同构时,$(g_1, g_2, \cdots, g_n)$为$K_0$到${K'}_0$的同构.
			\end{CST}
			证明:根据结构规则\ref{CST6}可证.
			
			\begin{exer}\label{exer198}
				\hfill\par
				令$S$为符号$P$、$X$、$x_1$、$x_2$、$\cdots$、$x_n$组成的集合,$P$的权重为$1$,$X$的权重为$2$,其他符号的权重为$0$.如果$L_0(S)$的单词$T$是平衡单词,则称$T$为在$x_1$、$x_2$、$\cdots$、$x_n$上的阶梯类.令$M$为比集合论强的理论,$E_1$、$E_2$、$\cdots$、$E_n$为$M$的项,定义$T(E_1, E_2, \cdots, E_n)$如下:
				\par
				第一,如果$T$为字母$x_i$,则$T(E_1, E_2, \cdots, E_n)$为集合$E_i$;
				\par
				第二,如果$T$为$PU$的形式,则$T(E_1, E_2, \cdots, E_n)$为集合$\mathcal{P}(U(E_1, E_2, \cdots, E_n))$;
				\par
				第三,如果$T$为$XUV$的形式,则$T(E_1, E_2, \cdots, E_n)$为集合$U(E_1, E_2, \cdots, E_n)\times\\V(E_1, E_2, \cdots, E_n)$.
				\par
				求证:对任意在$x_1$、$x_2$、$\cdots$、$x_n$上的阶梯类$T$,$T(E_1, E_2, \cdots, E_n)$是在$E_1$、$E_2$、$\cdots$、$E_n$上的阶梯,反之,任何在$E_1$、$E_2$、$\cdots$、$E_n$上的阶梯,都可以用唯一的方法表示为\\$T(E_1, E_2, \cdots, E_n)$.此时,称$T(E_1, E_2, \cdots, E_n)$为阶梯类$T$在$E_1$、$E_2$、$\cdots$、$E_n$上的实现.进而,试用在$x_1$、$x_2$、$\cdots$、$x_n$上的阶梯类$T$表示映射的规范扩展,并证明,对于在$n$个项上的阶梯构造模式$S$和$S'$,如果$S(x_1, x_2, \cdots, x_n)=S'(x_1, x_2, \cdots, x_n)$,则$\langle f_1, f_2, \cdots, f_n\rangle^S=\langle f_1, f_2, \cdots, f_n\rangle^{S'}$.
			\end{exer}
			证明:
			\par
			用数学归纳法可证$T(E_1, E_2, \cdots, E_n)$和阶梯可相互表达并具有唯一性;
			\par
			映射的规范扩展定义为:
			\par
			第一,$T$为字母$x_i$,则相应的映射为$f_i$;
			\par
			第二,$T$为$PU$的形式,则相应的映射为$U$相应的映射在子集上的规范扩展;
			\par
			第三,$T$为$XUV$的形式,则相应的映射为$U$和$V$相应的映射在乘积集合上的规范扩展.
			\par
			用数学归纳法可以证明映射的规范扩展用$T(E_1, E_2, \cdots, E_n)$表达的唯一性.故如果\\$S(x_1, x_2, \cdots, x_n)=S'(x_1, x_2, \cdots, x_n)$,则$\langle f_1, f_2, \cdots, f_n\rangle^S=\langle f_1, f_2, \cdots, f_n\rangle^{S'}$.

		\section{态射和派生结构(Morphismes et structures dérivées)}		
			\begin{STdef}
				\textbf{态射集合(ensemble des morphismes),态射(morphisme)}
				\par
				令$M$为比集合论强的理论,$X$是仅有一个基集合的结构种类.$\sigma$为项,字母$x$、$y$、$s$、$t$互不相同,且都不是X的代表特征和公理包含的字母.
				\par
				在理论$M$中,如果项$\sigma$满足下列条件,则称$\sigma$为$X$的态射集合:
				\par
				第一,在理论$M$中,$((s\text{是}X\text{在主要基集合}x\text{上的结构})\text{与}(t\text{是}X\text{在主要基集合}y\text{上的结构}))\\\Rightarrow (\sigma\subset \mathcal{F}(x; y))$;
				\par
				第二,令$M'$为比$M$强的理论,在理论$M'$中,$E$、$E'$、$E''$在理论$M'$中分别具有$X$的结构$K$、$K'$、$K''$,$E$、$E'$、$E''$、$K$、$K'$、$K''$都不含字母$s$、$t$、$x$、$y$,则$(f\in (K'|t)(K|s)(E'|y)\\(E|x)\sigma\text{与}g\in (K''|t)(K'|s)(E''|y)(E'|x)\sigma)\Rightarrow g\circ f\in (K''|t)(K|s)(E''|y)(E|x)\sigma$;
				\par
				第三,令$M'$为比$M$强的理论,在理论$M'$中,$E$、$E'$在理论$M'$中分别具有$X$的结构$K$、$K'$,$E$、$E'$、$K$、$K'$都不含字母$s$、$t$、$x$、$y$,$f$为$E$到$E'$的双射,则$((f)\text{为同构})\Leftrightarrow (f\in (K'|t)(K|s)(E'|y)(E|x)\sigma\text{与}f^{-1}\in (K|t)(K'|s)(E|y)(E'|x)\sigma)$.
				\par
				此时,令$M'$为比$M$强的理论,在理论$M'$中,对具有结构$K$的集合$E$、具有结构$K'$的集合$E'$,对任意$f\in (K'|t)(K|s)(E'|y)(E|x)\sigma$,称$f$为具有$K$的$E$到具有$K'$的$E'$的态射,在没有歧义的情况下,可以简称$f$为$E$到$E'$的$\sigma$态射,或$f$为$E$到$E'$的态射,或$f$为$\sigma$态射:
			\end{STdef}
			注:原书主要研究仅有一个基集合的结构种类的态射问题.对于多个基集合的结构种类的态射,也可用类似的方法下定义.

			\begin{CSTcor}\label{CSTcor9}
				\textbf{恒等映射为态射}
				\par
				令$M$为比集合论强的理论,$X$是仅有一个基集合的结构种类,$\sigma$为$X$的态射集合.令$M'$\\为比$M$强的理论,在理论$M'$中,$E$具有$X$的结构,则$Id_E$是$\sigma$态射.
			\end{CSTcor}
			证明:根据根据补充结构规则\ref{CSTcor6},$(f)$为同构,根据定义可证.
			
			\begin{STdef}
				\textbf{满态射(morphisme surjectif)}
				\par
				令$M$为比集合论强的理论,$X$是仅有一个基集合的结构种类,$\sigma$为$X$的态射集合.令$M'$\\为比$M$强的理论,在理论$M'$中,$E$、$E'$分别具有$X$的结构$U$、$U'$,$f$为$E$到$E'$的$\sigma$态射.如果\\$\langle f\rangle^S(U)=U'$,则称$f$为满态射.
			\end{STdef}
			
			\begin{CST}\label{CST8}
				\hfill\par
				令$M$为比集合论强的理论,$X$是仅有一个基集合的结构种类,$\sigma$为$X$的态射集合.令$M'$\\为比$M$强的理论,在理论$M'$中,$E$、$E'$为具有$X$的结构的集合,$f$为$E$到$E'$的$\sigma$态射,$g$为$E'$\\到$E$的$\sigma$态射,如果$g\circ f=Id_E$,$f\circ g=Id_{E'}$,则$(f)$为$E$到$E'$的同构,$(g)$为$(f)$的逆同构.
			\end{CST}
			证明:根据定理\ref{theo20}可证.
						
			\begin{STdef}
				\textbf{更细的结构(structure plus fine),更粗的结构(structure plus fine)}
				\par
				令$M$为比集合论强的理论,$X$是仅有一个基集合的结构种类,$\sigma$为$X$的态射集合.令$M'$\\为比$M$强的理论,在理论$M'$中,$K_1$、$K_2$为$X$在集合$E$上的结构,如果具有$K_1$的$E$到具有$K_2$\\的$E$的恒等映射是$\sigma$态射,则称$K_1$是比$K_2$更细的结构,或称$K_2$是比$K_1$更粗的结构.
			\end{STdef}
			注:在原书中,“更细”这个概念包括与自身相等的情况,即一个结构比自身更细.
			
			\begin{CSTcor}\label{CSTcor10}
				\hfill\par
				令$M$为比集合论强的理论,$X$是仅有一个基集合的结构种类,$\sigma$为$X$的态射集合.令$M'$\\为比$M$强的理论,则在理论$M'$中,$(K_1\text{为}X\text{在集合}E\text{上的结构})\text{与}(K_2\text{为}X\text{在集合}E\text{上的结构})\\\text{与}(K_1\text{比}K_2\text{更细})$是在$X$在$E$上的结构集合上的偏序关系.
			\end{CSTcor}
			证明:根据定义可证.
			
			\begin{STdef}
				\textbf{起始结构(structure initiale)}
				\par
				令$M$为比集合论强的理论,$X$是仅有一个基集合的结构种类,$\sigma$为$X$的态射集合.令$M'$\\为比$M$强的理论,在理论$M'$中,$(A_i)_{i\in I}$为集族,$E$为集合,并且,对任意$i\in I$,$K_i$为$X$在$A_i$\\上的结构,$f_i$为$E$到$A_i$的映射.对于$X$在$E$上的结构$F$,如果对任意集合$E'$、$X$在$E'$上的结构$F'$、$E'$到$E$的映射$g$,均有$(g\text{为}E'\text{到}E\text{的}\sigma\text{态射})\Leftrightarrow (\forall i)(i\in I\Rightarrow f_i\circ g\text{为}E'\text{到}A_i\text{的}\sigma\text{态射})$,则称$F$为关于三元组族$(A_i, K_i, f_i)_{i\in I}$的起始结构.
			\end{STdef}
			
			\begin{CSTcor}\label{CSTcor11}
				\hfill\par
				令$M$为比集合论强的理论,$X$是仅有一个基集合的结构种类,$\sigma$为$X$的态射集合.令$M'$\\为比$M$强的理论,在理论$M'$中,$(A_i)_{i\in I}$为集族,$E$为集合,并且,对任意$i\in I$,$K_i$为$X$在$A_i$\\上的结构,$f_i$为$E$到$A_i$的映射,关于三元组族$(A_i, K_i, f_i)_{i\in I}$的起始结构为$F$,则对任意$i\in I$,$f_i$为具有$F$的$E$到具有$K_i$的$A_i$的$\sigma$态射.
			\end{CSTcor}
			证明:根据补充结构规则\ref{CSTcor9},$Id_E$为$\sigma$态射,根据定义可证.
			
			\begin{CST}\label{CST9}
				\hfill\par
				令$M$为比集合论强的理论,$X$是仅有一个基集合的结构种类,$\sigma$为$X$的态射集合.令$M'$\\为比$M$强的理论,在理论$M'$中,对于集合$E$,如果存在$X$在$E$上的结构$F$,为关于三元组族$(A_i, K_i, f_i)_{i\in I}$的起始结构,那么,对任意$X$在$E$上的结构$F'$,如果对任意$i\in I$,$f_i$为具有$F'$的$E$到具有$K_i$的$A_i$的$\sigma$态射,则$F$比$F'$更粗,进而,$F$是唯一的.
			\end{CST}
			证明:根据定义,对于结构$F$,对任意$i\in I$,$f_i$均为$\sigma$态射.同时,具有结构$F$的$E$的恒等映射$Id_E$是$\sigma$态射,根据定义,$f_i\circ Id_E$为具有$F$的$E$到具有$F'$的$E$的$\sigma$态射,因此,$F$比$F'$更粗.进而,根据定义,$F$是唯一的.
			
			\begin{CST}\label{CST10}
				\hfill\par
				令$M$为比集合论强的理论,$X$是仅有一个基集合的结构种类,$\sigma$为$X$的态射集合.令$M'$\\为比$M$强的理论,在理论$M'$中,$(Ai)_{i\in I}$为集族,$E$为集合,并且,对任意$i\in I$,$K_i$为$X$在$A_i$\\上的结构.$(J_l)_{l\in L}$是$I$的划分,$(B_l)_{l\in L}$是集族.对任意$l\in L$,$h_l$为$E$到$B_l$的映射;对任意$l\in L$和$i\in J_l$,$g_{li}$为$B_l$到$A_i$的映射,并且$f_i=g_{li}\circ h_l$.如果,对任意$l\in L$,存在$X$在$B_l$上的结构${K'}_l$,为关于三元组族$(A_i, K_i, g_{li})_{i\in J_l}$的起始结构,则下列两个命题等价:
				\par
				第一,存在$X$在$E$上、关于$(A_i, K_i, f_i)_{i\in I}$的起始结构$U$;
				\par
				第二,存在$X$在$E$上,关于$(B_l, {K'}_l, h_l)_{l\in L}$的起始结构$U'$.
				\par
				并且,$U=U'$.
			\end{CST}
			证明:令$F$为具有结构的集合,$v$为$F$到$E$的映射,根据定义,$(h_l\circ v\text{为}F\text{到}B_l\text{的}\sigma\text{态射})\Leftrightarrow (\forall i)(i\in J_l\Rightarrow g_{li}\circ hl\circ v\text{为}F\text{到}A_i\text{的}\sigma\text{态射})$,后者即$(\forall i)(i\in J_l\Rightarrow f_i\circ v\text{为}F\text{到}A_i\text{的}\sigma\text{态射})$.故$(\forall l)(l\in L\Rightarrow hl\circ v\text{为}F\text{到}B_l\text{的}\sigma\text{态射})\Leftrightarrow (\forall i)(i\in I\Rightarrow fi\circ v\text{为}F\text{到}A_i\text{的}\sigma\text{态射})$.
			\par
			因此,两个命题等价,同时,$(v\text{是}F\text{到具有结构}U'\text{的}E\text{的态射})\Leftrightarrow (v\text{是}F\text{到具有结构}U\text{的}\\E\text{的态射})$,根据结构规则\ref{CST9},$U$、$U'$都是唯一的,得证.
						
			\begin{STdef}
				\textbf{结构的原像(image réciproque d'une structure)}
				\par
				令$M$为比集合论强的理论,$X$是仅有一个基集合的结构种类,$\sigma$为$X$的态射集合.令$M'$\\为比$M$强的理论,在理论$M'$中,关于$(A, K, f)_{i\in \{i\}}$的起始结构,称为结构$K$在$f$下的原像.
			\end{STdef}
			
			\begin{STdef}
				\textbf{导出的结构(structure induite)}
				\par
				令$M$为比集合论强的理论,$X$是仅有一个基集合的结构种类,$\sigma$为$X$的态射集合.令$M'$\\为比$M$强的理论,在理论$M'$中,$K$为$X$在$A$上的结构,$B\subset A$,$j$为$B$到$A$的规范映射,如果结构$K$在$j$下的原像存在,则称其为结构$K$在$B$上导出的结构.
			\end{STdef}
						
			\begin{STdef}
				\textbf{可采子集(partie permise)}
				\par
				令$M$为比集合论强的理论,$X$是仅有一个基集合的结构种类,$\sigma$为$X$的态射集合.令$M'$\\为比$M$强的理论,在理论$M'$中,$K$为$X$在$A$上的结构,$B\subset A$,如果$K$在$B$上导出的结构存在,则称$B$为$A$的可采子集.
			\end{STdef}
			
			\begin{CST}\label{CST11}
				\hfill\par
				令$M$为比集合论强的理论,$X$是仅有一个基集合的结构种类,$\sigma$为$X$的态射集合.令$M'$\\为比$M$强的理论,在理论$M'$中,$B\subset A$,$C\subset B$,$K$为$X$在$A$上的结构,$K'$为$K$在$B$上导出的结构,则当且仅当$K'$在$C$上导出的结构存在时,$K$在$C$上导出的结构存在,并且二者相等.
			\end{CST}
			证明:根据结构规则\ref{CST10}可证.
			
			\begin{CST}\label{CST12}
				\hfill\par
				令$M$为比集合论强的理论,$X$是仅有一个基集合的结构种类,$\sigma$为$X$的态射集合,$K$为\\$X$在$A$上的结构,$K'$为$X$在$A'$上的结构,$B\subset A$,$B'\subset A'$.$K$在$B$上导出的结构$J$存在,$K'$在$B'$上的导出的结构$J'$存在.令$F$为$A$到$A'$的$\sigma$态射,且$f\langle B\rangle \subset B'$,令$g=f|B$,则$g$是具有结构$J$的$B$到具有结构$J'$的$B'$的$\sigma$态射.
			\end{CST}
			证明:令$B$到$A$的规范映射为$j$,$B'$到$A'$的规范映射为$j'$,因此$f\circ j=j'\circ g$,由于$f$、$j$为$\sigma$态射,因此$j'\circ g$为$\sigma$态射,根据定义,$g$为$\sigma$态射.

			\begin{STdef}
				\textbf{乘积结构(structure produit)}
				\par
				令$M$为比集合论强的理论,$X$是仅有一个基集合的结构种类,$\sigma$为$X$的态射集合.令$M'$\\为比$M$强的理论,在理论$M'$中,$(A_i)_{i\in I}$为集族,并且,对任意$i\in I$,$K_i$为$X$在$A_i$上的结构.$E=\prod\limits_{i\in I}A_i$,$pr_i$为指标$i$的射影函数,则称关于$(A_i, K_i, pr_i)_{i\in I}$的起始结构为结构族$(K_i)_{i\in I}$的乘积结构.
			\end{STdef}
						
			\begin{STdef}
				\textbf{两个结构的乘积结构(structure produit deux stuctures)}
				\par
				令$M$为比集合论强的理论,$X$是仅有一个基集合的结构种类,$\sigma$为$X$的态射集合.令$M'$\\为比$M$强的理论,在理论$M'$中,$A$、$B$为集合,$K_A$、$K_B$分别为$X$在$A_i$上的结构.令$E=A\times B$,如果$X$在$E$上的结构$K$满足下列条件,则称$K$为$K_A$和$K_B$的乘积结构:
				对任意集合$E'$,令$F'$为$X$在$E'$上的结构,$g$为$E'$到$E$的映射,则$(g\text{为}E'\text{到}E\text{的}\sigma\text{态射})\Leftrightarrow ((pr_1\circ g\text{为}E'\text{到}A\text{的}\sigma\\\text{态射})\text{与}(pr_2\circ g\text{为}E'\text{到}B\text{的}\sigma\text{态射}))$.
			\end{STdef}
			
			\begin{CST}\label{CST13}
				\hfill\par
				令$M$为比集合论强的理论,$X$是仅有一个基集合的结构种类,$\sigma$为$X$的态射集合.令$M'$\\为比$M$强的理论,在理论$M'$中,$(A_i)_{i\in I}$为集族,并且,对任意$i\in I$,令$K_i$为$X$在$A_i$上的结构.$(J_l)_{l\in L}$是$I$的划分.对于$l\in L$,令$B_l=\prod\limits_{i\in J_l}A_i$ ,并且$(K_i)_{i\in J_l}$的乘积结构${K'}_l$存在,$K'$为$({K'}_l)_{l\in L}$的乘积结构,则当且仅当$({K'}_l)_{l\in L}$的乘积结构$K'$存在,并且$\prod\limits_{i\in I}A_i$到$\prod\limits_{l\in L}B_l$的规范映射为同构时,$(K_i)_{i\in I}$的乘积结构$K$存在.
			\end{CST}
			证明:根据结构规则\ref{CST10}可证.
			
			\begin{CST}\label{CST14}
				\hfill\par
				令$M$为比集合论强的理论,$X$是仅有一个基集合的结构种类,$\sigma$为$X$的态射集合.令$M'$\\为比$M$强的理论,在理论$M'$中,$(A_i)_{i\in I}$为集族,并且,对任意$i\in I$,令$K_i$为$X$在$A_i$上的结构,$B_i\subset A_i$,$K_i$在$B_i$上导出的结构为${K'}_i$.$(K_i)_{i\in I}$的乘积结构$K_0$存在,则下列两个命题等价:
				\par
				第一,存在$K_0$在$\prod\limits_{i\in I}B_i$上导出的结构$K$;
				\par
				第二,存在$({K'}_i)_{i\in I}$的乘积结构$K'$.
				\par
				并且,$K=K'$.
			\end{CST}
			证明:令$j_i$为$B_i$到$A_i$的规范映射,$j$为$\prod\limits_{i\in I}B_i$到$\prod\limits_{i\in I}A_i$的规范映射,令$p_i$为$\prod\limits_{i\in I}A_i$的指标$i$的射影函数,${p'}_i$为$\prod\limits_{i\in I}B_i$的指标$i$的射影函数,则$p_i\circ j=j_i\circ {p'}_i$.根据结构规则\ref{CST10},$(A_i, K_i, j_i\circ {p'}_i)$的起始结构为$K'$,$(A_i, K_i, p_i\circ j)$的起始结构为$K$,得证.
			
			\begin{CST}\label{CST15}
				\hfill\par
				令$M$为比集合论强的理论,$X$是仅有一个基集合的结构种类,$\sigma$为$X$的态射集合.令$M'$\\为比$M$强的理论,在理论$M'$中,$(A_i)_{i\in I}$为集族,并且,对任意$i\in I$,令$K_i$为$X$在$A_i$上的结构,$f_i$为$E$到$A_i$的映射,$K$为$(K_i)_{i\in I}$的乘积结构.则当且仅当结构$K$在$E$到$A$的映射$x\mapsto (f_i(x))_{i\in I}$下存在原像时,存在关于$(A_i, K_i, f_i)_{i\in I}$的起始结构.并且,两个结构相等.
			\end{CST}
			证明:令$E$到$A$的映射$x\mapsto (f_i(x))_{i\in I}$为$f$,则$f_i=pr_i\circ f$,根据结构规则\ref{CST10}可证.
			
			\begin{CST}\label{CST16}
				\hfill\par
				令$M$为比集合论强的理论,$X$是仅有一个基集合的结构种类,$\sigma$为$X$的态射集合.令$M'$\\为比$M$强的理论,在理论$M'$中,$(A_i)_{i\in I}$、$(B_i)_{i\in I}$为集族,并且,对任意$i\in I$,令$K_i$为$X$在$A_i$\\上的结构,${K'}_i$为$X$在$B_i$上的结构,$(K_i)_{i\in I}$、$({K'}_i)_{i\in I}$的乘积结构存在,对任意$i\in I$,$f_i$为$A_i$\\到$B_i$的$\sigma$态射,则$(f_i)_{i\in I}$为$A$到$B$的$\sigma$态射.
			\end{CST}
			证明:令$f=(f_i)_{i\in I}$,$p_i$为$(A_i)_{i\in I}$的指标$i$的射影函数,$q_i$为$(B_i)_{i\in I}$的指标$i$的射影函数,则$q_i\circ f=f_i\circ p_i$,根据补充结构规则\ref{CSTcor11},$p_i$为$\sigma$态射,故$q_i\circ f$为$\sigma$态射,因此$f$为$\sigma$态射.
			
			\begin{CST}\label{CST17}
				\hfill\par
				令$M$为比集合论强的理论,$X$是仅有一个基集合的结构种类,$\sigma$为$X$的态射集合,令$M'$\\为比$M$强的理论,在理论$M'$中,集合$A$、$B$分别具有结构$K_A$、$K_B$,$K_A$和$K_B$的乘积结构为$K$.令$f$为$A$到$B$的映射,$F\subset A\times B$,$p$为$A$到$F$的双射$x\mapsto (x, f(x))$,则当且仅当$K$在$F$上导出的结构$K'$存在,且$p$为$A$到具有$K'$的$F$的同构时,$f$为$A$到$B$的$\sigma$态射.
			\end{CST}
			证明:
			\par
			充分性:令$j$为$F$到$A\times B$的规范映射,则$f=pr_2\circ j\circ p$,因此$f$是$\sigma$态射.
			\par
			必要性:令$K_F$是$K_A$通过$p$的转换在$F$上得到的结构,$j$为$F$到$A\times B$的规范映射.根据定义,$j\circ p$为$\sigma$态射,由于$p^{-1}$为同构,故$j$为$\sigma$态射.令$E$为集合,$g$为$E$到$F$的映射,且$j\circ g$为$\sigma$态射,故$pr_1\circ j\circ g$为$\sigma$态射,即$p^{-1}\circ g$为$\sigma$态射,故$g$为$\sigma$态射.综上,$K_F$是$K$在$F$上导出的结构,且$p$为$A$到具有$K_F$的$F$的同构,得证.
			
			\begin{STdef}
				\textbf{最终结构(structure finale)}
				\par
				令$M$为比集合论强的理论,$X$是仅有一个基集合的结构种类,$\sigma$为$X$的态射集合.令$M'$\\为比$M$强的理论,在理论$M'$中,$(A_i)_{i\in I}$为集族,$E$为集合,并且,对任意$i\in I$,令$K_i$为$X$在\\$A_i$上的结构,$f_i$为$A_i$到$E$的映射.对于$X$在$E$上的结构$F$,如果对任意集合$E'$,$X$在$E'$上的结构$F'$,$E$到$E'$的映射$g$,均有$(g\text{为}E\text{到}E'\text{的}\sigma\text{态射})\Leftrightarrow (\forall i)(i\in I\Rightarrow g\circ f_i\text{为}A_i\text{到}E'\text{的}\sigma\text{态射})$,则称$F$为关于三元组族$(A_i, K_i, f_i)_{i\in I}$的最终结构.
			\end{STdef}
			
			\begin{CSTcor}\label{CSTcor12}
				\hfill\par
				令$M$为比集合论强的理论,$X$是仅有一个基集合的结构种类,$\sigma$为$X$的态射集合.令$M'$\\为比$M$强的理论,在理论$M'$中,$(A_i)_{i\in I}$为集族,$E$为集合;对任意$i\in I$,令$K_i$为$X$在$A_i$上的结构,$f_i$为$A_i$到$E$的映射.如果关于三元组族$(A_i, K_i, f_i)_{i\in I}$的最终结构为$F$,则对任意$i\in I$,$f_i$为具有$K_i$的$A_i$到具有$F$的$E$的$\sigma$态射.
			\end{CSTcor}
			证明:类似补充结构规则\ref{CSTcor11}可证.
			
			\begin{CST}\label{CST18}
				\hfill\par
				令$M$为比集合论强的理论,$X$是仅有一个基集合的结构种类,$\sigma$为$X$的态射集合.令$M'$\\为比$M$强的理论,在理论$M'$中,对于集合$E$,如果存在$X$在$E$上的结构$F$,为关于三元组族$(A_i, K_i, f_i)_{i\in I}$的最终结构,那么,对任意$X$在$E$上的结构$F'$,如果对任意$i\in I$,$f_i$为具有$K_i$的$A_i$到具有$F'$的$E$的$\sigma$态射,则$F$比$F'$更细,进而,$F$是唯一的.
			\end{CST}
			证明:类似结构规则论\ref{CSTcor10}可证.
			
			\begin{CST}\label{CST19}
				\hfill\par
				令$M$为比集合论强的理论,$X$是仅有一个基集合的结构种类,$\sigma$为$X$的态射集合,令$M'$\\为比$M$强的理论,在理论$M'$中,$(A_i)_{i\in I}$为集族,$E$为集合,同时,对任意$i\in I$,令$K_i$为$X$在\\$A_i$上的结构.$(J_l)_{l\in L}$是$I$的划分,$(B_l)_{l\in L}$是集族.对任意$l\in L$,$h_l$为$B_l$到$E$的映射;对任意$l\in L$和$i\in J_l$,$g_{li}$为$A_i$到$B_l$的映射,并且$f_i=h_l\circ g_{li}$.如果,对任意$l\in L$,存在$X$在$B_l$上的结构${K'}_l$,为关于三元组族$(A_i, K_i, g_{li})_{i\in J_l}$的最终结构,则下列两个命题等价:
				\par
				第一,存在$X$在$E$上、关于$(A_i, K_i, f_i)_{i\in I}$的最终结构$U$;
				\par
				第二,存在$X$在$E$上,关于$(B_l, {K'}_l, h_l)_{l\in L}$的最终结构$U'$.
				\par
				并且,$U=U'$.
			\end{CST}
			证明:类似结构规则论\ref{CSTcor11}可证.
						
			\begin{STdef}
				\textbf{结构的直像(image direct d'une structure)}
				\par
				令$M$为比集合论强的理论,$X$是仅有一个基集合的结构种类,$\sigma$为$X$的态射集合.令$M'$\\为比$M$强的理论,在理论$M'$中,关于$(A, K, f)_{i\in \{i\}}$的最终结构,称为结构$K$在$f$下的直像.
			\end{STdef}
			
			\begin{STdef}
				\textbf{商结构(structure quotient)}
				\par
				令$M$为比集合论强的理论,$X$是仅有一个基集合的结构种类,$\sigma$为$X$的态射集合.令$M'$\\为比$M$强的理论,在理论$M'$中,$K$为$X$在$A$上的结构,$R$为在$A$上的等价关系,$f$为$A$到$A/R$\\的规范映射,如果结构$K$在$f$下的直像存在,则称其为结构$K$对于等价关系$R$的商结构.
			\end{STdef}
			
			\begin{CSTcor}\label{CSTcor13}
				\hfill\par
				令$M$为比集合论强的理论,$X$是仅有一个基集合的结构种类,$\sigma$为$X$的态射集合.令$M'$\\为比$M$强的理论,在理论$M'$中,集合$A$、$B$分别具有$X$的结构$K$、$K'$.$f$为$A$到$B$的$\sigma$态射,$R$为等价关系$f(x)=f(y)$,$h$为$A$到$A/R$的规范映射,$j$为$f\langle A\rangle $到B的规范映射.设$K$对于$R$的商结构为$K_0$,$K'$在$f\langle A\rangle$上导出的结构为${K'}_0$,令$f$的规范分解为$j\circ g\circ h$,其中$g$为$A/R$到$f\langle A\rangle$\\的双射,则$g$为具有$K_0$的$A/R$到具有${K'}_0$的$f\langle A\rangle$的$\sigma$态射.
			\end{CSTcor}
			证明:根据定义,$j\circ g$为$A/R$到$B$的$\sigma$态射,因此,$g$为$\sigma$态射.
			
			\begin{CST}\label{CST20}
				\hfill\par
				令$M$为比集合论强的理论,$X$是仅有一个基集合的结构种类,$\sigma$为$X$的态射集合.令$M'$\\为比$M$强的理论,在理论$M'$中,集合$A$、$A'$分别具有$X$的结构$K$、$K'$.$R$为在$A$上的等价关系,$R'$为$A'$上的等价关系.$K$对于$R$的商结构为$K_0$,$K'$对于$R'$的商结构为${K'}_0$.如果$f$是$A$到\\$A'$的$\sigma$态射,并且是同等价关系$R$和$R'$相容的映射,$g$为$f$对于$R$和$R'$通过商导出的映射,则$g$\\为$A/R$到$A'/R'$的$\sigma$态射.
			\end{CST}
			证明:令$h$为$A$到$A/R$的规范映射,$h'$为$A'$到$A'/R'$的规范映射,故$g\circ h=h'\circ f$,由于$h'$、$f$为$\sigma$态射,故$g\circ h$为$\sigma$态射,根据定义,$g$为$\sigma$态射.
			
			\begin{CST}\label{CST21}
				\hfill\par
				令$M$为比集合论强的理论,$X$是仅有一个基集合的结构种类,$\sigma$为$X$的态射集合.令$M'$\\为比$M$强的理论,在理论$M'$中,$A$为集合,$K$为$X$在$A$上的结构.$R$、$S$为在$A$上的等价关系,$S$比$R$更细,$K$对于$R$的商结构为$K'$,当且仅当$K$对于$S$的商结构$K_0$存在,并且,具有$K_0$的\\$A/S$到具有$K''$的$(A/R)/(S/R)$的规范映射为同构时,$K'$对于$S/R$的商结构$K''$存在.
			\end{CST}
			证明:令$j$为$A$到$A/R$的规范映射,$k$为$A/R$到$(A/R)/(S/R)$的规范映射.根据结构规则\\\ref{CST19},$K''$为$K'$对于$S/R$的商结构,等价于$K''$为$(A, K, k\circ j)_{i\in \{i\}}$的最终结构.
			\par
			如果$K_0$存在,且$A/S$到 $(A/R)/(S/R)$的规范映射为同构,根据定义,$K''$为$(A, K, k\circ j)_{i\in \{i\}}$的最终结构.反过来,如果$K''$为$(A, K, k\circ j)_{i\in \{i\}}$的最终结构,令$g$为$A/S$到\\$(A/R)/(S/R)$的规范映射,则$K_0=g^{-1}(K'')$,则$g$为同构,并且,根据定义,$K_0$为$K$对于$S$的商结构,得证.

			\begin{exer}\label{exer199}
				\hfill\par
				令$S$为符号$P$、$P^-$、$X$、$X^-$、$x_1$、$x_2$、$\cdots$、$x_n$组成的集合,$P$和$P^-$的权重为$1$,$X$和$X^-$\\的权重为$2$,其他符号的权重为$0$.
				\par
				对于$L_0(S)$的单词$A$,按下列方式定义$A$的变异数:
				\par
				令任意字母$x_i$、$P$、$X$的变异数为$0$,$P^-$、$X^-$的变异数为1;
				\par
				如果$A$的符号有偶数个变异数为$1$的符号,则$A$的变异数为$0$,否则$A$的变异数为1.
				\par
				满足下列条件之一的平衡单词$A$,称为有符号阶梯类:
				\par
				第一,$A$为符号$x_i$;
				\par
				第二,$A$为$fA_1A_2\cdots A_p$的形式,其中$p=1$或者$p=2$,$A_i$($i\in [1, p]$)均为平衡单词且为符号阶梯类,同时,如果$f=X$,则$A_1$、$A_2$的变异数均为$0$,如果$f=X^-$,则$A_1$、$A_2$的变异数均为$1$.
				\par
				符号阶梯类的变异数为$0$的,称为协变的,变异数为$1$的,称为逆变的.
				\par
				将有符号阶梯类中的所有$P^-$替换为$P$,$X^-$替换为$X$,得到平衡单词$A^*$,$A^*$在$E_1$、$E_2$、$\cdots$、$E_n$上的实现,称为有符号阶梯类$A$在$E_1$、$E_2$、$\cdots$、$E_n$上的实现,记作\\$A(E_1, E_2, \cdots, E_n)$.
				\par
				令$E_1$、$E_2$、$\cdots$、$E_n$、${E'}_1$、${E'}_2$、$\cdots$、${E'}_n$为集合,对于$i\in [1, n]$,$f_i$为$E_i$到${E'}_i$的映射,求证:
				\par
				对任意有符号阶梯类$S$,按照下列规则,任何$\{f_1, f_2, \cdots, f_n\}^S$都是某个确定的映射:
				\par
				第一,如果$S$是协变的,$\{f_1, f_2, \cdots, f_n\}^S$是$S(E_1, E_2, \cdots, E_n)$到$S({E'}_1, {E'}_2, \cdots, {E'}_n)$的映射,如果$S$是逆变的,$\{f_1, f_2, \cdots, f_n\}^S$是$S({E'}_1, {E'}_2, \cdots, {E'}_n)$到$S(E_1, E_2, \cdots, E_n)$的映射;
				\par
				第二,如果$S$是$x_i$,$\{f_1, f_2, \cdots, f_n\}^S$是$f_i$;
				\par
				第三,如果$S$是$PT$(或$P^-T$)的形式,则$\{f_1, f_2, \cdots, f_n\}^S$是$\{f_1, f_2, \cdots, f_n\}^T$在子集上的规范扩展(或在子集上的逆扩展);
				\par
				第四,如果$S$是$XUV$或$X^-UV$的形式,则$\{f_1, f_2, \cdots, f_n\}^S$是$\{f_1, f_2, \cdots, f_n\}^U$和\\$\{f_1, f_2, \cdots, f_n\}^V$在乘积集合上的规范扩展.
				\par
				$\{f_1, f_2, \cdots, f_n\}^S$称为$f_1$、$f_2$、$\cdots$、$f_n$和有符号阶梯类$S$对应的规范扩展.对于$i\in [1, n]$,$f_i$为$E_i$到${E'}_i$的映射,${f'}_i$为${E'}_i$到${E''}_i$的映射,那么:
				\par
				如果$S$是协变的,则$\{{f'}_1\circ f1, {f'}_2\circ f2, \cdots, {f'}_n\circ f_n\}^S=\{{f'}_1, {f'}_2, \cdots, {f'}_n\}^S\circ \\\{f_1, f_2, \cdots, f_n\}^S$.
				\par
				如果$S$是逆变的,则$\{{f'}_1\circ f1, {f'}_2\circ f2, \cdots, {f'}_n\circ f_n\}^S=\{f_1, f_2, \cdots, f_n\}^S\circ \\\{{f'}_1, {f'}_2, \cdots, {f'}_n\}^S$.
				\par
				并且,如果对任意$i\in [1, n]$,$f_i$为双射,${f'}_i$为其反函数,则$\{f_1, f_2, \cdots, f_n\}^S$和\\$\{{f'}_1, {f'}_2, \cdots, {f'}_n\}^S$均为双射且互为反函数.
				\par
				同时,令$S^*$为将$S$中的所有$P^-$替换为$P$,$X^-$替换为$X$得到的阶梯类,则如果$S$是协变的,则$\{f_1, f_2, \cdots, f_n\}^S=\langle f_1, f_2, \cdots, f_n\rangle^{S^*}$,如果$S$是逆变的,则$\{f_1, f_2, \cdots, f_n\}^S=\\\langle {f'}_1, {f'}_2, \cdots, {f'}_n\rangle^{S^*}$.
			\end{exer}
			证明:
			\par
			用数学归纳法可证任何$\{f_1, f_2, \cdots, f_n\}^S$都是某个确定的映射.
			\par
			类似结构规则\ref{CST1}可以证明:
			\par
			如果$S$是协变的,则$\{{f'}_1\circ f1, {f'}_2\circ f2, \cdots, {f'}_n\circ f_n\}^S=\{{f'}_1, {f'}_2, \cdots, {f'}_n\}^S\circ \\\{f_1, f_2, \cdots, f_n\}^S$.
			\par
			如果$S$是逆变的,则$\{{f'}_1\circ f1, {f'}_2\circ f2, \cdots, {f'}_n\circ f_n\}^S=\{f_1, f_2, \cdots, f_n\}^S\circ \\\{{f'}_1, {f'}_2, \cdots, {f'}_n\}^S$.
			\par
			类似结构规则\ref{CST2}可以证明$\{f_1, f_2, \cdots, f_n\}^S$和$\{{f'}_1, {f'}_2, \cdots, {f'}_n\}^S$均为双射.
			\par
			并且,用数学归纳法可证$\{{f'}_1\circ f1, {f'}_2\circ f2, \cdots, {f'}_n\circ f_n\}^S$是恒等对应,故$\{f_1, f_2, \cdots, f_n\}^S$\\和$\{{f'}_1, {f'}_2, \cdots, {f'}_n\}^S$互为反函数.
			\par
			根据补充定理\ref{cor78}、补充定理\ref{cor88}及数学归纳法可以证明:
			\par
			如果$S$是协变的,则$\{f_1, f_2, \cdots, f_n\}^S=\langle f_1, f_2, \cdots, f_n\rangle^{S^*}$,如果$S$是逆变的,则\\$\{f_1, f_2, \cdots, f_n\}^S=\langle {f'}_1, {f'}_2, \cdots, {f'}_n\rangle^{S^*}$.
			
			\begin{exer}\label{exer200}
				\hfill\par
				令$M$为比集合论强的理论,$S$是在$n+m$个字母上的有符号阶梯类,$X$是结构种类,其主要基集合是$x_1$、$x_2$、$\cdots$、$x_n$,辅助基集合是$A_1$、$A_2$、$\cdots$、$A_m$,其代表特征为$s\in \mathcal{P}(S(x_1, \\x_2, \cdots, x_n, A_1, A_2, \cdots, \\A_n))$.
				\par
				令$M'$为比$M$强的理论,在理论$M'$中,$U$为$X$在主要基集合$E_1$、$E_2$、$\cdots$、$E_n$上的结构,$U'$为$X$在主要基集合${E'}_1$、${E'}_2$、$\cdots$、${E'}_n$上的结构,对于$i\in [1, n]$,$f_i$为$E_i$到${E'}_i$的映射.
				\par
				求证:按照下列条件定义的$(f_1, f_2, \cdots, f_n)$是态射:
				\par
				第一,如果$S$是协变的,则$\langle f_1, f_2, \cdots, f_n, Id_{A_1}, Id_{A_2}, \cdots, Id_{A_m}\rangle ^S(U)\subset U'$;
				第二,如果$S$是逆变的,则$\langle f_1, f_2, \cdots, f_n, Id_{A_1}, Id_{A_2}, \cdots, Id_{A_m}\rangle ^S(U')\subset U$.
				\par
				并且,可以通过适当的选择,定义偏序结构、代数结构和拓扑结构的态射.
			\end{exer}
			证明:
			\par
			根据定义可证$(f_1, f_2, \cdots, f_n)$的集合符合态射集合的三个条件.
			\par
			对于偏序结构,单增映射为态射;对于代数结构,令$A$、$A'$上的合成运算分别为$p$、$p'$,如果$p'(f(x), f(y))=f(p(x, y))$,则$f$为态射;对于拓扑结构,令$A$、$A'$上的拓扑分别为$V$、$V'$,如果$X'\in V'\Rightarrow f^{-1}(X')\in V$,则$f$为态射.
			
			\begin{exer}\label{exer201}
				\hfill\par
				令$M$为比集合论强的理论,$X$是仅有一个基集合的结构种类,$\sigma$为$X$的态射集合.令$M'$\\为比$M$强的理论,在理论$M'$中,集合$A$、$B$、$C$都具有结构种类$X$的结构,$A$到$B$映射$f$是满态射,$B$到$C$的映射$g$是态射,$g\circ f$是同构,求证:$f$、$g$都是同构.
			\end{exer}
			证明:类似定理\ref{theo21}(3)、定理\ref{theo21}(4)、定理\ref{theo21}(6)可证.
			
			\begin{exer}\label{exer202}
				\hfill\par
				令$M$为比集合论强的理论,$X$是仅有一个基集合的结构种类,$\sigma$为$X$的态射集合.令$M'$\\为比$M$强的理论,在理论$M'$中,集合$A$、$B$、$C$、$D$都具有结构种类$X$的结构,$A$到$B$映射$f$、$B$到$C$的映射$g$、$C$到$D$的映射$h$都是态射,$g\circ f$、$h\circ $g都是同构,求证:$f$、$g$、$h$都是同构.
			\end{exer}
			证明:类似习题\ref{exer51}可证.
			
			\begin{exer}\label{exer203}
				\hfill\par
				令$M$为比集合论强的理论,$X$是仅有一个基集合的结构种类,$\sigma$为$X$的态射集合.令$M'$\\为比$M$强的理论,在理论$M'$中,集合$A$具有结构种类$X$的结构$U$,集合$B$具有结构种类$X$的结构$U'$,$f$为$A$到$B$的态射,$g$为$B$到$A$的态射,令$M=\{x|x\in A\text{与}g(f(x))=x\}$,$N=\{y|y\in B\text{与}f(g(y))=y\}$,$M$具有$U$在$M$上导出的结构,$N$具有$U'$在$N$上导出的结构,求证:$M$同构于$N$.
			\end{exer}
			证明:类似习题\ref{exer84}可证.
			
			\begin{exer}\label{exer204}
				\hfill\par
				结构种类$X$的主要基集合为$A$,辅助基集合为$k$,代表特征为$s\in (\mathcal{P}(A\times A\times A)\times \mathcal{P}(A\times A\times A)\times \mathcal{P}(k\times A\times A))\times \mathcal{P}(A)$.公理为:$(pr_1s\text{为有单位元的}k\text{代数结构})\text{与}(pr_2s\text{为不可约理想})$.集合$A$、$A'$分别具有结构种类$X$的结构$(F, H)$、$(F', H')$,如果$f$是$A$到$A'$的$\sigma$态射,并且,其将单位元映射为单位元,同时$f(H)\subset H'$,则称$f$为$k$代数同态.在$A$上的$X$的结构的集合,按$(K_1\text{为}X\text{在集合}E\text{上的结构})\text{与}(K_1\text{为}X\text{在集合}E\text{上的结构})\text{与}(K_1\text{比}K_2\text{更细})$排序.
				\par
				试给出$X$在$A$上的结构族$(S_i)$,使$(S_i)$的最小上界存在,但不是$(A_i, S_i, Id_i)$的起始结构,其中$A_i$是具有结构$S_i$的集合$A$,$Id_i$是$A$到$A_i$的规范映射.
				\par
				同时,试给出$X$在$A$上的结构族$(S_i)$,使$(S_i)$的最大下界存在,但不是$(A_i, S_i, Id_i)$的最终结构,其中$A_i$是具有结构$S_i$的集合$A$,$Id_i$是$A_i$到$A$的规范映射.
			\end{exer}
			答:对于前半段,考虑多项式环$A=k[T]$,则$A$的不可约理想是极大理想的幂.令$F$为$k$\\代数结构,$p$、$q$为$A$的不同极大理想.考虑$X$在$A$上的结构$A_p=(F, p)$、$A_q=(F, q)$,其最小上界是$(F, (0))$.令$B= p\cap q$,考虑$B$到$A$的映射$f=(x\mapsto x)$,其不是同态,但$f\circ Id_p$、$f\circ Id_q$是同态,故$(F, (0))$不是起始结构.
			对于后半段,$A$的对偶空间和结构$A_p$、$A_q$的转置,符合题目条件.
			\par
			注:习题\ref{exer204}涉及尚未介绍的“代数结构”知识.
			
			\begin{exer}\label{exer205}
				\hfill\par
				结构种类$X$的主要基集合为$A$,辅助基集合为实数集$R$,代表特征为$s\in \mathcal{P}(\mathcal{P}(A))\times \mathcal{P}(R\times A)$.公理为:$(\forall V')((V'\subset pr_1s)\Rightarrow ((\bigcup\limits_{X\in V'}X)\in pr_1s))\text{与}(A\in pr_1s)\text{与}(\forall X)(\forall Y)((X\in pr_1s\text{与}Y\in pr_1s)\Rightarrow (X\bigcap\limits_Y\in pr_1s))\text{与}(\text{存在实数}a>0\text{,使}pr_2s\text{为区间}[0, a]\text{到}A\text{对拓扑}pr_1s\\\text{的连续单射的函数图})$.集合$A$、$A'$分别具有结构种类$X$的结构$(V, f)$、$(V', f')$,定义$A$到$A'$的\\$\sigma$态射为:$g$为对$V$、$V'$的连续映射,并且$g$的函数图$F$满足$F\circ f\subset f'$,则称$g$为$\sigma$态射.
				\par
				求证:该态射可以通过习题\ref{exer199}的方式来定义;并且,对具有任意$X$的结构的集合$A_1$、$A_2$,存在在$A_1\times A_2$上的乘积结构.
				\par
				同时,试给出一个例子,在$A_1\times A_2$上的乘积结构在$pr_1$下的直像所具有的结构,不是在$A_1$上原本的结构.
			\end{exer}
			证明:
			\par
			根据定义可证该态射可以通过习题\ref{exer199}的方式来定义.
			\par
			令$A_1$具有结构$(V_1, f_1)$,$f_1$的定义域为$[0, a_1]$,$A_2$具有结构$(V_2, f_2)$,$f_2$的定义域为$[0, a_2]$,则结构$(V_1\times V_2, f)$为其乘积结构,其中$f$为$x\mapsto (f_1(x), f_2(x))(x\in [0, a_1]\cap[0, a_2])$.
			\par
			如果$a_1>a_2$,则直像的第二射影,是$f_1$在$[0, a_2]$上的限制.
			\par
			注:习题\ref{exer205}涉及尚未介绍的“拓扑”知识.
			
			\begin{exer}\label{exer206}
				\hfill\par
				结构种类$X$的主要基集合为$A$,代表特征为$s\in \mathcal{P}(\mathcal{P}(A))\times A\times A$,公理为:$(\forall V')((V'\subset pr_1pr_1s)\Rightarrow ((\bigcup\limits_X\in V'X)\in pr_1pr_1s))\text{与}(A\in pr_1pr_1s)\text{与}(\forall X)(\forall Y)((X\in pr_1pr_1s\text{与}Y\in pr_1pr_1s)\Rightarrow (X\bigcap\limits_Y\in pr_1pr_1s))\text{与}prr2pr_1s\neq pr_2s$.集合$A$、$A'$分别具有结构种类$X$的结构\\$(V, a, b)$、$(V', a', b')$,定义$A$到$A'$的$\sigma$态射为:$f$为对$V$、$V'$的连续映射,并且$f(a)=a'$,$f(b)\\=b'$, 则称$f$为$\sigma$态射.
				\par
				求证:该态射可以通过习题\ref{exer199}的方式来定义;并且,对分别具有任意$X$的结构$F$、$F'$的集合$A$、$B$,存在在$A\times B$上的乘积结构,并且,该乘积结构在$pr_1$(或$pr_2$)下的直像所具有的结构,为$A$(或$B$).同时,同时,试给出$pr_1$的截面不存在,但存在$A$到$A\times B$的$\sigma$态射的例子.
			\end{exer}
			证明:
			\par
			根据定义可证该态射可以通过习题\ref{exer199}的方式来定义.
			\par
			令$A$的结构为$(V, a, b)$,$B$的结构为$(V', a', b')$,则$(V\times V', (a, a'), (b, b'))$为其乘积结构.
			\par
			令$A$为联通空间,$B$为离散空间.则$pr_1$的截面不存在,但存在$A$到$A\times B$的$\sigma$态射.
			\par
			注:习题\ref{exer206}涉及尚未介绍的“拓扑”知识.
			
			\begin{exer}\label{exer207}
				\hfill\par
				$X$为域结构种类,
				\par
				求证:
				\par
				可以这样定义$\sigma$态射:$f$或者为$K$到$K'$的群同态,或者为$f_0$,其中$f_0(0)=0$,$f_0(x)=1$($x\neq 0$).
				\par
				并且,该态射具有以下性质:对任何域$K$,存在$K$的结构在$\{0, 1\}$导出的结构(同构于$F_2$),令$R$为等价关系,其等价类为$\{0\}$和$K^*$,其中$K^*=K-\{0\}$,存在$K$的结构对于$R$的商结构(同构于$F_2$).
			\end{exer}
			证明:根据定义可证.
			\par
			注:习题\ref{exer207}涉及尚未介绍的“域”知识.
			
			\begin{exer}\label{exer208}
				\hfill\par
				$X$为有序阿基米德完全域结构种类.对任意具有$X$的结构的集合$A$,令$g_A$为$A$到$R$唯一同构.$A$、$B$为具有$X$的结构的两个集合,
				\par
				求证:可以按下列方式定义$A$到$B$的$\sigma$态射:
				\par
				对任意$x\in A$,均有$g_B(f(x))\geq g_A(x)$,则称$f$为$\sigma$态射.
				\par
				并且,尽管结构种类$X$是统一的,但存在不同构的双射态射.
			\end{exer}
			证明:
			\par
			根据定义可证$\sigma$符合态射集合的条件.
			\par
			令$k$为映射$x\mapsto a_x$(当$x\geq 0$时,$a=2$,当$x<0$时,$a=1/2$),$f=g_B^{-1}\circ k\circ g_A$,则$f$为$A$到$B$的双射态射,但不是同构.
			\par
			注:习题\ref{exer208}涉及尚未介绍的“域”知识.
			
			\begin{exer}\label{exer209}
				\hfill\par
				在理论$M$中,结构种类$X$只有一个主要基集合,其代表特征为$s\in F(x)$,公理为$R$.$A(x)$为$X$在$x$上的结构的集合.$\sigma$为项,其符合态射的前两个条件,并符合下列条件:
				\par
				令$M'$为比$M$强的理论,在理论$M'$中,$E$、$E'$分别是具有$X$的结构$K$、$K'$的集合,$E$、$E'$、$K$、$K'$都不含字母$s$、$t$、$x$、$y$,$f$为$E$到$E'$的双射,则$((f)\text{为同构})\Rightarrow f\in (K'|t)(K|s)\\(E'|y)(E|x)\sigma$.求证:
				\par
				(1)$s\in A(x)\text{与}t\in A(x)\text{与}Id_x\in (x|y)\sigma\cap(x|y)(s|p)(t|s)(p|t)\sigma$(其中$\sigma$不含字母$p$)是在$A$上关于$s$、$t$的等价关系.
				\par
				(2)令$B(x)$为商集$A(x)/q$,$g_x$为$A(x)$到$B(x)$的规范映射.假设$s'\in B(x)$是可转换的,$W$为结构种类,其代表特征为$s'\in \mathcal{P}(F(x))$,公理为$s'\in B(x)$.令$\sigma'$为满足下列条件的$x$到$y$\\的映射的集合:$s'\in B(x)\text{与}y'\in B(y)$,并且,存在$s\in A(x)$、$t\in B(x)$,使$s'=g_x(s)$、$t'=g_y(t)$、$f\in \sigma$.求证:对于结构种类$W$,$\sigma'$为态射,并且$\sigma\subset \sigma'$.
			\end{exer}
			证明:
			\par
			(1)根据定义可证.
			\par
			(2)根据定义可证.
			
		\section{普遍性映射(Applications universelles)}		
			\begin{STdef}
				\textbf{到具有结构的集合的映射(application dans un ensemble muni d'une structure)}
				\par
				令$M$为比集合论强的理论,$X$是仅有一个基集合的结构种类,$\sigma$为$X$的态射集合,$E$为项.在理论$M_X$中,$s$为通用结构,如果项$\alpha$满足下列条件,则$(F|x)(K|s)\alpha$的元素,称为$E$到具有$K$的$F$的$\alpha$映射:
				\par
				第一,在理论$M_X$中,$\alpha\subset \mathcal{F}(E; x)$;
				\par
				第二,令$M'$为比$M$强的理论,在理论$M'$中,$K$、$K'$分别为$X$在$F$、$F'$上的结构,如果$f$是$F$到$F'$的$\sigma$态射,则$g\in (F|x)(K|s)\alpha\Rightarrow f\circ g\in (F'|x)(K'|s)\alpha$.
			\end{STdef}
			
			\begin{STdef}
				\textbf{具有普遍性的集合和映射(ensemble et application universels),普遍性映射问题的解(solution du problème d'application universelle)}
				\par
				令$M$为比集合论强的理论,$X$是仅有一个基集合的结构种类,$\sigma$为$X$的态射集合,$E$为项.令$M'$为比$M$强的理论,在理论$M'$中,$f$为$E$到具有$K$的$F$的$\alpha$映射,如果对任意具有$X$的任意结构的集合$G$和任意$E$到$G$的$\alpha$映射$g$,存在唯一的$F$到$G$的$\sigma$态射$h$,使$g=h\circ f$.则称集合$F$和$\alpha$映射$f$具有普遍性.有序对$(F, f)$称为关于$X$、$\sigma$、$\alpha$对$E$的普遍性映射问题的解,在没有歧义的情况下也可以简称为对$E$的普遍性映射问题的解.
			\end{STdef}
			
			\begin{CSTcor}\label{CSTcor14}
				\hfill\par
				令$M$为比集合论强的理论,$X$是仅有一个基集合的结构种类,$\sigma$为$X$的态射集合,$E$为项.令$M'$为比$M$强的理论,在理论$M'$中,$(F', f')$和$(F'', f'')$都是对$E$的普遍性映射问题的解,则存在$F'$到$F''$的同构$g$,令其逆同构为$g^{-1}$,使$f'=g^{-1}\circ f''$,$f''=g\circ f'$.
			\end{CSTcor}
			证明:根据定义,存在映射$h_1$、$h_2$,使$f'= h_1\circ f''$,$f''= h_2\circ f'$.因此,$h_1\circ h_2=Id_{F'}$, $h_2\circ h_1=Id_{F''}$,根据结构规则\ref{CST8}可证.
			
			\begin{CSTcor}\label{CSTcor15}
				\hfill\par
				令$M$为比集合论强的理论,$X$是仅有一个基集合的结构种类,$\sigma$为$X$的态射集合,$E$为项.令$M'$为比$M$强的理论,在理论$M'$中,$F$为具有$X$的结构的集合.$f$为$E$到$F$的$\alpha$映射,则当且仅当$(F, f)$满足下列两个条件时,其为对$E$的普遍性映射问题的解:
				\par
				第一,对任意集合$G$和任意$E$到$G$的$\alpha$映射$g$,存在$F$到$G$的$\sigma$态射$h$,使$g=h\circ f$;
				\par
				第二,对任意集合$G$,$F$到$G$的任何两个$\sigma$态射,如果在$f\langle E\rangle$上重合,则相等.
			\end{CSTcor}
			证明:
			\par
			充分性根据定义可证.
			\par
			必要性:如果$(F, f)$为对$E$的普遍性映射问题的解,对任意集合$G$,$F$到$G$的任何两个$\sigma$态射$h$、$h'$,如果在$f\langle E\rangle$上重合,则$h\circ f=h'\circ f$,根据定义,$h=h'$,得证.
			
			\begin{CST}\label{CST22}
				\hfill\par
				令$M$为比集合论强的理论,$X$是仅有一个基集合的结构种类,$\sigma$为$X$的态射集合,$E$为项.令$M'$为比$M$强的理论,则在理论$M'$中,满足下列三个条件时,对$E$的普遍性映射问题的解存在:
				\par
				第一,对$X$在任意集族上的结构族,乘积结构存在;
				\par
				第二,令$(F_i)_{i\in I}$为集族,对任意$i\in I$,$f_i$为$E$到$F_i$的$\alpha$映射,则$E$到$\prod\limits_{i\in I}F_i$的映射$(f_i)_{i\in I}$也是$\alpha$映射;
				\par
				第三,存在具有以下性质的基数$m$:对任意集合$F$和$E$到$F$的$\alpha$映射,存在$F$的可采子集$G$,满足$f\langle E\rangle\subset G$、$Card(G)\leq m$、$f$通过$F$的子集$G$导出的映射也是$\alpha$映射并且任何两个以$G$为定义域的$\sigma$态射只要在$f\langle E\rangle$上重合则相等.
			\end{CST}
			证明:
			\par
			令$s\in S(x)$为$X$的类型化,$L$是符合下列条件的三元组$(C, Q, P)$的集合:
			\par
			$C\subset m$,$Q$是$X$在$C$上的结构,$P$是$E$到具有$Q$的$C$的$\alpha$映射的图.
			对任意$l\in L$,令$l=(C_l, Q_l, P_l)$,$f_l$为映射$(P_l, E, C_l)$,令$F=\prod\limits_{l\in L}X_l$,$f$为$x\mapsto (f_l(x))_{l\in L}$,因此$f$为$\alpha$映射.
			\par
			对任意$E$到$H$的映射$h$,令$G$为满足第三个条件的集合,$j$为$G$到$H$的规范映射,$g$为$E$到\\$G$的映射并且其图和$h$相等,则$h=j\circ g$,故$g$是$E$到$G$的$\alpha$映射.令$G'\subset m$,并和$G$等势,令$k$为$G$到$G'$的双射.因此,存在$l$,使$X_l=G'$.故$k\circ g= f_l$,$q=j\circ k^{-1}\circ pr_l$,进而$h=q\circ f$,因而补充结构规则\ref{CSTcor15}的第一个条件成立.进而,根据第三个条件,补充结构规则\ref{CSTcor15}的第二个条件成立,得证.
			
			\begin{CST}\label{CST23}
				\hfill\par
				令$M$为比集合论强的理论,$X$是仅有一个基集合的结构种类,$\sigma$为$X$的态射集合,$E$为项.令$M'$为比$M$强的理论,在理论$M'$中,$(F, f)$为关于$X$、$\sigma$、$\alpha$对$E$的普遍性映射问题的解,则当且仅当对任意$x\in E$、$y\in E$,均存在具有$X$的结构的$G$以及$E$到$G$的$\alpha$映射$h$使$h(x)\\\neq h(y)$时,$f$为单射.
			\end{CST}
			证明:根据定义可证.
						
			\begin{STdef}
				\textbf{分开元素的映射(application qui sépare les éléments)}
				\par
				令$M$为比集合论强的理论,$X$是仅有一个基集合的结构种类,$\sigma$为$X$的态射集合,$E$为项.令$M'$为比$M$强的理论,在理论$M'$中,$(F, f)$为关于$X$、$\sigma$、$\alpha$对$E$的普遍性映射问题的解,如果$f$为单射,则称$f$为分开$E$的元素的映射.
			\end{STdef}

			\begin{exer}\label{exer210}
				\hfill\par
				$E$为拓扑空间,结构种类$X$和态射按照习题\ref{exer205}或者习题\ref{exer206}定义.$\alpha$映射为$E$到具有结构种类$X$的某个集合的连续映射.求证:不存在$E$的普遍性映射问题的解.
			\end{exer}
			证明:
			\par
			按照习题\ref{exer205}定义的情况下,令$A$具有结构$(V, f)$,其中$f$为$[0, a]$到$A$的映射,设$(A', k)$为\\$A$的普遍性映射问题的解,$A'$具有结构$(V', f')$,其中$f'$为$[0, a']$到$A'$的映射.考虑带有结构\\$(V'', f'')$的集合$A''$,和$A$到$A''$的连续映射$g$,其中$f''$为$[0, a'']$到$A'$的映射且不是满射,$g(0)\notin f''([0, a''])$.设$h$为使$g=h\circ k$的态射,则$h\circ k(A)\subset f''([0, a''])$,矛盾.
			\par
			按照习题\ref{exer206}定义的情况下,令$A$具有结构$(V, a, b)$,设$(A', f)$为$A$的普遍性映射问题的解,$A'$具有结构$(V', a', b')$.考虑任意带有结构$(V'', f'')$的集合$A''$,和$A$到$A''$的连续映射$g$,如果$f(a)=f(b)$,令$g(a)\neq g(b)$,如果$f(a)\neq f(b)$,令$g(a)=g(b)$,则均不存在态射$h$使$g=h\circ f$,矛盾.
			\par
			注:习题\ref{exer210}涉及尚未介绍的“拓扑”知识.
			
			\begin{exer}\label{exer211}
				\hfill\par
				$E$为交换域,$X$为代数闭交换域结构种类.定义$\sigma$态射为同态,$\alpha$映射为$E$到代数闭域的同态.$F_E$为$E$的代数闭包.求证:$E$到$F_E$的规范单射,符合普遍性的映射问题的存在性条件,但不存在$E$的普遍性的映射问题的解.
			\end{exer}
			证明:根据定义可证.
			注:习题\ref{exer211}涉及尚未介绍的“交换域”知识.
			
			\begin{exer}\label{exer212}
				\hfill\par
				$X$为结构种类,$(A_i)_{i\in I}$为两两不相交的集族,对任意$i\in I$,$K_i$为$X$在$A_i$上的结构,$E$为\\$(A_i)_{i\in I}$的并集.定义$X$的$\sigma$态射,并定义$\alpha$为$E$到具有$X$的结构的集合$F$并符合下列条件的映射$f$的集合:
				\par
				对任意$i\in I$,$f$在$A_i$上的限制是态射.
				\par
				令$M'$为比$M$强的理论,求证:在理论$M'$中,如果$E$的普遍性映射问题的解$(F, f)$存在,并且$f$为满射,则$F$具有的结构$K$为族$(A_i, K_i, f_i)_{i\in I}$的最终结构,其中$f_i$为$f$在$A_i$上的限制.
				\par
				此外,令$G$为集合,对任意$i\in I$,$g_i$为$A_i$到$G$的映射,如果族$(A_i, K_i, g_i)_{i\in I}$的最终结构存在,则$g_i=g\circ f_i$,其中$g$为$F$到$G$的态射,$G$具有的结构是$F$具有结构在$f$下的直像.
			\end{exer}
			证明:
			\par
			对任意集合$F'$以及$F$到$F'$的映射$p$,如果$p$为态射,根据定义,对任意$i\in I$,$f_i$为态射,故$p\circ f_i$为态射;反过来,如果对任意$i\in I$,$p\circ f_i$为态射,则$p\circ f$为$\alpha$映射,则存在$F$到$F'$的态射$p'$使$p\circ f=p'\circ f$.因此$p=p'$,故$p$为态射.
			\par
			如果族$(A_i, K_i, g_i)_{i\in I}$的最终结构存在,则对任意$i\in I$,$g_i$为态射,故存在$F$到$G$的态射$g$,且$g_i=g\circ f_i$;同时,对任意$G$到$G'$的映射$q$,如果$q$为态射,则$q\circ g$为态射,反过来,如果$q\circ g$为态射,则任意$i\in I$,$q\circ g_i$为态射,故$q$为态射.
			\par
			注:原书习题\ref{exer212}遗漏“f为满射”的条件.
\end{document}